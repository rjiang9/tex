\chapter{字,行,段的格式化}
在上一章中,我们安装了LaTeX和TEXworks的编辑器来编写我们的
第一个文档。现在,我们将谈论的文档结构,我们
将重点放在文本的细节和它的格式。

在这一章中,我们将:
\begin{itemize}
	\item 探讨关于逻辑格式化
	\item 学习如何更改字体,文本的形状和风格
	\item 使用box来限制文档的宽度
	\item 学习如何断行和提高断字
	\item 探索对齐和格式化段落
\end{itemize}

过工作的例子,并尝试新的功能,我们将学习一些基本概念
乳胶。通过本章的结束,我们将命令和环境的熟悉。
你甚至可以定义自己的命令。
	\section{理解逻辑格式化}
在前面的章节中,我们写了一个小例子文件。让我们扩展它使得它变成一个具有
说明性的例子,以了解典型的文档结构。
		\subsection{Time for action-titling your document}
我们将使用第一个例子,插入一些命令使其产生一个漂亮的title。
\begin{lstlisting}[language={[LaTeX]TeX}]
\documentclass[a4paper,11pt]{article}
\begin{document}
\title{Example 2}
\author{My name}
\date{January 5, 2011}
\maketitle
\section{What's this?}
This is our second document. It contains a title and a section
with text.
\end{document}
\end{lstlisting}
\begin{itemize}
	\item 我们的文档使用article类。它将使用a4纸和11号基本字体。
	\item 标题是"Example 2".
	\item 你是作者。
	\item 时间是January 5, 2011
	\item concerning the content of the document:
		\begin{itemize}
			\item 开始与标题。
			\item 第一section具有标题"What's this?"
			\item 接下来的文本是"This is our second document."
		\end{itemize}
\end{itemize}
注意,我们并没有设置标题的字体大小;页眉有设置加粗和居中。这类的格式化是LaTeX
做的,但是你可以告诉LaTeX具体看起来效果如何。
		\subsection{探索文档结构}
让我们看看细节。一个LaTeX文档中并不是孤立的-通常是基于一个
多功能的模板。这类基本的模板被称为类(class)。它提供了可定制的
功能,通常是具有特定的目的性。有些类中有书(book),期刊文章
(journal articles);信件(letter),演示文档(presentation),海报
(poster)和其他很多;还有更多的数百个可靠的类可以在互联网存档,
在你已经安装了TeX Live之后,在您的计算机上也有很多,在这里,我们
选择了适用于小型文档的文章(article)类别。

第一行用\textbackslash documentclass开始。这个单词之前有反斜杠;这样的单词称为
命令(command)。我们使用命令制定类(class)和文档属性(document properties):
title,author和date.

文档的第一布冯称为文档的前言(preamble)。我们在这里选择类,指定属性,
一般地,在这里制作整个文档范围的定义。

\textbackslash begin\{document\}标记了前言的结束和真正文档的开始.\textbackslash end\{document\}标记了文档的结束。所有之后的东西都将被LaTeX忽略。在\textbackslash begin...\textbackslash end之间的这样一段代码,被称作环境(environment)。

在实际的文档中间,我们使用了\textbackslash maketitle来以一种美观的格式输出标题,作者和日期。
使用\textbackslash section命令,我们制作了一个比正常文字大且粗的标题。接着是一些文字,
在文档环境中的将被输出。相反的,前言从来不产生任何输出。

让我们仔细看一下这些命令。

		\subsection{理解LaTeX命令}
LaTeX命令以反斜杠开始,跟着是一串大写或小写字母。LaTeX命令通常使用一种
描述性的小写字符串命名。当然也有例外:你将会看到有些命令只有反斜杠和一个特殊
字符组成。

命令也可以有参数,含在大括号或方括号中。

调用命令格式如下:

\textbackslash command

或者:

\textbackslash command\{argument\}

或者:

\textbackslash command\[optionan argument\]\{argument\}

参数可以有多个,每个在大括号或方括号内。在大括号中的参数是必需的。如果一个
命令被定义为需要一个参数,那么就必须提供一个。例如,如果我们没有声明一个类名,
调用\textbackslash documentclass将是徒劳的,

%这些内容是大括号中的
方括号中的参数是可选的,如果不指定的话,会使用默认设置,例如\textbackslash documentclass\{article\}.
这份文档将使用10pt字体,因为这是这个类的默认值。
\textbackslash documentclass\[a4paper,11pt\]\{article\}将使用指定的属性。
有些命令是产生输出的-例如\textbackslash LaTeX 而有些命令是设置属性,更改字体,
或者布局的。通常,命令是根据目的来命名的。我们将在这一章中探究其细节,但是
这里先让我们看看LaTeX是如何处理我们输入的。
	\section{LaTeX如何读取你的输入}
在我们继续之前,让我们来看看如何LaTeX的理解你写在编辑器中的东西。
%此处有省略
	\section{输出特殊符号}
常见的文本大多含有大写和小写字母,数字和标点符号.这些只要简单的输入即可。
然而,一些字符是保留作为LaTeX的命令;它们不能被直接使用.我们已经遇到了
这样的字符,除了百分号,还有花括号等。有LaTeX命令来打印这样的符号。
Statement \#1:
50\% of \$100 makes \$50.
More special symbols are \&, \_, \{ and \}.
\textbackslash textbackslash用来输出反斜杠。\textbackslash\textbackslash
作为断行的shortcut。这有点奇怪,这是由于文章中经常需要断行而很少需要
反斜杠,所以使用了缩略命令。
%这里又省略了一点。。。
\section{格式化文本-字体形状和风格}
	\subsection{调整字体形状}
	\textbackslash emph 斜体
	\textbackslash textit 意大利斜体
	\textbackslash textbf 粗體
	\textbackslash textsl slanted
	\textbackslash textsc small caps
这些可以组合起来使用,例如 \textbackslash textsc\{\textbackslash textbf\{nested\}\}
也可以重复使用,例如\textbackslash textbf\{\textbackslash textbf\{nested\}\}
%又省略一点。。。
	\subsection{选择font family}
	\textbackslash textsf : sans-serif font
	\textbackslash texttt : typewriter font
	\textbackslash textrm : Roman text - the default font with serifs.
\textbackslash textsf\{LaTeX\ resources on the internet\}
\textbackslash texttt\{http://www.ctan.org\}.
\textsf{LaTeX\ resources on the internet}
The best place for downloading LaTeX related software is CTAN.
Its address is \texttt{http://www.ctan.org}.
	\subsection{字体切换}
	\textbackslash sffamily
	\textbackslash rmfamily
	Command		Declaration		Meaning
	\textbackslash textrm\{\}	\textbackslash rmfamily
	\textbackslash textsf\{\}
	\textbackslash texttt\{\}
	\textbackslash textbf\{\}
	\textbackslash textmd\{\}
	\textbackslash textit\{\}
	\textbackslash textsl\{\}
	\textbackslash textsc\{\}
	\textbackslash textup\{\}
	\textbackslash textnormal\{\}
	\subsection{Using environments}
	\subsection{节省事件和精力 自定义命令}
	\section{使用box来限制文字的宽度}
\section{Breaking lines and paragraphs}
通常情况下,当你写的文字,你并不需要关心自动换行。
只要输入文本与编辑LaTeX会使其适合线和照顾
的理由。如果你想开始一个新的段落,结果得到一个换行符
在输出时,只需插入一个空行,然后再继续你的文字。
现在,我们将找出如何控制自动换行。首先,我们将看到如何提高
自动断行。然后,我们将学习指令直接插入休息。
\subsection{提高断字(Imporving hyphenation)}
如果你看一下,你会发现在较长的文本,它是优秀的文字是完全有道理的
LaTeX和如何单词之间的间距被均匀的分布上的线。如果有必要,
LaTeX会划分的话,并把连字符的行结束时,为了打破中的行
更好的办法。 LaTeX的很好的算法已经使用连字符连接的词,但它可能发生
它不能找到一个可以接受的方式,将一个字。前面的例子指出了这一点
问题:打破了词的缩写,提高产量,但LaTex不知道
划分。我们将找出如何解决这个问题。
\subsection{Improving the justification further}
如今最流行的TeX编译器是pdfTeX,它直接以PDF格式输出。当xxx写了pdfTeX,他
使用micro-typographic的兼容性扩展了TeX。当我们直接输出pdf时,我们实际上
是用的pdfLaTeX,并且我们可以使用microtype包来使用它的特性。
\subsection{Breaking lines manually}
我们可能会选择结束压倒一切的全自动行。有几个命令
不同的效果。	
\subsection{Preventing line breaks}
命令\textbackslash linebreak有直接的对应:\textbackslash nolinebreak。此命令
在当前位置防止断行。像它的对手,它带有一个可选的的参数。如果你写\textbackslash nolinebreak[0],
则建议不断行。使用1,2,或3使得请求更强大和\textbackslash nolinebreak[4]完全禁止它。
如果你不提供一个参数,则使用的是后者。已经提到的命令,\textbackslash mbox[文字],
不仅要禁用断字,也避免了换行的完整文本。	

LaTeX可以在字与字之间的空间进行有意义的断行。符号〜标识词间没有空间断行:
如果你写Dr. ~Watson, 那么Rr.绝不会出现在行末。
\subsection{Managing line breaks wisely}
坏断字的文件仍然可以消失的增长,说明一些明智的断字规则将不会做任何伤害,
但可能被证明是有用的。但只使用\textbackslash\textbackslash,\textbackslash newline,
and \textbackslash linebreak 来调整您的文档的最终版本!当你还在编辑你的文字,
你不需要担心换行。他们仍然在写作过程中可能会改变。难看理由可能有所变动,
不干预变得更好。另一方面,如果你手动换行,但后来插入文本之前,其结果可能是
不必要的短行。

所以,不要浪费你的能量格式化而你正在写的文本。	
\subsection{Exploring the fine details}
印刷字体约定可能需要注意一些小细节,也有不同的破折号,且一个点周围的空间可能
会有所不同,这取决于上下文。之后的空间一些字母可能取决于以下,以至于一些字母,
甚至可能被接合到一个单一的。这种结构被称为连字。让我们来仔细看看在他们。
\section{Understanding ligatures}
	\subsection{Choosing the rightdash}
	\subsection{Setting dots}
	\subsection{Setting accents}
\section{Using special characters directly in the editor}
\section{Truning off full justification}
\section{Creating ragged-left text}
\section{Using environments for justification}
由于具有对应每个声明都有相对应的环境,我们可以在之前的文本中使用\textbackslash
begin\{centering\}...\textbackslash end\{centering\}。也可以做类似ragged-right
和ragged-left文本。这里也有一些预定义的环境做类似的事情,但是同时开始一个新的
段落。
\subsection{Time for action-centering verses}
让我们重用fragment of 诗"Annabel Lee"。这次我们将居中所有erses:
\begin{lstlisting}[language={[LaTeX]TeX}]
\documentclass{article}
\usepackage{url}
\begin{document}
\noindent This is the beginning of a poem
by Edgar Allan Poe:
\begin{center}
	\emph{Annabel Lee}
\end{center}
\begin{center}
	It was many and many a year ago,\\
	In a kingdom by the sea,\\
	That a maiden there lived whom you may know\\
	By the name of Annabel Lee
\end{center}
The complete poem can be read on
\url{http://www.online-literature.com/poe/576/}.
\end{document}
\end{lstlisting}
我们使用\textbackslash开始来避免段落缩进。\textbackslash\{center\}开始了
居中环境。它开始了一个新段落,留下一些空白to the preceding text。\textbackslash \{center\}
结束了环境。我们两次使用环境。第二次,我们插入了\textbackslash \textbackslash来
结束erses。

在center环境结束之后,有一些空白接着,下一个段落began at the left margin。

对应于ragged-right text的环境叫做flushleft,ragged-left的叫做flushright。
\section{显示引用}
想象一下你的作品中有其他作者的引用。如果只是简单的插入到文本中的话将会使得难以
阅读。一种常用的提高可读性的方式是:设置这些文本两边缩进。
	\subsection{Time for action-quoting a scientist}
	我们将引用著名物理学家的思想。
	\begin{lstlisting}[language={[LaTeX]TeX}]
	\documentclass{article}
	\begin{document}
	Niels Bohr said: ``An expert is a person who has made
	all the mistakes that can be made in a very narrow field.''
	Albert Einstein said:
	\begin{quote}
		Anyone who has never made a mistake has never tried anything new.
	\end{quote}
	Errors are inevitable. So, let's be brave trying something new.
	\end{document}
	\end{lstlisting}
首先我们使用了行内引用。产生了一个左引号;这个字符也称作backtick。产生了个右
引号。我们只是键入两个这样的符号来获得双引号。

然后我们使用quote 环境来显示引用。我们并没有开始一个新的段落,因为引用已经
设置了缩进。这就是我们在环境中前后不使用空行的原因。
\section{引用更长的文本}
当你写短的引用的时候,the quote environment 看起来非常好。然而,当你需要引用的
文本有好几段的时候,你可能会希望获得与环绕文字相通的缩进格式。The quotation
environment将会为你解决问题。
	\subsection{Time for action-quioting TeX's benefits}
\begin{lstlisting}[language={[LaTeX]TeX}]
\documentclass{article}
\usepackage{url}
\begin{document}
The authors of the CTAN team listed ten good reasons
for using \TeX. Among them are:
\begin{quotation}
	\TeX\ has the best output. What you end with,
	the symbols on the page, is as useable, and beautiful,
	as a non-professional can produce.
	\TeX\ knows typesetting. As those plain text samples
	show, \TeX's has more sophisticated typographical algorithms
	such as those for making paragraphs and for hyphenating.
	\TeX\ is fast. On today's machines \TeX\ is very fast.
	It is easy on memory and disk space, too.
	\TeX\ is stable. It is in wide use, with a long history.
	It has been tested by millions of users, on demanding input.
	It will never eat your document. Never.
\end{quotation}
The original text can be found on
\url{ http://www.ctan.org/what_is_tex.html}.
\end{document}
\end{lstlisting}
在这里,我们使用了quotation 环境来显示一些段落。就像在正常文本中一样,使用空行
分割段落。它们使用左缩进,就像在我们的正文中一样。

但是如果我们不希望缩进呢?让我们看看解决方案。
\subsection{Time for action-spacing between paragraphs instead of indentation}
我们希望避免段落缩进,相反的 ,我们使用一些垂直空白来分割段落。
\begin{lstlisting}[language={[LaTeX]TeX}]
\documentclass{article}
\usepackage{parskip}
\usepackage{url}
\begin{document}
The authors of the CTAN team listed ten good reasons
for using \TeX. Among them are:
\TeX\ has the best output. What you end with,
the symbols on the page, is as useable, and beautiful,
as a non-professional can produce\ldots
The original text can be found on
\url{ http://www.ctan.org/what_is_tex.html}.
\end{document}
\end{lstlisting}
第二行显示了我们加载了parskip包。它的唯一目的是完全消除段落缩进。同时,这个包
引进了段落之间的skip。但是这个包并不影响quotation 环境的定义-你仍然可以使用
quote环境。
为了区别段落,有两种方法可供选择。一种是缩进每个段落的开头;这个LaTeX的默认
风格。另一种是在段落之间插入垂直空白wihle omitting the indentation,这在那些
特别窄的列缩进会话费太多width的地方是比较适合的。
\section{Pop quiz-lines and paragraphs}
\section{Summary}
在这一章中,我们学习了一些基本技能:编辑、排列和格式化文本。

特别地,涵盖了:

更改字体的形状和风格
断行和提高断字
Controling justification of text

我们了解了基本的LaTeX概念:
命令和声明,必须参数和可选参数
定义新命令
使用环境
使用包,如何加载和包的选项

需要记住一点是,我们直接在文本中直接使用了格式化命令,你需要在在前言中命令定义
使用它们以利于将来修改。在你的学习和写作过程中,你可能会知道更多的命令和包可以
提高你之前写的命令。
我们学到了常用的技能:
经可能多地写自己的macros to archieve a logical structure. 收益将会是整个文档
都变得容易修改。
Deal with line or page breaking issues at the earliest when you go for your
final version.

现在既然我们学到了格式化文本的细节,我们就准备好进入下一章处理格式化和布局整个
页面和文档。
