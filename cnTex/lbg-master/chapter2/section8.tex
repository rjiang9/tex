\section{Truning off full justification}
	\subsection{Time for action-justifying a paragraph to the left}
	\subsection{Creating ragged-left text}
	\subsection{Time for action-centering a title}
	\subsection{Using environments for justification}
由于具有对应每个声明都有相对应的环境,我们可以在之前的文本中使用
\textbackslash begin\{centering\}...\textbackslash end\{centering\}。
也可以做类似ragged-right和ragged-left文本。这里也有一些预定义的环境
做类似的事情,但是同时开始一个新的段落。
	\subsection{Time for action-centering verses}
让我们重用fragment of 诗"Annabel Lee"。这次我们将居中所有erses:
\begin{lstlisting}[language={[LaTeX]TeX}]
\documentclass{article}
\usepackage{url}
\begin{document}
\noindent This is the beginning of a poem
by Edgar Allan Poe:
\begin{center}
	\emph{Annabel Lee}
\end{center}
\begin{center}
	It was many and many a year ago,\\
	In a kingdom by the sea,\\
	That a maiden there lived whom you may know\\
	By the name of Annabel Lee
\end{center}
The complete poem can be read on
\url{http://www.online-literature.com/poe/576/}.
\end{document}
\end{lstlisting}
我们使用\textbackslash 开始来避免段落缩进。\textbackslash \{center\}开始了
居中环境。它开始了一个新段落,留下一些空白to the preceding text。\textbackslash \{center\}
结束了环境。我们两次使用环境。第二次,我们插入了\textbackslash \textbackslash
来结束erses。

在center环境结束之后,有一些空白接着,下一个段落began at the left margin。

对应于ragged-right text的环境
叫做flushleft,
ragged-left的叫做flushright。

