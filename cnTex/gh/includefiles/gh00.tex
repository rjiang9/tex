
\chapter*{前\hspace{\ccwd} 言}
\begin{fzliukai}
1942 年冬春之际,在中共战时首府延安,随即在中共领导的各抗日
根据地,开始了一幕延续多年的大规模政治运动——整风运动,由于这
场运动是以延安为中心,又以在延安开展的运动最为典型,史称「延安
整风运动」。

延安整风运动是中共历史上进行的第一次全党范围的政治运动,这
个运动是和毛泽东的名字紧密联系在一起的,这是一场由毛泽东直接领
导,包括诸多方面内容的党内整肃和重建的运动,它包含:

党内上层的斗争与党的中央权力机构的改组;

全党的思想改造;

审查干部的历史和「肃反」;

新制度的创设。

在上述几个方面中,党内上层的斗争和领导机构的改组始终处于中
心地位。

延安整风运动发端于 1942 年初,但是它的真正起步却在 1942 年以前
的很长时间就已经开始。它最初表现为 1935 年遵义会议後至 1937 年间,
毛泽东运用其在中共领导层中所获得的相对优势地位对中共政策及领导
机构作出的局部调整,这种局部调整在 1938 年中共六届六中全会後迅速
转变为毛泽东对中共政治路线、组织机构、精神气质等方面所进行的一
系列重大改变。 1938 年秋在延安举行的中共六届六中全会对于毛泽东具
有决定性的意义,这次会议,将毛泽东于 1935 年後在军权、党权方面的
权力扩张予以合法化,
使毛泽东在中共核心层中的地位得到极大的加强。
从 1938 年未至 1941 年秋,是毛泽东操纵局势演变、并使其党内对手日益
虚弱的权力再扩张的重要阶段。
这个过程在 1941 年 9 月政治局扩大会议
上达到高潮,以毛泽东当面向王明发起挑战,并获得全胜而告结束。

在多年精心准备的基础上,延安整风的大幕终于在 1942 年初拉开。延
安整风运动是毛泽东运用其政治优势,彻底改组中共上层机构,重建以毛
为绝对主宰的上层权力再分配的过程。同时,延安整风运动又是毛泽东以
自己的理念和思想,彻底转换中共的「俄化」气质,将中共改造成为毛泽
东的中共的过程。

毛泽东在整风中运用他所创造的思想改造和审干、肃反两种手段,沉
重打击了党内存留的五四自由民主思想的影响和对苏俄盲目崇拜的气氛,
不仅完成了党的全盘毛泽东化的基础工程,而且还建立起一整套烙有毛泽
东鲜明个人印记的中共新传统,其一系列概念与范式相沿成习,在 1949 年
後改变了几亿中国人的生活和命运。

经过多年的斗争,毛泽东改变了他原先在中共领导层内孤立的处境,
在延安整风运动中,毛泽东与刘少奇的政治结合得到进一步的巩固,在刘
少奇的全力支持下,毛泽东使中共核心层完全接受了他的主张和他的至高
无上的个人权威。造成历史上毛泽东一度「孤立」的原因,是他的有别于
莫斯科正统理论之「异端」想法和他个人的专断性格。经过实践证明,毛
泽东在军事战略方面的「异端」主张,大大有利于中共实力的扩展,这种
结果,迫使中共党内的亲莫斯科派向毛泽东输诚,同时,也将中共高级军
事将领重新聚集在自己的周围。毛泽东的专断性格最早暴露于 1930-1931
年由他亲自参与领导的「肃 AB 团」大镇压,此事件曾造成江西红色根据
地的严重危机和中共力量的削弱。 1935 年後,面对复杂多变的严峻形势,
毛泽东暂时收敛了他的专断个性,但是随着毛泽东在中共党内控制力的不
断加强,其专断个性在 1941 年後又再度复苏,而此时,中共高层已再无可
能对毛泽东的专断行为予以有效的约束。

毛泽东在延安整风运动中有意放纵其专断的个性,使之有机地配合于
自己的政治目的。在上层,毛泽东以我划线,创造并利用一切机会打击异
己;在延安和各根据地,毛泽东策动整肃全党干部的「抢救运动」,放任
恐怖政治。由毛泽东植入中共肌体的极左的审干、肃反政策,经过整风运
动,演化为党的性格的一部分,对 1949 年後的中国带来长期不良的影响。

延安整风为毛泽东显现其复杂诡奇的政治谋略提供了舞台。毛泽东敢
于突破中共历史上的常规,其手法深沉老辣,对其对手心境之揣摩和制敌
谋略的运用,均达到出神入化、炉火纯青的地步。毛泽东的谋略既来之于
他对中国古代政治术的熟练运用,又源之于他对俄共「格伯乌」手段的深
刻体会。在毛泽东的强力驱动下,通过 1945 年中共七大,毛将中共所有权
力都集中于自己手中。毛泽东的公共关系形象在整风前後也得到充分展
现,伴随日益升温的对毛泽东个人崇拜的气氛,毛泽东有意识地在全党和
国人面前显现自己的领袖姿态。在公众场合和他与各方人士的会晤中,毛
泽东常常扮演礼贤下士、虚怀若谷的「明主」角色,接人待物诚恳、得体,
从而赢得中共广大党员和国统区社会各界人士对他个人的普遍好感。但在
党内高层,毛泽东放纵其刚愎自负、桀傲不驯的个性,对昔日政敌睚眦必
报,对党内同僚峭刻嘲讽,由于毛泽东随时调换他的两副面孔,致使外界
对他长期缺乏深切的了解。

发生在 1942-1945 年的延安整风运动,虽然已经过去五十多年,但
是在传统意识形态术语的演绎中,其全貌至今尚混沌不明。本书的目的,
并不在于对主流话语系统中有关整风运动的论断展开辩驳,而是试图通过
对远近各种有关延安整风运动史料的辨析和梳理,对延安整风运动进行新
的研究,拂去历史的尘埃,将延安整风运动的真貌显现出来,在官修的历
史之外,提供另一种历史叙述和解释,斯是吾愿,是否达到这个目标。还
有待读者评判吧!

\chapter*{重印自序}
拙著《红太阳是怎样升起的:延安整风连动的来龙去脉》由香港中文
大学出版社于 2000 年 3 月出版後,受到读者的欢迎。两年来,我收到了许
多读者的来信,对于他们的鼓励和支持,我谨致以衷心的感谢。

延安整风运动是影响二十世纪中国历史进程的重大事件,但学界的相
关研究却较为薄弱,我在本书中尝试对其作出研究和分忻,自是一家之
言,欢迎读者随时赐正为祷。

此次拙著重印,特别要感谢一些学界前辈的指教。杨振宁、王元化、
陈方正、吴敬链、韦政通,张灏、林毓生、张玉法、董健、魏良※等先生
以不同的方式与我探讨拙著中涉及的若干重要问题,并对我的研究给予了
宝贵的鼓励和嘉许。金观涛、刘青峰、熊景明、吕芳上、陈永发、刘小枫、
许纪霖、萧功秦、朱学勤、何清涟、陈彦、丁学良、徐友渔、黄英哲、唐
少杰、钱文忠、钱永祥、梁侃、毛丹、李杨、张文中、钱钢、吴东峰等先
生还就我所从事研究的如何深人发展提出了积极的建议,他们的看法使我
受惠甚多。

本书初版时,由于电脑转换简繁体字的功能不尽完善,虽然对文稿做
了多达七次的校核,仍留下若干文字的错误。此次重印仅限于文字错讹处
的更正。近两年来,围绕延安整风运动,又有若干新的史料问世,日後,
当拙著出增订本时,我将对其内容做全面的修正和补充。

在这里,我要对殷毅、马沛文、尉天纵先生和薛遴教授表示诚挚的谢
意。拙著甫出版,殷毅、马沛文先生就来信、他们不仅就拙著的内容和写
作与我进行了深人的讨论,还特别指正了书中的文字错误。尉天纵先生也
来信指正了书中一误值的地名的错谬。薛遴教授是南京大学的语言学专
家,她的语言学方面的知识对本书的修订有重要的启发。

本书的修订,由我的研究生黄骏协助做了电脑文字处理工作,特此致
谢。

\bigskip\mbox{}\large\hfill  高华\qquad \qquad \qquad\\
\bigskip\mbox{}\large\hfill
2002 年 5 月 12 日于南京龙江寓所
\end{fzliukai}
\cleardoublepage

\tableofcontents
