\chapter{锻造「新人」
:从整风到审干}
\section{教化先行:听传达报告和精读文件}
1942 年春,整风运动初起之时,其主要内容是号召全党干部学习中共
中央指定阅读的一系列文件。整风以学习文件先行,反映了毛泽东在开展
党内斗争的方法上所具有的独创性。与斯大林三十年代推行的大规模肉体
消灭政策相区别。毛泽东领导的整风并不单纯依赖暴力镇压,和斯大林相
比,毛更擅于交替使用教化与强制两种手段。毛泽东进行党内斗争手段的
多样化,主要乃是因为四十年代毛的政治目标与斯大林完全不同。斯大林
是在苏共执政的条件下,为强化自己的独裁地位而滥施暴力,毛政治上的
首要目标则是彻底打倒党内的留苏派,完全确立并巩固自己在中共党内的
领袖地位,进而谋取抗战胜利後取代国民党,建立共产党和他本人对中国
的统治。在中共尚未在全国执政的条件下,为了达到这一目标,毛必须将
党内的整肃斗争控制在一定范围内,而主要依靠自己的路线、方针、乃至
个人的作风和风格吸引追随者。其次,诉诸教化手段是中国儒家传统的基
本方法,它既有可操作性,又有易于被人接受的亲和性。毛相信,借用儒
家传统的若干概念和方法,再配之以列宁主义的部分内容,基本可以达到
转换人的意识的目标,
从而避免了单纯使用暴力手段可能给党带来的破坏。
在 1941 年 9 至 10 月召开的中央政治局扩大会议上,毛泽东已获得对王明
等的绝对优势,但在全党毫无思想准备的情况下,毛很难迅速将党内上层
的分歧向全党公开。然而为了彻底摧毁国际派在党内的基础和影响,又必
须在政治上「搞臭」对手,只有将上层与中、下层的斗争全面展开,才能
为全党转变思想、接受「王明是机会主义」这一命题扫清障碍。正是基于
这种考虑,发动全党思想改造——学习毛的论述和经毛泽东审定编辑的有
关文件,才成为整风初期的中心任务。

中共中央通令全党在整风运动中必读的文件通称
「二十二个文件」, 但
在 1942 年 4 月 3 日中宣部颁布的
《关于在延安讨论中央决定及毛泽东同志
整顿三风报告的决定》中,只规定了十八个文件为必读文件,在这十八个
文件中只有两份是斯大林的作品。可能是毛泽东感到如此编排文件,倾向
性过于明显,4 月 16 日,中宣部又增添了四份必读文件,除一份为季米特
洛夫的论述,其它三份均为斯大林、列宁的论述,这样就正式形成了「二
十二个文件」。

在「二十二个文件」中占据最重要位置、被列入首篇和第二篇的是毛
泽东的《整顿学风、党风、文风》和《反对党八股》。
被列入第三篇的文件
则是康生在延安两次干部大会上作的「关于反对主观主义,宗派主义的报
告」和「痛斥党八股的报告」。

康生作为毛泽东发动整风运动最坚决的支持者,从整风之初就发挥着
极其重要的作用。康生在 2 月 21 日、3 月 7 日这两次大型报告会上,竭力
发挥毛泽东对国际派及知识分子的嘲讽、
挖苦,
将毛泽东有关知识分子
「其
实是比较最无知识」的新概念在全党广泛地传播开来。

除了康生,
毛泽东在延安的其他盟友 1942 年春也纷纷行动了起来。
中
央政治局委员陈云、任弼时以及政治地位正在上升的彭真、李富春、陆定
一、胡乔木等人,或在《解放日报》发表阐释性文章,或亲赴中央党校作
学习「二十二个文件」的辅导报告。身为政治局候补委员兼中宣部代部长
的凯丰,为了立功赎罪,也打足精神,跟在康生等人的後面摇旗呐喊。至
于陈伯达、艾思奇、张如心、何思敬等理论家更是积极辛劳,不时在《解
放日报》上推出长文或短论。一时间,密集的理论灌输,犹如暴风骤雨,
在延安倾盆落下,就在这强大的宣传攻势下,延安干部的思想改造过程已
经开始。

如果与整风运动中後期大规模的审干、肃奸、抢救斗争相比,整风运
动发动之初的干部学习文件的活动就显得比较轻松了。尽管自毛泽东发表
演说和康生传达毛的报告後,延安出现了一段为时不长的「自由化」时期,
但为时不久,
「矛头向上」的风向就被迅速扭转。3 月下旬,毛泽东紧急刹
车,精心部署对王实味的「反击」
。中宣部「四三决定」更具体落实毛的战
略意图,明确强调广大中下层干部也和高级干部一样,同属整风对象
\footnote{参见《中共中央宣传部关于在延安讨论中央决定及毛泽东同志整顿三风报告的决定》
(1942 年 4 月 3 日)
,载中
央档案馆编:
《中共中央文件选集》
(1941—1942)
,第 13 册,页 364-65.},并
宣布研究文件的时间为五个月。4 月 18 日,康生在中共中央直属机关和军
委直属机关干部大会上作学习「四三决定」的动员报告,在这次有二千人
参加的大会上,康生要求各机关成立学习分委员会,由该组织统一领导各
单位的运动\footnote{《延安整风运动纪事》
,页 107、111-12.}4 月 20 日、21 日,中共中央书记处秘书处和陕甘宁边区系
统分别召开文件学习动员大会,中央办公厅秘书处主任王首道和负责领导
边区工作的任弼时作了和康生报告相类似的动员讲话。\footnote{《延安整风运动纪事》
,页 107、111-12.}。于是学习「二十
二个文件」的大规模活动在各单位迅速展开。

「二十二个文件」的学习包括三个阶段:

一、粗读文件的阶段

在这个阶段中,要求将「二十二个文件」全部浏览一遍,读後要做笔
记,并进行初步讨论。

二、精读文件的阶段

在这个阶段中,要求将所有文件分类反复精读,达到「眼到」
(精细研
读)、「心到」
(深思熟虑,领会文件的实质和精神)、「手到」
(写读书笔记)、
「口到」
(质疑、漫谈、开讨论会)。\footnote{阅读和研究文件的「四到」方法为王首道首先提出,参见:
《延安整风运动纪事》
,页 111.} 

三、考试阶段

从 1942 年 6 至 8 月,
延安各单位的文件学习进入到考试阶段。
中央党
校在 6 月 23 至 7 月 4 日举行了第一次考试,
所拟定的四个考题事先经毛泽
东审阅和修改\footnote{王仲清主编:
《党校教育历史概述(1921 一 1947)
,页 78、77.}。考试题目是:

(一)什么是党的学风中的教条主义?你所
见到的最严重的表现是哪些?你自己在学习和工作中曾否犯过教条主义的
错误?如果犯过,表现在哪些方面,已经改正了多少?

(二)什么是党的
学风中的经验主义?你所见到的最严重的表现是哪些?你自己在学习和工
作中曾否犯过经验主义的错误?如果犯过,表现在哪些方面,已经改正了
多少?

(三)你听了或读了毛泽东同志《改造我们的学习》的报告和中央
《关于延安干部学校的决定》、《关于在职干部教育的决定》以後,你对过
去党内的教育和学习反省的结果如何?有些什么意见?你如何改造自己的
学习或工作?

(四)你接到中央《关于调查研究的决定》以後,怎样根据
它来检查并改造或准备改造你的工作?\footnote{王仲清主编:
《党校教育历史概述(1921 一 1947)
,页 78、77.}

中央党校规定,
在考试期间,
学校
关闭,除星期天以外,停止接待来访。文化程度低不能执笔的学员,可以
口授,由文化教员代为执笔。

在中共历史上,由党的中央机关动用组织行政力量,安排大批干部暂
停日常工作进行如此大规模的文件学习,这是首次(以往中共党员也有组
织安排的政治学习,但为时一般较短,性质更与延安整风期间的文件学习
完全不同)。毛泽东利用新成立的各级学习委员会,
使这个新设组织成了各
级党组织的核心,借助于学习委员会高效、有力的组织措施,毛将自己一
系列新概念强制性地灌输进广大党员的头脑,初步打击了党内知识分子的
自我意识,为下一步的思想改造奠定了心理方面的条件。

\section{排队摸底:命令写反省笔记}
毛泽东密切注视着延安干部的「二十二个文件」的学习活动,尤其关
心高级干部和知识分子对文件学习的反应。为了及时掌握延安各级干部的
思想动态,1942 年春夏之际,毛泽东作出决定,命令所有参加整风的干部
必须写出具有自我批判性质的反省笔记,并且建立起抽阅干部反省笔记的
制度。

用检查私人笔记的方法,来了解干部的「活思想」,
这也是毛泽东的独
创,
这说明毛泽东对全党能否真正在思想上接受自己的主张并不十分乐观。
毛很清楚地知道,由于他不能用准确无误的语言来表示自己的真实想法,
全党在思想上极有可能造成大的混乱。毛的最大困难在于,他不可以公开
批评斯大林和共产国际,
相反必须对斯大林、
共产国际持完全肯定的态度。
毛暂时也不能将党内上层斗争的真相完全公开,用明确的语言直接批判王
明、博古等,从而暴露出党的核心层的分歧,相反,毛必须维护党的核心
层表面上的团结一致。面对如此复杂的局面,毛泽东只能小心行事,而决
不可对延安干部草率处之以粗暴手段,可供选择的最佳方法就是「文攻」
——不战而屈人之兵,要求干部写出反省笔记和建立抽阅反省笔记的制度
就是实现「文攻」的有效途径之一。
 
对于毛泽东而言,
建立抽阅干部反省笔记制度至少有两大好处:

第一,
可以就此观察全党接受自己新概念的程度如何,以因势利导。

第二,在干部反省笔记中搜寻异端,择其典型打击之,以起警戒之效,
用大棒配之以胡萝卜可纠「和风细雨」思想改造之弊,使全党对新权威顿
起敬畏之心。

提倡干部进行思想反省,并写出带有自我批评性质的反省笔记。对于
延安的广大干部固然是一种压力,但是这还不至于超出他们的心理承受范
围。因为全党对于「反省」一词并不陌生,刘少奇更在 1939 年作的《论共
产党员的修养》报告中借孔子「吾日三省吾身」之说,鼓吹共产党员应加
强「党性锻炼」
,事实上,许多共产党员已经按照刘少奇所要求的那样去做
了。中共元老吴玉章自述:他「恍然觉得我们现在的整风工作,就是中国
古圣先贤所谓『克己复礼』
『正心诚意』的修养」,「所谓『诚其意者,毋自
欺也』(《中庸》),虽然旧思想是唯心的,但他的严于自己省察,行为不苟,
是可宝贵的。」\footnote{《吴玉章文集》
,上(重庆:重庆出版社,1987 年)
,页 240.}由于列宁主义的「新人」概念与中国哲学中的「内省」、「修
身」并无明显矛盾,因此对于中共广大党员,接受这种兼顾新旧、融合列
宁主义与中国传统的思想改造方法并不十分困难。

毛泽东的方针已定,下一步的问题就是如何将文件学习与反省思想加
以结合并用来指导眼下的运动。1942 年 3 月 9 日,经毛泽东精心修改,由
胡乔木撰写的社论《教条和裤子》在《解放日报》正式发表。胡乔木在这
篇社论中第一次提出「脱裤子,割尾巴」——在全党进行思想反省的问题,
社论要求每个党员对照毛的讲话,勇敢地解剖自己,与旧我告别。继之,
中宣部的「四三决定」进一步明确提出,参加整风的干部「每人都要深思
熟虑,反省自己的工作及思想」。4 月 18 日,康生在中央直属机关和军委
直属机关联合举行的整风学习动员大会上重申必须「运用文件反省自己」,
并具体指导写反省笔记的方法:
「内容要多写自己阅读(文件)後的心得,
自己的反省」。
康生并且首次宣布:
「学习委员会有权临时调阅每个同志的
笔记」。\footnote{《延安整风运动纪事》
,页 107.} 

两天後,为了给秉承自己意志的康生撑腰,毛泽东亲自出马,在中央
学习组召开的高干会议上,动员全党自上而下「写笔记」。
毛泽东以十分强
硬的口吻说道:
\begin{quote}
{\fzwkai 中宣部那个决定上说要写笔记,党员有服从党的决定的义务,
决定规定要写笔记,就得写笔记。你说我不写笔记,那可不行,身
为党员,铁的纪律就非执行不可。孙行者头上套的箍是金的,列宁
论共产党的纪律是铁的,比孙行者的金箍还厉害,还硬,这是上了
书的,……我们的「紧箍咒」里面有一句叫做「写笔记」
,我们大家
都要写,我也要写一点……不管文化人也好,
「武化人」也好,男人
也好,女人也好,新干部也好,老干部也好,学校也好,机关也好,
都要写笔记。首先首长要写,班长、小组长也要写,一定要写,还
要检查笔记……现在一些犯过错误的同志在写笔记,这是是很好的
现象,犯了错误还要装老大爷,那就不行。过去有功劳的也要写笔
记……也许有人说,我功劳甚大,写什么笔记。那不行,功劳再大
也得写笔记\footnote{毛泽东:
《关于整顿三风》
(1942 年 4 月 20 日)
,载《党的文献》
,1992 年第 2 期。}。} 
\end{quote}

在 4 月 20 日中央学习组的会议上,
毛泽东甚至引述康生两天前在中直
和军属机关动员大会上的讲话。毛说:
\begin{quote}
{\fzwkai 康生同志在前天动员大会上讲的批评与自我批评,
批评是批评
别人,自我批评是批评自己。批评是整个的,但自我批评就是说领
导者对自己的批评是主要的。\footnote{毛泽东:
《关于整顿三风》
(1942 年 4 月 20 日)
,载《党的文献》
,1992 年第 2 期。}} 
\end{quote}
毛泽东表示自己也要「写一点」笔记,但事实上,他只是以此作一个
幌子。毛所谓「要反复研究自己的思想,自己的历史,自己现在的工作,
果不其然,
好好地反省一下」, \footnote{毛泽东:
《关于整顿三风》
(1942 年 4 月 20 日)
,载《党的文献》
,1992 年第 2 期。}完全是针对其他领导人和一般党员干部的。
5 月 1 日,中央党校在制定学习二十二个文件的计划中作出规定,参加整
风学习的学员必须「联系反省个人思想及与本身有关工作」明确宣布中央
党校的各级领导机构均有权「随时检查笔记、记录」。

经过约一个月的试点准备,到了 1942 年 5 月下旬,毛泽东认为,将学
习二十二个文件转入对照文件进行思想反省的时机已经成熟。5 月 23 日,
《解放日报》发表社论《一定要写反省笔记》
,至此,整风进入到思想反省
的阶段,调阅干部反省笔记的制度随之在各机关、学校迅速推广开来。

从现象上看,动员干部写反省笔记和建立抽阅反省笔记的制度,并没
有遭到来自任何方面的抵制和反抗,然而毛泽东并没有就此放松警觉。他
完全明白,
联系个人的思想与历史进行自我反省决不同于一般的阅读文件,
许多干部往往会避重就轻,不愿进行彻底的自我否定。为了引导干部作出
比较深刻的自我批判,必须及时推出一些有代表性的反省标本,作为引导
全党进行反省的示范。1942 年 6 月後,
《解放日报》陆续刊出一批反省文
章,这些文章大致包括四种类型。

一、犯有「经验主义」错误的中央领导干部政治表态性的反省。所谓
「经验主义」, 是毛泽东在整风运动期间给周恩来、
彭德怀等中共领袖贴上
的政治标签。
「经验主义者」因在政治上曾经支持留苏派,或虽未明确表示
支持留苏派,但曾一度与毛泽东的意见相左,因而也与「教条主义」同列,
是毛整肃的对象。但是,
「经验主义者」大多有较长的革命历史,在党内的
基础也较深厚,所以只是处在被整肃的第二层,而毛对「经验主义者」的
策略是分化他们与王明、博古等的关系,将他们争取到自己的一边。
「经验
主义者」只要能公开承认自己的「错误」,而不管这种「承认」及「反省」
是否表面化,毛泽东一般均放他们过关。中共元老王若飞的反省即提供了
经验主义领导干部自我反省的范例。

1942 年 6 月 27 日,中共中央副秘书长王若飞在《解放日报》上发表
《
「粗枝大叶自以为是的工作作风是党性不纯的第一个表现」
》的文章,王
若飞在该文中以毛泽东的立论为依据,对照检查自己:
\begin{quote}
{\fzwkai 是多少带有陶渊明所说的某些气质,
「好读书不求甚解」,「性嗜
酒造饮辄醉」,
这种粗疏狂放的作风,每每不能深思熟虑,谨慎其事
处理问题,即令自己过去曾是时时紧张的埋头工作,也常陷于没有
方向的事务主义,以致工作无形中受到很多损失。严格地说,这是
缺少一个共产党员对革命认真负责实事求是的态度。\footnote{《解放日报》
,1942 年 6 月 27 日。}} 
\end{quote}

王若飞的上述反省,严格地说,并不「深刻」。他不仅没有对自己的
过去历史作出严厉的自我批判,更没有将批评的矛头对准王明、博古等留
苏派,与此相反,王若飞甚至在作「自我批评」时也没忘了为自己评功摆
好,例如,王若飞反省道:
\begin{quote}
{\fzwkai 过去我对党性的认识,只注重从组织方面去看,认为党是有组
织的整体、个人与党的关系。个人一切言行,应当无条件的服从
党组织的决定,只要自己埋头为党工作,不闹名誉,不闹地位,不
出风头,不把个人利益与党的利益对立,便是党性,并以此泰然自
安。\footnote{《解放日报》
,1942 年 6 月 27 日。}} 
\end{quote}

人们从这些话中实在难于判断王若飞
「对党性的认识」, 究竟是属于缺
点,还是属于优点。尽管王若飞的反省只是检查自己「粗枝大叶自以为是」
的工作方法,但是仍然受到毛泽东的欢迎。王若飞属党的元老,因在 1926
至 1927 年担任中共中央秘书长期间与陈独秀关系密切,
长期遭受莫斯科与
国际派的排挤。王若飞与周恩来的关系也不紧密。抗战後王若飞获毛泽东
容纳,成为毛泽东核心圈外第二层的重要干部。王若飞平时对毛的态度十
分恭敬,现在又在报上进行自我反省,在政治上公开表示对毛的支持和效
忠,对于这样一位在党内享有较高声望的老同志的政治表态,毛泽东又如
何可以求全责备?此时此地,毛泽东所要求于中央领导层干部的就是像王
若飞这样在政治上表明态度。更重要的是,王若飞身为中央领导干部,带
头响应毛的号召进行自我反省,其影响不可谓不大,其他干部焉能不从?

二、犯有「教条主义」错误的高级文职干部的反省。

对于一批有留苏或留日、留欧美背景,在中央宣传部、中央研究院等
文宣系统工作的党的高级文职干部来说,理解延安整风的真正意图并不困
难。当传达了毛泽东的几篇演说和《解放日报》的《教条和裤子》社论发
表後,他们很快就知道了自己是这场运动首当其冲的目标。摆在他们面前
的道路只有两条:或拒绝反省,最终被他们寄托于生命全部意义的党所抛
弃;或遵循党的要求,彻底与过去告别,脱胎换骨,用毛的概念取代过去
被他们视为神圣的俄式马列的概念。习惯于听从上级指示的文职干部几乎
不加思索地就选择了第二条道路。然而这条道路并不平坦,首先,他们必
须对自己罪孽深重的过去痛加谴责;继之又须对毛泽东的新概念
表示心悦诚服。1942 年 8 月 23 日,
《解放日报》发表的王思华的反省文章《二十
年来我的教条主义》,
就堪称教条主义高级文职干部自我反省的标本。

担任中央研究院中国经济研究室主任的王思华原是三十年代颇有名气
的左翼社会科学家,曾留学德国专攻马克思主义政治经济学。他充分领会
了毛泽东发动整风的意图,在他的反省文章中,对自己以往二十年的理论
研究活动采取了全盘否定的态度,王思华写道:
\begin{quote}
{\fzwkai 我在大学和留学时,所学和研究的,不是英国的亚当·斯密与李嘉
图,便是法国的魁奈和萨伊,……所学的是外国的,自己在大学里教的,自然
也只能是这些外国的。这样做,不但省劲,而且受学生的欢迎。因为在一般的
大学生中,有一种反常的心理,对中国问题无兴趣,他们一心向往的,就是他
们从先生那里学外国。学生的这种反常心理,先生这种投机取巧的态度,普遍
的存在于中国大学生,这种轮回教育,不知害了多少青年!它是害了青年时代
的我,而我又拿来害青年!\footnote{《解放日报》
,1942 年 8 月 23 日。}} 
\end{quote}

王思华上述有关对中国现代教育制度弊端的批评,
在某种程度上是符
合历史实际的。但问题在于,王思华的兴趣似乎并不在对此种弊端展开严
肃认真的分析,而是企图以此作为迎合某种政治新风向的手段。为此,他
不惜将纷繁复杂的现象简单化,为毛泽东的论断提供具有个性特征的注
解:
\begin{quote}
{\fzwkai 十三年前,当我接受了马克思主义经济学後,又把它「生吞活
剥」地搬到中国来。……在对待马克思主义经济学的态度上还是主
观主义的。在这种态度下,还是只想懂得希腊,不想懂得中国;……
把马克思的一切东西当作千古不变,放之于四海皆准的教条了。
\footnote{《解放日报》
,1942 年 8 月 23 日。}} 
\end{quote}

紧接着,王思华使用了一系列羞辱性的词句进行自我贬损。他承认,
教学生「啃《反杜林论》则是为了迎合学生的好高骛远的奇特心理」,
\footnote{《解放日报》
,1942 年 8 月 23 日。}到延安後,
「在马克思主义中国化的口号之下,不得不联系到中国」,但这仅
是「以资装饰门面」,自己仍「只想在《资本论》本身上来翻筋斗」。\footnote{《解放日报》
,1942 年 8 月 23 日。} 

王恩华痛骂自己
「夸夸其谈」、
「不老实,
企图取巧」、
「只知背诵教条」、
「向马列主义开玩笑」。在对自己口诛笔伐的同时,
王思华竭力称颂毛泽东
对发展马列主义的贡献。他表示,
「为了彻底消灭『比屎还没有用处』的教
条」,「彻底打垮我这样根深蒂固的不正确的思想方法」,
自己已决定「到实
际工作中去,不仅是到实际研究中去,而且是真正变为一个实际工作者」
\footnote{《解放日报》
,1942 年 8 月 23 日。}。


王思华的反省开创了教条主义高级文职干部自我批判的模式。
范文澜、
王子野等的自我反省文章同属于这一模式。

中央研究院副院长、历史学家范文澜对前一阶段中研院出现的以王实
味为代表的自由化思潮严重泛滥的局面痛悔不迭。范文澜谴责自己「高唱
民主,忽视集中,形成放任自流的『领导』」,声称这是「难以忘怀的一件
痛苦经验」,
他对此「衷心抱疚」\footnote{《解放日报》
,1942 年 6 月 1 日。}。

中央政治研究室资料组和国际政策研究室成员王子野则专门检讨了自
己「夸夸其谈」的「不正派作风」
,他痛陈自己往往仅凭「一知半解」,「凭
着想当然」大发议论,现在回想起来,
「实在荒唐之致」\footnote{《解放日报》
,1942 年 6 月 1 日。}。

在毛泽东发起的劝导反省的巨大压力下,大批高级文职干部纷纷自我
批判,口诛笔伐「比屎还没有用处」的教条本本,那些当年翻译马列著作
的知识分子更因积极传播教条而首当其冲地成为被批判的对象。中央研究
院国际问题研究室主任柯柏年是一个老党员,早在二十年代末就是国内闻
名的红色社会科学家,曾翻译《经济学方法论》等多种马克思主义理论著
作,\footnote{参见《生活全国总书目》
(1935)
(上海:上海生活书店编印,1935 年)
,页 72.}但在整风之初,柯柏年并没有在《解放日报》发表自我谴责的文章,
于是柯柏年被攻击为「教条主义者」
,罪名是曾翻译过教条本本。此事给柯
柏年很大的刺激,他发誓以後再不搞翻译工作。1943 年春,毛泽东根据已
变化了的形势(教条主义者已被搞臭,苏联对德国已取得优势)
,认为有必
要恢复中共的马列著作翻译工作。可是当毛泽东征求柯柏年意见时,柯柏
年却向毛坚决表示,今後再不搞翻译了。
\footnote{参见师哲:
《在历史的巨人身边——师哲回忆录》
,页 247.}
柯柏年以後转入到周恩来领导
的中共外事系统,改行做对外统战工作,再也没回到中共马列著作编译部
门。毛泽东的「反教条主义」所要达到的效果极为显著,及至 1945 年春,
谢觉哉私下也感慨,
「自从反教条,有人不讲书本子了」。\footnote{ 《谢觉哉日记》
,下,页 791.} 

三、具有「经验主义」倾向的高级军职干部的反省。
和党的高级政治生活毫无牵涉的军队一般高级干部,他们不是、也不
可能是整风的重点整肃对象。然而整风既为全党性的运动,军队干部也不
能完全置身于外,他们同样应在运动中「提高认识」
。但是,对于来自不同
军队系统的干部,他们所需「提高」认识的内容并不一致。一般而言,原
红四方面军的干部有必要检讨自己在张国焘「另立中央」事件中的立场和
态度,而原红一方面军中的干部则只要检查一下自己的工作方法与思想方
法即可。我们以曹里怀的反省为例:

曹里怀是毛泽东创建井冈山根据地时期的老部下,他的「自我检讨」
重点反省了自己的四大缺点:一、在日常工作中,解决和处理问题不细心,
草率从事。二、爱面子。三、理论和知识的修养太差。四、自己的经验不
能很好地整理。曹里怀给自己贴的标签是「主观主义经验主义的倾向确是
浓厚地存在着」。

饶有兴味的是,曹里怀的「自我检讨」有一半的篇幅是歌颂毛泽东的
内容,若将其和「教条主义者」嘴里发出的赞美相比较,具有「经验主义」
倾向的军队高级将领对毛的赞美似乎更加诚挚和热烈。曹里怀径直将毛泽
东与列宁并列,声称毛的著作是「活的马列主义」
,言外之意其它均是「死
的马列主义」:
\begin{quote}
{\fzwkai (毛泽东的著作)告诉了我们提出问题,分析问题,解决问题的唯物辩证
法的方法。这些著作,是完全从客观的现实出发,而又向客观实际获得了证明
的最正确,最科学,最革命的真理。\footnote{《解放日报》
,1942 年 7 月 13 日。}} 
\end{quote}

曹里怀更进一步将歌颂毛泽东与谴责国际派结合起来,他写道:
\begin{quote}
{\fzwkai (毛泽东的)
这种有高度布尔什维克原则性和极丰富的革命斗争经验,
丰
富的革命内容的政策,不是主观主义教条主义者所能办得到的。
\footnote{《解放日报》
,1942 年 7 月 13 日。}} 
\end{quote}

四、革命历史雄厚,且不掌实权的党的元老的反省。

在四十年代的延安居住着几位德高望重、受到全党尊敬和爱戴的革命
老人,他们分别是林伯渠、吴玉章、谢觉哉、徐特立。除了这四老,张曙
时等尽管也年届六十,但是依当时的习惯,他们尚不够「革命元老」的资
格。在「革命四老」中,只有林伯渠担负边区主席的实际工作,吴玉章等
大多挂个虚衔,并不掌握具体部门的领导实权。整风运动初起,吴玉章等
也积极行动起来,以自己的反省现身说法,为毛发动整风的「正当性」、
为
知识分子必须进行脱胎换骨改造的论断,提供最具说服力的证明。吴玉章
写道:
\begin{quote}
{\fzwkai 中国旧时社会最坏的习惯,就是稍有聪明才智的人都变为
知识分子而脱离生产,结果,小的变流氓,大的变政客,都为
社会的毒害。而从事生产的广大群众则蠢蠢无知、任人鱼肉。

……如果我们不自欺欺人,则我们这些小资产阶级知识分子, 对于国家民 族尽
了什么责任呢?这样来一个反省, 恐怕不汗颜的没有几个。我自己一反省 就觉
得「才无一技之长,手无缚鸡之力」,而还往往「夸夸其谈」「哗众取宠」,
党八股的馀毒很深。这能免「欺世盗名」之诮吗?我虽从事革命四十馀年,只 有
力求前进到底不懈这一点足以自信自慰,其它能力太缺乏了! \footnote{吴玉章:
《以思想革命来纪念抗战五周年》(1942 年 7 月 7 日),载《吴玉章文集》
。上,页 241. }} 
\end{quote}

吴玉章的反省颇真实地反映了当时在延安的一些革命老人的共同心态。

李六如早
年参加辛亥革命,是五四前後湖南教育界的著名人物,毛泽 东在青年时代即与其
相识,二十年代李六如就加入了中共,延安时期,曾 任中央军委主席办公室秘书
长。1942 年李六如已经五十五岁,他对老友谢 觉哉说, 「以前自以为不错, 自
以为立场稳定, 整风後才知自己政治水平低, 『组织上入了党,思想上未入党』」
(此系毛泽东在整风中发明的名言)。谢 觉哉说,他对李六如所言「很有同感」,
\footnote{《谢觉哉日记》,上,页 456.}谢觉哉不仅自我反省,还在《解 放日
报》化名发表《一得书》短评,向教条主义发起攻击。谢觉哉指出, 教条主义
「如只放在案头上摆样,虽然比屎还没有用,不能肥田,不能喂 狗,但狗屎自享,
于人无干。若拿了去对付革命,那就为害非浅,容易把 革命弄坏」。
\footnote{焕南(谢觉哉)《感性与理性》: ,载《解放日报》,1942 年 8
月 10 日。} 

以上四种类型的干部反省的样本,
为全党展开思想反省提供了不同的
参照系统。毛泽东、康生、彭真、李富春、胡乔木、陆定一利用报纸,大
力推广这些反省经验,再结合于组织措施的落实,对延安干部的心理造成
了剧烈的冲击和震荡,尤其使有「教条主义」背景的干部自惭形秽,无地
自容。至此,毛泽东设计的思想改造工程的关键性步骤——清算过去的大
门已经打开。下一步就是广大党员挑选适合自己的政治标签,对号入座;
根据自己的具体情况,依照报上发表的反省样本,如法炮制各自的反省检
查。

\section{审查在後:动员填「小广播调查表」}
1942 年秋冬之交,延安各机关、学校正遵照中央总学委的部署,将学
风和党风学习纳入到干部自我反省的方向,就在广大干部和党员纷纷写出
个人反省笔记的时候,忽然间,运动的风向又发生了新的变化,1942 年 12
月 6 日,中央总学委发出《关于肃清延安「小广播」的通知》
,各单位又迅
速开展了以反对「小广播」为中心的反对自由主义的斗争。

所谓「小广播」,与「脱裤子、割尾巴」一样,是中共在延安时期创
造出的政治新词汇。「小广播」系指和党的宣传口径等「大广播」相对应
的,
在同志之间对党的政治、
人事关系的私下议论。
被中央总学委列为
「极
端危害党的大患」的「小广播」有下列五种类型:

一、泄露党的政治、军事、党务、组织、经济、教育、锄奸、情报等
秘密消息和行动;

二、
散布与党的宣传口径不一致的对国际国内战争形势的看法。
例如,
传播对苏德战争、中日战争的悲观言论;

三、
有关对整风运动目的的怀疑和议论,
「散布整风是为了打击某些人
的谰言」;

四、攻击党的领导,
「对党内同志任意污蔑,造谣中伤」;

五、同情「托派反革命的人性论、
蜕化论的宣传」,「替反革命分子『广
播』反党思想」\footnote{《中央总学委会关于肃清延安「小广播」的通知》
,载中央档案馆编:
《中共中央文件选集》
(1941—1942)
,第
13 册,页 468-70.}。

那么,最有可能散布这些反革命「小广播」、「实际上变成了敌人义务
的情报员」的又是哪些人呢?中央总学委的《通知》提示各学委会必须严
密注意下列对象:

一、在思想和组织上存在浓厚的自由主义,厌恶党的原则、组织纪律
和秘密工作制度的人。

二、
「讲温情私交,论友谊」的人。这些人敌我不分,
「对『私交』可
以无所不谈」,「就是反党的分子也可以作为他们的朋友」
。但他们「对党的
组织可以欺骗隐瞒,甚至听到反革命的言论,也可以不报告组织」。

三、
「喜欢溜门子」,「打听个人的生活起居,加以评头论足」的人\footnote{《中央总学委会关于肃清延安「小广播」的通知》
,载中央档案馆编:
《中共中央文件选集》
(1941—1942)
,第
13 册,页 468-70.}。

具有上述三种表现的人是运动的重点整肃对象,然而,中央总学委并
不想把运动仅限于这三种人中间。因为在广大普通党员中,因历史、职业、
地域、个性等背景的相近,
「讲温情私交,论友谊」的人比比皆是,而依照
中央总学委的逻辑,凡具有这种特性的人,皆有成为敌人「义务的情报员」
的可能性,所以《通知》明令:
\begin{quote}
{\fzwkai 每个党员深刻的反省自己与严正的批评别人,检查自己和别人
是否犯了「小广播」的错误,曾泄露了一些什么秘密,向外广播了
一些什么消息,向党隐瞒了一些什么问题,听到了一些什么不利于
党的消息没有向党报告,对于这些问题每个党员应向党诚恳坦白的
报告出来。\footnote{《中央总学委会关于肃清延安「小广播」的通知》
,载中央档案馆编:
《中共中央文件选集》
(1941—1942)
,第
13 册,页 468-70.}} 
\end{quote}

如何坦白?中央总学委这一次又创造出新的办法,每个干部必须填写
「小广播表」!

中央总学委规定,各机关学校应根据《通知》的精神和各单位的具体
情况,
「制发『小广播』调查表」
,并将其
分发每个同志填写,以调查本机关的工作人员向外广播了一些
什么及由外向本机关的工作人员广播了一些什么,这种调查材料,
应加以整理研究,并向总学委报告\footnote{中央档案馆编:
《中共中央文件选集》
(1941 一 1942)
,第 13 册,页 470.}。

用行政命令的方式,动员并强制广大党员交代自己的言行,涉及面如
此广泛,这在中共政治生活中尚属头一回。虽然在这之前,延安的党员和
干部已依照中央总学委的部署,普遍写出反省笔记,但反省内容大多属于
思想认识方面的问题,如今更深入到个人的私生活领域,调查党员的私下
言行和个人间的交往,这反映了毛的「思想改造」极端强制的一面。尽管
《通知》通篇都是「党的原则」、「党的纪律」、「党的团结」等意识形态术
语,但调查党员私下言行毕竟与要求党员反省思想不是一回事,其正当性
颇令人怀疑。于是,针对党员中有可能出现的对填「小广播调查表」的消
极不满情绪,中央总学委又「适时」提出了「反对自由主义」的口号。1942
年末,
围绕动员填
「小广播表」
一事,
各机关学校布置反复学习毛泽东 1937
年所作的《反对自由主义》的报告。毛的这篇演讲稿与其说是论述自由主
义的论文,勿宁说是毛在阐述其理想中的共产党员人生哲学的范式。毛在
这篇演说中。撇开「自由主义」一词的规定性,对「自由主义」一词作出
新的解释。把「自由主义」等同于中国传统的人际交往的一般习惯。毛所
要反对的自由主义,除了指政治思想上与党的路线背离外,重点是指党内
的「一团和气」
,换言之,就是在共产党员中所存在的「讲温情私交,论友
谊」的现象。现在,重新翻出毛泽东当年的报告,把「客观上帮助敌人」
的「自由主义」和眼下要肃清的「小广播」串联起来,为反对「小广播」
提供了理论的依据。

12 月 6 日中央总学委反对「小广播」的通知下达後,延安宣传媒介的
反自由主义的宣传攻势紧紧跟上。1943 年 1 月 19 日,陈伯达在《解放日
报》发表《应用辩证法,反对自由主义——在整风中纪念列宁逝世十九周
年》
,延安各机关学校除了动员每人填写「小广播调查表」外,还纷纷组织
以反对自由主义为中心的「学习会」和「讨论会」。

作为反「小广播」斗争的试点单位,陕甘宁边区师范学校学委会早在
11 月 20 日就布置了反对自由主义的「大讨论会」
。大会历时十九天,经历
了三个阶段:第一个阶段,由学校领导机关广泛搜集「犯自由主义」的材
料;第二个阶段,动员师生展开互相批评;第三个阶段,则将斗争重点转
移到「犯自由主义特别严重,错误思想特别顽固」的人和事件上。
\footnote{《延安整风运动纪事》
,页 337、338. 
}

12 月 6 日,就在中央总学委发出肃清「小广播」通知的当天,中共中
央材料室(即中央政治研究室资料组)向每个工作人员发出考试试题,要
求回答下列问题:

一、到今天为止你对党还有什么隐瞒的事情没有?还有什么不满意党
的地方没有?

二、你的自我批评精神如何?你对其他同志的批评还有不坦白的没
有?其他同志对你有什么批评没有?你的认识和态度如何?有无自由主义
的毛病?自己还有什么缺点需要揭发呢\footnote{《延安整风运动纪事》
,页 337、338.}?

上述试题与半年前中央党校学风考试的内容已完全不同,延安的干部
在遵命填写「小广播调查表」
,搜肠刮肚地检讨自己的自由主义错误时,愈
来愈对整风运动的转向感到迷惑。

\section{为运动重心的转移作准备:毛泽东、康生的幕後活动}
整风运动既以反对主观主义开场,运动展开以後,延安干部又相继经
历了整顿学风和党风的阶段,
但在 12 月份,
运动的风向明显地发生了变化。
中央总学委 12 月 6 日发布的肃清「小广播」的通知,强调党员和干部必须
彻底交待个人一切言行和日常人际来往情况,并运用组织力量对所谓「串
门子」、「爱打听」的情况进行大规模的调查,这早已和批判「主观主义」
与「党八股」风马牛不相及,而更类似于保卫机关的肃反侦察手段,尽管
1942 年 12 月 18 日,中央总学委又部署开展反对「党八股」的文风检查,
延安的报纸还在继续宣传整顿三风,然而事实上,原先以教化为重心的学
习运动,现在已迅速向以镇制为重心的审干、反奸运动转移。

将整风运动导入审干、反奸轨道的总策划者不是别人,正是毛泽东和
他亲密的助手康生。为了顺利地将运动重心转移,从 1942 年春开始,毛泽
东就在康生的协助下,作了长时间精心的酝酿和准备。

促使毛泽东萌发审干、肃奸念头的导火线是王实味的《野百合花》事
件。1941 年 9 月政治局扩大会议後,中共中央虽然已经成立了以康生为首
的「党与非党干部审查委员会」
,但在这之後的一段时间内,这个委员会似
乎还未正式开展工作。1942 年春,
《野百合花》的发表及其在知识分子中
激起的广泛共鸣引致毛泽东的高度重视和强烈反感,王实味文中所流露出
的强烈的人道主义和反特权思想,连同延安文化人蔑视领导权威的种种行
径,使毛感到自由主义、人性论已对自己的政治目标和个人权威构成了巨
大的威胁。本来毛泽东就对「和风细雨」进行思想改造的局限性有着充分
认识,委派康生作为领导整风运动的第二把手和自己的代言人,本身就蕴
含着整风运动所具有的惩戒性质,
现在王实味和延安文化人公开跳将出来,
更使毛相信,欲统一全党思想和确立自己的领袖地位,必须一手拿笔(教
化)
,一手持剑(镇制)
,使文武两手并行不悖。从这时起,毛泽东就产生
想法,要以王实味事件为突破口,在肃清王明等国际派的同时,一并消灭
党内的自由主义。

1942 年 4 月,延安《解放日报》开始陆续发表批判王实味的文章,从
表面上看,这个时期报上批王的言辞并不十分激烈,毛泽东其至还两次通
过秘书胡乔木,向王实味转达他个人希望王实味「改正错误立场」的意见。
\footnote{参见李言:
《对中央研究院整风运动的几点体会》
,载《延安中央研究院回忆录》
(北京:中国社会科学出版社、
湖南人民出版社,1984 年)
,页 19.}
但在内部,中共上层已决定对延安的自由化思潮采取强硬措施。

4 月上旬,中共中央政治局召开会议,听取中央杜会部部长康生作有
关国民党方面对延安动态反应的汇报。
康生在讲话中没有提及大後方对
《野
百合花》的反应,只举出了中央青委的《轻骑队》壁报已被敌人所利用。
康生说:
「国民党特务称赞《轻骑队》为延安专制下的唯一呼声。」\footnote{参见王秀鑫:
《延安「抢救运动」述评》
,载《党的文献》
,1990 年第 3 期。王秀鑫为中共中央党史研究室研究人
员,在此文中,王秀鑫使用了未曾公开的毛泽东在 1942 年的讲话。
}康生提
供的情报激起与会者的强烈反应,其中有一位「中央领导同志」在发言中
历数延安「特务分子」的种种罪恶:
\begin{quote}
{\fzwkai 特务分子利用党内自由主义乘机活动,在开展批评与自我批评 及检查工
作中,故意扩大党内的缺点和错误,散布思想毒素,反对 各学校机关的领导,并
写文章、出壁报、进行小广播,团结不坚定 的党员来反党。\footnote{参见王秀
鑫: 《延安「抢救运动」述评》,载《党的文献》,1990 年第 3 期。王秀鑫
为中共中央党史研究室研究人 员,在此文中,王秀鑫使用了未曾公开的毛泽东在
1942 年的讲话。}} 
\end{quote}

在会议上发言的这位「中共领导同志」究竟是谁?1942 年 4 月在延安
的中央政治局委员只有毛泽东、康生、任弼时、陈云、王明、博古、朱德、
凯丰、王稼祥、邓发等十人,王明因病住进了医院,早已不出席政治局会
议。当时经常列席政治局会议、属于「中央领导同志」之列的还有彭真、
李富春、陆定一、胡乔木和晋绥联防军司令员贺龙等人,在上述这些人中,
较有可能发表这番关于「特务分子」讲话的仅任弼时、彭真、李富春、凯
丰、贺龙等数人,这位「中央领导同志」发言的意义在于,他不仅预设了
中共党内存在「特务分子」的活动,并且具体描绘了「特务分子」的特征
和进行破坏活动的手段。

在这位「中央领导同志」的眼中,
「特务分子」具有三个基本特点:

一、故意扩大党内的缺点和错误。

二、散布思想毒素。

三、反对各学校机关的领导。

「特务分子」反党的活动方式主要有三种:

一、写文章。

二、出壁报。

三、进行「小广播」。

从这位「中央领导同志」所描绘的「特务」图谱看,早在 1942 年 4
月中旬,延安上层已将表达批评性意见的党内知识分子划入了「特务」之
列,不管王实味是否承认错误,其被扣以「国民党特务」、「探子」的帽子
早已是命中注定。

就在这次会议上,在听了康生的汇报和其他人的发言後,毛泽东讲了
话,他明确表示:
\begin{quote}
{\fzwkai 在学习和检查工作中,实行干部鉴定,对干部的思想与组织观 念,实行
审查工作:在审查工作中,发现反革命分子,加以扫除, 以巩固组织。
\footnote{参见王秀鑫: 《延安「抢救运动」述评》,载《党的文献》,1990
年第 3 期。王秀鑫为中共中央党史研究室研究人
员,在此文中,王秀鑫使用了未曾公开的毛泽东在 1942 年的讲话。}} 
\end{quote}

这是迄今为止所发现的毛泽东最早布置审干、肃奸的资料——时间是
1942 年 4 月中旬。

4 月 20 日,毛泽东在中央学习组会议上作的报告中,猛烈抨击自由主
义,他将自由主义比喻作「诸子百家」
,历数了延安「思想庞杂,思想不统
一,行动不统一」的种种表现:
\begin{quote}
{\fzwkai 这个人这样想问题,那个人那样想问题,这个人这样看马列主 义,那个
人那样看马列主义。一件事情,这个人说是黑的,那个人 则说是白的,一人一说,
十人十说,百人百说,各人有各人的说法。差不多在延
安就是这样,自由主义的思想相当浓厚\footnote{毛泽东: 《关于整顿三风》
(1942 年 4 月 20 日),载《党的文献》,1992 年第 2 期。}。} 
\end{quote}

毛泽东表示,一定要在整风中「统一思想」,「统一行动」,为此,付出
任何代价也在所不惜:
\begin{quote}
{\fzwkai 如果打起仗来,把延安失掉要哇哇叫,鸡飞狗跳。那时候, 「诸 子百
家」就都会出来的,那就不得了,将来的光明也就很难到来, 即使到来,也掌握
不了它。……总之,一定要搞,搞到哇哇叫也要 搞,打得稀栏也要搞。
\footnote{毛泽东: 《关于整顿三风》(1942 年 4 月 20 日),载《党的文
献》,1992 年第 2 期。}} 
\end{quote}

在这次报告中,毛泽东明确地表明了他要肃清党内自由主义的决心,
此时,他已将自由主义排在「主观主义」之前,将其列为头号清除对象。
在宣布人人必须「写笔记」之後,毛又向与会的各机关学校的领导干部部
署了新的任务,要求从政治上对延安的党员普遍进行一次排队,他指示:
\begin{quote}
{\fzwkai 要把干部中的积极分子,平常分子,落後分子分开,对思想有 问题的,
党性有问题的人要特别加以注意。\footnote{毛泽东: 《关于整顿三风》(1942
年 4 月 20 日),载《党的文献》,1992 年第 2 期。}} 
\end{quote}

毛泽东 4 月 20 日的报告, 虽然没有直接谈及王实味和审干问题, 但是 毛的讲
话已将反对自由主义异端的问题凸现出来,所谓对有问题的人「要 特别加以注意」
,实际上就是动员审干、肃奸的代名词。

毛泽东在 4 月中旬和 4 月 20 日的两次内部谈话已给审干、 肃奸开了放 行的绿
灯,康生及其领导的中央社会部雷厉风行,立即行动起来。本来, 康生的情报系
统在「搜集材料」、「钻研材料」方面,就已成效卓著。举凡  一切涉及政治、军
事、经济、文化、社会阶级关系等方面的相关材料,无 不在他们关注的视野之下。
1941 年 11 月末,康生的政治秘书匡亚明就曾 专门著文介绍他们那种「带着高度
科学性的」「调查研究」的方法,例如如 何「一点一滴地从各个方面,各个角落,
用各种方法去搜取」材料;如何 「随时留心,随机应变,善于适应环境,善于和
人接近,善于选择对象, 善于灵活运用调查项目,达到调查工作的目的」;如何
「掌握材料,分析材 料,运用材料,而不为材料所束缚」\footnote{匡亚明:
《论调查研究工作的性质和作用》,载《解放日报》,1941 年 11 月 29 日。
匡亚明 1926 年加人中共,1929 年曾被中共特科红队误认为是叛徒而遭枪击,子
弹从口中射入,穿过脖颈险而未死,1941 年任中央社会部(情报部)第四室(政
治研究室)副主任。参见罗青长: 《深切缅怀隐蔽战线的老前辈匡亚明同志》;
丁莹如(匡亚明夫人):《永远的怀念》,载《匡亚明纪念文集》(南京:南京
大学出版社,1997 年),页 10-11、97. 另据师哲称,匡亚明此时虽 是康生的
政治秘书,却不被康生信任,康生不允许匡亚明接触重要文件。参见《峰与谷——师
哲回忆录》(北京:红旗 出版社;1992 年),页 216-17. }。如今,中央社会
部的这一整套经验 全部被派上了用场。

1942 年 4 月底或 5 月初,康生在中央社会部宣布:王实味的《野百合
花》已于 4 月在香港的报纸上发表了。\footnote{宋金寿:
《关于王实味问题》
,载《党史通讯》
,1984 年第 8 期。}不久,康生正式宣称,王实味是托
派分子,也是复兴社分子,是兼差特务。\footnote{宋金寿:
《关于王实味问题》
,载《党史通讯》
,1984 年第 8 期。}对王实味的这个政治判决究竟依
据的是什么,康生避而不谈,但显而易见,最重要的证据便是香港的报纸
发表了王实味的《野百合花》
。另一个证据便是王实味于 1940 年,向中央
组织部提交的有关自己与托派分子历史来往的书面材料。如果依据王实味
自己撰写的书面材料,推测王有托派嫌疑,虽然武断,但仍有迹可寻;但
是指称王实味是复兴社分子则纯属向壁虚构,完全是「不为材料所束缚」
、
滥用想象力的结果。大概康生唯一可以拿得出手的证据就是香港的报纸发
表了王实味的文章。耐人寻味的是,康生为何执意「选择」王实味作「调
查对象」
,一心要将王实味制造成「特务」
,而有意放过了同样受到国民党
方面称赞的《轻骑队》?一个很重要的原因乃是,参加《轻骑队》的干部
先後都隶属陈云领导下的中央青委,而陈云与毛的关系十分密切,与康生
同属毛所倚重的核心圈子,康生不得不有所顾忌;王实味则是张闻天的属
下,将王实味揪出来,足以令张闻天难堪,正可说明教条主义与自由主义
乃一脉相承,是祸害革命的一对毒瘤!第二个原因则因为王实味的自由主
义异端思想更典型、更严重,更符合「领导同志」眼中的「特务」标准。

有了康生对王实味的政治结论,正式给王实味戴上「特务」帽子就只
是时间和手续问题了。1942 年 6 月 11 日,在康生的指导下,中央研究院
负责人罗迈在批判王实味的斗争暂告结束之际,公开宣布王实味是一个托
派分子\footnote{李维汉:
《回忆与研究》
,下册,页 492.}。

毛泽东对中央社会部和中央研究院配合作战,挖出王实味的战绩大为
赞赏。6 月 19 日,
毛在一次会议上对此加以充分肯定,
并对如何扩大审干、
肃奸的战果作了进一步的指导。毛说:
\begin{quote}
{\fzwkai 现在的学习运动,已在中央研究院发现了王实味的托派问题, 他是有组
织地进 行托派活动,他谈过话的人有二十多个。中直、军 委、边区机关干部中知
识分子 有一半以上,我们要发现坏蛋,拯救 好人。要发现坏人,即托派、国特、
日特三 种坏人。……各机关都 要冷静观察,此项工作应有计划的布置。
\footnote{参 见王秀鑫: 《延安「抢救运动」述评》,载《党的文献》,1990
年第 3 期。王秀鑫为中共中央党史研究室研究人 员,在此文中,王秀鑫使用了
未曾公开的毛 泽东在 1942 年的讲话。}} 
\end{quote}

在上述这段话中,毛泽东不仅首次将原先不太明确的「特务」范畴加
以丰富和补充,而且还明确划定了审干、肃奸的工作重点和目标:

一、与「问题人物」来往密切的人是审干的重点对象。

二、
「坏蛋」主要集中在知识分子中。

三、
「好人」
也有被
「坏蛋」
拉过去的可能,
因此需要通过审干加以
「拯
救」。

四、审干、肃奸工作应秘密进行,以免打草惊蛇。

中央社会部积极贯彻毛泽东 6 月 19 日讲话的精神,将「调查研究」的 重点集中
于王实味的社会关系和知识分子当中。果其不然, 又再次发现 「敌 情」:
1942 年 7 至 8 月,中央政治研究室揪出了成全(陈传纲)、王里(王 汝琪);
9 至 10 月,中央研究院又开展了对潘芳(潘蕙田)、宗铮(郭箴一)的批判斗
争。以上四人为两对夫妻 (潘芳为中央研究院俄文研究室副主任, 其妻宗铮,
原名郭箴一;成全是中央政治研究室工作人员,其妻王里在中 央妇委工作),他
们或因与王实味比邻而居,平时有些私人来往;或在历史 上曾与王实味相识; 或
因与王实味在思想上存有共鸣, \footnote{成全于 1942 年 2 月曾给任弼时上书,
提议不仅要整顿三风,而且要「整顿人风」「信的内容同王实味的(野百 , 合花)
中内容类似」。} 最後无一幸免, 全被网入「王实味五人反党集团」
\footnote{宋金寿: 《关于王实味问题》,载《党史通讯》,1984 年第 8
期。}。

一波未平,一波又起。就在宣布揭露出「王实味五人反党集团」的同 时,中央党
校也传出揪出了「党校的王实味」——李国华和吴奚如的消息。李国华曾任延安马
列学院党总支常委,是曾经留苏的红军干部;吴奚如原 名吴习懦, 大革命时期曾
任叶挺团的连党代表, 1933 年到上海参加 「左联」, 1934 年冬转入中央「特
科」,1938 年,吴奚如受中共长江局负责人王明、 周恩来的派遣,随同叶剑英
在南岳军事训练班为国民党军官讲授游击战, 以後又担任中共驻桂林办事处负责
人,後转入新四军,皖南事变中被俘, 後逃脱来延安。
他还是 1940 年成立的延安黄埔同学会的负责人,
现在竟被
宣布为国民党特务。消息传来,在不大的延安城激起了强烈的震动,惊悚、
戒惧的空气迅速弥漫开来。

4 月方部署的地下审干活动,到了 9 月,就已获得很大进展,
陆续破获的「特务案」为毛泽东的论断提供了「有力」证据——党内暗藏
有托派、国特和日特,而自由主义则是产生敌人的土壤。此一形势大大鼓
舞了毛泽东,促使他调整整风战略,将整风的重点加速转移到审干、肃奸
的轨道上来。

1942 年 10 月 19 日,
毛泽东在西北局高干会议开幕词中再次强调审干、
反特的问题,将原先秘密进行的审干扩大至半公开的状态。毛向与会者发
出号召:
\begin{quote}
{\fzwkai 我们各个机关学校,要好好注意清查王实味之类的分子,要客 观的、精
细的、长期的去清查。\footnote{参见王秀鑫: 《延安「抢救运动」述评》,
《党的文献》。1990 年第 3 期;另参见《延安整风运动纪事》,页 298.}} 
\end{quote}

毛泽东严厉指责各级领导思想麻痹,
斗志松懈,
丧失了阶级的警觉性,
他抨击道:
\begin{quote}
{\fzwkai 过去我们对这些是采取不看不查的自由主义\footnote{参见王秀鑫:
《延安「抢救运动」述评》,《党的文献》。1990 年第 3 期;另参见《延安整
风运动纪事》,页 298.  }!} 
\end{quote}

11 月 21 日和 23 日,毛泽东花了两个整天时间在西北局高干会议上作 《关于斯
大林论布尔什维克化十二条》的长篇报告,毛在报告中严厉抨击 他所称之为当前
存在的两种错误偏向——「闹独立性和自由主义的偏向」, 毛厉声谴责党内有「一
部分反革命奸细、托派分子,以党员为招牌」进行 反党活动,他说, 「吴奚如就
是这样一个人」, \footnote{毛泽东: 《布尔什维克化的十二条》(1942 年
11 月 21 日),载《毛泽东论党的历史》,页 116-17.}毛并且正式宣布:
\begin{quote}
{\fzwkai 整风不仅要弄清无产阶级与非无产阶级思想(半条心)
,而且更
要弄清革命与反革命(两条心)
,要注意反特斗争。\footnote{参见王秀鑫:
《延安「抢救运动」述评》,《党的文献》
。1990 年第 3 期;另参见《延安整风运动纪事》
,页 298. }} 
\end{quote}

毛泽东对整风目的的新解释,
清楚表明从 1942 年 4 月开始的教化与镇
制并重的整风正快速由教化向镇制全面转移(2 至 3 月,是整风的动员和
「鸣放」批评阶段)
。很快,原先交叉进行的两条战线的斗争——公开战线
是以学习文件至为中心的整顿三风运动,隐蔽战线是秘密进行的审干、反
奸试点活动——迅速汇合为汹涌的审干、反奸狂流。

整风既以「弄清」党员是否是「半条心」、「两条心」为目标,毛泽东
提出的这个主张固然不会遭到党内领导层和一般高干的反对,
但是如何
「弄
清」,即运用什么手段对广大干部进行清查,
却是一个有待解决的棘手难题。
自从毛泽东秘密部署开展审干、反特斗争後,也只是由中央社会部选择几
个重点单位作了小范围的试点,绝大多数机关和学校对于如何进行审干、
反特仍然是一知半解。因此当务之急是提高各单位领导的肃反意识。

1942 年 12 月 6 日,就在中央总学委发出肃清「小广播」通知的当天,
康生以领导整风的中央总学委副主任和领导审干反特的首脑机关——中央
社会部部长的双重身份,在西北局高干会议上作有关审干肃奸的动员和情
况介绍的报告。康生首先描绘了一幅特务猖狂活动的恐怖画面:敌人已经
大量渗入延安和边区,潜伏在各机关、学校,尤以经济和文化单位的敌清
最为严重,以致一年来,各种破坏和阴谋活动层出不穷。紧接着康生严厉
谴责对「反革命的麻木不仁态度和自由主义倾向」,警告领导干部,敌人可
能就隐藏在身边,必须提高警惕,不得姑息养奸\footnote{华世俊、胡育民:
《延安整风始末》
,页 66;另参见陈永发:
《延安的阴影》
,页 60.}。

1943 年 1 月 4 日,审干试点单位——中央党校负责人彭真,继康生之
後也前往西北局高干会议作反奸肃反报告。
彭真在会上结合党校破获的
「吴
奚如特务案」详细「介绍如何与反革命斗争的实际经验,给到会干部以锄
奸工作具体方法的启示」\footnote{《延安整风运动纪事》
,页 346;另参见《谢觉哉日记》
,上,页 377.}。

康生、彭真的报告为毛泽东有关「弄清」
「半条心」和「两条心」作了
形象化的解释,对于将审干、肃奸推向全党起了重要的作用。中共西北局
书记高岗不甘落在康生、彭真之後,他也积极响应毛泽东的号召,在 1943
年 1 月 13、14 日所作的西北局高干会议的总结中,正式将「反奸」列为西
北局当前的首要任务之一。高岗鼓动各级干部「要从深入整风学习、检查
工作、审查干部中,清查暗害分子」
,同时指示各级领导务必「自己抓紧对
于本部门的审查和防奸的领导」
,不得单纯依赖边区保安处与组织部,
「所
有干部都必须学会如何与反革命分子斗争的办法」。\footnote{参见陈永发:
《延安的阴影》
,页 60.}最後,西北局高干会
议作出决定,实行党员重新登记,并划出了清洗人数的比例,将占党员总
数中的 10\%,包括奸细在内的坏党员清除出党\footnote{《延安整风运动纪事》
,页 107.}。

于是,原先以打击党内自由主义异端思想与异端分子为主要目标的秘
密审干、反特斗争,在毛泽东、康生等的精心领导下,已经发展到普遍清
查党员的审干——肃奸运动,斗争的范围也从原先的重点人群扩大到延安
的所有党员干部,而动员党员干部填写「小广播调查表」就成了全面审干
肃奸的前奏曲和突破口。

\section{向党交心:交代个人历史}
1942 年 12 月 6 日,以康生在西北局高干会议上作肃奸报告和中央总
学委发出肃清「小广播」的通知为标志,整风运动已过渡到审干、肃奸阶
段。初期,审干仍在地下秘密进行,由中央社会部、边区保安处与各机关
学校首长负责对可疑人员进行背靠背的侦察,在公开的场合,则仍以整风
为号召。给人们的一般印象是,整风似乎进入到整顿文风(反对党八股)
的阶段,在许多单位,甚至一边布置干部填写「小广播调查表」,一边还在
动员干部检查各人身上表现出的「党八股馀毒」
。但是,进入 1943 年後,
反对党八股的学习检查活动很快就告结束,表面的遮盖一旦揭去,审干、
肃奸的主题顿时凸现出来。
随着审干的节奏加快,
中央总学委继动员填
「小
广播调查表」後,又发动了坦白运动,指令每个党员和干部以书面的形式
详细交代个人历史。

1943 年 6 月 6 日,毛泽东给在太行的彭德怀发出一份电报,具体传授
延安开展运动的经验、方法和步骤:

一、关于写反省笔记的问题。毛要求彭组织干部对照季米特洛夫的四
条干部标准进行反省,让「各人」都写一次反省笔记。

二、
关于写思想自传的问题。
毛指示,
「可三番五次地写,
以写好为度」。

三、
关于发动坦白运动及动员填
「小广播表」。毛要彭德怀
「发动填
『小
广播』表格及社会关系表,在这两个表上叫各人将平日所作一切带政治性
而不应泄露的『小广播』及本人历史上各种社会关系统统填上去」。

四、关于审干。毛指示,上述一切都搞完後,
「才实行审查干部」
(主要是清查内奸)。毛告诉彭德怀,这些工作做好,
「就算是了不起的成绩,
我党百年大计即已奠定」。
\footnote{见毛泽东 1943 年 6 月 6 日致彭德怀电,载《文献和研究》
,1984 年第 8 期。}
毛泽东这份电报的中心意旨是将调查干
部个人历史、审干提高到一个过去从未达到的高度。然而根据中共组织原
则,凡申请入党的人员在入党之前都须向党组织交上自己详细的履历以供
审查,非党人士也得经过这道审查手续,方可在中共根据地的经济、教育、
文化等部门工作。在中共党内,隔三差五,要求干部填写履历表更是常事,
在 1940 年的审干中,延安的党员干部都已向党组织再次交代了个人历史,
因此,仅就党员干部向组织提供自己的书面履历而言,此举并没有多少新
意,也谈不上是什么新创造。

但是此时此地重提此事却是别有一番深意的。这是毛泽东为了深化审
干,从思想上和组织上加紧控制全党而采取的一项重大行动。

首先,党员提供的个人历史材料可以立即用于审干斗争。通过分析个
人交代的历史材料,中央社会部和各机关学校的首长,能够迅速排查出可
疑分子。

第二,从长久的战略性眼光看,此举有利于在每一个党员心目中确立
毛泽东的绝对权威。整风之初,胡乔木秉承毛泽东的旨意,提出「脱裤子,
割尾巴」的口号,但在当时,其针对的对象主要是王明、博古、张闻天等
「教条主义者」和一批有留苏背景的知识分子,广大中下层党员普遍认为
与己无关。随着形势的变化,毛泽东意识到可以将「脱裤子,割尾巴」的
内涵丰富化,使「脱裤子,割尾巴」的对象从「教条主义大师」扩大到每
一个党员,让全党上下都「脱裤子,割尾巴」
。要求党员将自己的历史事无
巨细和盘向党交代,同时「将一切对不起党的事告诉党」
,就是「脱裤子,
割尾巴」的具体化。在这个过程中,一方面是党员不断的透明化,另一方
面随着党员自我意识的日益消失,毛作为党员良知和共产党道德判断的最
高存在,已在党员的精神世界中牢牢占据主宰地位。

正因为毛泽东对干部交代个人历史一事极为重视,
并寄以很高的期望,
各单位在 1943 年审干的过程中,
都对干部交代个人历史作了极其详细、
严
格的规定,所要求的范围几乎涉及个人的历史与现实的一切方面。

按照中共组织部门的要求,
干部交代个人历史的形式主要有两种:
一、
填写干部履历表。二、书写详细的个人自传。在这两种形式中,以个人自
传为重点。

一份合乎要求的个人自传通常由五个方面的内容组成:

一、个人的一般概况。包括年龄、出身、专业特长和配偶姓名,政治
面貌等。

二、个人的学历,参加革命前的经历,参加革命後的经历及受奖惩情
况。这一部分为自传的核心部分。传主必须按年月叙述,不得有任何省略,
并且需要提供每段时期的证明人及证明人的工作单位。

三、家庭状况和社会关系。传主必须详细交代自己的阶级出身,家庭
经济收人状况,家庭成员的姓名、职业、政治态度以及自己与家庭成员的
关系。传主也必须交代自己与一般同学、老师、同事的关系,他们的姓名、
职业、社会地位和政治面貌。

四、个人对革命的认识以及思想变化情况。在这一部分中,传主必须
详细交代白己参加革命的动机,以及对当时国内外重大事件的看法。尤其
需要提供入党时的详细情况:由谁在何时何地介绍入党,是否履行过审批
手续。更要提供是否曾经被俘、被捕及受伤脱队的详情,传主必须交代事
情的原因及所有细节,以及各项事件的旁证人。

五、党性检查。传主必须根据整风文件的精神,详细反省个人参加革
命後的一切言论,
工作表现及工作作风,
对上下级的态度等各方面的表现。

按照中共组织和干部管理部门的一般常规看,一份包含了上述五个方
面的个人自传应该算是合格的了。因为无论从调查项目的细密程度,或是
涉及干部历史背景的广泛和深入程度看,这样的历史交代材料都足以使中
共组织对干部个人情况有一个完全、彻底的了解。换言之,当某个党员向
组织交上这样一份自传後,他已无任何个人隐密,实际上已成了一个透明
体。然而,事情并非如此简单,毛泽东对干部交代历史还有更高的要求,
他提出此类材料「可三番五次地写,以写好为度」。

组织部门对干部自传的撰写已有严格要求,事实上已到了事无巨细、
包罗万象的程度,为何毛泽东还不放心?他的「写好」的标准又是什么?
根据现有的资料分析,毛泽东要求干部「三番五次」写自传至少基于
两个原因:

一、通过「三番五次」的写自传,给干部造成巨大的心理压力,以彻
底摧毁党员的「资产阶级、小资产阶级的意识」。
所谓「写好为度」,
其实
并没有一个明确的标准,无非是要求个人交代得更细致。更广泛、更深入。
问题的要害是干部在「三番五次」写自传的过程中,必然进一步否定自我,
而对党的领袖和各级领导愈加敬畏,因为最後判断是否「写好」
,除了要看
是否符合整风文件的要求,主要取决于各机关、学校首长的态度。

二、
在干部提交的不同版本的自传材料中,
发现漏洞和自相矛盾之处,
再结合个人的现实表现,对照传主的反省笔记、
「小广播调查表」和多次填
写的履历表,在多种材料中,进行排比、分析,从中确定可疑分子。
如此看来,干部交代历史一事的意义可谓大也。一方面,它可以用毛
泽东的新概念来锻炼干部,考察和提高干部的党性觉悟,使党组织持久、
全面、彻底地掌握干部的一切;另一方面,它又可以借此发现敌人。作为
审干的一个中心环节,干部交代个人历史,终于成了锤炼「新人」成长的
铁砧。

\section{「脱裤子,割尾巴」
:在双重压力下涤荡灵魂}

从写反省笔记,到填写「小广播调查表」
,再到「三番五次」写个人历
史自传,延安的党员干部所面临的思想和精神压力步步升级。对于毛泽东
的这一系列举措,
不仅 1937 年後入党的新党员完全陌生,
即使是老党员和
老干部一时也茫然不知所措:因为毛泽东的这一套毕竟与过去的审干肃反
方式大不相同。

毛泽东的「新」就在于融理论灌输和暴力威慑于一炉,配之以强有力 的组织措施,
给广大党员, 尤其是知识分子党员制造了一座强大的压力场, 使其在反复震荡中
蜕尽「旧我」, 换上一颗全新的灵魂。对于毛泽东的这套思想改造术,谢觉哉有
十分生动贴切的解释,他援 引王阳明临死前说的「此心光光地」一段话,要求共
产党员遵照毛泽东的 教导,把心中的一切杂念,连根除掉。\footnote{焕南(谢
觉翮: 《此心光光地》,载《解放日报》,1942 年 7 月 3 日。}谢觉哉说,
改造自己,就是「把自 己完全变个样」, 他写道,如此过程, 「好比生肉煮成
熟肉」:「『五个月』学
习是紧火煮;
『长时期思想上教育与行动上实践』
(四三决定)是慢火蒸。
煮过了,并不就算『熟』
,还得长时期的熬炼,一直到要『而今而後,吾知
勉夫!』」

谢觉哉用一首诗形象地描述了如何脱胎换骨的要诀:
\begin{quote}
	\fzwkai 紧火煮来慢火蒸,

煮蒸都要工夫深。

不要捏着避火诀,

学孙悟空上蒸笼。

西餐牛排也不好,

外面焦了内夹生,

煮是暂兮蒸要久,

纯青炉火十二分。\footnote{焕南(谢觉觊:
《拂拭与蒸煮》,载《解放日报》,1942
 年 6 月 23 日。}
\end{quote}


又是「蒸」,又是「煮」,广大党员犹如进了一座思想高压炉。

压力之一,是来自个人内心的白我压力。经过对「二十二个文件」的
逐字逐句的精读,和反复对照检查,个人的自我意识开始分裂。随着「发
掘本心」的逐步深入,学习者普遍产生了负罪意识,知识分子党员更自惭
形秽,认为自己确实如毛泽东所言,除了读过一些如同「狗屎」般无用的
书之外,对共产党和人民毫无价值,尤其严重的是,自己甚至还会在客观
上危害革命,简直是罪孽深重!这样的自我压力有如大山般沉重,使许多
知识分子党员原有的自尊心、自信心和自豪感荡然无存。

压力之二,是来自集体的压力。党公开号召党员在批评白己的同时还 需揭发别人,
因此每个人都必须接受来自其他人的揭发和批判,而这些都 是以集体和组织帮助、
关心同志的面目出现的。陇东驻军「大渡」部队政 治处别出心裁,甚至发动了
「小册子运动」鼓励每人准备一个小本子,封 面上写着 「请为帮助同志而提上意
见吧」, 让持本者挨门挨户征求意见。\footnote{抗战期间驻陕甘宁边区八路军
各部队皆有代号,如「团结」部, 「澳洲」部等。参见《延安整风运动纪事》,
页 352. } 中 央党校一学员先後征求了所在支部三十多人对他的意见。
\footnote{《延安中央党校的整风学习》,第 1 集,页 101.}集体帮助的形式
也有两种,一种是温情式,另一种为斗争式,在更多的情况下,两种方式
交替使用。一般而言,领导同志和整风小组的骨干分子会不厌其烦、亲自
登门,苦口婆心地启发、引导当事者反省自己的思想和历史问题,其态度
之热情、诚恳。往往使当事者为之感动,于是将自己的「坏思想」和盘托
出。如果当事者仍然冥顽不化,拒绝深刻反省,基于「治病救人」的目的,
党组织立即采取下一步行动,布置小组批评会,让所有的同志,包括与当
事者有同乡、同学关系的人,对你展开面对面的揭发和「同志式的斗争」
。
首先「帮助」你端正态度,继而批评你的「错误言行」
,众口铄金,千夫所
指,使你孤立无援,有口难辩,直至当事者彻底「向无产阶级缴械投降」。


在自我压力与集体压力的双重重压下,个人的灵魂受到强烈的震撼和 撞击,犹如
历经一次漫长的心理炼狱的过程。在整风审干期间,干部们普 遍食不甘味,夜不
能寐。许多人因思虑用度, 「头痛、失眠、减少饭量,面 色发黄」, 以至「旧
病复发」。\footnote{《中央党校二部学风学习总结》(1944 年 9 月 17 日)
,载《延安中央党校的整风学习》,第 2 集(北京:中共中 央党校出版社;
1989 年),页 278-79.}  更有个别人因神伤气虚,心情极度焦虑、 紧张,以致
「午睡遗精」。\footnote{《中央党校二部学风学习总结》(1944 年 9 月 17
日),载《延安中央党校的整风学习》,第 2 集(北京:中共中 央党校出版社;
1989 年),页 278-79.} 为了使自己的反省获得组织的首肯,绝大多 数干部都
竭尽全力,反复撰写有关材料,惟恐对自己的挖掘、批判不够深 入而难于过关。
中央党校有个学员检讨自己的「小广播」, 竟写出八百多条。
\footnote{ 《延安中央党校的整风学习》,第 2 集,页 140.} 中央党校二部学
员的反省材料一般都「修改了三五遍」, 有的学员的材料 「修改了八次才完
成」, 少数人甚至 「修改了十三遍」。\footnote{《中央党校二部学风学习总
结》(1944 年 9 月 17 日),载《延安中央党校的整风学习》,第 2 集(北
京:中共中 央党校出版社;1989 年),页 278-79.} 与 工农干部相比,知识分
子干部所承受的精神压力更大,中央党校三部学员 刘白羽自陈, 「在那些难熬的
日日夜夜里」, 他「惶恐不安,彻夜难眠」,「产 生过种种幻灭之感」, 後来在
党校三部副主任张如心的具体指导下,
竟写下
「数十万字之多」的自传资料。刘白羽回忆道:
\begin{quote}
{\fzwkai 我受到审干运动的冲击,才从孤悬万丈高空,落到真正平实的
地面。在这个基础上,使我受益最深切,真正从精神领域进行一场
自我革命的,是用整风文件精神对照重新写自传,这是使知识分子
客观地认识世界,对症下药良好的方法。当时张如心同志是党校三
部的副主任,由他负责对我进行了审查。我详详细细从诞生之日起
一点一点严格剖析自己,对自己进行再认识。我写了一稿,自以为
不错,谁知张如心同志看了却不以为然,一方面严正地指出不正确
之处,一方面推心置腹耐心交谈,于是我又从头到尾写了第二稿,
还是不能通过,最後写了第三遍稿,张如心同志才点头认可。
\footnote{刘白羽: 《我的人生转折点》
,载《延安中央党校的整风学习》
;第 1 集,页 134-36.}} 
\end{quote}

刘白羽的回忆为人们提供了一幅精神炼狱的逼真画面,尽管他没有说
明为什么他的两稿自传都没被通过的原因,也没有具体描述张如心是如何
指导他抛弃「旧我」的,但我们仍可以从上述文字中窥见当年审干严厉之
一斑。问题是,如此酷烈的灵魂搏杀,能否产生毛泽东所预期的效果,答
案是肯定的。据刘白羽称,他就是经由审干的洗礼,
「在党的热切关怀,强
大威力推动之下」
,才犹如一只小船,
「终于漂向真理的彼岸」。\footnote{刘白羽:
《我的人生转折点》
,载《延安中央党校的整风学习》
;第 1 集,页 134-36.} 

和刘白羽的情况相类似,丁玲也经历了这种思想转变的过程。丁玲在
整风运动中一度是延安文抗机关整风领导小组的组长,也曾写下两本学习
心得:一本名为《脱胎换骨》
,另一本叫《革面洗心》
。1950 年,丁玲曾含
蓄地描述了当年她的那段心路历程:
\begin{quote}
{\fzwkai 在陕北我曾经历过很多的自我战斗的痛苦,
我在这里开始来认
识自己,正视自己,纠正自己,改造自己。……我在这里又曾获得
了许多愉快,
……我完全是从无知到有些明白,
从感情冲动到沉静,
从不稳到安定,
从脆弱到刚强,
从沉重到轻松……走过来这一条路,
是不容易的……凡走过同样道路的人是懂得这条道路的崎岖和平
坦的……。\footnote{丁玲的这两本整风笔记以後佚失,参见陈明:
《丁玲在延安——她不是主张暴露黑暗派的代表人物》
,载《新文
学史料》
,1993 年第 2 期,页 35-36.}} 
\end{quote}

不言而喻,不管是刘白羽,还是丁玲,要想「得救」,到达「真理的彼 岸」,
都是「不容易的」。这必须付出代价,这个代价就是「将一切对不住 党的事通通
讲出来」, \footnote{见毛泽东 1943 年 6 月 6 日致彭德怀电,载《文献和研究》
,1984 年第 8 期。}向党献上一颗赤诚的心,最後彻底埋葬「旧我」,走 向新生。

\section{「得救」:「新人」的诞生} 
 
对于已在组织内的一般中共党员和干部,能否「得救」
,即获得党组织
的真正信任和被组织完全接受,首先取决于党员个人对党组织的态度,而
判断其态度的重要标志,是看他(她)是否向党敞开心扉,将自己的一切
向党和盘托出。换言之,一个普通党员若想从孤立、苦闷、绝望的困境中
摆脱出来,唯一的出路就是向党忏悔。对于个人而言,寻求「得救」一旦
成为内心的强烈冲动,就使原本带有强迫性质的坦白反省挟有了一丝愉悦
的快意,许多干部为了赢得组织的好感,忽然变得异常积极、主动,甚至
不惜以精神自虐的方式渲泄个人的隐秘。一时间,延安出现了群体性的自
我悔过的热潮,在坦白内容的广泛性和自我鞭挞的严厉性方面,都达到了
空前的程度。

我们以中央党校三部女学员朱明的反省为例\footnote{ 朱明:
《从原来的阶级中解放出来》
,载《延安中央党校的整风学习》
,第 1 集,页 255-81.}。

朱明原先是一个「出身于剥削阶级」的学生,1938 年到延安後参加了
中共,被分配在文化单位工作,以後进入王明担任校长的延安中国女子大
学学习,继而调入中央研究院,最後被送入审干、肃奸重点单位——中央
党校第三部「学习」。

朱明反省的最大特点是她的坦率性、深刻性和广泛性。

一、首先,朱明直言不讳地坦承自己在一系列重大政治问题的看法上
与党的观点相左,承认自己同情蒋介石和国民党,仇视新生阶级,仇视共
产党,怀疑毛主席。朱明说:
\begin{quote}
	
{\fzwkai 回忆北伐前,我们住在安徽,当时在军阀统治下,不仅财产要 受勒索,
就连精神也受威胁,尤其是太太小姐们,不敢抛头露面, 总是坐在家里。……
当时蒋介石军队到南京後,我们可高兴了,因 为我们现有的资财不仅有了保障,
就连安徽的财产也被蒋介石解放 了。……精神上的成胁,同时也被解除了。}
\end{quote}

接春朱明反省了自己对十年内战的看法:
\begin{quote}
	
{\fzwkai 十年内战究竟是谁打谁,对这
个问题我也怀疑。书里说蒋介石 要坚决消灭共产党, 可是我在外面听说共产党
「捣乱」, 想要得天下, 不让蒋介石统一国家,复兴民族,所以才打。当时我认
为应该打, 因为共产党不安份守己,不让蒋介石统一国家,国不统一,民族 焉能
复兴?所以应该打。} 
\end{quote}
 
朱明其至坦白了自己原先对毛泽东作为中国人民领袖的怀疑:
\begin{quote}
	
{\fzwkai 「毛泽东同志是中国人民的领袖」,开始我听这句话,也是怀疑
的。因为在我思想中一贯认为蒋介石是中国人民的「领袖」
。他统治
中国,他领导抗日,共产党也是在他领导下抗日的,为什么要说毛
主席是中国人民的领袖呢?说他是边区人民的领袖还差不多,因为
只有这样大的一块地方,大後方的人民,我想还是承认蒋介石是领
袖吧。} 
\end{quote}

对于中共所宣称的蒋介石是「假抗日」之说,朱明也表示了强烈的反
感:
\begin{quote}
	
{\fzwkai 到底为了什么东西大家都说蒋介石抗日是为了消灭异己,不是
为了中华民族?在这种气氛中,我口里也不得不跟着大家一样说,
可是心里却想蒋介石抗日虽然是要消灭异己,但也是为了中华民族。
记得「八一三」我在上海的时候,亲眼看见中国飞机和日本飞机战
斗。晚上也听到中国飞机去轰炸日本军舰。我也看到过前线运下来
的伤兵。能说蒋介石不是抗日?……过去我一听到说共产党代表中
华民族,我就反感。我想共产党是国际主义者,哪里代表什么民族,
代表民族的是蒋介石,他要复兴民族。} 
\end{quote}

二、朱明反省的另一特点是她将自己作为反面典型,执意以自己的错
误来证明王明「右倾投降主义」与知识分子劣根性存在着密切关系。

朱明来延安後曾入中国女子大学学习,受校长王明的影响,女大较注
意对学生进行统一战线与国共合作的教育,对此,朱明专门结合自己的思
想进行了反省。朱明一方面检讨自己对毛泽东的态度,另一方面不指名地
批评了女大的「负责同志」:

几年前读毛主席《湖南农民运动考察报告》
,里面有一句:
\begin{quote}
	
{\fzwkai 
「土
豪劣绅的小姐少奶奶的牙床上,也可以踏上去滚一滚。」我很反感,
我想你们要打土豪分田地就打就分好了,为什么要去糟踏那些小姐
少奶奶们呢?从这里可以看出我的阶级立场,警惕性很高。

……我对毛主席的文件,是这样反感。但是,我对蒋介石的东
西怎样呢?在这里我附带地反省在女 大时的投降主义。当每年
「七·七」的时候,蒋介石发表的宣言,女大总是配合着时事来讨
论的。有时候负责同志还帮助我们指出宣言里哪些是比较进步的:
「譬如说团结吧,虽然提到,但还不具体,所以他的进步还不够」,
我呢,总是希望在宣言中找出一些「进步」的东西,因为我不希望
国共关系不好,负责同志有时还说:
「我们党在抗战中是发展了,可
是国民党呢?只要他和我们合作抗日,也是有前途的。有些工作他
做不好,
我们还可帮助它。
譬如保卫大武汉,
我们还帮助他动员哩。」

我听了这些话,就很能接受,我总是希望共产党能帮助国民党,这
样两党不会分裂,统一战线也才能持久。因为我有一个中心思想,
当我想革命的时候,我又怕吃苦,我总想过资产阶级生活,但又要
无产阶级的事业,这个矛盾如何能统一呢?我就想到了统一战线工
作,所以我不希望国共分裂,从我自己讲,我就希望国共长期合作,
我的矛盾也就长期统一了。
从家庭来说,
我也不愿意国共分裂。……
从国家民族的前途上讲,我也不愿意国共分裂,因为我把希望完全
寄托在国民党身上。} 
\end{quote}

三、朱明坦承自己由于在思想上与党不断抵触,最後发展到在政治与
组织上与党对抗,
承认自己对党组织阳奉阴违,
「总是采用合法的手续来超
越组织」。

朱明交代自己羡慕周恩来、林伯渠,总想做「特别党员」,希望到大後 方作统一
战线工作, 「也坐汽车,也住大饭店」,所以经常以要求学习为名, 逃避具体
工作:
\begin{quote}
	
{\fzwkai 比如我在大众读物社的时候,那里的会计因为生孩子到医院去
了。支部书记和我谈话,让我暂时代理一下,我心里不高兴,可是
没有办法,因为自己是候补党员,又是支部书记和我谈话,只好勉
勉强强地答应了。因此在工作中,手里拈着账条子,心里想着统一
战线,我做梦也没想过我会做这个工作。} 
\end{quote}

为什么不安心做具体工作呢?朱明坦承自己对延安的生活感到了厌
倦:
「在延安老是风平浪静的」,「也感觉不出什么阶级友爱」,
在几次申请
随林伯渠前往重庆做统战工作的要求被拒绝後,经过个人的顽强努力,排
除了种种障碍,终于以「要求学习」为名,转换了工作单位,先进了延安
大学俄文系,又转到中央研究院,到了研究院後,「一心想进国际问题研
究室,将来好做外交家」,「从来也没有想到,白己要服从组织」。


四、朱明从自己的「剥削阶级」出身挖掘思想的根源,认为自己之所 以「与党一
切都是分歧」,与她的阶级出身有密切关系。

对于阶级出身对自己带来的严重影响,
朱明分
「家庭教育和社会教育」
两部分进行反省。朱明说:
\begin{quote}
	
{\fzwkai 我的家庭是百年以上的剥削阶级,直到民国初年,才逐渐没
落。……外祖父本来是前清学者,其家庭也是百年左右的剥削阶级,
是地主式的书香门第。……

我的母亲和姨母们都懂得一些封建的艺术,早晨起来,还临帖
临《灵飞经》
,什么《高山流水》、《桐叶舞秋风》等曲谱也都懂得一
些。我自幼即生长在这样的环境中,所谓是三岁念唐诗,十二岁看
《红楼梦》的角色。……记得小时候听故事,从来就没听过工人和
农民的事情,专门是歌颂统治阶级的人物,就连外国故事也是一样,
总是讲什么公主、王子的遭遇,飞行船、玻璃鞋等神乎其神的事情。} 
\end{quote}

为了「说明阶级斗争不仅限于武装斗争」,朱明列举了她在衣食住行
方面所受到的「家庭的阶级教育」:
\begin{quote}
	
{\fzwkai 吃饭要慢,要不带声音。否则就骂你象饿死鬼,没有吃过饭
的,……说话更要轻声慢语。尤其是女孩子,要温存,还要深沉含
蓄,所谓轻声浅笑,不准张开嘴,哈哈大笑。否则就要为你是莽张
飞,……连走路你也不能随便,站也没有自由,必须要按他们那套
去做。比如走路要稳重,不能东张西望,一步一步走,走要端正,
站要站得笔直。否则就要骂你小家碧玉,像牵牛花一样依靠墙壁。
所谓大家闺秀,像梅花,像牡丹,不仅要风骨凛然,而且还要仪态
万方。} 
\end{quote}

接着,朱明以自己的亲身经历,控诉资产阶级学校教育的罪恶;声称
「资产阶级教育的中心,就是培养大私无公的个人英雄主义。不管科学也
好,艺术也好,就是为了这么一个目的」。
朱明检讨道:
\begin{quote}
	
{\fzwkai 我在资产阶级的学校中,受了一些什么样的教育呢?……因为
我是一个女的,家里希望我懂得一些文学,学一些艺术,所以我从
小就很喜欢文学。斯大林同志说,文学家是人类灵魂的工程师。这
是说无产阶级的文学家,资产阶级也有雕刻灵魂的工程师。我的灵
魂就是被他们雕刻过的。我喜欢反映自然的印象派的东西,什么月
亮怎么亮,花怎么香,……可是鲁迅的东西,我就不喜欢。……对
于旧俄时代的东西,我是喜欢的,如托尔斯泰、屠格涅夫的作品,
社会主义的东西我就不大喜欢。譬如《安娜·卡列尼娜》
,我很喜欢,
觉得很熟悉。
《钢铁是怎样炼成的》、 《被开垦的处女地》
,我就不感
兴趣。什么牛、猪,我觉得没有意思。再说音乐,我到延安後,就
很少唱歌,因为我喜欢「山在虚无飘渺间」这样一类东西。劳动人
民的歌声,我是不感兴趣的。……讲到画呢,我也喜欢资产阶级的
一套,比如鲁迅介绍的版画,我虽然买了,但不喜欢。我喜欢古典
的画,如意大利画家画的「蒙娜丽莎」像,我非常喜欢。} 
\end{quote}

五、朱明给自己戴上一串政治大帽子,用自唾自责、自我鞭笞的方法
表达她「从原来的阶级中解放出来」的决心。

朱明的反省通篇充满自我责骂的词句,诸如:自己一贯坚持大地主、
大资产阶级的政治立场,
「站在蒋介石方面,替大地主资产阶级说话」,「没
有一点劳动人民的感情」,
「在政治上、
思想上、
组织上都是与党不一致的」,
「对国民党有感情」,
自己希望「站在广大人民的头上」,「总想做一个特殊
人」,到延安是出于个人英雄主义等等。
朱明甚至将自己贬低到贱民的地步,
她痛悔由于自己出身于「剥削阶级」的家庭,致使「我的血液都带有剥削
阶级的成分」
。为了表达她的思想转变,朱明干脆直接歌颂「血统论」的合
理性:
\begin{quote}
	
{\fzwkai 今天我明白了,党为什么珍惜无产阶级出身的干部和革命後代,
以及先烈遗孤,因为不仅他们的思想有传统,就连他们的血液也是
干净的,党为什么不珍惜他们呢?} 
\end{quote}

最後朱明表示,从今以後,自己将心甘情愿「做无产阶级的牛」\footnote{朱明在 1945 年 5 月与林伯渠结婚,文革中因不堪迫害自杀身亡。}。

朱明的反省提供了延安干部自我反省的合格样本。也许朱明的反省确有
若干真实的内容,但是,这份反省人工雕琢的痕迹太重,它简直堪称标准
的「反革命百科全书」。它几乎具备毛泽东所要批判的资产阶级知识分子劣
根性的所有表现,并为毛的有关知识分子的论断提供了充足的证据:

一、出身剥削阶级的党员知识分子,其思想与行为明显烙有反动阶级
的烙印,他们虽然在组织上入了党,但在思想和感情上并未真正入党。

二、受过系统的资产阶级的教育而未经改造的知识分子党员,在一切
问题上与党和革命格格不入。

三、知识分子党员极易由思想上与党的对立发展到在组织上对抗党。

四、资产阶级和小资产阶级知识分子是中共党内机会主义错误路线的
社会基础。

五、只有经过长期艰苦的改造,知识分子党员的资产阶级世界观才可
能转变。

问题是,知识分子所有的劣根性难道都集中在朱明这「一个人」身上
吗?从朱明的反省所暴露出的思想看,
她岂止是一个未改造好的知识分子,
而更像一个十恶不赦的反革命,可是人们又不禁发出疑问,朱明既然有那
么多的反动思想,为何又要投奔延安?

种种迹象表明,朱明的反省是在领导的诱导下写成的。为了给毛泽东 的论断提供
具有个性特征的实证资料,一定有某些「灵魂工程师」对朱明 的反省进行精心的
设计和加工,使其符合毛所需要的一切特征,而这种设 计加工现象在坦白运动期
间是普遍存在的。在康生等人的速成训练下,许 多单位的审干领导小组的成员已
学会了 「政治诱导术」, 他们巧妙地利用干 部们急于解脱的心理,交替使用大
棒与胡萝卜,劝导被审查的对象给自己 层层加码,上纲上线,使他们相信,非经
自唾自责,不足以证明自己对革 命的忠诚。在这种巨大压力下,当事者只得依据
诱导者的逻辑推论,硬将 自己描绘成极反动、极卑鄙的「两条心」,以满足领导
者的「关心」和「爱 护」。在精神几近崩溃的状态下,当事者从「你要什么,我
就给什么」,逐 渐发展到主动的忏悔交代,然後进入到亢奋性的渲泻状态,最终,
外力与 内力产生了奇妙的结合,当事者开始出现旧皮蜕尽的喜悦。因此朱明的反
省,不仅是思想改造能工巧匠手上创作的一件工艺品,也是朱明心理状态 复杂变
化的产物。

向组织彻底交代自己的一切,只是标志着当事者已初步迈入无产阶级
的门槛,如何巩固既有的成绩,防止旧思想复辟,还须经由一定的形式才
能完成,这就是当事者必须在公开的场合暴露自己的丑恶思想,接受同志
们的批判。

在思想改造过程中,个人进行坦白的形式也是一个极为重要的环节。庄重的、 富
有礼仪色彩的坦白形式对加强党组织的权威, 增强党的凝聚力, 教育当事者和其
他党员起着重要的作用。在一般情况下,党支部或党小组 召集会议,让当事者当
众宣读自己的书面交代,接受每一个出席会议的党 员的质疑和批评。在这种会议
上,每个人都应积极发言,为了表明自己党 性强、觉悟高,宁可说过头话,也不
可显出「小资产阶级的温情主义」。当 事者则必须拿出小本子,虚心记下同志们
的批评,而决不可作申辩。如果 某人急于解释,强调各种客观原因,那么大家就
会众口一声,指责此人态 度很不老实,坦白极不深刻,结果这位党员必须再次写
出交代,直至领导 和与会者一致认可才能过关。如此循环,该支部或小组的各个
成员,每人 都需经历这一程序。会议的组织者,由于他受到上级的信任,负责所
有人 的最後鉴定,他的个人权威在这样的过程中得到充分的体现,与会者都众 星
拱月般地簇拥在他的周围, 以他的意见为自己的意见。而他本人的交代, 往往
一次就能通过。党支部或党小组的负责人,还负有发现、培养坦白典 型人物的责
任,一旦某人的反省被认为具有典型意义,负责人将推荐这个 党员到更高一级的
会议上当众忏悔,以现身说法的方式,引导更多的人走 坦白道路,同时,也以此
向上级表明本支部开展运动所取得的成绩。

经由这套程序,
「新人」终于诞生了,随着党组织给每个党员作出政治
鉴定,个人有了新的归属——此生不仅思想上,而且身体和生命都无保留
地属于党。
从此,
这个世界上就少了一个个人主义者或小资产阶级分子(「半
条心」),而又多了一个无产阶级的革命战士( 「一条心」)。

个人一经获组织的肯定和接纳,就「像越过一道阴阳分界线」,精神面 貌顿时焕
然一新,萎靡之气一扫而空。有人形容这种感觉如同「穿过黑夜 , 走向黎明」
「看到鲜红的晨光」。\footnote{ 刘白羽:
《我的人生转折点》
,载《延安中央党校的整风学习》
,第 1 集,页 136.} 

经由坦白、审干运动铸造的新人已具有某种特有的气质。他们中的绝
大多数人确实已牢牢记住了毛泽东的一系列重要概念,并学会用这套概念
来观察世界和指导个人的言行。表现在行动上,则是彻底抛弃资产阶级人
性论和温情主义的任何表现,非党性勿言、非党性勿听、非党性勿动,绝
对服从领袖、组织、上级的命令和指示。

然而就在「新人」成批产生的同时,保守苟且、浑浑噩噩的「机械人」作为「新
人」的伴生物,也开始在革命队伍中出现。为了服从现实生存和 政治发展的需要,
根据地内的许多党员学会了隐瞒真实想法,而随声应和 上级的指示。根据地内的
人际关系也发生了深刻的变化:原先基于共同政 治理想而结合的「同志」关系,
慢慢向人身依附的关系转变,冷漠、猜忌、 互相防范逐渐取代了同志间的亲爱、
坦诚。\footnote{ 伴随着毛泽东新权威的确立,根据地内的等级制度也基本形成,
在这个过程中,根据地内的人际关系发生了静 悄悄的改变,讲人情,谈私交开始
受到遏制。早在 1939 年,曾三对此就有过议论,他认为, 「除同志关系外,不
许有 私人感情,这话不尽对。只能说私人朋友感情是次要,不允许超过或并重于
党的利益,而不能说私人感情须一笔抹杀, 这是违反人情的。这一趋向的发展,
可能走到人间的冷酷。」参见《谢觉哉日记》,上,页 284-85. 对于根据地内
的反 「人性论」的气氛,初入延安的从事白区地下斗争的党员和外来知识分子都
觉得不甚习惯。因为在白区,同志之间的 关系都十分亲密,到了「家」,情况反
而不一样了,所以一时间, 「延安缺少同志间的友爱」成了许多人的共同感受
(丁 玲、王实味、萧军、朱明)。王世英作为党的高级干部在进入延安後,感触
最深的就是「人际关系越来越难处」。他对 刘少奇说:回到延安,我感觉学了一
些坏东西,自己不愿做的不愿说的,也得去做去说感觉没有在秘密工作时期那样
。他告诉王世英: 「所谓好坏之分,应从党的工作,党的利益 出发, 吹牛拍马
不好, 但对工作有利就是好的, 就要做。」参见段建国、 贾岷岫著, 罗青长
审核: 《王世英传奇》, 页 244-45.  这种革命的「吹牛拍马」,以後伴随着
等级森严的干部级别制度,逐渐发扬光大,成为某种新政治文化的基本特征,以
至于长期从事国统区工作、从未去过根据地的夏衍在 1949 年听到别人称他「高干」
,看到革命文艺家马寒冰向他敬礼 报告时,还很不习惯。参见夏衍: 《懒寻旧梦
录》,页 62-22、640.  }口是心非、投机钻营、见风使 舵、趋炎附势之辈渐渐
充斥中共党内。由于人性毕竟非强力和说教所能完 全改变,作为一种整体性的现
象,具有双重人格的党员在整风审干後开始 出现。

勿庸置疑, 交替使用教化与强力方法锻造 「新人」是毛泽东的一项 「伟 大」
的发明,与斯大林的清党和肃反运动相比,坦白审干运动在触及和改 造人的灵魂
的深度和广度上都达到了前所未有的程度。但是一经细致考察, 我们仍可发现,
这场基于明确政治目的而发起的运动,除了套用列宁、斯 大林的若于概念和方法
之外,在其运作方式和操作实践的背後,还有着浓 厚的中国内圣之学的痕迹。干
部坦白交代和自我剖析与宋明新儒家的「格 物致知」,寻求「天人合一」的路向
几乎异曲同工,只是词汇和解释系统不 同,而在手法上更具强制性。
