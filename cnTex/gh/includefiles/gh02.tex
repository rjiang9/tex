\chapter{遵义会议後毛泽东的权力扩张和来自莫斯科的政
策干预}

\section{毛泽东逐步掌控军权、党权}
毛泽东自诩为「以其道易天下者」, \footnote{参见梁漱溟:
〈再忆初访延安〉 ,载《我的努力与反省》
(南宁:漓江出版社,1987 年) ,页 319.}「道」者,个人对改造中国社
会和世界所持的理想抱负也。
那么,
三十年代前期,
毛所企盼实现的
「道」,
其具体内涵又特指哪些方面呢?作为一个已接受共产主义基本概念的中共
领导人,致力于结束国家分裂混乱局面,创建一个以共产主义为价值符号
的公平、正义的社会,这或许距毛当时所要实现的「道」不至相差太远。
然而,此「道」与彼时一般共产党人之「道」并无多少差别。毛的「道」
之特殊性,即此时的毛已开始萌发若干有别于莫斯科「正统」理论之片断
想法。毛基于多年在乡村领导农民革命之体验,已具体感受到在共产国际
指挥下之中共中央诸多政策和实践与中国社会环境之间存在严重冲突,而
由此冲突显示出的中国社会环境对莫斯科理论之拒斥,将严重阻碍中共在
中国社会扎根、断送中共取国民党天下而代之的大业宏图。

对于胸怀济世之志,如毛泽东这样的聪秀之人,
「道」之产生并非太
难, 其乃源于对现实的直接感悟,
只要将其略加提升, 「道」
即可了悟于心。
困难者,实现其「道」必先有其凭籍,即所谓有道无恃,道乃虚空,有恃
无道;其恃也忽。只有融「道」 (思想、理念) 、术(策略、方法)、
势(地 位、权力)于一体,方可出现运动中的良性循环,并渐次向理想境界迈进。

如果说,1927 年以前的毛泽东对上述三者之有机关系尚无直接感触,
那么到了 1935 年,在历经开创红色根据地的万般辛苦和多年党内斗争的
沉浮後,毛对其间关系之体认就深镂于心了。所幸天佑中共,毛泽东这个
「本来很灵」,但被扔到茅坑里去,
「搞得很臭」的「菩萨」,在遵义又开始
「香起来」\footnote{《毛泽东接见佐佐木更三、黑田寿男等日本社会党人士的谈话》
(1964 年 7 月 10 日,载《毛泽东论党的历史》
(南京:南京大学印行,无出版日期) ,页 4.},并被大家捡了起来,原就素有「救小人」之志的毛,果然义
不容辞,在遵义会议後立即就行动了起来。
 
极具现实感的毛泽东深知,在 1935 年,他实现其道的唯一凭藉就是
中共及其领导的军队。然而,遵义会议及以後陆续发生的中共核心层人事
变动,
只是使毛在中共最高决策和指挥系统第一次获得了发言权和决定权,
离执掌党和军队「最後决定权」的距离尚远,这种状况虽非令毛满意,但
在当时,也只能如此。在中共面临危急存亡的紧急关头,毛选择了「见好
就收」的方针,主动放下党内分歧,将全副精力用于对外,此既是毛的明
智,也是形势使之然。

1935-1936 年,面对国民党的军事追击,维持中共及其军队的生存,
是压倒一切的头等大事,
但是对毛而言,
事实上却存在着并行的两条战线。

第一条战线是对付国民党的外部战线。不言而喻,
「易天下」即是缚
国民党之「苍龙」 。在中共未夺取政权之前,威胁中共生存和发展的主要力
量只能是蒋介石政权。因此,如何回击并打败国民党,不仅是毛须臾不能
忘怀的首要问题,也是毛用以凝聚、驾驭和统一全党的最重要的政治理念
和驱动力量。

与第一条战线相比,第二条战线虽不那么凸显,却同样重要——这即
是党内斗争的战线。显而易见,欲易蒋介石政权之天下,若不牢牢掌握中
共及其军队,则一切免谈。

众所周知,毛泽东是中共军队的主要创造者之一,但是,毛对江西中
央红军的实际控制力在 1932 年後的中共上层斗争中逐渐被削弱,以至最
终丧失。军队高级领导人受到党的影响,对毛泽东渐趋疏远,直至遵义会
议召开前,毛泽东对军队的影响力仍是晦暗不明。将毛与军队领导人联系
在一起的唯一共同点,就是双方对 1934 年後中共军事指挥的不满。因此,
毛泽东在遵义会议获胜後的首要任务就是将其分散在军队中的影响力重新
聚合起来,将他在遵义会议上所获得的政治优势迅速落实到对军队的掌握
与控制上。

1934 年 10 月,中央红军号称十万,实则八万,自江西突围,行至遵
义时,因战斗伤亡、脱队,人数已减至三万多人。领导这支军队的红军将
领,分别是红一军团总指挥林彪和红三军团总指挥彭德怀。而由张国焘、
陈昌浩、徐向前领导的红四方面军和由贺龙、任弼时、萧克领导的红二方
面军,分别处于单独作战状态,林、彭所率军队实际上是此时中共中央唯
一可以依靠的力量, 林、
彭也是毛泽东急欲驾驭的军方两个最重要的将领。

毛泽东在遵义会议後期改组中央机构阶段,进入了中央常委会,但是
红军的指挥权仍归周恩来。毛泽东开始利用战争的紧急环境,有步骤地扩
张自己在军事指挥方面的影响力。1935 年 3 月 4 日,在张闻天的提议下,
中革军委以主席朱德,副主席周恩来、王稼祥的名义签发命令,决定成立
前敌司令部,
「委托朱德同志为前敌司令员,
毛泽东同志为前敌政治委员」,
\footnote{《毛泽东年谱》
,上卷,页 450.}
这个建制与任命,虽然没有改变周恩来的最後决策者的地位,但是,毛泽
东事实上已开始以政治委员的身份,担负前敌总指挥的职责。

毛泽东就任前敌总指挥一职,是迈向掌握军权的关键一步,但是几天
以後,围绕是否攻打打鼓场,毛泽东的意见与周恩来、张闻天、王稼祥及
一军团的林彪、聂荣臻等发生了分歧。在 3 月 10 日由张闻天主持召开的
中央政治局扩大会议上,毛泽东刚担任没有几天的前敌总指挥被撤消,由
彭德怀暂代。\footnote{张闻天:
〈我的思想检讨〉
(1969 年 6 月 28 日、
〈关于反革命分子林彪的一点材料〉
(1972 年 3 月 28 日,转引自
程中原:
《张闻天传》 ,页 218.} 

在这决定毛泽东前途的关键时刻,毛毫不退缩,他于当晚找周恩来详
谈,使周恩来接受了他不进攻打鼓场的主张。 \footnote{《周恩来年谱》
,页 277.}紧接着,毛泽东以日常军
事指挥需要完全集权的理由,向张闻天提议:成立「三人团」,全权指挥军
事,并得到了张闻天的赞成。\footnote{张闻天:
〈1943 年延安整风笔记〉
,引自程中原:
《张闻天传》
,页 219.}1935 年 3 月 12 日,张闻天在苟坝附近召开
中央政治局扩大会议,将成立「三人团」的提议提交会议讨论。会议批准
成立由周恩来、毛泽东、王稼祥组成新的「三人团」
(又称「三人军事领导 小组」,至此,毛泽东才正式进入军委最高领导机构。 )
毛泽东在短短的一、两天时间内,以坚强的意志力,挽回颓势,更进
一步扩大战果,
当仁不让,
毛遂自荐,
提议组成有自己参加的新
「三人团」,
并通过党的会议的形式,正式予以合法化,使自己名正言顺地成为当时最
重要的军事领导决策者之一。

毛泽东进入新「三人团」後,迅速使自己处于核心决策的重要位置。
在 3 月至 5 月的两个月时间里,毛泽东以大踏步回旋转移的指挥战略,集
中中央红军在黔、滇、川之间穿插运动,其间,既有胜利,也有战斗失利。
频繁的战斗和就地打圈的战略,使部队疲惫不堪,更遭致红军高级将领的
埋怨和不满。到了 1935 年 5 月上旬,由毛泽东部署的攻打会理城的战斗,
屡攻不下,这时红军领导层对毛的不满已到了一触即发的地步。
 
林彪正是在这种背景下,给周、毛、王「三人团」写了一封信,提请
毛泽东、周恩来、朱德随军主持大计,请彭德怀任前敌指挥,迅速北进与
四方面军会合。\footnote{《彭德怀自述》
,页 198.} 
 
林彪的这封信是在中共及中央红军处于紧急状态下写出的,
并无与毛
泽东有意作对的念头。在以往的历史上,林彪与毛从无个人过节,林彪的
这封信纯系出自他对红军前途的考虑,表达的是当时红军中普遍存在的情
绪。

无独有偶,类似林彪信中所反映的对毛泽东指挥才能的怀疑、不满情
绪在其他中共领导人中也不同程度地存在。1935 年 4 月中旬,刘少奇到红
三军团担任政治部主任不久,很快就感觉到部队中弥漫着对「只走路不打
仗」的强烈埋怨的情绪。他将了解到的军中情绪,结合自己的意见,给中
央军委发了一个电报,刘少奇拿着电报要三军团军团长彭德怀和三军团政
委杨尚昆签字,彭德怀认为电报所述内容与他的看法不同,拒绝签字,杨
尚昆则在电报上签了字\footnote{《彭德怀自述》
,页 198.}。

在新「三人团」中,王稼祥与毛泽东关系较为密切,但是他对毛的指
挥方法也存有疑窦。还在新「三人团」成立之前,王稼祥就常因作战指挥
问题, 与毛发生争论。
王稼祥还经常要求中央开会, 讨论军事行动。\footnote{参见程中原:
《张闻天传》
,页 218.} 新 「三人团」成立後,王稼祥对毛指挥部队大幅度运动不以为然,
他向张闻天表
示,老打圈圈不打仗,可不是办法。\footnote{张闻天:
〈1943 年延安整风笔记〉
,引自程中原:
《张闻天传》
,页 221.} 

林彪的信和刘少奇的电报,对毛泽东是一个危险的信号,是毛泽东在
遵义会议後所面临的最严重的挑战。这个事件性质之严重,不仅在于它显
示了对毛泽东的不满已蔓延至当时中共中央所赖以依靠的唯一军事力
量——红一军团和红三军团,而且这种不满正向中央核心层蔓延,如不立即予
以消除,毛刚刚获得的军权极有可能被再度削夺。

毛泽东迅速采取行动,正面反击这股由林彪领头的反毛风潮,他既不
采取与林彪等私下沟通的方式,也不逐个向中央核心层成员解释、征询意
见,而是将问题直接挑明。毛向张闻天建议,召开政治局扩大会议,得到
了张闻天的同意。毛的目的非常清楚,用召开政治局扩大会议的方式,以
党的名义将对自己的不利舆论打压下去。

1935 年 5 月 12 日,旨在批评林彪等人「右倾」、「动摇」的中央政治
局扩大会议在会理郊外召开。与会者仅有张闻天、周恩来、毛泽东、王稼
祥、朱德及一、三军团司令员和政委林彪、聂荣臻、彭德怀、杨尚昆。这
次会议名义上的主角是张闻天,他按照会前与毛泽东、王稼祥商量的报告
大纲,严厉指责林彪等人对毛军事指挥的怀疑是右倾。在这个时刻,毛颇
需要张闻天所擅长的理论语汇,只要将这股对自己不满的风潮压下去,扣
什么「帽子」都无所谓。

也许考虑到张闻天仅是一介书生,还不足以震慑林彪等武将,毛泽东
全然不顾自己是当事人的身份,亲自出马。严厉指责林彪、刘少奇,称彼
等信和电报是对失去中央苏区不满的右倾情绪的反映。 \footnote{《彭德怀自述》
,页 199;另参见《毛泽东年谱》
,上卷,页 455.}毛并认定林彪是
「娃娃」, 不明事由,
而将事先毫不知内情的彭德怀看成是幕後挑唆者。\footnote{《彭德怀自述》
,页 199.} 毛
在利用了张闻天作为反林彪、刘少奇的主攻手以後,迅速再将矛头转向张
闻天。毛不能容忍张闻天扮演党内最高仲裁者的角色,决意利用这个机会
打击一下张闻天的威信。毛在讲话中暗指张闻天去三军团,与彭德怀勾结
反毛。\footnote{张闻天:
〈1943 年延安整风笔记〉
,引自程中原:
《张闻天传》
,页 232.}这次会议以肯定毛的军事指挥、毛大获全胜而结束。面对毛泽东的
无端攻击,彭德怀、张闻天抱着「事久自然明」的态度,一切以大敌当前,
内部宜安定为考虑,在会上和会後都未予以说明和解释\footnote{《彭德怀自述》
,页 199;另参见程中原:
《张闻天传》
,页 223.}。

会理会议对毛泽东在党内和军内地位的确立具有决定性的意义。
如果
说遵义会议意味着毛泽东获得了政治上的胜利,使毛进入了中共最高核心
层,那么会理会议则标志着毛已将他在政治上的胜利具体落实到对军权的
控制之上,从此,毛成为事实上的军队最高领导人。毛用其坚强的意志力,
将红军最重要的将领林彪与彭德怀牢牢掌握在手中,并使自己在核心层中
处于不可批评的地位。毛泽东在会理会议前夕及会议上的行动,将刚刚获
得在「党内负总责」的张闻天置放于一个尴尬的境地,使代表党的张闻天
成为某种点缀和不具实际权威的象征人物。毛实际上用合法手段使自己成
为中共第一号人物,而这一切都是在周恩来对毛的妥协、退让下实现的。
 
会理会义也给日後中共核心层的内部关系造成复杂影响,
埋下了毛泽
东对彭德怀、张闻天怀疑、猜忌的种子。毛与彭德怀共事很久,两人个性
殊异,双方虽在 1931 年後关系疏远,但是并无明显矛盾和冲突。但是林
彪信一事,使毛认定彭德怀城府很深,从此对彭深藏防忌之心。会理会议
後,毛将曾参与刘少奇电报一事的杨尚昆调出三军团,而改派自己的老故
旧李富春任彭德怀的政委,实负监军之责。毛对张闻天素无好感,仅是为
了推翻博古,才使毛、张暂时联合。毛对张的固有成见,使他在指责过刘
少奇以後,将刘轻轻放过,并听信了刘少奇对张闻天参与和彭勾结的猜测
和判断。刘与军队素无渊源,在军中不具资望,毛并不认为刘少奇有在军
中掀风作浪的能力。刘少奇为脱身,迅速将责任推到张闻天身上,使刚刚
开始的毛与张的政治结合蒙上了阴影,\footnote{毛泽东固执己见,认定张闻天在会理会议前夕唆使彭德怀、林彪反对自己,1941 年後多次在核心层会议上就此
事指责张。被毛无端指责的张闻天一直忍辱负重,不予辩白,直到 1943 年 9 月政治局会议上,毛再次提及此事时,张
才在呈毛阅读的《整风笔记》中作了自我辩解。张写道,
「现在大致可以判明,说我曾经煽动林、彭反对三人团的话,
是 XXX 同志的造谣!
(林、彭二同志关于此事有正式声明)」。见程中原:
《张闻天传》
,页 233. 张在此处提到的 XXX 同
志极大可能是指刘少奇。从三十年代初开始,刘少奇就与张闻天长期处于对立状态,有许多资料证明,刘少奇利用一
切机会散布对张的不满。1966 年中共八届十一中全会期间,胡乔木揭发刘少奇在延安时曾在私下谈话中影射攻击毛泽
东,刘少奇当即加以澄清,说自己当时只是针对张闻天,而非毛泽东。
}也使毛对刘与张的对立关系有了新
的认识。但是刘少奇电报一事,还是使毛多存一分心计,为避免刘少奇在
军中培植影响,会理会议以後,刘少奇也被调出三军团。

1935-1936 年,毛泽东将其侧重点主要放在对付国民党的第一条战
线,在毛的努力下,红军阻遏了国民党对陕北的军事进攻,使中共的生存
环境获得了明显改善。
毛在军事上的成功,
对其政治生涯有极重要的意义:
在一个相当长的时期内,毛只是以擅长指挥军事而著称于中共党内,人们
看重毛,主要也因他深谙中国传统兵法并将其灵活运用于开创中共根据地
和发展中共武装。毛在遵义会议上之所以复出,最重要的原因也是中共军
事行动屡屡受挫,军事指挥已捉襟见肘,党和军队的前途万分危殆,中央
政治局一班人迫于无奈,请毛出山,试看毛能否使中共脱离险境。而在当
时,党的上层,从来也未将党领袖之名义与毛的名字联系起来,更遑论想
象毛登上军事指挥岗位即不再下来,并将其在军事指挥上的影响力迅速向
政治和党务领域延伸。

从主要担负军事领导责任到一身兼负党和军队的决策以及指挥责任,
毛泽东在党和军队中发挥的作用日益突出,此既是中共领导体制在战时环
境下变化的产物,又与毛所占据的特殊地位、他所拥有的独特的政治资源
有关。同时,这也是毛顽强努力的结果。

中共领导体制在战时状态下发生的变化,
对毛泽东顺利地将其在军事
指挥领域的权力延伸至党的领域有着直接的影响。中共在江西瑞金时期,
曾模仿苏联体制,建立起以党为核心的党、军队、政府三套相对独立的系
统,在这三个系统中,党机关的权力至高无上。博古虽是一介书生,对军
事指挥完全外行,但他主持的中央政治局和书记处却完全将军事系统置于
自己的领导之下。担任军事领导的周恩来、朱德、项英等严格遵循共产党
纪律,在作出任何重大军事部署前,均请示征得博古的同意。李德发挥的
作用虽然极大,但他并不参与政治决策,其对红军的军事指挥往往也是首
先向博古通报,并知会周恩来後,再发出作战命令,尽管他的个人意见一
般均是最後意见。长征前夕,战况瞬息万变,形势极端危急,为了适应战
略大转移的战时需要,党和政府系统全部并入军队,全部权力完全集中于
博古、李德、周恩来领导的「三人团」。遵义会议虽取消了「三人团」,但
在 1935 年 3 月,又根据毛的提议,重新由周恩来、毛泽东、王稼祥组成
「新三人团」。然而, 「新三人团」的体制却不同于「老三人团」,代表党的
张闻天并不在「新三人团」之列。遵义会议原来决定,周恩来是代表党在
军事上下最後决心者,毛泽东辅助周工作,\footnote{参见陈云:
《遵义政治局扩大传达提纲
(1935 年 2 月或 3 月),
》 载中共中央党史资料征集委员会、
中央档案馆编:
《遵义会议文献》
(北京:人民出版社,1985 年)
,页 42.}但到 1935 年春夏之交,周恩
来与毛泽东调换了角色——周成了毛的辅助者!本来,王稼祥因伤重,很
少参与决策,这样毛就成了事实上的中共最高军事指挥者。在紧张的战时
状态下,军队与党已溶为一体,当毛置身于领导军队的关键地位时,实际
上他已处于随时可以领导党的有利位置。
 
毛泽东作为中共军队的主要缔造者和中共最大一块根据地——中央
苏区的开辟者,在中央红军中拥有广泛的干部基础。毛所拥有的与军队的
这种特殊关系能够确保毛即便在政治上失意之时,也可以对军队发挥一定
的影响力。与绝大多数中共领导人不同,毛还是参与建党的元老,他是硕
果仅存的几个中共一大代表之一,
其在党内历史之长,
在军中基础之深厚,
除张国焘之外,1935-1936 年中共领导层中的任何人都无法与其比肩。毛
完全可以依靠其在党内的资历和地位,就党的全局性的方针政策和其它非
军事性的问题提出自己的主张,而不致遭到越权的指控。

正是基于上述因素的合力作用,毛泽东在 1935-1936 年使自己在中
共领导层中越来越处于最有实力、最具影响力的地位。

在大敌当前,全力指挥军事的同时,毛泽东对党的大政方针保持着高
度的关心。一方面,毛不敢冒任何风险,谨慎地在莫斯科划定的禁区前穿
插迂回,努力维持着中央领导层的稳定;另一方面,毛又不失时机,利用
战时状态提供的机会,巧妙地运用自己的影响力和特殊地位,有条不紊、
小心翼翼地对党的重要机构进行局部调整。

一、在中央核心层,毛继续保持同「教条宗派分子」的合作,至少在
形式上,中共六届四中全会、五中全会形成的政治局的格局保持不变(在
正常情况下,大规模调整政治局需事先报经共产国际的批准)
。但是,从莫
斯科返国干部的具体工作,大多只限于党的宣传系统、技术性的党务工作
系统和地方工作系统。
「教条宗派集团」基本失去了对军队的影响力。与此
同时,个别军队领导被吸收参加了政治局,而一批重要的军事干部经常列
席政治局会议则逐渐成了惯例。1935 年 12 月 27 日,毛泽东扩张党权的第
一个大动作出台,由毛而非张闻天,在党的活动分子会议上,作传达瓦窑
堡会议的报告《论反对日本帝国主义的策略》。1936 年 5 月,由毛主见美
国记者斯诺。

二、毛将与周恩来等关系密切、且和莫斯科有较深情感联系的原国家
政治保卫局局长邓发调作其它次要工作,\footnote{1935 年 9 月下旬,邓发改任由原中央机关和红军总政治部机关组成的陕甘支队第三纵队政委,11 月,红一方面
军番号恢复後,邓发主要负责红军的筹粮工作。1936 年 4
月,他被委以中央代表的身份派往苏联。参见《周恩来年谱》, 页 293、306. }
将原由政治局直接领导、
因长征
而不复存在的国家政治保卫局易名为方面军政治保卫局,并派自己在江西
瑞金时的秘书王首道担任该局领导,将这个关键机构予以恢复,并划归于
自己管辖之下\footnote{长征开始,国家政治保卫局除少数负责人随首脑机关行动外,其他工作人员均被并入各军团,国家政治保卫局
只留下名义,工作权限已大大缩小。1935 年 10 月,王首道被任命为国家保卫局执行部部长,原执行部长李克农被调作
联络西北军和东北军的统战工作。该年年底,国家政治保卫局建制被正式取消,其工作由方面军政治保卫局接替。}。

三、毛任命王首道取代邓颖超负责刚刚恢复建制的中共中央秘书处,
并同时领导中央军委机要科、原国家政治保卫局机要系统,将原由邓颖超
负责的党、军队、秘密工作等全部机要通讯系统置于自己的统一管理和严
密监督之下\footnote{1934 年 10 月,
红军长征前夕,
中共中央秘书处,
军委秘书处均被裁撤,
其遗留工作由中央军委机要科承担。
1935
年中共中央迁到陕北瓦案堡後,中共中央各部委及秘书机构渐次恢复,原来仅有的机要机构——中央军委机要科一分
为三,成立了中共中央秘书处机要科、中央军委机要科和中央社会部机要科(其实当时尚无中央社会部这个机构,此
处所讲的中央社会部机要科实际上就是原国家政治保卫局管理的机要系统一笔者注)
,上述三个单位统归王首道领
导。参见费云东、余贵华:
《中共秘书工作简史(1921 一 1949)(沈阳:辽宁人民出版社,1992 年)
》
,页 186-87、204;
另参见《王首道回忆录》
(北京:解放军出版社,1988 年) ,页 197.} 。

四、毛深知掌握与莫斯科来往秘密电讯对其政治生涯的极端重要性,
从 1935 年末开始,毛就直接控制与莫斯科的电讯交通,而不容其他任何
领导人插手,\footnote{参见张国焘:
《我的回忆》 ,第 3 册,页 345;另参见师哲:
《在历史巨人身边一师哲回忆录》
(北京:中央文献出
版社,1991 年)
,页 203.} 从而确保自己在信息掌握、
研判及利用上获得任何人无法得
到的优势与便捷。

毛泽东对军权与党权的蚕食,
是在张闻天的配合及利用了张闻天的党
的领袖地位,以公开的形式进行的,遵义会议後产生的毛泽东与张闻天的
政治结合,为毛泽东扩张自己的权力提供了合法的掩护。

\section{从毛、张(闻天)联盟到毛、刘(少奇)联盟}
建立在反对原中共最高权力核心「三人团」基础上的毛泽东和张闻天
的政治结合是在遵义会议及其後形成并逐渐得到巩固的。为了反对博古等
人的极左的领导,从 1934 年 10 月起,毛泽东就加紧了与张闻天的联络,
在毛的启发和诱导下,张闻天和王稼祥相继从原中央政策的拥护者转变为
批评者,成为毛要求改变中央领导的重要支持力量。在遵义会议上,毛泽
东鼓励张闻天与博古、李德展开面对面的斗争,让张闻天在会议上起了十
分重要的作用。会议决定由张闻天起草决议,此举使张闻天在党的核心层
内的作用明显突出。1935 年 2 月 5 日前後,中央政治局常委会决定张闻天
取代博古在党内负总的责任,至此张闻天成了事实上的中共中央总书记。
这是在当时的形势和条件下,中共核心层所能作出的最佳安排。由于张闻
天与莫斯科有较深的历史渊源,
且是中共一个较长时期的主要领导人之一,
此举不仅可以减缓莫斯科对中共领导层变动可能产生的不安和疑虑,更可
以向全党,尤其是向那些与近几年党的方针、政策有较多牵涉的军政干部
显示党的路线的连续性,从而尽量减少中央改组在党内引起的震动,加强
党在极端困苦条件下的团结和统一。在张闻天成为中央总负责人之後,
1935 年春夏之交,毛泽东也取代了周恩来在红军中的最高军事指挥地位。
至此,毛泽东与张闻天,一个全力掌管军事,一个集中精力于党务,两人
开始了政治上的密切合作。
 
毛泽东和张闻天是性格完全不同的两类人。毛是中共元老,又多年在
乡野打游击,深受中国农民造反文化的浸染,身上既有源于高度自信的沉
稳和持重,又带有颇为浓厚的「山大王」气息;而张则是「红色教授」型
的知识分子。1935 年以前,张闻天对毛虽未予以高度重视,但亦无明显的
成见;然而,毛对张则有一种类乎出自本能的排斥。毛鄙夷张等仅凭背了
一麻袋马列教条,却在莫斯科支持下来苏区夺权;毛更反感张以理论家自
居,隔三差五即有大报告问世,给毛造成巨大的心理压力。毛和张虽在反
对博古中央的基础上,达成了一种战略合作的关系,但毛从未将张放在眼
里。张闻天在军中毫无基础,其政治资源主要来自莫斯科以及他在政治上
对毛的支持,因而在毛、张联盟中,张只是一个弱势的合作对象,天平必
然向毛的方面倾斜。遵义会议後,毛泽东开始注意掩饰其个性中的不良方
面,但是一触及到利害关系,毛性格中的那种刚愎自用、猜忌防范心重的
特征立时就暴露出来,毛在会理会议上的表现即是明显的例证。
 
从毛这方面看,毛张联盟的最重要成果就是通过与张闻天的合作,联
合了暂时还占据党机关的「教条宗派分子」博古、凯丰等,运用党的权威
挫败了当时毛的头号对手——张国焘「另立山头」的分裂活动。在毛、张
合作共事的几年里,对毛个性已有了解的张闻天尽量避免与毛发生正面冲
突,对毛的咄咄逼人和峭刻嘲讽一再忍让。\footnote{参见张国焘:
《我的回忆》,
第 3 册,页 332. 毛泽东对洛甫的轻蔑态度在五十年代後期完全公开,其代表性的文
字是 1959 年 8 月 2 日《给张闻天的信》
,直至七十年代初,毛还不断数落洛甫。} 张闻天之对毛奉命唯谨,主要
是出于对共产党事业的考虑,在另一方面也与其个性温厚有关,但同时亦
是因为他已为自己创造了毫无依托的虚弱地位。
张闻天乃一书生型领导人。
置身于严酷的战争环境,只能唯毛马首是瞻,尽管张闻天还坚持着最後一
两个阵地不愿轻易放弃。

1936 年底至 1937 年初,毛在求「势」的过程中,熟练操用各种谋略,
已将许多大权集中于自己的手中。然而毛的胸臆仍难以抒解——在以其
「道」易中共路线方针方面,毛面临着巨大的困难,在党内同僚的压力下,
他只能隐忍内心的不满,违心接受对中共过去政治路线的评价。

「中共的政治路线是正确的」,这是横亘在毛泽东面前一座难以逾越
的高山,这不仅因为它来自莫斯科,也因为它是遵义会议参加者所一致拥
护和接受的正式结论,
它同样也是毛泽东与张闻天政治结合的基础。\footnote{张闻天在 1943 年整风期间写的笔记中指出,
「遵义会议没有提出过去中央政治上的路线错误,而且反而肯定了
它的正确......这在毛泽东同志当时只能如此做,不然我们的联合曾成为不可能。
」参见〈从福建事变到遵义会议〉
(1943
年 12 月 16 日)
,载中共中央文献研究室编:
《文献和研究》
,1985 年第 1 期。
} 在军
事压力紧迫和毛急于出山的 1935 年 1 月,他为了长远目标同时也出于现
实的考虑,可以同意这个结论,但是到了 1937 年,再继续维持这个结论,
就愈发显得强人所难了。
这个结论之所以要修正,是因为它关系到毛泽东能否实现其「道」,
从而为党及其个人在政治前途上开辟一新的境界。不推翻此结论,便无从
摧毁「教条宗派集团」的政治合法性基础,毛就难以顺利地推行他改造党
的一系列设想,毛的新概念的地位更无从建立。

然而,推翻此结论存在很大的难度,除了共产国际这一外部障碍外,
在国内最大的障碍就是毛的政治合作者张闻天。作为六届四中全会後产生
的中共领导人,张闻天几乎本能地将自己的政治前途与这个评价联系在一
起,断言「党的政治路线是错误的」将直接打击他和其他一大批领导干部
的威望,
严重动摇张闻天在党内的地位,
因此必然遭到张闻天的强烈反对。

1937 年初,党的发展、毛泽东和张闻天的政治结合,以及毛的思路
皆处在一个十分微妙的变化过程中,随着国内和平局面的到来,国民党军
事压力的舒缓,
解决战时状态下无暇顾及的若干重大问题的机遇正在出现。
与此同时,遵义会议後确立的毛掌军、张闻天管党的格局早已发生重大变
化,张闻天显示出他的作用仅限于党的理论和宣传教育领域。经过几年的
磨合,毛与周恩来已建立起融洽默契的合作关系,博古也安于自己在中共
核心层中新的角色,张国焘在党内斗争中的失败已成定局。现在毛已十分
具体地感受到张闻天给他带来的困窘。对于毛而言,在新的时空环境下,
继续违心地接受令其厌恶的对过去政治路线的评价将越发勉强,可毛又惧
于在条件尚未成熟之前,和盘托出自己的真实想法,从而将自己置于和张
闻天及一批党的高级干部发生正面冲突的尴尬境地。就在这关键的时刻,
1937 年春夏之际,刘少奇站了出来,就党的十年路线问题向张闻天发起挑
战。刘的出现打破了中枢层沉闷多时的僵局,并最终导致了毛、刘政治结
合的确立。

促成毛刘政治结合的契机是 1937 年 2 月 20 日、3 月 4 日刘少奇就中
共历史问题向张闻天陈述自己意见的两封信。刘少奇在这两封各长达万言
的类似政治意见书的长信中,大胆地突破了共产国际和遵义会议关于「中
共政治路线是正确的」结论,尖锐批评了 1927 年之前及 1927 年以来,尤
其是中共六届四中全会以後中共的极左错误。

刘少奇的长信触及了当时中共中央的几个禁区:

一、刘少奇认为大革命失败的主要原因不仅是「右倾的陈独秀主义」,
而且还有「右倾机会主义之反面的错误——『左倾』错误」, \footnote{参见刘少奇:
《关于过去白区工作给中央的一封信》
(19373 月 4 日)
,中央档案馆编:
《中共中央文件选集》
(1936
——1938)
,第 11 册,页 802.}刘少奇以自 
己亲身经历为例,猛烈抨击了广州、武汉时期工人及民众运动中已达「骇
人」地步的「左倾」狂热。
\footnote{参见刘少奇:
〈关于大革命历史教训中的一个问题〉
(1937 年 2 月 20 日), 载中固革命博物馆编:
《党史研究资料》 , 1980 年第 5 期。}
刘的上述看法与共产国际和中共六大以来
的历次决议严重相违。

二、
刘少奇虽然没有直接宣布中共十年来执行的是一条错误的政治路
线,但反复抨击中共「十年来一贯地犯了『左倾』错误」
,并且强调十年错
误已形成「一种传统」
。刘少奇特别集中抨击了中共有关白区工作的方针,
用釜底抽薪的办法全盘否定了十年政治路线。

三、刘少奇要求在党内公开讨论党的十年历史,并且详细述说自己因
坚持「正确」主张而遭「打击」的经历,把批评的矛头直指中共六届四中
全会後的中央政治局,暗示中央有关领导人要为错误承担责任,透露出要
求改组中共中央领导机构的明显意图\footnote{参见刘少奇:
《关于过去白区工作给中央的一封信》
(19373 月 4 日)
,中央档案馆编:
《中共中央文件选集》
(1936
——1938)
,第 11 册,页 805-17.}。

刘少奇在 1937 年 2 月 20 日、3 月 4 日给张闻天写信之前是否征询过
毛泽东的意见,或得到毛的鼓励,至今虽无确切的史料证明。但根据现有
资料分析,不能完全排除这种可能性。1935 年 12 月 29 日,中共中央常委
会议派刘少奇为中央驻北方局代表,1936 年春,刘少奇偕其妻谢飞,从陕
西临潼乘火车前往北方局机关所在地天津,于 1936 年 3 月抵达。1937 年
2 月,刘少奇又随北方局机关移往北平。至 1937 年 4 月底返回延安。这期
间刘少奇虽未返陕北,\footnote{1937 年 3 月 4 日,刘少奇致张闻天的信写于北平,3 月 18 日前刘仍在北平。周恩来在 3 月 13 日、3 月 18 日于
西安两次致函刘少奇,并转河北省委,要刘少奇等负起对留平、津地区的东北军的统战工作,参见《周恩来年谱》 ,页
358-59.}但是 1936 年後,在北方局和陕北之间已建立了电
台和信使联系。 1996 年出版的
据 《刘少奇年谱(1898-1969) 》披露,1936
年 10 月 1 日、12 月 2 日毛分别三次致电刘少奇,毛还在 10 月 22 日写信
给刘少奇,\footnote{ 参见中共中央文献研究室编:
《刘少奇年谱(1898 一 1969),上卷(北京:中央文献出版社,1996 年)
》 ,页 160、169、163. 以下简称《刘少奇年谱》
。}毛、刘通过电台交换有关对全局性问题的看法,已具备基本条
件\footnote{刘少奇前往天津就任中共中央驻北方局代表时握有与陕北联络的无线电密码本。据当时担任刘少奇译电员的郭
、恩来,
「有时也直接接发给毛主席」
,署名则
明秋回忆,她经手翻译的刘少奇给陕北的电报,上款大都是洛甫(闻天)
是胡服(这是刘少奇在党内长期使用的化名)
。参见郭明秋:
《少奇同志在北方局》
,载《怀念刘少奇同志》
(长沙:湖
南人民出版社,1980 年)
,页 185. }。

且不论毛泽东是否曾对刘少奇写信的举动表示过支持,
刘少奇给中央
写信都应被视为是一个重大举动。刘少奇决定向张闻天陈述自己政治意见
的动机,一方面是刘少奇多年来就不满中共中央的一系列政策,另一方面
与刘少奇在北方局工作期间遭遇到党内左倾分子对新政策的强烈抵抗有
关。\footnote{1936 年 3 月,刘少奇以中共中央代表身份到达天津,不久又担任了北方局书记,着力纠正北方局「左的关门主
义」倾向。刘少奇领导的纠偏工作,除了思想纠偏之外,还包括纠北方局领导机构的改组,因而引起北方局内部的争
论。刘少奇上任後,任命彭真(1928 年彭真与刘同在天津的中共顺直省委工作)取代柯庆施担任北方局组织部长,任
命陈伯达为宣傅部长。刘的这些措施激起柯庆施等一批原北方局领导人的不满,刘在党内频频发表文章,不指名批评
柯庆施等的「关门主义与冒险主义」的错误,从而埋下柯庆施与刘少奇长期不和的种子。}除此之外,还有另一个重要的原因,即 1937 年的中共中央正处在调
整政策的关头,中共中央内还未真正形成某个人的绝对政治权威。张闻天
虽是党的总负责人,但其权力有限,其他中共领导人大都是独当一面;毛
泽东尽管处于上升状态,
但当时也并非是大家一致公认的唯一领袖。\footnote{刘少奇在 1937 年 3 月 4 日给张闻天的信中只字未提毛泽东,刘且写道:
「我国还没有中国的斯大林,任何人想
作斯大林,结果都划(画)虎不成。」参见中央档案馆编:
《中共中央文件选集》
(1936-1938)
,第 11 册,页 817. 由
此可见,安时毛的权威并未得到包括刘少奇在内的中共领导人的一致承认。
} 因此
刘少奇给张闻天写信,不仅不会遭遇到党的历史上屡屡发生的党员因向中
央陈述意见而被打成「反党分子」的厄运,相反,却有可能得到毛泽东的
支持。刘少奇很清楚,在对党的十年历史的看法上,毛泽东与自己有很多
共同语言。

刘少奇的长信在中共中央核心层引起轩然大波。1937 年 3 月 23 日、
4 月 24 日中央政治局两次开会都讨论了白区工作问题。
张闻天对刘少奇的
意见极不以为然,一些同志随声附合,认为刘少奇对大革命失败原因的分
析是替陈独秀洗刷,是陈独秀的「应声虫」
。还有人指责刘少奇受到了张国
焘的影响。\footnote{张国焘:
《我的回忆》
,第 3 册,页 375.}对于刘少奇有关中共犯了十年「左」的错误的看法,中央政治
局大多数成员也都认为言过其实。在一片责难声中,只有毛泽东一人站出
来替刘少奇讲话, 称「刘并没有反对中央的野心」。 毛的态度鼓励了刘少奇,
使刘少奇敢于采取下一步重大行动,在 1937 年 5、6 月间中共中央召开的
白区工作会议上,当面向张闻天发起挑战。
 
1937 年 5 月 17 日—6 月 10 日,
中共中央在延安召开了白区工作会议。
这次会议由于刘少奇与张闻天的激烈争论,其间曾一度中断,後在毛泽东
有倾向的调和下,会议才得以继续进行。
 
从 5 月 17 日到 5 月 26 日,是白区工作会议的第一阶段,会议围绕刘 少奇
〈关于白区的党和群众工作〉的报告展开了激烈的争论。刘少奇报告 的主要内容
是重复 3 月 4 日给张闻天信中的精神,着重批评十年来党在白 区工作指导中的
「左」的传统。刘的报告激起强烈反响,张闻天、博古、 凯丰、陈赓等都表示难
以接受刘少奇的看法,认为刘少奇的批评,充满托 陈取消派攻击共产国际,攻击
中共的论点。\footnote{参见郭华伦: 《中共史论》 ,第 3 册(台北:国际关
系研究所、国立政治大学东亚研究所,1971 年) ,页 189;另 参见中共中央文
献研究室编: 《刘少奇传》上(北京:中央文献出版社,1998 年) ,页 258-
59.}柯庆施在发言中,更是指着刘 少奇的鼻子骂他是「老右」 。\footnote{杨尚
昆在 1987 年改定的〈怀念少奇同志〉一文中,虽末点出柯庆施名。但他所称的
「那个坚持『左』倾机会主 义路线的人」明龃指柯庆施。参见《缅怀刘少奇》
(北京:中央文献出版社,1988 年) ,页 5. }许多代表反对刘少奇提出的白区
工作「损失 几乎百分之百」的观点,不同意刘对白区工作的总体评价,坚持认为
党的 六届四中全会後,白区工作的「总路线是正确的」\footnote{出席白区工作
会议的代表主要是北方局及所属北平、天津、河北、河南、山东、绥远等地党组织
的负责人及广 东代表约三十人。彭真作为华北代表团团长,是主持会议的刘少奇
的助手。华北代表团的代表有柯庆施、高文华(原 河北省委书记,兼原北方局书
记职能) 、吴德、李昌、李雪峰、黎玉、乌兰夫等。据参加过这次会议、1936 年
5 月被任 命中共山东省委书记的黎玉回忆,刘少奇的报告「对『左』的错误提得
很急,提得也很高」 。有关「白区损失百分之百」 的说法「有点过头」,因为
参加会议的「北方党组织的代表这幺多,就说明白区的损失不能说成百分之百」
。黎玉的看 法反映了当时参加会议部分代表的观点。参见黎玉: 〈抗战前夕在延
安召开的白区工作代表会议〉 ,载《革命回忆录》 增刊(1) (北京:人民出版
社,1981 年) ,页 42-43;另参见陈绍畴: 〈党的白区工作会议述略〉 ,载中
共中央文献研究室编:《文献和研究》 (1987 年汇编本)
(北京:档案出版社,1991 年) ,页 295.} 。

由于会议上出现的紧张激烈的争论,中央书记处宣布会议暂停。6 月
1 日至 4 日,中央政治局就白区工作会议讨论中提出的一些基本问题召开
会议,博古、凯丰在发言中都否认刘少奇提出的有关白区工作存在着一贯
的左倾盲动主义和关门主义传统的说法,只有列席会议的彭真支持刘少奇
的意见。\footnote{中共山西省党史研究室编:
《彭真生平大事年表》
(北京:中共党史出版社,1995 年)
,页 8.} 一时形势对刘少奇明显不利,
然而毛泽东在关键时刻助了刘一臂
之力。在 6 月 3 日政治局会议上,毛作了支持刘少奇的重要发言,他一反
不久前回避刘张争论的态度,明确表示刘的报告「基本上是正确的」称赞
刘在白区工作方面
「有丰富的经验」, 说刘系统地指出了党在过去时间在这
个问题上所言过的病症,是一针见血的医生。毛甚至称赞刘少奇「他一生
很少失败,今天党内干部中像他这样有经验的人是不多的,他懂得实际工
作中的辩证法」\footnote{《刘少奇传》
,上,页 26.}。毛避而不谈十年政治路线问题,而是针对反对派集中批
评刘少奇只讲缺点、不讲成绩,首先谈了中共所「取得的伟大成绩」,在谈
论了党的优秀传统後,毛着重指出党内「还存在着某种错误的传统」。强调
「这是不能否认与不应否认的事实」,从而全面肯定了刘少奇的观点,
在刘
张争论中有力地支持了刘少奇\footnote{《刘少奇年谱》
,上卷,页 183;另见程中原:
《张闻天传》
,页 372.}。由于毛泽东在发言中回避了容易引起分歧
的有关过去白区工作指导方针是否犯了十年一贯的「左」的错误这个敏感
问题,因此毛的意见得到包括张闻天在内的与会者的一致同意,并成为下
一阶段白区工作会议的主调。

1937 年 6 月 6 日,白区工作会议继续开会,会议进入第二阶段。张 闻天有意识
淡化毛泽东在 6 月 3 日讲话的倾向性,抓住毛讲话中对自己有 利的内容,坚持
自己的观点。\footnote{参见程中原:《张闻天传》,页 371}他根据自己理解的
6 月 1 至 4 日中央政治 局会议的精神,代表中共中央在会上作了《白区党目前
的中心任务》的报 告。张闻天强调「实践中的某些错误是不可避免的」,中共在
白区工作所犯 的错误性质不是政治路线错误, 「不是由于什么一定的政治路线或
政治倾 向」,「而是在领导斗争中有时犯了策略上的错误」,这种错误「不过是
整个 领导群众策略与群众工作方式中的某些部分错误,而不是整个领导的错 误」
,党「坚决领导的方针是完全正确的」 。张闻天坚决反驳刘少奇对中共 中央在白
区工作中反对「合法主义」的批评,坚持认为「过去党反对合法 主义的斗争,仍
然是对的」,强调指出,「过去一切非法斗争,是必要的与 正确的,而且过去
主要的斗争方式只能是非法的」。张闻天不无影射地批评刘少奇像俄国的普列
汉诺夫一样,「以每次革命斗争的成败的结果来判断革命斗争的价值」,把失败
的斗争看成「无意义」或「谩骂一顿『盲动主义』 完事」 指责刘少奇 , 「看不
到每一次革命群众的革命斗争, 就是结果失败了, 仍然有着他的巨大意义」 。
张闻天批评刘少奇把「关门主义」、「冒险主义」 作为「钢鞭」,全盘否定中共
十年白区工作的成就,强调指出, 「每一斗争 在胜利或失败之後,必须详细的研
究其经验与教训,切不要拿简单的空洞 的帽子(如盲动主义、冒险主义、机会主
义)去代替对于最具体问题的具 体分析」 。对党内所存在的「各种不正确思想,
应有确当的估计,不要夸大 或缩小,或任便给同志们『戴大帽子』\footnote{张
闻天: 〈白区党目前中心任务〉 (1937 年 6 月 6 日在白区党代表会议上报告
之一部分) ,载中央档案馆编: 《中 共中央文件选集》 (1936-1938) ,第
11 册,页 234-36、238-39、261、263.}。」

张闻天的报告获得与会代表的一致拥护,
在暂时不利的
形势下,
刘少奇被迫退却。6 月 9 日和 10 日,
刘少奇在会议上作结论报告,
表示同意张闻天的报告,
并且对自己前一报告作了解释和自我批评:
「我
在会上作的报告,着重是批评『左倾关门主义与冒险主义』,并不是否定过
去的一切,因为主要是批评错误这一方面,没有说到其它方面,并且对某
些问题缺乏具体分析,有些地方说过火了。」\footnote{参见陈绍畴:
〈党的白区工作会议述略〉,
载《文献和研究》
(1987 年汇编本),
页 298;另参见《刘少奇年谱》,
上卷,页 183.}

1937 年 6 月正是全面抗战爆发前夕,刘少奇、张闻天围绕党的十年
历史和白区工作评价问题展开的争论并没有获得任何实质性的结果,刘少
奇试图通过检讨党的历史问题,改变对十年政治路线评价的目的暂时遭到
挫折。但是刘少奇、张闻天的争论给中共带来了深远影响,它是延安整风
运动的前奏和一次针对「教条宗派分子」的短促突击,为以後毛泽东、刘
少奇全面批判六届四中全会政治路线,联手打倒王明等制造了舆论。毛泽
东从这次争论中吸取了丰富的经验,他终于体会到「教条宗派分子」在党
内的影响绝非一朝一夕经过一次会议就能清除。打倒「教条宗派集团」除
了需要在理论上进行细致深入的准备外,还要在组织上进行精心的安排。

刘少奇与张闻天争论的另一结果是扩大了刘少奇在党内的影响和知
名度。刘少奇虽属党的老资格领导人之一,但因长期从事白区工作,在江
西时期的两年仅负责领导全国总工会执行局,较少参与重大军政问题的决
策,刘与当时重要的政治领导人周恩来、张闻天等的关系又较为疏远,因
而在一个时期内,刘少奇在党和军队的影响力不大。与张闻天的争论充分
展现了刘少奇的思想深度和理论水平,使全党、尤其是党的高级干部对刘
少奇有了新的认识。

对于刘少奇与张闻天的争论,毛泽东的态度既明确又微妙。初期,毛
置身于争论之外,但对刘少奇明显表示同情,毛希望刘少奇的意见能被中
央领导层所接受;
後期,
则担心刘少奇承受不住张闻天和党内的巨大压力,
于是,在 6 月 3 日的政治局会议上,发表了声援刘论点的重要讲话。但是
当毛看到刘的有关看法遭到普遍反对,遂决定从长计议。毛在这次论战中
发现了刘少奇的杰出才干,首先是刘在白区工作方面的丰富经验;其次,
毛也看到了刘的理论能力,刘甚至能够引人入胜地分析十年「左」的传统
和根源之一乃是「思想方法与哲学方法上的错误」
,即「形式逻辑」是造成
「许多错误的根源」, 这给毛耳目一新的感觉。
然而毛并没有把自己的威望
全部投放在对刘少奇的支持上,因为时机还不成熟。现在毛更愿意做党内
争论的最高仲裁者,既然僵局已经打破,矛盾的盖子已被揭开,张闻天受
到了强烈震动,自己的威信反而因对争论所持的折衷调和态度而得到进一
步的提高,因此下一步的目标就是团结张闻天。为了防止张闻天和将要回
国的王明重新结合,加速「教条宗派政治组织上的分裂」
,维持和加强与张
闻天的合作,既有必要,又有可能。由于有了这些考虑,白区工作会议结
束後,刘少奇并没有立即被提拔进中共中央书记处(常委会) ,而于 7 月
28 日被派到太原,继续担任已从北平迁至太原的中共北方局书记的职务。

刘少奇虽然离开中枢,但毛张联盟从此走向解体,而毛、刘长达三十
年政治结合的基础却因此争论而告奠定。毛、刘与毛、张同是政治上的结
合,但是两种政治结合之间却有显著区别。

第一,毛张结合是战时非常状态下的临时组合。1935 年初,为了共
同的政治目标,毛泽东和张闻天有意忘却昔日政治上的分歧走到一起;毛
刘结合也是一种政治结合,但在毛泽东与刘少奇之间不存在政治观点上的
分歧,正是对原中央政治路线及其领导人的不满,使毛与刘走到了一起。
而早在 1932 年,毛刘之间就围绕此问题彼此交换过意见,并达成了一致
的看法。与毛张结合相比,毛刘结合具有更深的思想基础\footnote{据 1931 年秋至 1932 年底与刘少奇同在上海从事秘密工作、一度与刘少奇夫妇同住的张琼的回忆,刘少奇曾在
1932 年底就白区工作的策略问题写信给毛泽东,批评中共中央的左倾错误。不久毛泽东给中共中央「写来一封很长的
回信」
,表示赞成刘少奇提出的稳健主张。参见张琼:
〈刘少奇同志在上海革命活动片断〉
,载《党史资料丛刊》
,1980
年第 2 辑(上海:上海人民出版社,1980 年)
,页 48、47.}。

第二,毛泽东与张闻天没有很深的历史渊源,但是毛刘不仅有同乡之
谊,而且,早在 1922 年毛、刘就有工作上的密切联系。

第三,毛张结合是两个地位相近的政治人物的平行结合。但 1937 年,
刘少奇在党内的地位和影响则远逊于毛和张闻天。因此毛刘结合是一种以
毛为核心、刘为辅助的政治结合,而非两个地位相当人物的平行结合。
毛刘结合的上述特点保证了毛以後在向「教条宗派分子」发起挑战时
可以得到刘少奇的全力支持。毛刘的结合也预示着毛与六届四中全会後产
生的领导人两年多的合作已接近尾声。

然而,历史的发展并非直线。就在毛泽东一路凯歌行进的 1937 年,
也有坏消息传来,远在莫斯科的中共驻共产国际代表团团长王明即将携共
产国际新方针返国,
正待毛泽东加紧对中共重大方针、
政策进行调整之际,
半路上却杀出了一个程咬金,毛泽东面临着 1935 年以来最严重的政治危
机。

\section{1931-1935 年王明对毛泽东的认识}
毛泽东在遵义会议之後,逐步控制了中共军队,并大大加强了他对中
央机关的影响力,但是,在一个相当长的时期内,毛泽东尚未能将他的势
力延伸到中共领导机构的另一组成部分——中共驻莫斯科代表团。以中央
政治局委员王明为团长的中共驻共产国际代表团,因享有法理上和精神上
的巨大优势,在中共党内获有崇高的威望,毛泽东将不得不与从未谋面的
王明合作共事。
 
以王明为团长的中共驻莫斯科代表团是中共六届四中全会产生的中
共中央派出的,代表团中有四名政治局委员,他们是 1931 年 11 月 7 日抵
达莫斯科的王明,1933 年春抵达的康生(六届五中全会政治局委员) ,和
1935 年 8 月抵达的陈云(六届五中全会政治局委员) 。工人出身的陈郁虽
是六届四中全会上任命的政治局委员,但陈郁因在 1930 年末一度参与罗
章龙派的活动而在政治上不被信任,被罚去斯大林格勒拖拉机厂劳动,并
不参加代表团的实际工作。除了这四名政治局委员,代表团成员还包括吴
玉章、李立三、林毓英、饶漱石、赵毅敏,和 1933-1935 年赴苏的中国
苏区代表团成员高自立、滕代远、白区代表孔原。中共代表团成员同时还
兼任中国各赤色组织驻莫斯科的代表,黄药眠、饶漱石先後任中国共产主
义青年团驻少共国际代表,林毓英任中国赤色工会驻赤色工会国际代表。
中共驻共产国际代表团集中了中共在苏区以外最庞大的领导阵容。

以王明为团长的中共驻莫斯科代表团在苏联期间,
正是国内的中共中
央领导机构向江西苏区转移、共产国际蕴酿建立国际反法西斯统一战线新
方针的时期,保持与国内联系信道的畅通成为代表团最重要的活动之一。
中共代表团通过两个渠道与国内的中共中央保持着密切的联系:

一、开通大功率无线电秘密电台、中共代表团通过共产国际的电台和
在上海的中共中央地下电台,以及共产国际远东局的地下电台保持经常的
秘密联系。因距离遥远和技术手段限制的原因,莫斯科与江西瑞金没有直
接的电讯联系,
而必须通过在上海的秘密电台中转。
远东局和上海中央
(包
括中共中央迁江西後成立的中共上海中央局)分别有自己的秘密电台,远
东局给瑞金的电报须由中共中央上海地下电台翻译转送。代表团与国内的
电讯联系在红军长征後中断。
1935 年末林毓英携密码本自苏联秘密返回陕
北,国内与莫斯科的电讯初步恢复,而当 1936 年刘长胜再携密码本回到
陕北後,在陕北的中共中央与代表团的电讯联系就得到完全恢复。

二、派遣秘密信使。中共代表团通过回国的中共党员向国内的领导机
构传递重要的信息,1933 年公开赴苏访问的著名新闻记者、中共秘密党员
胡愈之就曾为中共中央和中共驻莫斯科代表团传递情报。共产国际并借助
在中国国内的中共组织的协助,招募中共党员为其搜集情报,这些直属莫
斯科指挥的中共党员,间或也为莫斯科与上海的中共中央传递消息。
\footnote{三十年代初,共产国际远东情报局在上海秘密成立,一度由苏联著名间谍佐尔格主持,1932-1933 年佐尔格
曾去北京、南京活动,佐尔格去日本後,远东情报局由华尔敦主持。远东情报局于 1935 年春被国民党破获。参见夏衍:
《懒寻旧梦录》
(北京:生活·读书·新知三联书店。1985 年)
,页 279;另见于生:
〈轰动一时的神秘「西人案」,载
〉
《革命史资料》
,第 3 辑(北京:文史资料出版社,1981 年)
,页 156-64.}

中共驻莫斯科代表团是中共中央的派出机构,按照中共的组织原则,
代表团的主要职责是代表中共与共产国际联络,
向中共传达莫斯科的指示,
在莫斯科与中共之间起上传下达的桥梁作用。代表团的另一项工作职责是
领导在苏联学习、工作的中共党员。从 1931 年 11 月王明赴苏至 1937 年
末王明返回延安,中共驻莫斯科代表团的活动主要集中在下述三个方面:

一、在共产国际内展开对中共及中国工农红军的大规模宣传。王明自
抵苏联後,以共产国际执委和中共代表团团长的身份经常在共产国际机关
刊物《共产国际》和联共(布)机关刊物《布尔什维克》发表文章,宣传
中共主张,介绍苏区各方面情况。1932 年王明指派萧三以诗人身份参加在
苏联哈尔科夫召开的国际革命作家联盟大会,与高尔基、巴比塞等著名左
翼作家联络,以扩大中共的影响。1935 年,王明又指派吴玉章、饶漱石等
前往巴黎,创办中共报纸《救国报》
(後易名为《救国时报》)。王明并以其 
在共产国际分工主管拉美共产党事务的便利,指导美国共产党内的中共支
部,在美国创办华文报刊。

二、领导在苏联的中共党员。三十年代在苏联仍有不少中共党员,分
散在莫斯科的列宁学校和莫斯科的外国工人出版局中国部等单位。在远东
地区也有一批中共党员在苏联各单位工作。由于在苏联的许多中共党员同
时又是苏共党员,中共代表团所能领导的只是在莫斯科的部分中共党员,
莫斯科以外的党员基本上归苏共领导。

三、
配合苏共清党,
在莫斯科中共党员中厉行肃反。
早在 1927 至 1929
年王明在莫斯科中山大学学习期间,就曾密切配合校长米夫和苏联秘密警
察「格伯乌」,将持不同意见的中国学生投人监狱,或送至西伯利亚和北极
地区劳改。\footnote{参见庄东晓:
〈莫斯科中山大学与王明〉
,载《广东文史资料》
,第 33 辑(广州。广东人民出版社,1981 年)
;陈
修良:
〈莫斯科中山大学里的斗争〉
,载《革命回忆录》增刊(1)
(北京:人民出版社,1981 年)
;江泽民:
〈回忆在莫
斯科中山大学时期〉
,载《革命史资料》
,第 17 辑(北京:中国文史出版社,1987 年)
。有关反映中国留苏学生在苏联
流放、劳改的资料有马员生的《旅苏纪事》
(北京:群众出版社,1987 年)
;唐有章的《革命与流放》长沙:湖南人民
出版社,1988 年)
;姚艮:
《一个朝圣者的囚徒经历》
(北京:群众出版社,1995 年)
。}三十年代初,斯大林开始大规模镇压在苏华人,莫斯科华侨商
人被大批逮捕、流放、处死,
「新经济政策」後一度兴起的华人商业繁荣的
局面顿时消失殆尽,
中国人在苏联的处境日益艰难。\footnote{莫斯科广播电台,1993 年 1 月 3 日 23:2O 华语广播。} 远东地区的镇压则更
为残酷,许多进入苏联境内的东北抗日游击队员被当作「日本间谍」投放
到劳改营。\footnote{参见姚艮:
《一个朝圣者的囚徒经历》
(北京:群众出版社,1995 年)
,页 315.} 1934 年後,苏联因基洛夫被刺,再掀清党运动,王明、康生
紧紧跟上,
在莫斯科的中共党员内也展开类似运动,
代表团成员杨之华
(瞿 秋白之妻) 、曾涌泉、孔原等均曾遭受打击\footnote{参见孔原:
〈怀念敬爱的稼祥同志〉
,载《回忆王稼祥》
(北京:人民出版社,1985 年)
,页 98;另见蹇先任:
〈38
年留苏纪事〉
,载《革命史资料》
,第 15 辑(北京:中国文史出版社,1986 年)
,页 139.}。

四、援救西路军。1937 年初,中共代表团争取到共产国际的大量军
火援助,以接济准备进入新疆的西路军。计有五万支步枪,上百挺轻重机
枪和几十门大炮。陈云、滕代远、冯铉、段子俊、李春田押送这批武器,
存至阿拉木图,等待西路军的讯息,後因西路军失败,此事告吹
\footnote{黄火青:
《一个平凡共产党员的经历》
(北京:人民出版社,1995 年)
,页 142-43.}。

中共代表团除了上述四个方面的工作以外,还有一项特别的工作,这
就是指导中共满洲省委。1932 年後,因日本侵占东三省,中共满洲省委已
无法与上海中央局正常联络,转而受中共驻共产国际代表团直接领导。王
明等曾多次发出给满洲省委的指示信,并派人潜入东北。满洲省委和抗联
也多次派人去苏联向代表团汇报工作。
中共代表团领导满洲省委是特殊形势下的一个例外——满洲省委与
莫斯科的联络远比与上海的联络更便捷。按照中共组织原则,中共驻莫斯
科代表团无权干预中共国内的事务,但是在事实上,以王明为团长的代表
团在很大的程度上具有对国内领导层的影响力。王明之所以具有对国内的
影响主要源之于他的共产国际背景和当时他在国内领导层中所拥有的政治
优势。

王明是深受斯大林信任、由共产国际一手扶植起来的中共领导人。
1931 年 1 月,
在共产国际代表、
原莫斯科中山大学校长米夫的强力安排下,
王明被中共六届四中全会指定为政治局委员,紧接着王明赴苏,担任中共
驻共产国际代表团团长,不久便被任命为共产国际执委会委员和政治书记
处候补书记,东方部副部长、部长,因而被公认是「国际路线」的代表。
一方面王明是中共代表,另一方面又是共产国际的化身,王明所拥有的这
种双重身份使他可以随时向共产国际的下属支部——中共,表述其个人的
意见。

王明在国内有一批盟友,
在六届四中全会上,
博古、
王稼祥等皆以
「反
立三路线」的英雄,与王明一道进入中共领导层,稍後返国的张闻天在政
治上也属于王明、博古集团。在王明赴苏後,博古等人都成为中共的主要
领导人,他们与王明的关系是一种「声气相求」、「共存共荣」的政治盟友
关系。由莫斯科一手操办的中共六届四中全会得到莫斯科的高度评价,将
其称之为「中共布尔什维克化的开始」,成为王明、博古等领导中共的全部
法理依据。王明作为六届四中全会上台的一批人的精神领袖,对在国内的
博古等人无可置疑地具有影响力。

王明在莫斯科是否对国内的中共中央实行遥控?从现在已披露的历
史资料看,王明和代表团一般不对国内的具体活动进行直接干预,但是在
某些时候,王明也曾就重大政策问题向江西表达自己的意见。在 1931 至
1935 年,王明与国内的中共中央既有一致的方面,也有分歧的方面。导致
王明与国内产生分歧的原因乃是王明捕捉到共产国际即将开始新的策略方
针从而提出了新的政策主张,但是,国内的博古等人仍坚持旧有路线,拒
不同意调整方针。

王明在莫斯科对他在国内的盟友一直持坚定的支持态度,
对毛泽东则
在一个相当长的时期内多有贬低和冷淡。

王明在共产国际的讲坛上高度称赞六届四中全会後的中共中央
「对共
产国际总路线百分之百的忠实」 。1932 年 3 月 31 日,王明在共产国际执委
会第十二次全会上发言,他宣称:
\begin{quote}
	\fzwkai 中共代表团在共产国际执委这个全会上,完全有权利高兴地向
一切兄弟党说:我们党在其布尔塞维克中央领导之下,现在达到了
其十三年存在的历史当中空前未有过的统一,团结和一致\footnote{参见王明:
〈中国目前的政治形势与中国当前的主要任务〉
,载《王明言论选辑》
。页 312.}。
\end{quote}

与王明对中共中央全力支持形成鲜明对照的是他对毛泽东的排斥态
度。据王明在中山大学的同学陈修良等人回忆,早在 1928 年,王明等就
在中大散布过「山沟沟里出不来马列主义」 \footnote{陈修良:
〈莫斯科中山大学里的斗争〉
,载《革命回忆录》增刊(1)
,页 56.} 。

1932 年 10 月,苏区中央局主持召开「宁都会议」,在这次会议上,
毛泽东的军权被剥夺,毛的军事主张也被指责为「右倾」和「保守主义」 。
会议後,在上海的博古、张闻天迅速向共产国际作了汇报。王明很快在共
产国际的讲台上,对博古等作出呼应。他用几乎与博古、张闻天完全一样
的语言,指责「党内一部分分子,对于国民党军事围剿和日渐逼近的帝国
主义对苏维埃革命的公开武装干涉表示悲观、失望和消极的情绪和观点」,
表示坚决支持「党在中央的领导之下」,对「目前阶段的主要危险——右倾
倾向」进行斗争\footnote{王明:
〈革命,战争和武装干涉与中国共产党的任务〉
(1933 年 12 月)
,载《王明言论选辑》
,页 361、364.}。

王明与博古的完全一致在 1932 年逐渐发生变化,王明开始在一系列
重大问题上和国内的中共中央产生了分歧。王明在征得共产国际东方部部
长米夫的同意下,以共产国际的名义致电中共中央,提议调整中共的土地
政策、对富农的政策和工商业、劳动政策,但遭到博古等的拒绝。

1932 年 3 月,王明发表文章,第一次公开批评中共的土地政策,指
责国内苏区「时常不断地(甚至一年三、四次)重新平分那些已经由基本
农民群众分配过的土地」是一种「表面好象『左』的。实际上非常有害的
倾向」 。王明还指出「对中农的关系不正确」,是中央「最重要的」
「弱点和错误」 \footnote{王明:
〈革命,战争和武装干涉与中国共产党的任务〉
(1933 年 12 月)
,载《王明言论选辑》
,页 361、364.} 。

1933 年 1 月,王明进一步批评国内苏区对富农采取的全面没收的政
策。王明指出,采取这种「左」的立场是混淆了革命的阶段,
「认为在苏区
由资产阶级民主革命已经转变到社会主义革命了」。王明还尖锐批评中央苏
区禁止自由贸易,
严重损害了苏区的经济,
明确要求纠正上述政策方面
「左」
的观点,制定灵活的、能够反映各地区差别的经济政策\footnote{王明:
〈中国苏维埃区域的经济政策〉
,转引自周国全、郭德宏等:
《王明评传》
(合肥:安徽人民出版社,1989
年)
,页 222-23.}。

博古对远在莫斯科的王明的上述意见完全置之不理,
与一般人所想象
的情况绝然不同,博古并非在所有问题上都对王明亦步亦趋。此时的博古
正青春年少,位居中央苏区第一号人物的地位,在日益严峻的形势下,博
古更加坚持原有的僵硬政策。本来博古的立场在苏区内部就已受到张闻天
的质疑,\footnote{程中原:
《张闻天传》
,页 168-70.}现在连王明也提出批评,但是,博古对所有这类批评都采取了坚
决「挡回去」的态度。博古的僵硬立场引致王明的强烈不满,正是在这个
时刻,王明对毛泽东的态度也从冷淡转向热烈。

王明对毛泽东态度的转变,大致以 1934 年为界。在这前几年,王明
完全支持博古等对毛泽东的「批评」、「帮助」。
王明自认为在党内的基础巩
固,对毛泽东的重要性尚缺乏全面认识,也还没有将毛视为是自己政治上
的对手。在这个阶段,王明对毛泽东轻视、忽略有之,但认为王明出于防
范毛泽东的个人动机,在莫斯科处心积虑贬损毛,则未免言过其实,也缺
乏事实依据。1934 年後,随着王明对博古不满的加深,王明对毛泽东的态
度发生了变化,王明开始在共产国际的讲台上宣传毛对中共的贡献。1933
年下半年至 1934 年春,王明曾多次试图返回国内进入中央苏区,但最终
因知晓王明返国计划的上海地下电台台长被国民党逮捕,王明返国计划被
迫取消。\footnote{据 1934 至 19SS 年担任中共上海中央局书记的盛岳(盛忠亮)回忆,1933 年下半年至 1934 年春,共产国际多次
来电,要求为王明进入中央苏区加紧准备香港——汕头——闽西秘密信道,上海中央局为此曾两次派人前往香港进行
布置,但最终因上海地下电台台长被国民党逮捕,王明返国计划被迫取消。参见盛岳:
《莫斯科中山大学与中国革命》
(北京:现代史料编刊杜,1980 年)
,页 269.}为了修补因长期脱离国内艰苦斗争而对自己政治上所造成的损
害,在毛泽东已受到党内批评、权力被削弱的情况下,王明向毛泽东援之
以手,不仅可以进一步扩大自己在党内核心层中的影响,更可使自己在党
内矛盾中处在仲裁者的有利地位。正是基于这些原因,从 1934 年 4 月起,
王明在莫斯科陆续做出一些姿态,试图建立起和毛泽东较为亲善的关系。

1934 年 4 月 20 日,王明、康生致函中央政治局,批评中共中央在一
系列重大问题,诸如在苏区发动针对毛泽东的「反罗明路线」斗争中所存
在的「不可忽视的严重弱点」,信中指出:
\begin{quote}
	\fzwkai (中央政治局)A、对于缺点和错误的过份和夸大的批评,时常
将个别的错误和弱点都解释成为路线的错误,......没有一个白区主
要的省委或直接在中央领导之下的群众团体的党团,
不被指出过
(甚
至不只一次的)犯了严重的或不可容许的机会主义、官僚主义的、
两面派的错误,......决没有领导机关的路线正确,而一切被领导的
机关的路线都不正确的道理,此种过份和夸大的批评,既不合适实
际,结果自不免发生不好的影响,......

B、对于党内斗争的方法有时
不策略,比如在中央苏区反对罗明路线时,有个别同志在文章中,
客观上将各种的错误,都说成罗明路线的错误,甚至于把那种在政
治上和个人关系上与罗明路线都不必要联在一起的错误,都解释成
罗明路线者。这样在客观上不是使罗明孤立,而恰恰增加了斗争中
可以避免的纠纷和困难\footnote{王明、康生致中共中央政治局信(1934 年 4 月 20 日)
,转引自周国全、郭德宏等:
《王明评传》
,页 226-17.}。
\end{quote}

王明、康生虽然没有完全否定「反罗明路线」的斗争,但是这封信还
是使已进行一年的「反罗明路线」的斗争停了下来。

1934 年 8 月 3 日,王明、康生又就当年 1 月 18 日中共六届五中全会
通过的《政治决议案》给中央政治局写了一封长信,在这封信中,王明、
康生指出中央政治局在对第五次反围剿政治意义的评估、扩大百万红军、
以及有关对「一省数省首先胜利」的解释等三个重要问题上,都存有「问
题」,「很容易引起不正确的结论」 \footnote{〈王明、康生致中共中央政治局信〉
(1934 年 8 月 3 目)
,转引自周国全、郭德宏等:
《王明评传》
,页 255-57.}。具有讽刺意味的是,在延安整风运动 
中,王明对博古的这类批评,不仅未被承认,反而将其定为是王明的主张,
王明当年对政治局的批评意见,几乎被毛泽东全盘接受下来,只是已被当
作批判王明自己的武器。

1934 年 9 月 16 日,王明、康生再一次写信给中央政治局,这封信可
能是中共中央大转移前来自远方的最後一次信息。王明在传达了共产国际
关于准备召开七大以及对西北问题的指示後,专门谈及共产国际出版毛泽
东文集的事宜:
\begin{quote}
	\fzwkai 毛泽东同志的报告
(指毛在
「二苏」
大会上的报告——引者注),
中文的已经出版,绸制封面金字标题道林纸,非常美观,任何中国
的书局,没有这样美观的书,与这报告同时出版的是搜集了毛泽东
同志三篇文章(我们这里只有他三篇文章) ,出了一个小小的文集,
题名为《经济建设与查田运动》 ,有极大的作用\footnote{〈王明、康生致中共中央政治局的信〉
(1934 年 9 月 16 日)
,转引自向青:
《共产国际和中国革命关系史稿》
(北
京:北京大学出版社,1988 年)
,页 184.}。
\end{quote}

随後,在中共代表团的协助下,共产国际又出版了《中国苏维埃第二
次代表大会》一书,收有毛泽东在「二苏」大会上的报告等文件,并译成
俄、英、德、日等文字,在苏联和世界各国发行。这样,在整个三十年代,
中共领导人能够有资格在苏联出版文集的,除了王明,只有毛泽东。

在王明、康生 9 月 16 日来信後不久,1934 年 10 月 10 日,中共中央
和苏维埃政府及八万六千名红军开始撤离中央苏区。对于这一决定,王明
事先是知道的。1934 年 5 月,中央苏区军事战略重镇广昌被国民党军攻占
後,中央书记处在瑞金召开会议,决定将主力撤离江西,进行战略转移,
并将这一决定报请共产国际批准。\footnote{《周恩来年谱》
,页 262.} 在共产国际覆电批准转移计划後,
中央
书记处成立了以博古、周恩来、李德组成的「三人团」
,负责战略转移的全
部准备工作。从 1934 年 10 月上旬红军长征开始,瑞金与莫斯科的电讯联
络就已中断。直到 1934 年 11 月中旬,王明在莫斯科通过上海日文新闻联
合通讯社 11 月 14 日发布的消息,才知道红军开始长征。

王明在这种形势下,进一步加强了他对国内中共中央的批评。1934
年 11 月上旬,
王明向莫斯科外国工人出版局中国部全体工作人员作
(六次 战争与红军战略)的报告,11 月 14 日,又起草了致中共中央的信(这封
信因为红军已开始长征, 未能传送至中共中央)。 王明的报告及信件都提出
中共中央犯了忽略国内形势
「新特点」 的错误,
尤其在军事问题上存在
「许 多错误和弱点」。 王明还批评了中共中央处理福建事变的方针,
宣称由于没
能援助十九路军,最终导致闽变的失败,从而加剧了红军冲破蒋介石围剿
的严重困难。在中央红军撤出江西苏区的背景下,王明对中共中央的这些
批评,与已经形成的严重危机有密切关系,同时,也是他与博古等在一系
列问题上分歧的合乎逻辑的发展。

1935 年 1 月在长征途中召开的遵义会议,在一段时间内,王明并不 知晓(王明
是在 1935 年 8 月 20 日陈云一行抵达莫斯科後,才获知有关遵 义会议的详情
的), 王明尽了很大努力, 试图恢复与长征途中的中共中央的 电讯联系。1935
年初春,王明派李立三、段子俊和一个熟悉无线电通讯的 波兰人前往中亚的阿拉
木图,李立三专门派了两批人,携带无线电密码本 经新疆回国寻找红军, 但都未
获成功。\footnote{参见唐纯良: 《李立三传》 (哈尔滨:黑龙江人民出版社,
1984 年) ,页 115.}此时的王明并不知道博古已经下台、 毛泽东重新出山的消
息, 他仍然继续在一些重大场合中向毛泽东表示敬意。 1935 年 8 月 7 日, 王
明代表中共在共产国际七大上作关于中国革命形势与 党的任务的报告,他在报告
中例举了十三个中共领导人的名字,将他们称 之为「党内领袖和国家人材」。在
这份名单中,毛泽东位居第一,而博古仅 排在第十二位。\footnote{参见王明:
《论反帝统一战线问题》 (1935 年 8 月 7 日,载《王明言论选辑》 ,页 449.
1937 年王明返国後,对原 文作了修改,在被列为「党内领袖和国家人才」的中共
领导人中,删去了张国焘的名字,增补了董必武、徐特立两人, 博古由原排行第
十二位上升至第五位。参见《陈绍禹(王明)救国言论选集》 (汉口:中国出版
社,1938 年) 。} 

纵观王明在 1931-1935 年对毛泽东的认识及态度变化的过程,可以
发现,
王明对毛泽东看法的转变是与他和博古分歧的逐渐扩大互相联紧的。
1932 年後,王明受到共产国际调整政策的影响,其原有的极左思想发生明
显变化,而在国内的博古因消息闭塞,兼之头脑僵化,却继续恪守共产国
际旧时的政策。王明随着年龄和阅历的增长,对党内高层关系的复杂性也
有了进一步的认识,于是改善并加强与国内毛泽东的关系,就成了 1934
年後王明在莫斯科的主要活动之一。王明相信自己在党内所处的地位是不
可替代的,他已作好准备,和毛泽东等其他领导人携手合作。

\section{在「反蒋抗日」问题上毛泽东与莫斯科的分歧}

1935 年 3 月,中央政治局委员陈云奉命离开正在长征中的中央红军,
于 5 月辗转到达上海。在上海稍事休息後,陈云与已在上海的陈潭秋、杨
之华、何实山等会合,作为中共参加共产国际七大的代表,经由共产国际
远东局的秘密安排,在沪搭乘苏联货轮,前往海参崴,于共产国际七大闭
幕之日的 8 月 20 日到达莫斯科。在这之前,上海中央局派驻北方的代表
孔原和上海中央局重要干部,前中央提款委员刘作抚,化名陈刚也抵达莫
斯科。从陈云那里,王明第一次了解到有关长征和遵义会议的全部详情。
从此,在王明与毛泽东之间,开始了长达十年的错综复杂的关系。

在毛泽东与王明之间,围绕统一战线问题产生的意见分歧,始终占据
突出的位置,成为日後毛、王公开冲突的导火索。

1931 年「九一八」事变爆发後,中共中央奉行共产国际的关门主义
政策,号召「武装拥护苏联」,建立下层统一战线,
王明对此政策的推行负有
完全的责任。从 1931 年 11 月王明抵苏至 1932 年底,王明全力支持这项
政策,但是从 1933 年初开始,随着共产国际政策的调整,王明在统一战
线问题上的观点发生了明显的变化,
进而成为中共领导层中倡议转变政策、
提出抗日民族统一战线主张的第一人。

1932 年 8-9 月,王明出席了共产国际执委会第十二次全会。这次会
议鉴于欧、亚法西斯主义崛起的严重形势,开始修正过去的一些僵硬的观
点,认为存在着争取社会民主党下层群众、建立工人统一战线的可能性。
王明受到这次会议的启发,逐步酝酿在中国也调整政策。

1933 年 1 月 I7 日,王明以毛泽东、朱德的名义,起草了著名的〈中
华苏维埃临时中央政府工农红军革命军事委员会为反对日本帝国主义侵入
华北愿在三条件下与全国各军队共同抗日宣言〉, 明确宣布,
中共愿与国民
党外的一切拥护民族革命战争的政治党派进行合作,共同抗日。

1933 年 1 月 26 日,王明又以中共中央的名义,起草了给中共满洲各
级党部及全体党员的信,史称「一二六指示信」,第一次提出在东北建立广
泛的抗日民族统一战线的主张。同年春,王明还参与指导国内的中共组织
与冯玉祥的联络活动。

1933 年 10 月 27 日,王明、康生致信中共中央,提醒应关注「民族
革命战争的策略问题」, 并随信附上他们起草的
〈中国人民对日作战的具体
纲领〉 。这份呼吁「立即停止一切内战」的文件,经宋庆龄等 l779 人签名,
于 1934 年 4 月 20 日发表後,在国内外产生了巨大的影响。

到了 1934 年春共产国际预备召开七大期间,王明思想转变的步伐进
一步加快。该年春,共产国际加紧酝酿建立反法西斯统一战线,共产国际
的这一新动向,对王明产生了重要影响。1934 年 4 月 20 日、9 月 16 日、
11 月 24 日,王明、康生在致中共中央的信中都提出打破关门主义、改变
打击中间阶层的过左政策的主张。而到了 1934 年 11 月,王明在他的〈新
条件与新策略〉一文中,正式提出建立抗日民族统一战线的口号。1935 年
8 月,共产国际七大号召建立广泛的反法西斯统一战线。是年 10 月,王明
在与中共驻莫斯科代表团广泛协商讨论後,起草著名的《为抗日救国告同
胞书》即〈八一宣言〉,在巴黎的中共报纸《救国时报》发表,把统一战线
的范围扩大到除蒋介石以外的国内一切党派,包括国民党内的爱国分子。
而到了该年底,王明在《救国时报》撰文,宣传「联蒋抗日」
,将蒋介石也 纳人到统一战线的范围。

然而,根据现有的资料看,1935 年前的中共中央对于王明有关建立
抗日民族统一战线的意见,并没有给予充分的重视。由于国内反应冷淡,
王明还托从莫斯科返国的同志向国内领导机关传达口头信息。

1933 年秋,王明与即将返国的中国共青团驻青年共产国际代表黄药
眠谈话。王明说,中共应在战略上实行转变,逼迫蒋介石抗日。王明又说,
国民党虽是我们的敌人,但已不是最主要的敌人,由于日本已损害了国民
党的利益,损害了民族资产阶级、英美派利益,国民党中下层,甚至高级
军官都可能赞成统一战线。王明进而分析道,中共工作之所以不能打开局
面,就是因为党的政纲与最广大群众的利益不一致,而得不到群众的掩护
和支持\footnote{黄药眠:
《动荡:我所经历的半个世纪》
(上海:上海文艺出版社,1987 年)
,页 219-20、243;221.}。

王明的这番谈话给黄药眠留下深刻的印象,
因为此时的共产国际东方
部部长米夫的观点与王明并不一致。当黄药眠向米夫辞行时,米夫要他转
告国内「还是照旧的方针领导」。\footnote{黄药眠:
《动荡:我所经历的半个世纪》
(上海:上海文艺出版社,1987 年)
,页 219-20、243;221.} 黄药眠返回上海後,迅速把王明的意见
转告给当时的中共上海中央局负责人黄文杰,上海局又通过地下电台将此
意见向江西苏区作了传达,\footnote{黄药眠:
《动荡:我所经历的半个世纪》
,页 244. 黄文杰自 1933 年春中共上海中央局成立至 1934 年 4 月担任组
织部长,
1934 年 10 月至 1935 年 2 月 19 日,
黄文杰接替叛变国民党的原中央局书记盛忠亮担任中央局书记兼组织部长,
1935 年 2 月 19 日黄文杰被捕,1937 年「七七事变」後获释,参加中共长江局工作。}
但是王明的建议如同石沉大海, 没有得到博古
等中共领导人的任何响应。

由此可见,从 1933 年初开始,随着共产国际酝酿策略转变,王明的
思想发生了重大的变化,他联系中国正在发生剧烈变化的形势,为中共设
计了一条新的政治路线。这条路线有别于旧时以国共两党斗争为主题的路
线,其核心是,共产党在国内阶级关系发生新变化的形势下,应加紧建立
广泛的抗日民族统一战线;
中共应改变过去的关门主义和一系列过左政策,
投身到民族救亡运动中去,
并在这场运动中发展壮大自己。
王明的新思想,
一方面来自于共产国际,另一方面,也有他个人的思考,因而走得比共产
国际远一些,这也是国内的博古等拒不接受王明意见的一个重要原因。

当王明在莫斯科频频谈论统一战线问题时,
远在中央苏区的毛泽东正
处于没有发言权的地位,
故而未见毛泽东有任何谈论统一战线问题的论述。
红军长征抵达陕北後,面对红军严重被削弱及国内的新形势,毛泽东正急
谋中共的出路,恰在此时,张浩(林毓英)化装潜入陕北,带来共产国际
七大和王明的新精神,毛的思考与王明的思路一拍即合,故而有旨在推动
建立抗日民族统一战线的瓦窑堡会议的召开。然而,毛、王虽一致同意建
立抗日民族统一战线,两人的侧重点却大相径庭,毛对抗日民族统一战线
的思考极具现实主义色彩,而王明则对之过于理想化。

毛泽东迅速接过了抗日民族统一战线的旗号,但是,他最关心的问题
是如何利用统一战线,首先缓解陕北的剿共危机,解决红军和共产党的生
存问题,继而谋求共产党和红军的更大发展。在瓦窑堡会议後,毛决定立
即成立旨在瓦解进攻陕北苏区的东北军、西北军的白军工作委员会,以谋
求「不战而屈人之兵」的战略成效。

王明在万里之遥的莫斯科,则远比毛泽东「高蹈」,他的眼光并不在
陕北,而是在南京。王明把统一战线的重点放在争取全国范围内实现国共
合作抗日,头号争取对象就是蒋介石。1935 年 8 月 20 日,陈云抵达莫斯
科後,王明才真正了解到红军的实力已大大受挫,紧接着,中共驻共产国
际代表团于 8 月 25 至 27 日连续召开会议,决定把反蒋抗日统一战线,改
为联蒋抗日统一战线。1935 年底,王明在巴黎的《救国时报》不断刊文,
呼吁国共合作抗日。1936 年 1 月 4 日至 9 日《救国时报》连载王明的文章
〈第三次国共合作有可能吗?〉,正式提出「逼蒋抗日」的主张。只是当传
来蒋介石在 1935 年 12 月镇压北京学生抗日游行的消息後,王明才被迫重
提「反蒋抗日」的口号。

毛泽东和王明在统一战线问题上的差异自 1936 年後逐渐显现出来。
毛泽东力主利用一切反蒋矛盾,改善中共的地位;王明则强调支持蒋介石
为全国抗战的领袖,坚决反对各地方派的反蒋活动。1936 年下半年,围绕
「两广事变」,毛泽东和季米特洛夫、王明的矛盾终于爆发。

1936 年 6 月 1 日,陈济棠、李宗仁以「反蒋抗日」为由,发动「两
广事变」, 中共闻之消息,
立即表示支持,
称其具有 「进步的与革命的性质」 。
\footnote{《中央关于两广出兵北上抗日给二四方面军的指示》
(1936 年 5 月 18 日)
,载中央档案馆编:内部本《中共中央
文件选集》
(1936 一 1938)
,第 10 册,页 25.}
6 月 13 日中共中央发出〈关于当前政治形势〉的决议,提出以中共为中
心,与西南建立抗日联军的主张,并且强调「在目前形势下,抗日战争与
反蒋战争是分不开的」 。\footnote{中共中央于 1936 年 6 月 13 日发出的党内文件〈关于当前政治形势〉的中文原件迄今仍未公布。此处引文转引
自苏联科学院远东研究所 K·库库什金的《共产国际和中国共产党的抗日民族统一战线策略》
,载徐正明、许俊基等译:
《共产国际与中国革命——苏联学者论文选译》
(成都:四川人民出版社,1987 年)
,页 332. 此段引文的真实性可以
从 1936 年 6 月 12 日中共中央以毛泽东、朱德名义发表的〈中华苏维埃人民共和国中央政府中国人民红军革命军事委
员会为两广出师北上抗日宣言〉
中得到确定,
该宣言激烈抨击蒋介石
「处处替日本帝国主义为虎作伥」, 表示中共愿
「同 两广当局缔结抗日联盟」。
只是这份对外发表的公告没有像党内文件那样,直接表明抗日应以中共为「中心」。
参见中央档案馆编:内部本《中共中央文件选集》
(1936-1938) ,第 10 册,页 30、31.  }与此同时, 
毛泽东积极推动与包围陕北的东北军、
西北军的谈判,6 月,中共方面已与张学良、杨虎城部签订了停战秘密协
定。然而,中共联络西南的活动并不顺利,西南方面拒绝了中共的建议。7
月,两广方面与蒋介石妥协,事变得到平息。尽管如此,中共方面仍取得
重大收获,毛泽东不费一兵一卒,解决了陕北的生存危机,抗日民族统一
战线的策略,帮助毛泽东做成了这笔「无本生意」。

但是,毛泽东的上述活动却遭到共产国际总书记季米特洛夫的指责。 两广事变爆
发後,苏联政府机关报《消息报》发表社论,谴责事变是「日 本人试图煽起中国
内战,以便利于掩盖对华北新的进攻」的一场阴谋。 \footnote{ A·康托洛维奇:
《是烟雾还是挑衅》 ,载苏联《消息报》 ,1936 年 6 月 10 日,转引自向青:
《共产国际与中国革 命关系论文集》 (上海:上海人民出版社,1985 年) ,页
191.}7 月 23 日, 季米特洛夫在共产国际执委会书记处讨论中国问题会议上发表
批 评中共的讲话,他说, 「不能说,在政治方面,在我们在中国所遇到的这种
复杂的情势下,他们完全成熟了(指中共领导人——引者注)和做好了准 备」 。季
氏强调将抗日与反蒋并举是「错误的」,中共和西南方面联合反蒋 也是「错误的」
。他要求中共采取「逼蒋抗日」的方针,并给国民党发出公 开信,表示自己愿与
国民党合作抗日的立场。季氏重申,中国现阶段一切 均应服从反日斗争,他并建
议中共以「民主共和国」的口号代替「苏维埃 人民共和国」的口号\footnote{参
见 A·季托夫:〈中国共产党在抗日民族统一战线问题上的两条路线斗争〉 ;K·库
库其金: 《共产国际和中国共产党的抗日民族统一战线策略》,载《共产国际与
中国革命——苏联学者论文选译》 ,页 370-72、334-35. }。

季米特洛夫对中共的指责得到了王明的支持。 王明在莫斯科撰文,婉 转含蓄地批
评国内同志恪守过时的反蒋抗日的政策, 王明问道: 「为什么中 共不可以与蒋
介石建立统一战线?」\footnote{王明: 〈为独立自由幸福的中国而奋斗〉 (为
中共成立十五周年纪念和中共新政策实行一周年而作) (又题为〈新 中国论〉,
见《共产国际》 ) (中文版) ,第 7 卷,第 4、5 期合刊,1936 年 8 月。} 

这场围绕两广事变而引发的「反蒋抗日」问题的争论,以毛泽东完全
接受季米特洛夫、王明的意见而告结束。1936 年 8 月 25 日,中共中央发
出致国民党的公开信,倡议建立国共统一战线。9 月 1 日,中共中央发出
党内通知,决定采用「逼蒋抗日」的方针。9 月 17 日,中共中央政治局决
定,以「民主共和国」代替「苏维埃共和国」的口号。

这是毛泽东主政中共後第一次与莫斯科打交道, 它给毛留下了很深的 印象。从此,
毛泽东有了自己对付莫斯科的一整套方法,这就是对「远方」 的指示,适合口味
的就办,不适合口味的就拖延不办;如果「远方」的压 力太大,则采取偷梁换柱
的方法,对其做过加工後再执行。总之,务求莫 斯科的指示与中共的发展不致有
太大的冲突,更不能与加强自己在中共党 内的地位相冲突。毛泽东最後接受「逼
蒋抗日」的方针及和平处理西安事变,\footnote{ 1936 年 8 月 15 日,共产国
际执委会致电中共中央书记处,指示中共「必须保持同张学良的接触」,但明确
表示 反对中共关于吸收张学良入党的打算。 〈共产国际执委会书记处致中共中央
书记处电〉 参见 (1936 年 8 月 15 日), 载《中共党史研究》 ,1988 年
第 2 期。西安事变发生後,共产国际于 1936 年 12 月 16 日致电中共中央,命
令中共必须「坚决主张和平解决这一冲突」,并提出「张学良的发动,无论其意
图如何,客观上只会有害于中国人民的各种力量结成抗日统一战线,只会助长日
本对中国的侵略」 。1937 年 1 月 19 日,共产国际再次致电中共中央,批评中
共以前对蒋介石采取的错误方针,敦促中共必须「彻底摆脱这种错误方针」,并
且认为中共直至 1937 年 1 月还在「执行分裂国民党而不是同它合作的方针」。
参见〈共产国际执委会书记处致中国共产党中央委员会电〉 ,载《中共党史研
究》 ,1988 年第 3 期。} 就是依据了这种策略。结果,莫斯科虽对毛泽东阳奉
阴违不满,但 都因毛泽东最後还是贯彻了「远方」的意图而原谅了毛。

毛泽东因有求于莫斯科而对季米特洛夫无可奈何,但是,对王明则是
另一回事了。1936 年之前,毛泽东对王明虽无好感,但两人并没有直接打
过交道,与毛交恶的是博古、张闻天等人。现在王明跟着季米特洛夫的後
面鹦鹉学舌,指责国内同志,这就与毛泽东发生了对抗。只是 1936 年毛
泽东的领袖地位还未完全确立,王明在国际国内均享有很高的威望,毛泽
东还无力与王明正面冲突,但是,王明的举措已引起毛泽东的高度警惕。
为了防范王明影响的扩大,从而危及自己的地位,毛开始在核心层散布对
王明的不满\footnote{Edgar Snow:Rcd Star
Over China(London:RandomHouse;1979), P505.}, 公开向党内的同志表明自己的心迹,
毛已预感到他在党内的
真正对手是王明。
