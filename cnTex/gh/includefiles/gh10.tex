\chapter{整风在深入:宣传和干部教育系统的重建}
\section{重建「党的喉舌」:延安《解放日报》的整风}

在中共的政治组织构成中,党的意识形态宣传部门一直占据极其
重要的位置。意识形态宣传对于中共之重要,不仅在于它可为党的政治、
军事等一切实践提供全套的解释,使中共全部活动奠定在学理和道德基础
之上;而且还可以被党的领袖运用作为对付党内政敌的有力工具。正因如
此,三十年代末之後,毛泽东一直在为控制中共意识形态部门而奋斗,由
于在当时及以後相当长的时期内,报刊在中共意识形态宣传系统中占据着
特殊的地位,毛泽东对中共报刊寄予了最大的关注。1941-1942 年,毛泽
东依靠坚强的组织机构:中央高级学习组——中央总学委全面占领中共舆
论阵地,最终将其置放于自己的绝对控制之下,——中共中央机关报《解
放日报》的改版就是毛泽东这一战略行动的重要组成部分。

毛泽东改组
《解放日报》
是为推动全面整风而精心策划的一个攻坚战,
此举标志着从 1938 年中共六届六中全会後开始的毛个人控制舆论工具过
程的最後完成。六届六中全会後,虽然毛泽东已在制定中共文宣政策方面
起着决定性的作用,但他对文宣部门的控制尚未达到绝对化的程度,中共
文宣部门尚留有一些国际派人物在主持工作,对毛仍存有一定的制约,因
此毛对中共文宣系统还是左右看不顺眼。

首先,毛泽东对由博古担任主任的中央党报委员会根本不信任。中央党报委员
会是一个在中共历史上存在很长时间的组织,它的主要
职责是代表中央政治局领导、监督中共所有报刊的言论活动。自三十年代
初期以来,中央党报委员会主任一职长期由张闻天担任,1938 年後博古虽
继任此职,但由于博古留在重庆南方局工作,中央党报委员会实际上仍由
张闻天掌握,直到 1940 年 11 月博古返回延安,中央党报委员会才由博古
真正负责。中共六届六中全会以後,延安的报刊开始把毛的言论及活动置
于突出的地位,但对毛泽东也就做到这一步为止。在张闻天、博古的安排
下,毛泽东只是比较突出的一名政治局委员而已,为了体现集体领导的原
则,延安的各种报刊仍然大量刊登张闻天、王明、凯丰等人的理论文章。
这种舆论导向使广大党员无从减退对王明等人的崇仰,也无从增添对毛泽
东的认识。对于这种局面,毛泽东虽然气愤,但一时也无可奈何,毕竟毛
当时还不便主动提示别人来歌颂自己,然而毛泽东执意搬去张闻天、博古
这两块石头的决心却已下定了。

毛泽东的另一个不满对象是由王明、周恩来直接领导的《新华日报》。《新华日
报》作为中共机关报,于 1938 年 1 月 11 日创刊于武汉,以後随 中共代表团迁
至重庆,是唯一不受毛直接控制的中共重要报刊,在一个相 当长的时期内,实际
上起着中共中央机关报的作用。对于《新华日报》的 中央机关报性质, 1938 年
的中共党内是无人会公开提出疑问的, 在 事实上 延安也予以默认。1938 年 4
月 2 日,中共长江局以中共中央的名义向各地 方党委发出指示,要求各级党委、
各地方支部订阅《新华日报》,并且在党 的会议上讨论《新华日报》上发表的社
论和中央负责同志的文章。\footnote{〈中共中央关于党报问题给地方党的指示〉
(1938 年 4 月 2 日),载《群众》,第 1 卷,第 22 期。}在技术 方面,
《新华日报》也无可争辩地处在所有中共报刊的首位。《新华日报》的采编人员
大多为著名的中共文化人。王明、周恩来甚至还邀请了中间派 人士陆诒参加《新
华日报》工作。和大型日报《新华日报》相比,延安的 周二报《新中华报》只是
一张小报,很不符合中央机关报的身份。1938 年 7 月初, 王明指示 《新华日
报》暂不刊登毛泽东的 《论持久战》, 引起毛泽东的极大愤怒,尽管不久王明、
周恩来即指令长江局以「新群丛 书」第十五种的形式另出了《论持久战》单行本,
但这并没有消弥毛对王 明、周恩来等的怨愤。1939 年 5 月 17 日,毛泽东抓住
周恩来同意《新华 日报》暂时停刊、参加国民党提出的《联合版》一事,严厉指
责周恩来: 「你们未征求中央书记处意见,即同意停版,实属政治上一大疏忽。」
\footnote{1939 年 5 月 3 日至 4 日,重庆遭日本飞机轮番轰炸,十多家报馆均
遭破坏,国民党当局以疏散为借口,下令重 庆各报停刊,共出一张《联合版》。
为了维持统一战线的大局,周恩来说服《新华日报》社内部持反对意见的同事,接
受了重庆当局出版《联合版》的指令,但周恩来向国民党中宣部部长叶楚伧申明,
一俟将有定所,即恢复出刊《新华 日报》,然而此事却受到毛泽东的严厉批评。
参见《中国共产党新闻工作文件汇编》上卷,页 89;另参见韩辛茹: 《新 华日
报史 1938 一 1947》,上(北京:中国展望出版社,1987 年),页 92-94. }
在 毛泽东眼中, 《新华日报》不仅成了王明、周恩来用之于和延安分庭抗礼的
工具,更成了「第二政治局」指导全党,对外代表中共的舆论喉舌,实属 不能容
忍。\footnote{1938 年 12 月 12 日,王明在由重庆返回延安途中抵达西安,12
月 19 日给《新华日报》主编潘梓年以及吴克坚、 华岗写信,表达他对报社工作
的关心,而代表南方局领导《新华日报》的凯丰也曾于 1938 年 12 月 15 日、
1939 年 2 月 14 日给王明写信,向他汇报《新华日报》情况。1939 年 9 月至
10 月,王明赴渝参加国民参政会期间,多次在《新华 日报》社作报告,9 月 29
日,王明在重庆南方局,作〈目前国内外形势与党的任务〉的报告,在发表时改为
〈目前国 内外形势与参政会第四次大会的成绩〉,并注为「9 月 20 日在《新华
日报》工作人员会上的报告」,发表于《解放》周 刊总 89 期(1939 年 11 月
7 日),其中把「党的任务」部分全部删除(王明作报告的时间是《解放》周刊
有意变动的, 其目的是为了迷惑国民党)。以上情况一方面说明 1938 年後王明
对南方局和《新华日报》仍有一定的影响,另一方面 也表明延安并不乐意看到王
明就全党范围的问题发表看法。} 尽管毛泽东对《新华日报》强烈不满,但是对
于国统区这唯一份中共公开发行的报纸,毛当时尚无法直接支配。同时,毛也
相信, 如果能促使周恩来改变观念,并对《新华日报》施加毛个人的影响,使之
在政治和思想上改弦易辙, 《新华日报》自有其继续存在下去的必要,只是 《新
华日报》作为中共唯一机关报的地位必须改变。

1941 年春,毛泽东整肃中共新闻机构的措施相继出台。第一步、毛泽 东以「技术
条件的限制」为由,提议暂时裁并延安的大部分党刊。财政困 难固然是事实,但
毛的着眼点却并不在此,因为遭受裁撤的刊物在经济形 势好转以後并没有恢复。
裁并结果是,由张闻天主编的《解放》周刊、 《共 产党人》等一批报刊纷纷关门
大吉。\footnote{1941 年 3 月 26 日,中共中央作出〈关于调整刊物问题的决定〉
。决定《中国妇女》、《中国青年》、《中国工人》自 1941 年 4 月起暂时停刊,
四个月後恢复。1941 年 6 月 15 日,复宣布中共中央政治理论刊物——《解放》周
刊停刊, 1941 年 8 月, 《共产党人》停刊,以上刊物以後均未复刊。}中央党
报委员会名存实亡,几乎不再 有党刊党报需要管理。第二步、
在 1941 年 5 月 15 日, 宣布将《新中华报》和新华社内部刊物《今日新闻》合
并,于次日正式创 办大型中共中央机关报——《解放日报》,毛并通知全党,今後
中共「一切 党的政策,将经过《解放日报》与新华社向全国宣达」。
\footnote{毛泽东:
〈关于出版《解放日报》和改进新华社工作的通知〉1941 年 5 月 15 日,
载《毛泽东新闻文选》,页 54.} 


创办《解放日报》是毛泽东在统一中共全党宣传舆论机构方面获得的
一项重大进展。毛对《解放日报》倾注了强烈的关心,他亲自为该报撰写
了〈发刊词〉
,然而时隔不久,
《解放日报》的表现就令毛大失所望。毛发
现该报在言论方面与已经停刊的延安党刊并无任何实质区别。毛泽东选择
博古担任《解放日报》社长兼新华社负责人,虽属是对博古的降级使用,
但也是对曾担任中央党报委员会主任的博古的一项安抚。毛相信,在他本
人的鼻子底下,博古绝不敢违抗自己的旨意,
《解放日报》将忠实贯彻自己
的意志。可是情况却非毛泽东设想的那般顺利,博古与该报主编、另一国
际派人物杨松配合默契,竟将《解放日报》办成了苏联《真理报》的中国
版!

《解放日报》创刊後不久,苏德战争爆发,此事自是延安共产党员关
心的头等大事,于是博古、杨松等在报纸上以最显著的地位,突出报道苏
德战争和苏联红军抵抗人侵德军的战况,有关国际问题的新闻、社论和文
章在报纸版面上占了压倒优势;而对中共及其军队及陕甘宁边区的报道,
一时则数量相对较少,位置也不显著。毛泽东对《解放日报》直接使用外
国通讯社的电讯稿极为恼火,认为这种「有闻必录,不加改写」
,将外电直
接桶上报纸的方法,使读者看不出党对某一国际国内重大事件的立场和倾
向性,是在为别人做「义务宣传员」。\footnote{参见〈
《解放日报史》大纲〉
,载中国社会科学院新闻研究所编:
《新闻研究资料》
,第 17 期(北京:中国社会科
学出版社,1983 年)
,页 12. 另参见《胡乔木回忆毛泽东》
,页 449.} 

《解放日报》对有关毛泽东个人活动及整风运动的宣传,也令毛泽东
强烈不满。
《解放日报》创刊後,毛几乎不加任何掩饰地要求杨松在报上突
出宣传自己,他开导杨松,
「讲中国历史,要多讲现代,少讲古代,特别是
遵义会议以後,党如何挽救危局,要多加宣传,让大家知道正确路线是怎
样把革命引向胜利的」。\footnote{杜青(杨松遗孀)〈回忆杨松同志〉
:
,载《中共党史人物传》
,第 25 卷(西安:陕西人民出版社,1985 年)
,页
192.}但博古、杨松领导的《解放日报》却对毛的这番
话置若罔闻,未将有关他的活动置于特别重要的地位。
《解放日报》只是在
第三版右下角以三栏题报道了毛泽东于 2 月 1 日在中央党校作整顿三风报
告的消息,遂被毛看成是对他个人权威严重的藐视。于是,毛就认为博古
领导下的《解放日报》对整风运动的宣传,既无广度,更无深度。

对《解放日报》经常刊载张闻天、吴亮平等人所撰写的谈论马列和宣
传苏联的「又臭又长」的文章,毛泽东更是感到难以容忍。1941 年 5 月 16
日《解放日报》创刊号上就刊登了戈宝权译的苏联作家爱伦堡的〈另一个
法国〉。张闻天等人除了自己在报上发表文章,
还指使马列学院的干部从苏
联一些理论刊物翻译大块文章,然後移植于《解放日报》,
致使本来版面就
紧张的《解放日报》充斥大量「洋八股」
,显得更加「贫乏无味」,「面目可憎」
(1941 年 5 月 16 日至 9 月 15 日,
《解放日报》只出版两个版面)。

凡此种种,都成了毛泽东决心整肃《解放日报》的理由,毛多次尖锐批评《解放
日报》,强调:我们在中国办报,在根据地办报,应该以宣传我党的政策,八路
军、新四军和边区、根据地为主。\footnote{参见〈 《解放日报史》大纲〉,载
中国社会科学院新闻研究所编: 《新闻研究资料》,第 17 期,页 13.}在毛泽
东的提议下,中 共中央决定《解放日报》进行改版,并把此事作为延安整风运动
的一部分。1942 年 2 月,毛泽东整肃《解放日报》的行动正式开始,他亲自调
派 陆定一进入《解放日报》社,在暂时维持报社领导班子的情况下,陆定一成了
凌驾于博古之上、事实上的《解放日报》最高负责人。1942 年 3 月 16 日,中宣
部发出〈为改进党报的通知〉,要求中共各地组织「根据毛泽东同志整顿三风的
号召,来检查和改造报纸」。1942 年 3 月 31 日,毛泽东又亲自主持《解放日
报》改版座谈会,在会上鼓励与会的七十多人,对《解放日报》的缺点展开批评。
4 月 1 日, 《解放日报》发表(致读者〉,宣布「从今天起,报纸的版面加以
彻底的改革」,「要使《解放日报》能够成为真正战斗的党的机关报」。至此,
《解放日报》完全被纳人毛泽东的直接控制之下。

凭心而论,毛泽东对《解放日报》的批评,有一些是符合事实的,但
由于毛批评的出发点是为着巩固个人对报纸的控制,因此毛的批评在许多
方面又是夸大其辞,攻其一点,不及其馀。

即以外国通讯社做「义务通讯员」一事而论,
毛的批评也是有欠公允的。
《解放日报》
在第二次世界大战战火弥漫全球之
际,突出报道苏德战争和美英盟军反法西斯战场的战况,不仅完全符合读
者的阅读需要,而且是抗战期间持爱国立场的任何一家严肃报纸都必须履
行的职责。

对毛泽东个人活动的报道保持一定的适度,这与当时毛在党内的地位
有关。从中共组织原则上讲,毛仍是政治局和书记处的一名成员,并不是
名正言顺的「总书记」,「多宣传集体,少宣传个人」的集体领导原则,至
少是毛泽东表面上也赞同的。

《解放日报》大量刊载「洋八股」也是事出有因,概因毛泽东本人在
1938 年中共六届六中全会上号召全党开展学习马列的运动,
才引燃了这场
火。至于国际派借机「掉书袋」
,那是另外一回事,因为毛泽东的鼓动客观
上为他们的卖弄提供了机会。张闻天等人只是在奉命办事的过程中顺带一
些私货借以自炫罢了。

其实,在延安日益低迷的政治高压下,博古、杨松为办好《解放日报》
一直小心翼翼,战战兢兢。尤其是杨松,几乎达到披肝沥胆、呕心沥血的
地步,生怕工作中稍有失误,引致毛的不满,使自己本来就黯淡的政治前
途更加险恶。

杨松原名吴绍镒,曾用过吴平、瓦西里、戈里等化名,湖北黄安(现
大悟县)人,是一位与毛泽东毫无历史与工作渊源的老共产党员。杨松自
1927 年 1 月进入莫斯科中山大学学习後,
长期留驻中共驻共产国际代表团
工作,与王明有密切的关系。1931-1933 年,杨松被调入苏联远东海参崴
任太平洋职工会中国部主任,从事华工教育及搜集日本情报的特殊工作。
1933 年夏秋,
杨松又被调驻莫斯科赤色职工国际东方部。
1934 至 1935 年,
杨松奉王明命令,多次代表共产国际,冒着生命危险秘密潜人被日本占领
的东北,
向中共领导的东北抗联游击队传达指示,
协调东北抗联内部关系。
杨松还多次保护即将遭判刑和流放的被调入苏联接受审查的东北抗联干
部,使之免遭康生的荼毒,曾结怨于康生,受到党内警告处分。\footnote{李范五:
〈回忆杨松同志〉
,载《中共党史人物传》
,第 25 卷,页 187.}杨松并因
长期在极其艰苦的东北地下环境中从事秘密工作而患上严重的肺结核。
1938 年 2 月,杨松辗转来到延安,在张闻天领导下做理论宣传工作,曾任
中宣部秘书长兼宣传科长,
并在马列学院讲授
「中国现代革命运动史」。
《解
放日报》创刊後,博古、杨松有意将《解放日报》办成像《真理报》、《大
公报》、《新华日报》那样具有广泛影响的权威报纸,他学习《真理报》、《大
公报》重视社评、每日一篇社论见报的模式,在博古的要求下,也坚持每
天亲自撰写社论一篇,由于写作任务繁重,工作环境极差,杨松已痊愈的
肺结核再度复发,但杨松仍奋力带病工作。

博古、 杨松虽然对毛泽东发动整风的意图有所疑虑, 但在报社工作中, 却不敢
稍有怠慢。\footnote{丁玲回忆道,博古主持《解放日报》给她的印象是「极为
审慎」,博古曾告诫她,不能把《解放日报》文艺栏办 成「报屁股」、「甜点心」,
也不能搞成《轻骑队》。黎辛说,博古强调「自由主义不能在报纸上存在」,
「报纸不能闹独 立性,一个字也不能闹独立性」。参见丁玲: 〈延安文艺座谈
会的前前後後〉,载艾克恩编: 《延安文艺回忆录》,页 57; 黎辛: 〈丁玲
和延安《解放日报》文艺栏〉,载《新文学史料》,1994 年第 4 期,页 59.
}杨松更是极其谨慎,每天从社论到消息报道都逐字逐 句的仔细检查, 经常通宵
达旦地工作。\footnote{毛泽东于 1941 年 5 月作〈改造我们的学习〉演说後,
杨松就已预感到自己将面临被淘汰的命运,他对昔日的同 事张仲实说, 「我对于
外国的事情,还可谈几句。对于本国情形,的确一点都不熟悉。今後我要下定决心,
把自己改造 一下,不然对党实在没什么用处」。参见张仲实: 〈悼杨松同志〉,
载《解放日报》,1942 年 11 月 27 日。} 1941 年 9 月政治局会议开始後,
《解 放日报》紧紧跟上毛泽东的步伐,频频发表反教条主义、主观主义的社论 和
专论。9 月 2 日, 《解放日报》发表〈反对学习中的教条主义〉社论,9 月 16
日, 《解放日报》发表毛泽东政治秘书胡乔木的文章〈为什么要向主 观主义宣布
坚决无情的战争〉10 月 14 日又发表艾思奇的 , 〈主观主义来源〉一文,提
出主观主义具有书本教条主义与狭隘经验主义两种基本形态,将 毛发动整风意欲
整肃的两种对象正式揭示出来。

尽管《解放日报》对整风宣传尽心尽力,但博古、杨松再努力也是白 费劲。1942
年 2 月, 奉毛泽东命接管 《解放日报》的陆定一甫抵清凉山 (延 安《解放日
报》所在地),就拿杨松开刀祭旗。陆定一等批评《解放日报》每日撰写社论是
虚应故事,徒费劳力,杨放之等人甚至指斥杨松「粗制滥 造」\footnote{杨放之
又名吴敏,1937 年抗战爆发从国民党监狱释放後,即参加创办《新华日报》的工
作,是《新华日报》最 早的编委会成员之一。他完全清楚《新华日报》自创办至
1941 年初每天必发社论的传统,只是因 1941 年「皖南事变」爆发,国共关系严
重恶化, 《新华日报》才改而决定不再每天发表社论。但是杨放之在 1941 年调
入延安《解放日报》後,迅速站到了陆定一一边,成了反对《解放日报》每天发
表社论的主要人物。参见杨放之:《解放日报》改版与〈延 安整风〉,载《新闻
研究资料》,第 18 期,页 3. }。陆定一等的行动并非就事论事,而是以社论事
为突破口,谋求一举 改组《解放日报》。不言而喻,博古、杨松决非是有毛
作後盾的陆定一的对 手,陆定一的意见最後被采纳, 《解放日报》很快取消了一
日一篇社论见报 的惯例。1942 年 8 月 15 日,陆定一正式取代了杨松,担任了
《解放日报》主编。而杨松则在抑郁中于 1942 年 11 月 23 日病故。



\section{陆定一、胡乔木与毛氏「新闻学」原则的确立}
改版後的《解放日报》虽仍由博古任社长,但在很大程度上已属挂名,
尽管博古仍在报社继续负一些具体事情的领导责任,然而《解放日报》的
实权已在主编陆定一和中宣部代部长胡乔木的手中。

《解放日报》在陆定一和胡乔木的具体领导下,积极贯彻毛泽东的有
关指示,从报道内容和版面设计等一系列环节,对原《解放日报》进行了
大幅度的改造,逐渐确立了一些基本原则,从而建构了毛氏「新闻学」的
框架。

毛泽东「新闻学」最显著的特征是将政治功利性视为新闻学的本质,
而否认新闻具有超阶级性属性的观点。
毛泽东早年对新闻学就有强烈兴趣,
在北大图书馆工作期间,听过民国初年新闻大家邵飘萍的演讲,曾参加北
大学生社团「新闻学会」的活动,以後也曾创办《湘江评论》
,并一度担任
过大革命时期国民党中宣部刊物《政治周报》
的主笔。但是,身受五四「自
由办报」思潮之惠的毛泽东,并未接受自由主义新闻学的观点,尽管毛泽
东一贯喜好阅读政治倾向性较为中立的《大公报》、《申报》等报刊,然而
他始终将自己的阅读偏好与出于政治功利而对中共新闻宣传工作的要求划
分得一清二楚:毛要了解一切公开的和内幕的新闻,而中共一般干部和普
通百姓只需知道党要他们知道的那部分新闻。毛泽东的这种政治功利主义
的新闻观正是通过陆定一和胡乔木的解释,最早在延安《解放日报》体现
出来,它以五个核心原则为中心,包含了一系列互相联系的概念:

一、
「党性第一」的原则。

毛泽东、陆定一、胡乔木认为,任何报纸都是一定阶级的政治斗争的 工具,世界上
绝不存在超阶级的客观报道,中共创办的报纸无可争议的应 是反映党的政治路线
的党报。由于共产党代表了历史发展的方向和人民的 根本利益,党报不仅应是
「党的教科书」,而且也是「人民的教科书」。为 了不使人民失望——胡乔木说,
「人民的希望就是读教科书」,中共应把党 报办成像「章章都好」的《联共党史》
那样,使人民能「读一辈子」。\footnote{胡乔木: 〈报纸是人民的教科书〉,
载《解放日报》,1943 年 1 月 26 日。九十年代初,胡乔木将此篇文章改名
为 〈报纸是教科书〉,收入《胡乔木文集》,对为何改名,胡未作任何说明。
参见《胡乔木文集》,第 3 卷(北京:人民出版社,1994 年),页 303.  }为
此,党报「要在自己一切篇幅上,在每篇论文,每条通讯,每个消息......  中都
能贯彻党的观点,党的见解」。\footnote{〈致读者〉,载《解放日报》,
1942 年 4 月 1 日。}举凡一切评论、消息、照片都必须以 是否符合党的利益为
标准而加以取舍和编排,并以党的立场来判断一切。党报绝 不是「有闻必录」、
单纯报道消息的新闻纸,而是为了实现党的任务 而奋斗的宣 传工具,为了保证
党报的性质,必须将党报置于党的领导机关 的绝对领导之下。

二、反对「虚假真实性」的原则。陆走一等提出一个有名的的口号:「把尊重事
实与革命立场结合起来」, \footnote{陆定一: 〈我们对于新闻学的基本观点〉,
载《解放日报》,1943 年 9 月 1 日。}虽然从字面上,陆定一也强调新闻必
须完全真实,然而这个「事实」必须 置于「革命立场」的统帅之下。于是,陆定
一等从列宁那儿引进了「两种 真实性」的观点:一种是所谓「本质真实性」即代
表了历史发展方向的事 实,尽管它尚处于萌芽状态或尚未发生,但从本质上讲它
却是真实的;相 反, 「虚假真实性」只反映事物的「表像」和「假像」,而不
反映事物的本 质,因此它必定是不真实的。如果以为它是新近发生的事实, 「把
个别现象 夸大成为整体现象」而加以报道,那就必然会犯「客观主义」和「自由
主 义」的错误,而无产阶级的「真实性」和「客观主义」、「自由主义」是截 然
对立的\footnote{参见〈给党报的记者和通讯员〉,载《解放日报》,1942 年
11 月 17 日。}。

三、
「新闻的快慢必须以党的利益为准则」。

「抢新闻」是「资产阶级新闻学」的恶劣表现,正确的「无产阶级新
闻观」将发布新闻的快慢完全服从于党的需要,
「该快的快」,「该慢的慢」;
「有的压一下才发表,有的压下来不发表」
,总之,一切必须听命于党的领
导机关和最高领袖,绝不允许报纸和记者有丝毫的「独立性」和「自由主
义」。

四、运用报纸指导运动的原则。

党的领导机关必须善于「利用」报纸,\footnote{毛泽东:
〈在《解放日报》改版座谈会上的讲话〉
(1942 年 3 月 31 日)
,载《毛泽东新闻工作文选》
,页 90.}尤其要学会运用报纸指导政治
运动,在运动初起和达到高潮的一段期间内,集中报道,形成宣传规模,
用以教育干部和群众,震慑和打击敌人。

五、新闻保密和分层次阅读的原则。

抗战前,
《红色中华》报和以後改名的《新中华报》便开始抄收国民党
中央社的电讯。一部分在报纸上发表,一部分编印《参考消息》
,每天印五
十至六十份,供中共中央各部门负责人阅读。
《解放日报》改刊後,正式出
刊了供领导干部阅读的《参考消息》
,阅读范围较前有一些扩大。出版《参
考消息》的指导思想在于进一步明确新闻保密和分层次阅读的原则。因为
群众有左、中、右的划分,党员干部中也有左、中、右之区别。既然人之
有区别,在「知」的权利上就不能不反对「绝对平均主义」
。中低级党员干
部的政治觉悟和理论水平不足以抗御国内外新闻报刊散布的「毒素」的侵
袭,因此只有久经考验的少数高级干部才有资格被告知某些重要的新闻消
息,干部级别越高,阅读限制就越小,由此逐级而递减。至于一般普通老
百姓,为了保证他们思想上和政治上的纯洁性,就没有必要让他们知道党
报以外的其它消息了。当然,党员和群众还是有区别的,即便是普通党员,
组织上也会给其比普通百姓多一些的信息,这主要通过阅读党内读物,听
上级的传达报告来体现,以显示党员在「知」的方面所享有的特殊权利。
只是一般党员「知」的权利和高级干部相比,早已不能以道里计。

实际上, 毛泽东新闻学的基本观点与王明、 博古等并无实质性的分歧, 追根溯
源,毛泽东与王明、博古一样,师承的都是列宁——斯大林的新闻 理论,只是毛泽
东比王明、博古更加党化、更加斯大林主义化,甚至青出 于蓝而胜于蓝,超过了
斯大林。与毛相比,或许博古所受的五四影响稍大 一些。早在 1925-1926 年,
博古就曾在其家乡无锡和上海主编过一份影响颇广的政治刊物《无锡评论》
\footnote{参见〈秦邦宪与《无锡评论》,载《江苏出版史志》〉,1991 年第
3 期。}。1941 年 5 月,博古将边区最有名的女作 家丁玲调入《解放日报》,放
手让其主持文艺栏,正是由于博古的宽容,丁 玲才有可能在 1942 年的 《解放日
报》上推出王实味、 丁玲本人, 以及萧军、 罗烽、艾青等人撰写的一系列批评
性的杂文和短论,及至《解放日报》改 版,所有这类「暴露性」的言论被斩草除
根,彻底实现了毛泽东所要求的 「舆论一律」。毛泽东所达到的对新闻的垄断,
在某种程度上,甚至连斯大 林都难望其项背。在文网严密的苏联报刊,偶而还有
几篇批评官僚主义的 小品文问世,而在延安,1942 年後的报刊上已不复有任何
「暴露性」的文
字。在抢救运动期间,延安还揪出了一些「写不真实的新闻」的「特务分
子」\footnote{陆定一在〈我们对于新闻学的基本观点〉一文中称边区的特务分于常常写不真实的新闻,企图降低《解放日报》
的信用,已经被「查出来」了。参见《解放日报》1943 年 9 月 1 日。}。
 

改版後的《解放日报》在陆定一、胡乔木的领导下,面貌发生了重大 的变化,成了
一份地地道道、名副其实的「党报」。在版面安排上, 《解放 日报》彻底改变
了「一国际,二国内,三边区,四本市」的惯例,而变为 「一边区,二解放区,
三全国,四国际」的报道和版面安排的顺序,将国 际和国内新闻降至次要地位。
对刊登国际新闻更是从严掌握,所有国际新 闻,一律须重新编写,严禁照登外电
原文。

《解放日报》既为「党报」,
它就必然同时又是毛泽东的
个人喉舌。1942 年 4 月後,报社遵从毛的指示,多次发表经毛修改的讲话
和文稿,而发表这类讲话的时间一般都较毛作演讲的时间推迟很久。例如
毛在延安文艺座谈会上发表的演讲,其文字稿推迟约一年半才经修改整理
完毕,于 1943 年 10 月 19 日刊登在《解放日报》。

作为毛泽东严密控制的宣传工具,
《解放日报》
忠实地执行了毛利用该
报指导整风的意图。陆定一调入《解放日报》後,奉毛泽东之命,将工作
重点放在新辟专刊〈学习〉上,使〈学习〉很快成为指导整风的信风标。
〈学习〉专刊于 5 月 13 日出版,共办了八个月,出版了二十四期,对于如
何学习文件、如何开展小组讨论,怎样写反省笔记,都针对性地发表各类
文字予以指导。当整风进入到干部思想反省阶段後,
〈学习〉专刊还配合登
载了一批各类干部的自我反省文章作为示范。至 1943 年初,
延安整风转入
审干、肃反阶段,
〈学习〉专刊的使命遂最後完成,终于在 1943 年 1 月 16
日宣布终止。

《解放日报》还开创了中共利用报刊整肃「异端」知识分子的新模式。
1942 年 6 月报纸用两版篇幅集中登载批判王实味的文章,
范文澜、
张如心、
罗迈(李维汉)、
温济泽、李伯钊、陈道、蔡天心等纷纷「口诛笔伐」
,陈
伯达更在大批判文章中将王实味称之为
「王屎味」。但报纸绝不为王实味提
供为自己辩护的版面,使《解放日报》成为一边倒围剿王实味的主要战场
之一。1942 年 6 月 20 日,
《解放日报》发表冠之以「延安文艺界」名义
的〈关于王实味的文艺观与我们的文艺观〉的总结性长文,最终将王实味
扫入「反动派」的行列。

《解放日报》为贯彻毛泽东的意图尽心尽责,全面发挥了其作为党与
领袖喉舌的功能。
然而即使是训练有素的驯马,
难免也有马失前蹄的时候。
1942 年 4 月 10 日和以後一个短时间,
《解放日报》竟忘了「反对虚假真实
性」的原则,居然在整风运动的高潮中分别报道了中央党校一男学员自杀
和延安大学一女生自杀的消息,此「错误」被毛泽东迅速抓住。毛严厉指
责《解放日报》
「仍不和中央息息相关」
,报纸「尚未成为真正的党的中央
机关报」。
他称,
「有些消息如党校学生自杀是不应该登的」,
并表示,
《解
放日报》的几篇社论仍有错误。毛再次重申,
「以後凡有重要问题,小至消
息,大至社论,均须与中央商量」。\footnote{ 1942 年 9 月 5 日,陆定一在《解放日报》
、新华社第二十二次编委会上,传达毛泽东对改版後的《解放日报》
的上述批评意见。参见〈延安《解放日报》史大纲〉
,载〈新闻研究资料〉
,第 17 期,页 18.}和毛泽东相比,
《解放日报》编辑们头
脑毕竟简单一些,他们耳闻目睹在延安不时发生的干部、学生自杀事件,
以为选择一两条消息刊登也无妨,却未料道无意中他们已犯下「暴露阴暗
面」的严重政治错误。在毛泽东大喝一声後,从此在《解放日报》上就再
未有任何有关延安消极面,诸如自杀事件的报道了。

\section{邓发被贬黜与中央党校的三次改组}
1941 年 9 月中央政治局扩大会议後,
毛泽东四面出击,
向王明、
博古、
张闻天等国际派发起全面进攻,毛的进攻的步骤是先行夺回被国际派长期
控制的意识形态宣传部门和延安各类学校的领导权,继而全面整肃党和军
队中的留苏分子,用自己的思想彻底改造全党,中共中央党校的改组就是
毛为实现这一目标,将两个战役一并进行的一次重大战略行动。

中央党校是中共为对党的中高级干部进行马克思主义基础训练而设置
的干部教育机构,1933 年 3 月以「马克思共产主义学校」之名创立于中央
苏区「红都」瑞金。由于处在战争环境,设置极其简单,学制也较短,分
别从两个月到六个月不等。1935 年 11 月,中央红军长征结束後不久,因
长征而停办的「马克思共产主义学校」在陕北瓦塞堡复校并易名为「中共
中央党校」。
虽然自 1933 年後,董必武长期具体领导中央党校,但是张闻
天作为党在意识形态方面的最高负责人和前
「马克思共产主义学校」
校长,
他在中央党校具有广泛的影响力,所以中央党校的教学安排或讲授内容,
一直都处在张闻天的直接或间接的控制下。1937 年 5 月,董必武调任陕甘
宁边区政府代主席,由李维汉接任中央党校校长。1938 年 3 月政治局会议
後,又改由康生担任校长一职。康生就任中央党校校长虽仅半年左右,但
他在党校营造出的神秘紧张的气氛却给当年在党校学习的干部留下深刻的
印象\footnote{宋平:
〈张闻天同志对于干部理论教育的贡献——重读
《中央关于办理党校的指示》〉, 载《延安马列学院回忆录》,
页 38.}。
1938 年 11 月中共六届六中全会後,
中央党校校长一职空缺下来,
陈云以中央组织部部长的身份代管中央党校,
直至 1939 年底邓发担任中央
党校校长为止。

作为中央政治局候补委员,原政治保卫局局长的邓发被任命为中央党 校校长,是
邓发在中共核心层的地位进一步衰落的反映。遵义会议後,毛 迅速疏远与周恩来、
博古关系密切的邓发。1935 年 10 月,中央红军长征 一抵达陕北,毛就将邓发贬
为中华苏维埃政府西北办事处粮食部长,1936 年 6 月, 又以向共产国际汇报为
借口, 将邓发打发去莫斯科。1937 年 9 月, 邓发自苏联返国後,毛泽东不准
邓发返回延安,而让邓发在迪化作地位较 低的中共驻新疆代表和八路军驻新疆办
事处主任。在中共实现战略大转移, 加紧调兵点将,竭力发展军事力量的关键时
刻,毛把邓发箍在远离国内政 治中心的迪化, 使邓发彻底丧失了在八路军、 新
四军建功立业的机遇。1939 年末,邓发奉命返回延安,随即就任中共中央党校校
长,1940 年初又被任 命为中共中央职工委员会书记。具有讽刺意味的是,这两个
职务均是 1938 年春毛为考察刚从苏联返国的康生而有意让他担任的闲职,随着康
生获得 毛的信任,被任命为中央社会部部长,康生留下来的这两个闲职就改由昔
日权倾一时的中共「契卡」首脑、今天正在走下坡路的邓发来填补了。在 这段时
期, 邓发应张闻天的邀请, 还曾前往马列学院, 在张闻天主持的 「十 年苏维
埃运动」全院大课上讲过苏区保卫工作\footnote{雪苇: 〈在延安马列学院三班
的听课回忆〉,载《延安马列学院回忆录》,页 123.}。

邓发是参加过 1922 年香港海员大罢工和组织 1925 年省港大罢工的中
共党内少数出身工人阶级的著名领袖,一身兼有早期共产主义者清教徒式
的理想主义和狂热的苏联崇拜者的性格特征。1935 年後,因被剥夺了情报
肃反大权和在中共核心层影响力的下降,邓发的狂热性大为减弱,开始对
党内高层政治生活的复杂性逐渐有所认识。1936 至 1937 年,邓发居苏联
一年,参加了以王明为团长的中共驻共产国际代表团,和王明有了较多的
工作联系。返回延安後,邓发对毛表现出有尊严的承认态度,但是并无曲
意的奉承和吹捧,与毛的关系始终限于正常的工作范围。

邓发上任後不久,很快就因中央党校的课程设计问题与毛泽东发生了
分歧。邓发就任中央党校校长时,已是毛提出「马克思主义的中国化」口
号的一年後,中共文宣部门的气氛正在发生急剧的变化。1940 年 2 月,中
央党校的教学方针已被中央书记处规定为「由少到多,由浅入深,由中国
到外国,由具体到抽象的原则」, \footnote{〈中央关于办理党校的指示〉
(1940 年 2 月 15 日),
载王仲清主编:
《党校教育历史概述(1927-1947)》(北
京:中共中央党校出版社,1992 年),
页 212.}然而这个时候的中共文宣大权仍由张闻
天掌握,而邓发对莫斯科原教旨主义的敬意并未因自己境遇的改变而稍有
减退。中央党校的教学计划尽管已作了较大的变动,但是作为中共干部理
论教育的重镇,中央党校的学员,尤其是高级班的学员仍需学习政治经济
学、历史唯物论与辩证唯物论、近代世界革命史等课程。尽管教学课目已
一再精简,但是毛泽东仍然很不满意,因为只要讲授这些课程就必然给那
些留苏、留日的「红色教授」提供「掉书袋」的机会。毛泽东另一个说不
出的恼怒,则是邓发对学习毛的著作未予以足够的重视。毛相信,自己的
著作是不大被「红色教授」看得起的。中央党校虽然也组织学员学习讨论
毛的论文,但在毛眼中,这些大多属应景之举。特别令人生气和无可奈何
的是,即使深受毛泽东青睐、被毛指定干部必读的《联共党史》也不得不
由那些号称精通俄文,熟悉苏联情况的「红色教授」来讲授,于是在毛泽
东的眼里,中央党校不啻是一座被冥顽不化的「教条主义者」统治的堡垒。

毛泽东原先对中央党校并不十分重视,和军队相比,中央党校一类文
宣单位在毛的政治天平上只占较轻的份量。对于信奉「枪杆子万能」的毛
泽东而言,掌握并牢牢控制军队是其考虑一切问题的出发点——「有了军
队可以造党」这句名言,最典型地反映了毛在军队与党关系上的「唯枪杆
子」
主义的倾向。
因此,
1935 年後毛有意让国际派继续控制中共文宣阵地,
作为对方让出军队领导权的一笔政治补偿。但是,随着毛泽东权力不断得
到加强和巩固,
毛已不满足于仅仅领导中共军队,
毛现在已基本控制了党,
并正努力以自己的意志全面改造党,正是到了这个时候,原先不甚被毛泽
东看重的中央党校等一类单位就变得极其重要了,现在毛不仅有时间而且
有精力来过问中央党校的「教条主义教学方法」了。

当然,毛泽东关心的绝不仅仅是中央党校的「教学改革」
,他对中央党
校之所以有强烈兴趣还基于另一现实考虑:根据中共中央原先的决定,中
共七大将于 1940 年在延安召开,
各根据地和国统区党组织推选的七大代表
已陆续来到了延安。但是毛泽东并不愿在无绝对胜利的把握下召开七大,
他迫使中央政治局接受自己的主张,将中共七大推迟举行。但是毛又不愿
放这些代表返回原地,他要利用这一难得的时机对七大代表进行逐一的考
察和清理。将七大代表安排进入中央党校学习,是毛泽东要求七大代表留
在延安的最能说出口的理由,而中央党校又是安置这批干部的最佳场所。

在这种形势下,中央党校的责任不可谓不大也。一方面,中央党校要
进行自身的改造;另一方面,它又被毛赋予了特别任务。显然,邓发将不
足以承担如此重大的使命,无论是邓发对毛的态度,抑或是邓发与王明等
国际派的关系,都使毛对邓发不能放心。但是,邓发毕竟是中央政治局的
成员,毛一时又没有充足的理由可将邓发的校长职务罢免,于是,毛泽东
再次施出他惯用的「掺沙子」办法——继续保留邓发的校长职务,但调彭
真任中央党校教育长,让彭真掌握中央党校的具体实权,从而架空邓发。

1941 年上半年,
毛泽东将担任中共晋察冀分局书记的彭真从华北调回
延安。同年 12 月,毛泽东对中央党校进行第一次改组,邓发虽然继续留任
校长一职,但却又在中共中央内专设一个中央党校五人管理委员会,用以
分散邓发的领导权限。该管理委员会除邓发为委员外,毛的两个重要助手
彭真、陆定一,以及中央组织部的王鹤寿、军委总政治部的胡耀邦也名列
委员之中\footnote{王仲清主编:
《党校教育历史概述(1921-1947)》,页 126、255.}。

离中央党校第一次改组不到三个月,整风运动就在延安全面展开,一
时延安各机关、学校高干约三百至四百人,被集中进中央党校。为了落实
毛的整风计划,1942 年 2 月 28 日,中央政治局发布〈关于党校组织及教
育方针的新决定〉,宣布对中央党校进行第二次改组。
这次毛泽东亲自出马,
任命政治地位远逊于邓发的彭真为中央党校主管整风运动的最高负责人。
毛又一次施出釜底抽薪的谋略,将中央党校的领导权进一步分散,宣布将
党校划归中央书记处直接领导,由毛泽东负责对中央党校的政治指导,任
弼时负责组织指导,日常工作由邓发、彭真、林彪组成的管理委员会主持,
取消 1941 年 12 月成立的党校管理委员会\footnote{王仲清主编:
《党校教育历史概述(1921-1947)》,页 126、255.}。

这次改组後,邓发虽继续
留任党校校长,但他的权限已被削夺殆尽,仅负责主持教务会议。彭真则
被委以主持权力极大的政治教育会议,并和陆定一一道主编指导整风的刊
物《学习报》
。至于林彪,虽被责成主持中央党校的军事会议,但林彪并没
到中央党校就职,而是前往重庆代表毛与蒋介石会面,并留在重庆中共代
表团,直至 1943 年 7 月才返回延安。

经过第二次改组,原先对党校工作事无巨细「都事必躬亲」的邓发在 中央党校已
成了一个光杆校长。从现有资料看,邓发没有对毛作出任何抵 抗,他平静地接受
了这种屈辱性的安排,不仅如此,邓发在公开场合对毛 泽东的整风计划还表示了
一般性的支持。\footnote{1942 年 2 月 1 日,邓发主持了中央党校开学典礼,
就是在这次大会上,毛泽东作了有名的〈整顿学风、党风、 文风〉的演说,在毛
报告前,邓发在会上作的开场白里提出将以克服教条主义与主观主义作为党校教育
的新方针。整 风运动开始後,邓发经常在一些场合以自己经历讲述「工农分于与
知识分子结合的必要」。参见《中共党史人物传》, 第 1 卷(西安:陕西人民
出版社,1980 年),页 363.} 然而他仍被完全排除在彭真领 导的整风领导核心
之外,而他所负责的教务会议,也因中共中央下令中央 党校废除原有的所有课程,
事实上已名存实亡。\footnote{中共中央政治局于 1942 年 2 月 28 日作出的
〈关于党校组织及教育方针的新决定〉明确规定,中央党校「停止 过去所定课程,
在本年内教育与学习党的路线」,参见王仲清主编: 《党校教育历史概述(1921
一 1947)》,页 255.}这样,邓发这位被中央党 校炊事员、勤务员亲切称呼为
「邓大哥」的政治局候补委员兼中央党校校 长,在党校已处于无事可干的境地,
邓发除了过问一下党校的蔬菜生产和 扩大猪圈等一类杂事,\footnote{《中共党
史人物传》,第 1 卷,页 364.  邓发、项英、彭德怀是中共局面改善後,少数
几个仍保留有早期共产主 义者清教徒式工作和生活习惯的高级领导人。1939 年底
邓发自新疆返回延安後,发现一些人已开始追逐生活享受。对 此,邓发十分气愤,
他感叹道: 「我们党是一个劳动阶级的党,但是现在,一小部分人已经忘本了」
。参见司马璐: 《斗 争十八年》(节本),页 74.  }只得将工作重点转入延
安的中央职工运动委员会。

1943 年 3 月,邓发挂名中央党校校长的日子正式结束。邓发被解除了
中央党校校长的职务,被打发到由刘少奇担任书记的中央组织委员会的下
属单位——民运工作委员会作一名空头书记,毛泽东亲自兼任中央党校校
长,彭真、林彪担任副校长。林彪之被毛泽东挑选作党校副校长具有重要
的象征意义,毛需要借用林彪的军人身份,震慑党机关和党的高级干部,
然而林彪对此新职似乎并无兴趣,他固然全力支持毛泽东,却不愿多出头
露面,更不愿以自己的手去惩治别人。对林彪的倦怠,毛泽东表示宽宏大
量,他允许林彪只挂副校长的头衔,不到中央党校去视事,而将中央党校
的日常工作交彭真主持。至此,中央党校完成了它的第三次改组。

\section{彭真与中央党校的彻底毛化}
毛泽东为什么挑选彭真作中央党校的总管,让这个既非中央委员,又
未参加过长征,长期在白区从事地下工作的「城市职业革命家」来主持审
查包括参加过长征的老干部这样一个涉及要害的工作?答案只有一个:彭
真是中共党内刘少奇系统的第一号大将,毛意欲借助刘少奇的支持,利用
彭真在中央党校整肃异己,以巩固自己在党内的至高无上的地位。彭真又
为什么在中央党校为贯彻毛的整风部署日夜辛劳?答案也只有一个:刘少
奇、彭真要借助毛泽东的力量,利用主持中央党校整风运动的难得机会,
整肃其在党内的政敌,为刘少奇作为「白区正确路线代表」奠定基础,这
就是彭真之被毛泽东责成领导中央党校,和彭真在中央党校大树毛泽东权
威的全部底蕴。

彭真调入中央党校後,立即将校长邓发撇在一边,当仁不让地将党校
的所有重要权力抓在手中,
并采取一系列措施全面落实毛泽东的整风意图,
使中央党校的面貌和风格都发生了根本性变化。

为适应中央党校作为大规模干部「再教育中心」的需要,彭真将延安
的等级差序制全面引人中央党校,实现了中央党校的机关化和官僚化。中
央党校原先就有按入学干部的原有级别分班的惯例,
但是在 1942 年前,
由
于学制较短,高级班与中、初级班学员的政治待遇差别并不十分明显,学
校的管理机构也较为精干。但是随着来延安准备参加中共七大的代表和延
安各机关、学校的高中级干部陆续进入中央党校,原有的管理机构已不能
适应新的局面。在党校学习的高级干部,他们的身份兼具两重性质,一方
面是接受教育和审查的学员;另一方面又是曾掌管一方或一个部门的负责
干部;如何使这批干部既能接受审查,又不致过份影响他们的情绪,这就
成了一项急待解决的问题。为了解决这两者之间的矛盾,彭真制订了两项
制度:第一、将依照干部级别分班的原则固定化和制度化,在中央党校分
别成立代表班(七大代表)
、旅级地委及旅级以上干部班、团级及县级干部
班等不同班次,使即将展开的干部间的「批评与自我批评」局限于同级干
部范围内。第二、在中央党校确立优待高干的政策,明文规定凡中央委员、
旅级及地委书记以上干部,其妻子愿意随丈夫入中央党校学习者,一律可
照顾入校,而免除其家属的资格审查程序。\footnote{〈中共中央关于中央党校学生人学与调动问题的规定〉
(1942 年 3 月 11 日)
,载王仲清主编:
《党校教育历史概
述(1921 一 1947)》,页 260.}同时,原配有勤务员的高干,
准许其将勤务员带入学校归自己使用\footnote{彭真:
〈中央党校计划〉
(1941 年)
,载王仲清主编:
《党校教育历史概述(1921-1947)》,页 124.}。

与上述两项制度相配套,中央党校还增设了新的管理机构,从学校总
部机关、各部及附属的组织教育科、秘书科直至各支部,建立起垂直的组
织系统,并配备了专职政工干部(部组织教育科在各支部都派有负责联络
的组织教育干事)
,形成了一个严密的组织网络。经过这番改造,党校的机
构迅速扩大,在 1944 年初至 1945 年夏党校的鼎盛时期,全校人数共约六
千多人,
其中一半为教职员工。\footnote{王仲清主编:
《党校教育历史概述(1921-1947)》,页 165、127.} 伴随着机构扩增,
官僚化趋向急剧增长,
原中央党校曾经存在着的学员与学员之间、学员与学校管理干部之间的那
种相对平等的关系基本被扭转了过来。

彭真领导中央党校後党校发生的第二个重大变化,就是彻底废除了党
校系统讲授马列基础知识的传统,而代之以学习毛泽东、刘少奇、康生等
的论著,以及经过选择的斯大林著作和《联共党史》
,并将「学习」与展开
党内斗争紧密地结合起来。
按照 1941 年底中共中央制定的
〈关于中央党校
计划〉,第一次改组後的中央党校,
仍然必须讲授经过重新编排的马列基础
知识,以及中国和世界近代革命史。此计划还对中央党校学员的学习时间
作了明确规定,
将原先半年至一年的学习时间延长为两年。\footnote{王仲清主编:
《党校教育历史概述(1921-1947)》,页 165、127.} 然而到了 1942
年 2 月 28 日中央政治局作出〈关于党校组织及教育方针的新决定〉时,毛
泽东干脆宣布自即日起,中央党校停止过去所定课程,对学员的学习期限
也不再作出硬性规定。显然党校学制长短必须服务于毛的政治目标,当毛
泽东要打击张闻天等人时,
他一再抨击延安的干部教育制度既繁琐又费时;
而当毛泽东要利用党校达到自己功利主义的目的时,他又执意将大批干部
长期集中在党校,不把他们的头脑「洗干净」绝不善罢甘休。彭真对毛泽
东的意图心领神会,他巧妙地将中央党校的「学习」引向对王明、博古等
国际派的怀疑和攻击,又使这种「矛头向上」的批判和对干部本人的清算
挂起钩来,结果无休无止的学习、批判、审查将学员拴在党校长达三至四
年。
在毛泽东的总策划下,
彭真使中央党校完全变成了政治斗争的大舞台。

中央党校由彭真担任领导後所发生的第三个变化,也可以说是最重大
的变化,是彭真和中央社会部、中央组织部密切合作,使中央党校在长达
二、三年的时间内,成为中共最大的干部审查中心。

把党创办的军政学校权充
「有问题」
干部的审查收容所在 1942 年前即
曾有过先例。1937 年 3 月 31 日中共中央政治局作出〈关于张国焘同志错
误的决定〉
,在这前後,一批红四方面军高级干部被送入抗大「学习」
,抗
大校为红四方面军高级干部专门编了三个班。\footnote{《李先念传》编写组编,朱玉主编:
《李先念传(1909 一 1949)》(北京:中央文献出版社,1999 年)
,页 310-11;
另参见《成仿吾传》编写组编:
《成仿吾传》
,页 112.} 许世友、
王建安等著名将领
都曾被安置在该校「揭发、清算国焘主义」
。因不堪忍受株连,许世友、王
建安等曾议论率在抗大「学习」的红四方面军干部返回鄂豫皖或川陕打游
击,但被人打了「小报告」
,为此许世友等曾受到以董必武为主席的审判委
员会的审判,并被处以徒刑。毛泽东权衡利弊後采取怀柔政策,最终陆续
开释了被拘押的许世友和在抗大「学习」的红四方面军高级将领,这样抗
大作为「有问题干部」的审查和收容中心的历史才告一段落。

中央社会部渗入中央党校作政治情报工作,
也不是自 1942 年始,
然而
在整风运动前,中央社会部在党校的活动处于极其秘密的「地下」状态。
抗战爆发後,中央党校对外的代名一度称作「中山图书馆」,康生领导下的
中央社会部主办的
「敌工训练班」,将毕业的学员,
作为中央社会部的耳目,
秘密打入
「中山图书馆」。这些潜伏在党校各个部门的耳目必须定期回社会
部全面汇报在中央党校的地下侦察活动\footnote{司马璐:
《斗争十八年》
(节本)
,页 77.}。

中央党校与中央组织部存在密切关系则完全是公开的。中共中央规定
凡进入中央党校的干部,除来自军队系统和边区系统先由军委总政治部和
西北局审查其政治条件、再报中央组织部统一审查外,中央直属系统及外
地来延安的党政军干部一律得由中央组织部审查其政治条件,
才可入党校,
而干部在中央党校「结业」後,也统由中央组织部分配工作\footnote{〈中共中央关于中央党校学生入学与调动问题的规定〉
(1942 年 3 月 11 日)
,载王仲清主编:
《党校教育历史概
述(1921 一 1947)》,页 256-57.}。

1942 年,康生领导的中央社会部公开进入中央党校,与彭真须导的中
央党校,陈云、李富春领导的中央组织部密切配合,结成了一个完全效忠
于毛泽东的「铁三角」:

中央党校作为高干的「思想改造」中心,入校条件十分严格,须由中
央组织部逐个审查认可後方能入学;

中央党校又是审查中心,经中央组织部审查批准入校的学员,还得再
次接受中央党校的政治审查,学员在校期间非有特殊情况一概不得调出;

中央社会部配合党校的政治工作,凡经审查认为「有问题」的人,
「情
节严重」者移送中央社会部,
「一般问题」者留校隔离审查。

因此,除了一部分与毛泽东关系密切的高干和一般中下级干部外,进
入中央党校的干部还包含下里两类人:

一类是虽无重大政治历史问题,但曾在不同时期,程度不同地与国际
派或「经验主义者」有较多关系的干部;

另一类是政治历史有「嫌疑」的干部。

例如 1943 年 8 月 16 日,中央书记处在致邓小平转太行分局各同志的
电报中就要求送「有造就前途的高级上级干部四百至五百人」和「犯错误
难处理的干部」来延安学习\footnote{王仲清主编:
《党校教育历史概述(1921-1947)》,页 158.}。

由于进入中央党校的干部情况各异,中央党校在学员编班问题上采取
了特别的安排:在党校内,既有按干部级别分类组成的班;也有依「政治
可靠性」的类别组成的班。1943 年後,送入中央党校的干部人数大增,其
中大量是所谓「有问题」的干部,在彭真的主持下,分别将这类干部集中
在下列单位:

中央党校第三部:其成员多为被解散合并至中央党校的前中央研究院的
知识分子干部。

中央党校第六部:其成员多为来自国统区的干部和青年知识分子。

上述两个部是中央社会部在党校深入活动的重点单位,也是「挖」出
「特务」、「叛徒」最多的单位。中央党校三部和六部学员的最重要工作就
是「交待问题」。

在彭真的主持下,中央党校整风领导班子借助保安和组织两股势力在
校内「大破大立」
,破对国际派和「经验主义者」的迷信,立对毛泽东、刘
少奇的赞颂和服从。从 1942 年春始,中央党校就充斥着捧毛、捧刘的浓厚
空气,到 1945 年,校名也曾一度改为「中共中央毛泽东党校」。
\footnote{参见卢弘:
《李伯钊传》
(北京:解放军出版社,1989 年)
,页 417.} 

1942 年 5 月 14 日和 5 月 16 日,
《解放日报》分别发表彭真撰写的社
论:
〈领会二十二个文件的精神与实质〉和〈怎样学习二十二个文件〉
,文
章对毛泽东的整风意图详加阐述,强调每一个共产党员都要解决立场、观
点、方法问题,并用来「反省自己的工作,反省自己的思想,反省自己的
全部历史」。
彭真尤其要求党校学员要反复精读《二十二个文件》中所收载
的刘少奇的文章,以加强学员对刘少奇的认识。

彭真在中央党校担任主要领导期间,
和毛泽东保持着极其密切的联系。
1942 年 7 月,中央党校学风学习阶段结束,中央党校拟定的干部考试的四
个题目,是报请毛泽东亲自修改後才确定下来的。彭真对党校运动中所发
生的争论,事无巨细,都向毛泽东及时汇报,甚至一些鸡毛蒜皮事,诸如
某个军队学员对结了婚的干部每周六过夫妻生活表示不满这类的事,毛都
知道。\footnote{王仲清主编:
《党校教育历史概述(1921-1947),页 77-78;另参见《延安中央党校的整风学习》
》
,第 1 集(北
京:中共中央党校出版社,1988 年)
,页 92.}毛泽东也经常到党校彭真处了解情况,会见党校干部
\footnote{ 1943 年冬,毛泽东在井冈山时期的老部下江华返回延安,遵刘少奇命进入中央党校学习,毛泽东在彭真的住处
接见了他,参见江华:
《追忆与思考——江华回忆录》
(杭州:浙江人民出版社,1991 年)页 204.}。

彭真的忠诚和高效率的工作能力赢得了毛泽东的称许,相比之下,张
鼎丞、江华等毛的嫡系干部虽然也在中央党校,但是张鼎丞只担任了党校
的二部主任,
而江华仅为一部的普通学员。
毛泽东的信任使彭真信心倍增,
干劲十足。1943 年 10 月党校在肃奸、抢救高潮中转入路线问题「学习」,
彭真重翻历史老帐,
率先批判 1928 年的顺直省委和 1935 年的中共北方局,
明批柯庆施、
高文华,
实际上将矛头暗指 1928 年底代表中央政治局处理顺
直省委问题的周恩来,堂而皇之地将刘少奇树为白区正确路线的代表。勿
庸置疑,树立刘少奇就是树立彭真自己,刘少奇既然是「正确路线」的化
身,彭真作为当时刘少奇的副手自然也身居正确路线之列,这样彭真就为
自己进入中共最高核心层准备了充分的法理依据。

中央党校的整风运动为彭真加强自己在中共党内的地位提供了最佳的
机会,1943 年 7 月,刘少奇、彭真在北方局时期的老部下安子文被调入党
校,担任二部副主任,作张鼎丞的副手,但安子文实际上是彭真最得力的
助手。彭真和安子文在党校细心观察,仔细物色「忠于正确路线」的干部。
1944 年,陈云卸去了担任七年之久的中央组织部部长的职务,彭真立即填
补空缺,晋升为中组部部长。在毛泽东的支持和默许下,彭真和安子文利
用筹备召开中共七大的时机,将一批刘少奇担任北方局书记时的部下,或
者安排担任中央委员和候补委员,或者委以党政军关键部门的领导职务,
于是,刘少奇系统羽翼渐丰,成了中共党内最大的「山头」之一。

延安整风运动期间,由中央党校开创的学校官僚化、机关化的管理体
制,贬低理论知识的反智主义倾向,和动用政治保安力量开展思想斗争和
组织整肃的方式,构成了正在形成的毛氏新传统的重要组成部分。中央党
校的整风经验不仅逐渐在各根据地的党校得到推广,成为开展党内斗争的
一种基本形式。1949 年以後,它的基本精神还得到进一步的发展和完善,
在经过若干修正和补充後,成为中共在机关、文宣部门和高等院校进行持
续不绝的政治运动的传统方法,其影响一直延续到现在。而延安时
期中央党校创立的官僚化、机关化的学校管理体制在一定程度上仍是今天
中国高等院校管理体制的基础。

1942-1945 年,是中共历史上大动荡、大改组的关键时期,毛泽东、 刘少奇、彭
真互相支持,携手合作,将中央党校搅得天翻地覆。毛泽东、 彭真利用中央党校
整肃异己的方法, 其实和 1929-1930 年间斯大林利用莫 斯科红色教授学院,整
肃布哈林等所谓「右倾机会主义分子」的方法并无 多大的差别。所不同的是,斯
大林在莫斯科红色教授学院搞清洗的时间不 到两年,而毛在中央党校搞整风竟长
达三年半,中央党校成了名副其实的 整风的「风暴眼」。
