\chapter{上层革命的开始:毛泽东与王明的首次公开交
锋
}
\section{穷途末路的国际派
}
1938 年 9 月,王稼祥传达的季米特洛夫支持毛泽东为中共领袖的「口
信」,在中共上层犹如引发了一场七级强地震,对王明等产生了巨大的冲
击,中共核心层随即发生急剧的分化和新的组合。毛泽东虽未正式出任党
的总书记一职,但从此他已理直气壮、当仁不让地独揽中共党、军一切大
权。
在毛的分化瓦解下,
原先就已显现分裂迹象的王明营垒更加分崩离析,
王明等开始丧失与毛争斗的意志,在政治上逐渐衰竭下去。

作为毛泽东头号政治对手的王明,自 1938 年底从重庆返回延安後,实
际上已被毛泽东锁在笼中。从莫斯科返国後,王明在政治上的风光岁月仅
仅维持一年,王明初抵延安时的那种热烈、欢快的气氛早已烟消云散,现
在王明只能依照毛泽东的安排,在某些节庆场合出来点缀一番,其政治上
的权力已被大大削弱。

在毛泽东的控制下,王明在政治上的失势是一步步进行的。六届六中
全会後,王明仍是书记处和政治局成员,在六中全会後还新兼任了中央统
战部部长。从重庆返回延安後,在 1941 年初,王明又被任命为中央南方工
作委员会、东北工作委员会和中央党校委员会等三个机构的主任,王明并
担任了中央妇女工作委员会书记和中国女子大学校长。从表面上看,王明
一时担任了许多重要职务,然而,王明的这些职务大多为空头闲职。1939
年後的中央政治局和书记处已完全由毛泽东控制,开会时间、会议议程等
一切皆由毛决定,旁人不得置喙。至于中央统战部,则是延安的一个冷清
衙门。统战部下辖三个科:干部科、友军科、各党派科,然而中共有关与
国民党统战的所有大政方针,全由毛泽东亲自掌握,周恩来则起着辅佐毛
及执行毛指示的作用,王明在决策过程中所起的作用极为有限。延安的中
央统战部只有王明和副部长柯庆施、南汉宸(1939 年 9 月至 1941 年初任统
战部副部长,1941 年 2 月调往边区政府任财政厅厅长)以及少数工作人员。
除了有时配合延安交际处出面接待几个来访的国统区知名人士外,统战部
的主要工作是负责指导延安的中国女子大学,统战部的领导,从王明、柯
庆施到干部科科长徐一新(即徐以新)都在女大任职。原先中央统战部对
边区党委统战部也负有指导责任,1933 年,时任边区党委副书记兼统战部
部长的王观澜表示反对王明提出的边区也是统战区的观点,引起与王明的
争论。毛泽东一锤定音,决定新设边区统战委员会,由王观澜任主任委员,
宣布今後边区统战问题由该委员会领导,大事直接请示中央,将王明任部
长的中央统战部的最後一点权力也彻底剥夺干净。\footnote{参见赵来群:
〈毛泽东与王观澜〉
,载《党的文献》
,1996 年第 6 期。
} 中央南方委员会、东
北委员会这两个机构更是形同虚设。南方工作委员会成立于 1939 年底,本
来是为了加强对国统区中共地下党的领导,但事实上有关南中国党的工
作,基本由设在重庆的中共南方局领导。在抗战阶段,延安与东北的直接
联系很少,特别是东北抗联失败後,中共在东北的工作几乎完全停顿,为
此,1942 年中组部曾挑选 177 名东北籍干部前往东北开辟工作。被派往东
北从事地下工作的党员主要为苏联情报机构服务,间或也与延安发生电讯
联系,但是在日军严密控制下的伪满洲国,从延安派出的中共地下党员很
难开展工作,多数秘密机关被日军破获,因此设立中央东北工作委员会,
在很大程度上也是为了虚应人事。任命王明负责中央党校工作委员会,却
是毛泽东的别出心裁之举。毛以此举有意挑起王明与张闻天等的矛盾,指
望坐收渔人之利。王明真正负责的工作岗位,只是中央妇委和中国女子大
学。
而安排王明担任中央妇委书记和女大校长,
则有明显羞辱王明的含意。

对于江河日下的王明,毛泽东无丝毫顾惜之意,反而谋求对王明的进
一步打击。王明领导的中央妇委共有六名常委,包括其妻孟庆树,担任常
委的蔡畅、帅孟奇,不时就工作中的一些问题和王明发生冲突。一度在国
际和国内政治舞台上风光十足的王明,到了 1941 年初,竟不得不主持召开
妇委保育工作会议。
王明任校长的中国女大也受到种种限制,
在解决生源、
人员借调、学生分配等许多问题上,都碰到不少麻烦。1941 年 2 月 13 日,
王明为女大毕业生分配事写信给陈云,提议中央组织部将调做其它工作的
女大毕业生的比例下降到 25\%,希望把来延的女学生均送入女大学习。次
日,陈云覆信给王明,规劝王明「彼此以服从中央书记处多数同志通过的
决定为好」,陈云在信中说,「女大是我党的学校,全部学生都应归中组
部在中央总的意图下分配工作」,暗指王明将女大视为自己的私人范围。
陈云向王明明确表明了自己的态度,「妇女工作是全党工作的一部分,我
是党的工作者,我的责任和我的要求,也仅仅是『一视同仁』四个大字」。
\footnote{ 参见刘家栋:
《陈云在延安》
,页 170;另参见《王明年谱》
(合肥:安徽人民出版社,1991 年)
,页 138-39. }1941 年 6 月,王明被免去中央妇委书记的职务,由蔡畅接任。9 月 1 日,创
办两年、
在海内外享有盛誉的中国女大被合并进延安大学。
蔡畅上任伊始,
马上将妇委中原在中国女大学习的女干部召集在边区政府交际处会议室
开会,「了解一下王明当女大校长时,有什么错误言论」。\footnote{参见勉之:
〈革命圣地承教泽〉
,载《延安马列学院回忆录》
,页 148. 
} 在蔡畅的领
导下,中央妇委开始批判王明在领导妇委工作中所犯的「主观主义与形式
主义」的错误。

毛泽东对付王明最厉害的措施是切断王明与莫斯科的联系渠道。
毛独
握与斯大林的通讯系统,除毛之外,任何人不得染指。据师哲透露,1940
年 2 月任弼时在莫斯科时,共产国际机要处交给他两套机要密码,由周恩
来于 3 月随身带回延安。同年 11 月,延安与莫斯科新的通讯系统正式开通,
「效果良好,通讯准确无误」,但「只有毛主席一人有权使用」。掌握这
个绝密的通讯系统的机构,对外称「农村工作部」,又叫「农委」,设在
中央警卫团附近的小砭沟,部长为吴德峰,副部长是帅孟奇。毛泽东为了
避人耳目,任命王观澜为中央农村工作委员会主任,其实王观澜并不参与
此事,吴德峰才是真正的负责人。但是,作为中央机要局局长的吴德峰也
不能与闻毛与斯大林电报往来的内容。参与电报翻译的只有极少数中央社
会部工作人员,如师哲等一、二人。极端机密的电报则由任弼时翻译,直
呈毛泽东,从而避开了师哲。

在毛泽东的严密封锁下,身为共产国际执委、主席团委员、政治书记
处候补书记的王明,获知共产国际的信息,只能通过毛泽东的口头传达。
而是否向政治局委员传达共产国际的来电,是全文传达还是部分传达,是
向个别人传达,
还是向全体政治局委员传达,
这些全凭毛的个人意愿决定。

为了防范王明与苏联和外界联系,毛还严格限制王明前往重庆出席国民参
政会。只是因为得到周恩来的支持,王明才出席了 1939 年 9 月在重庆举行
的国民参政会第一届第四次会议,\footnote{参见《周恩来年谱》
,页 446.}而在这以後,王明就再没有去过重庆。

王明在六中全会後的一段时期内,似乎并无政治上受挫的明显迹象,
但随着毛针对王明的一系列措施的陆续出台,他已完全明白毛的意图所
在,只是有口难言,无力还手。1939 年春夏之际,当毛泽东携带江青,由
两名贴身卫士随从,乘坐当时延安唯一的一辆小包车——海外华侨捐赠八
路军前线将士的救护车,风驰电掣般穿行于延安街道时,人们看到的王明
则是形单影只,「时常独自漫步街头,也不带一名警卫,低着头,不发一
言地,沉重的脚步声中若有所深思」。
\footnote{参见司马璐:
《斗争十八年》
(节本)
,页 88.}

王明并不甘心自己在政治上的日趋败落,从 1939 年始,他就开始调整
对毛泽东的态度,试图以向毛示好的行动,改善自己的处境,在政治上重
新振作起来。1940 年,王明连续发表两篇文章,不惜对毛阿谀奉承,吹捧
毛对「发展马列主义理论所作出的巨大贡献」,甚至加封给毛三个头衔:
「中国革命的伟大政治家和战略家」、「伟大的理论家」。王明认为,自己
的言行至少会化解毛对他强烈的敌意,从而缓和并改善与毛的关系。

王明的这番表演十分拙劣,这种丧失了意志力和自尊的行为,使其在
毛泽东心目中原本就不高的形象,更加一落千丈。毛在得意之馀,不仅毫
不领情,反而视王明为一具可以任意摆弄的政治僵尸。1940 年,因热烈吹
捧毛泽东,王明的处境有所改善,毛泽东对王明还作出某种善意的姿态。
该年,原红一方面军干部黄火青从新疆返回延安(黄曾加入西路军,後进
入新疆),毛在接见黄火青时,特别关照黄一定要去看望王明。
\footnote{参见黄火青:
《一个平凡共产党员的经历》
,页 158.}
同年
3 月,王明在延安重新再版了 1931 年写的《为中共更加布尔什维克化而斗
争》一书,企图论证自己在党的历史上的地位。王明还凭籍其对马列原典
的熟稔在延安各机关、学校广作报告。在当年的延安知识分子中「王明同
志」是一个令人敬仰的名字,其受尊敬的程度和「毛主席」不相上下。口
若悬河的王明,作起报告来条理清晰,出口成章,几个小时的报告可以不
要讲稿。报告完毕,「再从头到尾归纳一遍,一二三四大项,下边又分甲
乙......再分,大家对照记录竟能丝毫不错」,有时一席演讲,竟受到数十
次掌声的欢迎。王明的「口才」和「理论水平」赢得了延安广大青年知识
分子的尊崇,人们普遍认为王明是「天才」,被公认为是「活马列主义」。
\footnote{参见《延安马列学院回忆录》
,页 112;刘家栋:
《陈云在延安》
,页 130;司马璐:
《斗争十八年》
(节本)
,页 73.}
这一切不仅把王明捧毛的效果冲得一干二净,更增添了毛对王明的憎恶。

和心情抑郁的王明相比,
1937 至 1940 年,
是博古心情比较舒畅的几年,
主要乃是博古远离延安,长期与周恩来在南京、武汉、重庆从事对国民党
的统战工作。六中全会後,王明被调回延安,但是博古继续留在重庆,担
任中共南方局常委兼组织部长。在周恩来的领导下,重庆的南方局继续保
持着长江局时期机关内互相配合、互相尊重的空气,丝毫没有在延安和其
它根据地普遍存在的「书记专政」的现象。生活和工作在这种与延安迥异
的氛围中,\footnote{邓颖超在  1984 年的一次谈话中,从另一个角度承认了上述事实。她说:
「当时长江局的民主集中制不健全,分
工如划线,个人管个人的工作」
,换言之,长江局没有任何领导人可以统掌一切。参见《抗战初期的中共中央长江局》
(武汉:湖北人民出版社, 1991 年)
,页 473.}博古暂时忘却了自己不甚光明的政治前途。

博古心情舒畅的日子到  1940 年 11 月终于结束。这时,国共关系正因新
四军北调问题而处于紧张状态。毛泽东判断国民党即将与中共全面决裂,
11 月 3 日,毛及中央书记处覆电周恩来及南方局,命博古、凯丰即回延安。
博古返回後,重新回到他原先担任的中央党报委员会主任的工作岗位,
1941 年 5 月,被责成领导新华社和《解放日报》,从此,毛泽东将博古牢
牢地置于自己的直接控制下。博古与毛打交道多年,早已领略毛那种「一
朝权在手,便把令来行」的领导气势。对他而言,忍受毛的权势,虽比王
明相对容易一些,但终难气平。所以博古的反应就具有了两重性:一方面,
博古自知气力不抵毛泽东,在返回延安後,很快就解除了精神武装,心如
枯井,任由毛随意摆布;另一方面,博古对苏联的干预还抱有一线希望,
在毛和康生的严密监视下,仍和苏联派驻延安的情报人员保持断断续续的
来往,将自己对党内问题的看法悄悄传送给莫斯科。苏联当时派驻在延安
的代表弗拉基米洛夫在他的《延安日记》一书中曾描述过,博古如何躲过
\footnote{弗拉基米洛夫:
《延安日记》
,页 125、136-37.}
康生情报机构的监视,来见苏联人时的那种胆战心惊的状态。弗拉基米
洛夫的叙述得到师哲则面的证实。据师哲回忆,弗拉基米洛夫对他不止一
次地说,「博古对他帮助最大,使他弄清了国际、中国国内和党内许多复
杂问题的来龙去脉」。
\footnote{师哲:
《在历史巨人身边——师哲回忆录》
,页 219、179. 
}

作为原国际派第三号人物的张闻天,由于长期和毛泽东合作,他的处
境要明显好于王明和博古。但是在 1940 年後,张闻天在中共核心层中的地
位已大不如从前,而有朝不保夕之危。早在 1937 年毛泽东、刘少奇政治联
盟刚建立时,毛与张闻天的关系就发生了重大变化,张闻天作为毛的第一
合作者的地位业已丧失。在 1937 年 12 月政治局会议上,张闻天失去了中共
「负总责」的地位。到了六届六中全会後,张闻天原拥有的主持中央书记
处会议的权力,也正式由毛泽东所取代。但在 1939 至 1940 年,当刘少奇不
在延安时,
毛为了分化张闻天与王明的关系,
有意倚重张闻天而疏远王明。
因此,在一段时期内,张闻天似乎仍是延安的第二号人物。然而好景不长,
张闻天担任的这种角色,在任弼时返回延安後便宣告结束了。张闻天虽然
仍是政治局委员和中央书记,但其工作范围只限于延安的党的意识形态部
门。毛泽东开始对张闻天「横挑鼻子竖挑眼」了,据师哲称,四十年代初
期,毛曾针对张闻天「大而化之」的「工作作风」,对其「提出过批评」。
\footnote{师哲:
《在历史巨人身边——师哲回忆录》
,页 219、179.}
1940 年 3 月,周恩来从苏联返国,在政治局会议上传达共产国际领导人曼
努伊尔斯基对中共领导人的评价,曼氏认为张闻天是中共党内优秀的马克
思主义理论家。毛泽东闻之勃然大怒,他讥讽道:什么理论家,张闻天从
苏联背回的是一麻袋教条,\footnote{刘英:
《在历史的激流中——刘英回忆录》
(北京:中共党史出版社, 1992 年)
,页 126. }当场将周恩来挡回去。1941 年春之後,毛对
张闻天更加咄咄逼人,对张闻天负责的工作百般挑剔。张闻天起草的关于
干部教育的几个指示都曾给毛看过,再用中央书记处的名义发出,均是经
中央决议通过的,毛却出尔反尔,又来训斥张闻天。毛动辄出口伤人,经
常骂张闻天「一事不懂」,使他满腹委屈,又无处倾诉。\footnote{参见程中原:
《张闻天传》
,页 480-81. 
}在这个时期,
张闻天仍领导延安马列学院等文宣单位,他去马列学院的次数不多,但留
给学生的印象却十分深刻。有一份回忆资料这样描述当年张闻天的形象:
高挑身分......头戴列宁帽......胸前别着一枚约伍分硬币大小的、用红色玻
璃镶嵌的马、恩、列、斯像,不时地闪闪发光。\footnote{江围:
(难忘的岁月)
,载《延安马列学院回忆录》
,页 102. 
}张闻天的学生们无法想
象到,他们所尊敬的导师在经历了与毛共事几年的风风雨雨後,即将面临
被毛抛弃一边的命运。

从表象上看,在 1938 至 1941 年,原国际派中最风光的人物是王稼祥。
毛泽东为酬谢王稼祥传达季米特洛夫「口信」的功劳,在王稼祥返国後,
立即封他中央军委副主席和八路军总政治部主任两个头衔。原国际派人物
参与军委,在长征结束後,这是首次,也是最後一次。由于毛的特别防范,
王明、博古、张闻天均未能进入军委。派任王稼祥为军委副主席,至少在
名义上,使王稼祥排名在毛泽东、朱德、周恩来之後,成为中共军队第四
号人物,表明了毛当时对王稼祥的特殊感激和信任。
但是,
王稼祥担任中央军委副主席只是毛泽东给予的一种酬劳性质的
安排,并不表明王稼祥拥有实际的军权。从未独当一面指挥过军队、在军
内毫无根基的王稼祥,虽然足以使毛完全放心,但毛仍然留有後手。王稼
祥在军委的职责主要集中于八路军的政治教育方面,并不能过问军中高级
干部的任免事项。王稼祥只是在毛需要借助党的力量,弹压他所认为的军
中分离倾向时才派上用场。1941 年 7 月 1 日,在毛的直接授意和指导下,王
稼祥与王若飞合作,起草了针对彭德怀等军中将领的〈关于增强党性的决
定〉。从这个意义上说,身为军委副主席的王稼祥,只是承办毛书案工作
的一名高级幕僚而已。
王稼祥政治上的顺境在 1942 年结束,
从此,
王稼祥开始走下坡路。
1941
年 9 月後,遵照毛泽东的意见,任弼时在中央办公厅设立了国际问题研究
室、政策研究室等机构。设置这类研究机构,毛有其独特的运思。毛的最
主要的目的是将一批党的高级干部打发进这类机构,将他们原先担任的职
务予以冻结和事实上的终止,同时,切断他们与党内外群众的联系,诱使
这批老干部在「研究」的过程中,逐步反省自己的「路线错误」。对于毛
的这个计谋,王稼祥似乎有所觉察。虽然毛亲自动员王稼祥担任国际问题
研究室主任,并出席了研究室成立大会,但是王稼祥对这个只有三个成员
的国际问题研究室,丝毫「不感兴趣」。由于王稼祥的消极,这个研究室
的工作基本处于停顿状态。
\footnote{师哲:
《在历史巨人身边——师哲回忆录》
,页 165. }

属于国际派另一个重要人物的凯丰(何克全),其境遇在 1941 年居然
发生了柳暗花明的转机。身为政治局候补委员的凯丰,由于在遵义会议上
曾发言支持博古,在长征结束後主动与毛接近。1938 至 1940 年,凯丰先後
随同王明、周恩来在长江局和南方局担任重要职务, 1940 年 11 月,奉毛的
命令,与博古一同返回延安。对如何安排凯丰的工作,毛深有考虑。一方
面,为了离间、分化国际派的关系;另一方面,也是作为对凯丰在长江局
工作期间未积极参加王明「闹独立性」活动的一种奖赏,\footnote{有资料显示,凯丰在  1938 年武汉工作期间不同意王明与延安对立,尽管在长江局与延安的电报中,均有凯丰的
署名。} 1941 年底,毛
任命凯丰接任张闻天,担任中央宣传部代部长(在这之前,凯丰被任命为
中宣部副部长)。毛一箭双雕,既名正言顺罢免了张闻天的中宣部部长的
职务,又赢得了凯丰的感激,同时也避免将凯丰安置在更重要的工作岗位
上。

在毛泽东的巨大压力和分化瓦解之下,国际派更加四分五裂、溃不成
军。王明等为了保住在党内的地位,不惜互相攻讦,争相向毛讨好。
1940 年 11 月 20 日,王明在延安《共产党人》杂志第 12 期发表《论马列
主义决定策略的几个基本问题》一文,该文在歌颂毛泽东革命策略思想的
同时,把批评的矛头指向了博古。王明不点名地批评了在博古领导下的中
共在三十年代中期所犯的错误,声称「苏维埃革命後半期,我们不能利用
反动统治阶级各派别及各种军事、政治力量的许多矛盾和冲突,以利于苏
维埃革命的发展」。固然,王明的上述看法并非首次发表,早在 1934 年王
明就有类似的意见,但当 1940 年博古正面临毛的巨大压力时,王明再次重
申这类批评,无疑具有与博古划清界限、着意摆脱干系的明显意图。对于
王明此番表演,毛只会暗中称好。因为至此以前,毛还没有充分的把握,
公开批判苏维埃後期的错误。王明对博古的攻击,使毛看到了国际派即将
土崩瓦解的景象。

和王明竭力向毛泽东靠拢有所不同,王稼祥自恃有功于毛,早就以为
自己已是「主席」的人马了。1938 年後,王稼祥更是刻意与王明、博古、
张闻天等拉开距离,以示自己与王明毫无瓜葛。王稼祥几乎断绝了与王明
等的一切私人来往。王稼祥自莫斯科返回延安後,与毛泽东的故旧、原长
沙周南女校校长朱剑凡之女、担任中央政治局保健医生的朱仲丽结婚,使
与毛的来往又增加了一个渠道。王稼祥并在中国女子大学借调教员等枝节
问题上,「抵制」王明的「错误」,以显示自己立场坚定,爱憎分明。
\footnote{参见朱仲丽:
《黎明与晚霞》
,页 326. 
}

看似淡泊名利,恂恂有君子风度的张闻天其实早已和王明、博古分道
扬镳。张闻天与博古有宿怨,在瑞金时期的最後阶段,博古曾公开批评过
张闻天主持的人民委员会的「文牍主义」。1934 年春夏,博古还排挤张闻
天,将其派往闽赣几个县巡视,不让张闻天参加最高决策机构「三人团」,
博、张关系早在长征之前就已出现严重裂痕。\footnote{张闻天在延安整风笔记上写道,六届五中全会以後,派他去担任人民委员会的工作,对于博古等人来说,是一
箭双雕的妙计:
「一方面可以把我从中央排挤出去,另一方面又可把毛泽东同志从中央政府中排挤出去。
」参见张闻天:
〈从福建事变到遵义会议〉
(1943 年 12 月 16 日)
,载《文献与研究》
,1985 年第 1 期。
}

张闻天对王明更是抱有强烈不满。王明返国後,为了打击张闻天,曾
散播「张闻天在莫斯科中山大学任支部书记期间,其所在支部党员都是托
派」的流言。这一切都决定了当毛泽东向王明等发起进攻时,张闻天将只
求自保,绝不会助王明、博古一臂之力。

在原国际派中,表现出最大团结意愿的只有博古一人。遵义会议後,
博古一改过去骄横、咄咄逼人的态度,对所有的同事都表示善意和尊重。
随着毛泽东权势的急剧膨胀,博古天真地希望原国际派人马能重新聚合起
来,对毛加以一定的制衡。1937 年,博古急切地盼望王明能早日返国,以
为王明能担负起这个领头的使命。1937 年 9 月初,博古作为中共代表前往
南京同国民党谈判,在南京期间,他还十分关心打探有关王明在苏联的消
息。10 月 21 日,博古致电张闻天,告知苏联《少共真理报》(即《共青团
真理报》)刊有王明一论文的消息,要求延安将是否收到王明文章的消息
迅速转告他,「以便译介」。\footnote{ 参见《抗战初期的八路军驻南京办事处》
,页 29.}但是王明返国後的表现却令博古失望。虽
然博古在武汉长江局期间和王明相处堪称愉快,可是只要一涉及过去的历
史问题,王明马上就表现出一副与己无关、自己一贯正确的姿态,使博古
对昔日的同事彻底寒了心,于是只盼望能长时间留在重庆工作。博古已预
感到自己迟早将成为昔日朋友献给毛泽东祭坛上的一只羔羊。

国际派大将所暗存的侥幸和自保的心理,被毛泽东一一看在眼里,毛
充分利用他们之间的各种芥蒂和矛盾,施用区别对待、各个击破的策略,
将王明等分别孤立在各个被动的地位。当王明等正在为与毛关系的改善而
暗自庆幸时,毛泽东已将他们引人早已布设好的包围圈,正待一举「歼灭」
之。

\section{进退失据的周恩来}
1938 年 11 月中共六届六中全会闭幕後不久,
周恩来看到毛泽东已获莫
斯科的「承认」,随即开始调整与毛的关系,其最重要的步骤就是与昔日
关系紧密的王明迅速拉开距离。

1939 年 6 月,周恩来离开重庆返回延安,准备出席预计于 7 至 8 月召开
的讨论国共关系的中央政治局会议。7 月 7 日,重庆《新华日报》为纪念抗
战两周年发表中共领导人撰写的文章,毛泽东、周恩来、张闻天、王稼祥、
刘少奇、博古、凯丰、董必武、吴玉章、叶剑英、邓颖超等皆报上有文,
唯独缺少王明的文章。《新华日报》不登载王明文章,并非是一般的工作
疏忽。周恩来此时虽不在重庆,直接领导《新华日报》的是南方局宣传部
长凯丰,若无重要领导人的指示或暗示,凯丰是没有理由、也无权力拒绝
在《新华日报》上刊登王明文章的。

1939 年 8 月下旬,周恩来为治疗臂疾,启程离开延安,转道兰州、迪
化前往苏联,于 9 月中旬到达莫斯科。周在苏联居留的半年时间,除了住
院治疗外,周此行的主要使命是向共产国际汇报中国抗战形势及中国党的
工作。周恩来起草了长达数万言的《中国问题备忘录》,于 1940 年 1 月分
发给共产国际各执行委员阅读。1940 年 2 月,共产国际执委会主席团作出
关于中共代表报告的决议,肯定中共的政治路线是正确的。季米特洛夫将
这份决议当面交给了周恩来。

周恩来对共产国际的汇报并没有直接涉及毛泽东与王明的分歧。
但明
显突出了毛在中共党内的作用。周了解共产国际领导人对王明的同情态
度,避免在正式场合批评王明,却在私下谈话中,对王明进行了抨击。季
米特洛夫在与周恩来个别交谈时问到王明回国後的表现及与毛泽东的关
系,周恩来回答,不够好,甚至有一个时期,王明跑到武汉,企图组织自
己的班子。\footnote{师哲:
《在历史巨人身边——师哲回忆录》
,页 141. }在这里,周若无其事,将自己与王明在武汉的密切合作全部
勾去。当季米特洛夫对中共远离工人阶级、以乡村为中心表示担心时,周
的答复则是,中共在农村经过长期斗争的锻炼,有毛泽东领导,完全可以
无产阶级化。\footnote{参见《周恩来年谱》
,页 452;周恩来:
〈关于党的「六大」的研究〉
,载《周恩来选集》
,上卷(北京:人民出版
社,1980 年)
,页 178-79.}周恩来的这番言论十足像一个毛泽东的政策解说员,他以
这种态度向共产国际表明他个人支持毛泽东的立场。

周恩来在莫斯科的又一个重要活动是参加共产国际监察委员会对李
德的审查。1939 年 8 月下旬,李德在华居留七年後随周恩来等同机前往苏
联。同年 12 月,共产国际「根据中国党的意见」,开始了对李德的审查,
参加者有周恩来、毛泽民、刘亚楼以及共产国际监委主席佛罗林。据李德
回忆,周恩来在发言中批评了他 1934 年对闽变的态度以及「短促突击」
的作战策略。但是,在李德与红军的「消极防御」关系问题上,周的态度
却「极其谨慎」。李德认为,这是因为此问题牵扯到周在内战时期的军事
主张。\footnote{奥托·布劳恩(李德)《中国纪事(1932 一 1939)》,页 359-60.}共产国际最後对李德问题作出结论,免予处分,改派与中国无关
的工作。1940 年後,李德被调往莫斯科外国文学出版社,从事苏联文学和
列宁著作的德译工作。以後李德长期在苏联克拉斯诺戈斯克任教,直至
1949 年才返回苏占区——「德意志民主共和国」。

尽管周恩来已经拉开了与王明的关系,翩然来归,毛泽东却对周很不
满意。毛对周的不满既有历史的原因,也有眼下现实的因素。1938 年後,
毛对周的不满主要集中在周对国民党的温和态度,以及周对项英表现出的
宽容。

如前所述,
毛泽东与王明在理论上的重大分歧即是如何评价国民党抗
战,以及中共在抗战阶段应对国民党持何种方针的问题。中共六届六中全
会批准了毛泽东将共产国际加强统一战线的精神与毛自己有关大力发展
中共力量相调和的政治路线,但在如何掌握这两者的关系问题上,却留下
了极具伸缩性的空间。从六届六中全会通过的正式文件中,很难真正看清
中共对国民党的方针、策略已发生了重大转变。中共对国民党的方针基本
取决于中共领袖对瞬息万变的形势随时作出的灵活反应。由于毛泽东在六
届六中全会上大大加强了自己的地位,党的最高决策权已集中于毛个人之
手,这就造成了极为尴尬的局面:负责具体处理与国民党交涉的周恩来,
根据六中全会的精神,继续强调中共与国民党的合作;而毛泽东则拥有对
六中全会的解释权,
在毛的解释下,
六中全会的路线应是反击对国民党
「妥
协」的「右倾投降主义」。

周恩来没有全程出席六届六中全会。1938 年 9 月 30 日,周在六中全会
作完统一战线工作报告後,随即离开延安于 10 月 1 日返抵武汉。周对六中
全会精神的理解主要依据的是 9 月中旬至下旬召开的政治局会议所通过的
基本方针,这次会议实际上是六中全会的预备会议,其主要内容是讨论毛
泽东即将在六中全会上宣读的《论新阶段》政治报告。周恩来就是根据这
次会议和《论新阶段》的精神,在六中全会後,继续强调拥蒋抗日,加强
国共两党合作。

然而,
周恩来为贯彻六中全会精神采取的一些措施并不被毛泽东所欣
赏,在毛的眼中,周恩来在处理与国民党关系上的不少作法都烙有「右倾
投降主义」的印记。在毛的「账簿」上,周恩来为国民党办训练班一事,
就是周犯下的一个严重「错误」。

1938 年 11 月 25 至 28 日,
周恩来和叶剑英在蒋介石主持召开的南岳军事
会议上,就中共为国民党举办西南游击干部训练班一事同国民党方面达成
协议。此项动议出自中共方面,1938 年 10 月 22 日,周恩来陪同抵达武汉的
朱德面见蒋介石时,向蒋正式提出此议。蒋介石对此表示同意,要求中共
提供教官。于是,周恩来担任了该训练班国际问题讲师,叶剑英任副教育
长,中共教官、工作及武装人员共三十多人参加了该训练班的工作。对于
此类「为他人作嫁衣裳」事,毛泽东一向深恶痛绝。尽管周恩来此举最终
得到毛的同意,但对毛而言,这不啻是在外力强迫下,对既成事实的无奈
批准,而依其本性则是极不情愿的。

周恩来的这类「错误」1938 年底达到了高峰。1938 年 12 月 12 日,周
恩来、叶剑英等在给廖承志并报毛泽东、张闻天的电报中,似乎忘记了毛
对国民党一向采取分化瓦解的既定方针,而提出中共应以「大党风度」调
和国民党各派系的矛盾。\footnote{参见《周恩来年谱》
,页 428.}周恩来的这种「抗战至上」的态度 1939-1940
年继续发展。
周在就统战问题与毛的往返电报中,
也是多强调
「击敌和友」
。
\footnote{参见《周恩来年谱》
,页 447、465.}

1940 年 10 月以後,在围绕新四军北移问题突然升温的两党冲突中,周和
毛的态度更是存在根大的区别。与精神高度紧张、提出「作与蒋介石决裂
的准备」的毛截然相反,
周恩来、
博古多次建议中央应作必要的妥协,
「也
让一步」,对国共分裂持相当谨慎的态度。\footnote{参见《周恩来年谱》
,页 472-475、479.}
在处理与国民党的关系上,毛泽东、周恩来的态度差异,并不涉及任
何实质性的路线分歧。毛的强硬与周的稳健都是出于维护中共利益的同一
目标。事实证明,抗战期间,毛泽东在制定对国民党的方针过程中,十分
重视并经常采纳周提出的外柔内刚、留有馀地的主张,毛和周构成的互补
关系,
有效地维护了中共的利益。
尽管周的中和作用对毛的决策极为重要,
然而在毛泽东心目中,周恩来作为一个「调和主义者」的角色也就固定化
了。
毛泽东对周恩来的另一不满是周对项英的态度。
1937 年 12 月政治局会
议决定,新四军受中共中央和长江局的双重领导,在党的关系上,长江局
直接领导东南分局。这样,王明、周恩来和项英就有了比较密切的工作联
系。毛泽东对项英原本就无好感,除了三十年代初期两人在江西结下的深
刻矛盾外,王明返国後,项英与王明关系的接近,更加深了毛对项英的不
信任。1939 年春,项英指令新四军军部秘书、抗战前曾任中共上海剧联书
记的扬帆,写出江青在沪演艺界表现情况的书面材料,署名项英,将材料
发电给延安,此电报明确提出:江青不宜与毛泽东结婚。\footnote{参见张重天:
《共和国第一冤案》
(北京:华艺出版社,1989 年)页 18-20.}项英此举更进
一步增添了毛对项英的反感。1937 年後,毛泽东十分担忧项英因领导新四
军而加强其在党内的地位,对项英明显表现出冷漠和排斥,对项英提出的
一些工作方面的请求,也多加以敷衍和事实上的阻拦。

1938 年春夏之际,项英为新四军组建事,多次致电毛泽东,请求调派
得力的军政干部前往江南。
项英在电报和信中提出,
延安集中甚多的干部,
「应抽调大批派到南方」,项英还提议中央派刘少奇、陈云到南方工作。
在给毛的电报中,项英建议「中央负责同志也应轮流到南方巡视」,强调
此举对「加强对于全国领导,实属必要」。项英的上述言论,在极度敏感
的毛泽东那里,有影射其据守延安、从不前往前线视察的寓意。毛对项英
提出的这类要求仅作一般性的敷衍,只派去周子昆、袁国平、李一氓等少
数高级军政干部和五六十名团连级干部。
毛在覆项英的电报中,
语含轻慢,
声称「如你处不要,则交长江局使用。究竟你处是否要这批干部」?
\footnote{在《毛泽东年谱》中没有记载毛泽东 1938 年 3 月 18 日致项英电,该电报和项英 1938 年 5 月 13 日致中共中央
信,载《抗战初期中共中央长江局》
,页 181、236.}对项
英毫不掩饰自己的厌烦。但是在一段时间内,毛泽东慑于项英在党内、
军内所享有的崇高声望,以及项英一身凛然的「共产主义清教徒」的气质
(项英是少数反对任何特殊照顾,在生活上坚决与普通战士同甘共苦的中
共领袖),对项英也无可奈何,只能在其背後多方加以掣肘。1939 年後,
随着毛政治地位的巩固,毛频频开始了对项英的指责,并采取一系列步骤
削夺项英的权力。对于毛泽东的狭隘胸襟和毛、项之间的纠葛,周恩来完
全清楚,他既附和毛对项英的部分批评,又试图维护项英的威信,在关键
时刻助项英一臂之力。

1939 年 8 月,中央政治局在延安连续举行会议,听取周恩来就统战问
题和南方党与军队工作情况作的政治报告。项英未出席这次会议,由张鼎
丞代表东南局和新四军出席会议。周恩来在谈到新四军问题时,高度评价
了项英的领导作用。8 月 25 日,会议由周恩来而非毛泽东作结论,周指出,
项英领导的东南局取得了许多成绩\footnote{在中共党史编纂学中,长期宣称这次政治局会议是由毛泽东作会议总结,参见中共中央党史研究室:
《中共党史
大事年表》
(北京:人民出版社,1987 年)
,直到八十年代末、九十年代初《周恩来传(1898-1949)
》和《周恩来年
谱(1898-1949)
》出版,才恢复了历史真相。}。

时隔一年,周恩来 1940 年 6 月 17 日举行的讨论新四军工作的南方局
常委会上,再次肯定新四军执行了中央为新四军制定的「向北发展,向东
作战和向南巩固」的方针。在这之前,1940 年 5 月 4 日,毛泽东在给东南局
发出的指示中,对项英领导的新四军工作未尽展开,实力发展不大提出批
评。项英感到委屈,于 5 月 9 日、12 日致电延安,表示同意中央路线、方针、
策略,但希望中央指明他的错误性质和具体内容,并公开宣布撤销他的职
务。周在 6 月 17 日南方局常委会的发言中虽然对项英提出了委婉的批评,
指出新四军没能利用有利时机大胆发展,但仍然强调「一年来东南局的工
作在项英的领导下是正确的」。周恩来并面告新四军政治部主任袁国平
\footnote{参见《周恩来年谱》
,页 457-58.}和
东南局副书记饶漱石,中央仍以项英为东南局书记。

1940 年 8 月 4 日,周恩来出席在延安召开的中央政治局会议,周一方面
批评新四军未积极执行中央向北发展的方针,在与国民党的谈判中也未坚
持党的立场;但同时,周再次提出,东南局仍以项英为书记。周恩来的这
番表态具有重要的意义。1940 年,毛泽东在远离江南的延安,隔着千山万
水,凭着电报对新四军实施具体指导。惯于为自己留退路,以显示自己在
任何情况下都料事如神的毛泽东,在给项英的电报指示中,前後矛盾,变
化无端。毛既要项英尽速作好北渡转移的准备,又要项英维持「拖」的局
面,来配合延安部署对国民党的谈判斗争。毛忽而判断国民党大军进攻在
即,要求皖南部队立即分批移动;忽而又命项英向重庆要开拔费,要饷弹,
「再拖一两个月」,\footnote{参见王辅一:
《项英传》
(北京:中共党史出版社)
,页 435-39.}致使项英无所适从,始终难以最後确定新四军北上
的时间和路线。而毛泽东只看结果,不管过程,有理三扁担,无理扁担三,
在延安对项英吹毛求疵,百般指责。中央政治局的大多数成员也随着毛泽
东的调子,对项英多方挑剔。江南新四军所处的环境与华北八路军完全不
同,
国民党在江南留有强大的兵力,
新四军的活动与发展受到很大的限制,
加之项英的领导确有缺失,对国民党顾忌较多,这些都严重削弱了项英在
党内的地位。1940 年 11 月中旬,延安决定成立华中新四军、八路军总指挥
部,令项英在部队移动安排就绪後,返回延安参加「七大」。在项英地位
岌岌可危的时刻只有周恩来能体谅项英进退两难的苦衷,强调项英在东南
局和新四军的领导作用,竭力维护项英在新四军中的威信。

周恩来在处理与国民党关系上的稳健态度及对项英的善意和宽容,
都
被毛看在眼里,毛只是需要选择适当的时机向周恩来表明自己的不满。
1939 年 1 月 5 日,在没有任何特殊迹象的情况下,毛泽东公然推出一项
旨在削弱周恩来权力的行动。是日,毛隐身其後、以延安中央书记处的名
义,致电在重庆的周恩来,无端改 1938 年 9 月 26 日中央政治局关于设立
南方局的决定,提议将华南及西南各省合并成立的中央局(即南方局)易
名为西南局。\footnote{《中央书记处会议记录》
,载南方局党史资料——党的建设》
(重庆:重庆出版社,1990 年)
,页 3.}此决定如果一经最後确定、必然大大降低南方局作为中共
在延安之外领导南中国党最大的派出机关的地位,而将南方局等同于地位
较低的北方局(杨尚昆)、中原局(刘少奇)等其它派出机构。周恩来立
即对延安的这封电报作出回应,1 月 7 日,周与在重庆的另两位政治局委员
博古、凯丰共同联名,覆电中央书记处,强调新设立的中央局「以南方局
名称为好」。只是在周恩来、博古、凯丰的强烈要求下,毛泽东才作出让
步,于 1 月 13 覆电,同意仍沿用南方局的名称\footnote{《毛泽东年谱》
,中卷,页 102-103.}。

对于周恩来直接领导的南方局的工作,毛泽东的态度也是模棱两可
的。1939 年 8 月 24 日,毛在中央政治局会议的发言中,一方面称赞南方工
作「做得好,各省工作有成绩」,并指出「这是在恩来领导下的成绩」。
但是,毛紧接着话锋一转,开始尖锐指责南方「党不巩固」,「群众运动
不深入」,「统战没有中层阶级更大的发展」。\footnote{参见《周恩来年谱》
,页 448.}周恩来对于毛的批评迅
速作出反应,在当天周所作的会议结论中,周承认南方局在巩固党、利用
合法机会做群众工作不多,周还检讨了在统战中「偏重了联蒋」,「对中
产阶级团结不够」\footnote{参见《周恩来年谱》
,页 448.}。

和周恩来小心翼翼、努力协调与毛泽东的关系全然不同,毛则不时示
周以颜色,让周明确地体会到两人的关系已今非昔比。1940 年 3 月 26 日,
周恩来、任弼时、蔡畅等从莫斯科返抵延安,尽管在抵达延安的前一天,
周恩来在甘泉就和毛通了电话,但是当周抵达时,毛却仍在睡觉,「只有
李富春前来迎接,而且主要还是接他的夫人蔡畅」。\footnote{师哲:
《在历史巨人身边——师哲回忆录》
,页 331-32.}两年後,毛却亲自
迎接从莫斯科返延的地位远低于周恩来的林彪。这一回毛不睡觉了,他挽
着林彪的手步入杨家岭他居住的窑洞。\footnote{师哲:
《在历史巨人身边——师哲回忆录》
,页 154.}毛对周、林的亲疏之别,一目了
然。尽管毛有意怠慢周恩来,周却处处关心、体贴毛。1940 年 2 月下旬,
周恩来一行自莫斯科启程返国时,共产国际为周等准备了两大箱西式食品
和烟酒,以备旅途食用。周却提议旅途中改用中餐,将这两箱洋式点心万
里迢迢带回延安。返回延安後,周又亲自检查食品箱,命令将其送给毛泽
东夫妇享用。\footnote{师哲:
《在历史巨人身边——师哲回忆录》
,页 154.}毛固然对周之细心周到心领神会,但对周的态度始终未越
过公事公办的界限,这就使周永远感到头顶上有一个紧箍咒存在,至死不
敢稍有懈怠。

1938 年後,毛泽东准确地利用了周恩来的性格特点,对待周,有扬有
抑,有紧有松,成功地瓦解了周恩来与王明的联盟,并重新理顺了毛、周
关系。对于毛而言,周恩来是须臾不可缺少的股肱,周的崇高声望、非凡
才干和对中共事业的献身精神,使周成为任何人都难以取代的人物。更重
要的是,毛了解周没有争当中共领袖的权欲,对毛毫无威胁。周既非王明
集团成员,又与苏联和共产国际关系深厚,是可以代表毛与斯大林打交道
的最合适人选,这一切都可使毛继续重用周。但是,毛绝非是一个能轻易
忘记过去的人,周 1931 至 1935、1937 至 1938 年两次与王明联手孤立毛的
往事,使毛刻骨铭心,难以忘怀。周恩来身上散发着的那种儒雅、高贵的
气质,也与毛一身的「山大王」气质格格不入。因此,毛时不时就要借机
敲打一下周恩来,并执意提拔刘少奇作中共第二号人物,将刘作为制衡周
恩来的工具,以防周恩来在党内形势发生突变的关头,再一次弃毛而去。

为了打下中共的江山,毛泽东必须借重周恩来的才干和忠诚;周恩来
也在与毛的长期共事中,发现毛具有自己所缺乏的可作「人主」的霸气,
断定凭藉这股霸气,毛能够一统江山。因此,周以其独有的机敏和灵活,
很快适应了与毛的合作关系。在这种关系中,毛居于主导地位,周则安于
作一个辅佐性的角色。由于毛、周都重视对方对自己的价值,各守分际,
因而两人的合作虽时兴波澜但仍弥久而不坠。只是三十年代末至四十年代
初,毛、周合作的关系还刚刚建立,毛对周还有太多的怨气没能尽兴宣泄,
故而毛有意让周体会上下不落地的虚空,使周恩来在很长时间内搞不明白
自己在中共核心层中究竟处于何种地位。这种由毛一手制造的对周恩来的
慢性精神虐待长达七年,直 1945 年中共七大召开才告一段落,使周恩来
进退失据,有苦难言,为自己昔日的「过失」付出了沉重的代价。
\section{
初战告捷:1941 年 9 月政治局扩大会议}
从中共六届六中全会後,毛泽东为在政治上彻底摧毁王明、博古等国
际派,小心翼翼,稳扎稳打,将王明等成功地加以分隔,逐步缩小包围圈,
已取得了对王明、
博古等的绝对优势。
经过三年的精心策划和细致的准备,
毛泽东 1941 年 9 月召开政治局扩大会议,正式向王明等下战书。

毛泽东向王明等挂牌的行动发生 1941 年 9 月 10 日至 10 月 22 日于延安
召开的政治局扩大会议上。据当时担任会议记录的胡乔木的记载,这次会
议「实际上只在 9 月 10 日、11 日、12 日、29 日和 10 月 22 日开了五次会」。
参加者有在延安的政治局委员毛泽东、任弼时、王明、博古、康生、陈云、
张闻天、王稼祥、朱德、邓发、凯丰等十一人,政治局委员周恩来、彭德
怀、刘少奇缺席会议。毛泽东为了确保自己在会议上取得完全的优势,批
准自己的支持者李富春、
高岗、
陈伯达、
彭真以及杨尚昆、
罗迈
(李维汉)
、
林伯渠、王若飞、叶剑英列席会议。毛并指定王首道、胡乔木担任会议记
录。

1941 年 9 月政治局扩大会议是在严格保密的状态下进行的,以至外界
长期难以窥其真相。
只是到了八十年代中期,
为反击王明对毛泽东的攻击,
\footnote{八十年代初,王明的《中共五十年》虽由内部出版,控制发行,但该书内容仍有所扩散。
}
中共党史研究部门才有选择性地披露了这次会议的若干资料。直到 1994
年《胡乔木回忆毛泽东》一书出版,才使得外界对历史上这次著名会议的
实情有了初步的了解。

毛泽东为这次会议确立的目标是具体和明确的,
这就是重新讨论 1931-1935 年中共的历史,从根本上摧毁王明、博古等国际派的政治合法性基
础,逼王明、博古彻底下台。

1941 年 9 月 10 日,毛泽东在会议上作基调发言,以谈「主观主义」对
党的危害为突破口,迅速切人到「苏维埃後期『左倾』机会主义」的命题。
毛严厉指责苏维埃後期的主观主义,自称为「国际路线」,穿上马克思主
义的外衣,其实是假马克思主义。毛说,1933 年中央苏区反「邓毛谢古」
实际上是「指鸡骂狗」,「在苏维埃运动後期,五中全会精神......这些都
比立三路线的『左倾』在政治上表现得更完备」。为了防止王明等抬出莫
斯科作挡箭牌,毛主动出击,采用釜底抽薪的办法,声称国际派师承的并
非是斯大林正宗,而是布哈林、季诺维也夫。毛说,主观主义来源之一即
是「外国的传统,过去共产国际如布哈林、季诺维也夫等人的影响」,一
下子就将王明等与已被斯大林消灭的「人民公敌」捆在了一起,把他们一
同打进「假马克思主义」之列。毛在集中攻击国际派的时候,没有忘了把
周恩来捎上,他在讲话中批评了周恩来领导的苏区中央局 1932 年 5 月 11
日作出的《关于领导和反对帝国主义进攻苏联瓜分中国与扩大民族革命战
争运动周的决议》,「是完全主观主义的东西」(当时博古、张闻天仍在
上海,苏区中央局书记为周恩来)。\footnote{毛泽东:
〈反对主观主义和宗派主义〉
(1941 年 9 月 10 日)
,载《文献和研究》
,198S 年第 1 期;另参见《毛泽
东文集》
,第 2 卷(北京:人民出版社,1993 年)
,页 372-75.}

毛泽东 1941 年 9 月主动挑起党的历史问题的讨论,是经由他精心策
划,
不断试探,
并判断在核心层已不会遭到多数人反对的情况下才进行的。
 1935 年 1 月召开的遵义会议上,毛为了夺取军权,无心在有关党的政治
路线的问题上纠缠。但是到 1937 年 6 月,当毛已初步站稳脚跟,他却支
持刘少奇向张闻天挑战,企图在对过去党的政治路线的评价问题上,打开
一个缺口。只是在遭到多数人的激烈反对後,毛才被迫退却。1938 年毛又
想重新挑起对党的历史问题的讨论,被共产国际明令阻拦,但毛不甘心就
此罢休,再次放出试探气球。10 月,毛在〈论新阶段〉政治报告中,有意
含混其辞,避免直接评 1931-1935 年党的政治路线,甚至重复〈遵义会
议决议〉,再次肯定遵义会议所纠正的党的错误,「并非是党的总路线的
错误,而是执行当时总路线所犯的战争策略与战争方式上的严重原则错
误」。但是毛同时宣称,这种错误具有「左倾机会主义性质」,为日後推
翻原有结论预埋了伏笔。1940 年 12 月,毛再次出击,在 12 月 4 日的政治局
会议上提出要总结苏维埃运动後期的政策错误问题。毛说,遵义会议的决
议须有些修改,决议只说那时是军事上的错误,没有说是路线上的错误,
实际上是路线上的错误。然而毛的这种说法又一次受到张闻天等的反对,
毛只能被迫妥协,在 12 月 25 日为中共中央起草的关于时局与政策的指示
中,没有提出「路线错误」的正式概念。半年多後,形势向有利于毛的方
面发生了重大的转变,1941 年 6 月 22 日,苏德战争爆发,毛已无後顾之忧。
毛编辑的供中共高级领导干部「对号人座」的《六大以来》已经发至各政
治局委员和其他负责干部。毛对党的组织、军队、保安力量的控制也达到
坚不可摧的地步,现在他再无需违心地附和众议了,于是义无反顾,破门
而出。

毛泽东是一位高明的战略家,深谙开展党内斗争也需像军事作战那
样,要讲究以虚就实、声东击西等一套战法。毛要师出有名,更要撇清个
人争权的色彩。
在九月政治局会议上,
毛将整肃国际派的意图暂时模糊化,
他说「要实行学制的改革,把过去的一套彻底打碎。......以《联共党史》
作为学习的中心......」。在他列举的「宗派主义」的各种表现中,首先是
「首长至上」。(「在延安,首长才吃得开」),「排挤非党干部」(「许
多科学家、文学家都被人看不起」)。明明毛要讨伐的是王明等的「宗派
主义」,却故意施用障眼法,将「宗派主义」的外延尽量扩大。然後,毛
宣布要「实行两条路线的斗争,反对主观主义和宗派主义」\footnote{毛泽东:
〈反对主观主义和宗派主义〉
(1941 年 9 月 10 日)
,载《文献和研究》
,198S 年第 1 期;另参见《毛泽
东文集》
,第 2 卷(北京:人民出版社,1993 年)
,页 372-75.}。

尽管毛泽东说得冠冕堂皇,但是与会者个个都明白他所指的「主观主
义」和「宗派主义」究竟是什么。王明作为毛泽东的头号政治对手,对毛
的意图洞若观火,可是他的嘴却被自己和毛泽东双重地封死。王明多年前
就曾批评「苏维埃後期左的错误」,现在毛只不过是重复王明昔日的指责,
王明已无任何理由对毛的批评提出异议。因此当毛抨击「苏维埃後期左倾
机会主义」
时,
王明明知其中隐藏凶兆 (「左的错误」与「左倾机会主义」
,
在共产党语汇中有质的区别),却也无可奈何。王明表示赞成毛的报告,
承认苏维埃後期的错误是路线错误。王明说,他对博古、张闻天在中央苏
区的政策是「不同意的」,对五中全会提出的「苏维埃与殖民地两条道路
决战」的主张,他也是「不同意的」。王明强调自己在莫斯科期间,就曾
反对过博古的错误,博古应是「苏维埃後期最主要的错误负责者」
\footnote{参见《胡乔木回忆毛泽东》
,页 199;另参见《任弼时传》
,页 470-71;中央档案馆党史资料研究室:
〈延安整风
中的王明——兼驳王明的《中共五十年》,载《党史通讯》
〉
,1984 年第 7 期。
}。王
明在发言中还爆出一个大冷门——他揭发博古道:1931 年 9 月中央临时政
治局成立时曾有过约定,将来到政治局委员多的地方要将权力交出来,因
为临时中央负责人博古、张闻天皆非中央委员。\footnote{莫斯科当代历史文献研究中心档案,全宗号 495,目录号 74,卷宗号 333;转引自杨奎松:
〈毛泽东发动延安整
风的台前幕後〉
,载《近代史研究》
,1998 年第 4 期。页 14-15.}而事实上博古、张闻天
抵达中央苏区後,并没有正式传达这个意见。王明所述基本符合事实,但
是他自己对此也有责任,因为王明并没有从莫斯科打电报来纠正这个问
题。再说,王明这番揭发也把周恩来、陈云、康生牵扯了进来,因为临时
政治局成立时,周恩来仍在上海,陈云、康生均为临时中央政治局成员,
周恩来等有可能也与闻此事。博古、张闻天到达苏区与苏区中央局合并成
立中共中央局时,没有资料证明,周恩来曾向博古等提过这个问题。当然,
博古、张闻天、周恩来完全可以为自己辩护 1933 年在中央苏区成立的只
是中共中央局,当时在上海也存在一个中共中央局。最後,1934 年 1 月六
届五中全会成立的政治局的名单也得到了共产国际的批准。在博古落难之
际,王明的这些话无疑是对博古落井下石,同时也进一步把事情搞复杂化
了。诚然,对于王明而言,是没有什么「朋友」概念的,只要能保护自己,
随时可以「翻脸不认人」。

博古在会议上处于被告者的地位,「墙倒众人推」,没有任何人向他
援之以手。在一片指责声中,博古两次发言作了检讨。他承认自己「完全
没有实际经验,在苏联学的是德波林主义的哲学教条(德波林是布哈林派
的哲学家——引者注),又搬运了一些苏联社会主义建设的教条和西欧党
的经验到中国来」。博古对王明的「揭发」也作出了反应,他没有为自已
辩护,相反承认临时中央进入中央苏区後没有交权「确有纂位之嫌」,但
又认为对此共产国际和中共驻莫斯科代表团也有责任,\footnote{莫斯科当代历史文献研究中心档案,全宗号 495,目录号 74,卷宗号 333;转引自杨奎松:
〈毛泽东发动延安整
风的台前幕後〉
,载《近代史研究》
,1998 年第 4 期。页 14-15.}最後,博古表示
他有勇气研究自己过去的错误,希望在大家帮助下逐渐克服\footnote{参见《胡乔木回忆延安整风》
(上)
,载《党的文献》
,1994 年第 1 期。
}。

张闻天在会议上的表现在原国际派中是最突出的。
他是被批判的主要
对象之一,毛泽东在 9 月 10 日讲话中,多次把矛头指向张闻天——在国际
派中,张闻天力量最为虚弱,毛先捡软的柿子捏。毛尖锐抨击张闻天负责
的干部学习活动是「同实事求是的马克思主义相对抗的」。毛嘲弄张闻天
道,「对于理论脱离实际的人,提议取消他的『理论家』资格」。
\footnote{毛泽东:
〈反对主观主义和宗派主义〉
(1941 年 9 月 10 日)
,载《文献和研究》
,198S 年第 1 期;另参见《毛泽
东文集》
,第 2 卷(北京:人民出版社,1993 年)
,页 372-75.}面对
毛的挖苦讽刺,张闻天第一个缴械投降。他在发言中除了表示拥护毛的报
告外,还不惜自我贬损,称「过去国际把我们一批没有做过实际工作的干
部,提到中央机关来,这就给党的事业带来很大损失」,表示「现在要补
课」。9 月 29 日,张闻天再次发言,这一次他干脆把王明也拖下水,张闻
天仿效博古,承认临时中央到苏区「确有纂位问题」,但马上强调王明对
此也有关系,因为「五中全会的名单也是国际批准的,这些事情王明当时
为什么不起作用」?\footnote{莫斯科当代历史文献研究中心档案,全宗号 495,目录号 74,卷宗号 333;转引自杨奎松:
〈毛泽东发动延安整
风的台前幕後〉
,载《近代史研究》
,1998 年第 4 期。页 14-15.}在这天的会议发言中,张闻天一改过去固守「中央
政治路线是正确的」
态度,
宣布同意毛对苏维埃後期党的路线性质的评价,
承认「当时的路线是错误的」,并且主动表示,
「我是主要的负责者之一,
应当承认错误,特别在宣传错误政策上我应负更多的责任」。
\footnote{参见《胡乔木回忆毛泽东》
,页 195;另参见程中原:
《张闻天传》
,页 481-83.}张闻天的
这番表白究竟是出于真心,抑或是迫于毛的巨大压力的违心之论,看来两
者兼而有之。张闻天在和毛的长期共事中,对毛的性格体会甚深,了解毛
不做则已,一干则不择手段,不达目的,誓不罢休。在另一方面,也不能
排除张闻天的大转弯与其对毛认识的加深有密切联系。毛的文韬武略给张
闻天以深刻的印象,也许张闻天已从内心对毛「服气」,正是鉴于内力、
外力两方面因素的作用,张闻天决定向毛主动认输,以求早日解脱。

王稼祥、凯丰、邓发等几位政治局候补委员,虽然都知道毛泽东迟早
要清算过去的历史,但是乍听到毛的严厉指控,仍不免受到强烈的震动。
因为,毛在讲话中 1934 年中共六届五中全会的否定,直接威胁到他们在
那次会议上当选政治局候补委员的地位。

在这三人中,王稼祥早已归顺毛,因此王稼祥不可能转而为「苏维埃後期错误」辩护。在
9 月 10 日的会议上,王稼祥检讨自己,「我也实际工作经验很少,同样在
莫斯科学习一些理论,虽也学了一些列宁、斯大林理论,但学得多的是德
波林、布哈林的机械论。学了这些东西害多益少」\footnote{参见徐则浩:
《王稼祥传》
(北京:当代中国出版社,1996 年)
,页 361;另参见《胡乔木回忆毛泽东》
,页 196.}。

在九月政治局会议上,凯丰的心理压力也许是与会者中最重的一位。
遵义会议上反对毛泽东复出的旧事,足以使凯丰心惊肉跳,後怕不已。在
会议紧张的气氛下,凯丰除了作检讨外,别无其它选择。

邓发不属于国际派,也不在毛泽东的「亲密战友」之列,此时不握任
何实权,他在会上也作了检讨发言。
在国际派各员大将和邓发相继检讨的同时,
其他政治局委员也陆续发
言,除了表示拥护毛泽东的报告外,还采取主动,纷纷作出自我批评。在
毛的设想中,这次会议不仅要解决国际派的问题,即使那些现在属于毛营
垒的人也应对他们过去程度不同的「错误」表明态度,因为只有这样才能
显示在中共党内唯有毛泽东才是唯一正确的,也只有通过这种形式才能确
立毛在党内至高无上的地位。

9 月 12 日,毛泽东的重要盟友任弼时在会议上作检讨发言,承认自己
当年
「反对所谓
『狭隘经验主义』
是错误的」并说自己
,
「毫无军事知识」
,
却在当年中央苏区召开的南雄会议上对毛所坚持的苏区内部也能打仗的
正确主张不以为然\footnote{参见中共中央文献研究室编:
《任弼时传》
,页 470;另参见《胡乔木回忆毛泽东》
,页 197.}。9 月 11 日、29 日,陈云和康生分别就过去白区工作的
「错误」在会议上作检讨发言。陈云提出,刘少奇同志是代表了过去十年
来的白区工作的正确路线,有些干部位置摆得不适当,要正位,如刘少奇
同志将来的地位要提高。\footnote{《胡乔木回忆毛泽东》
,页 197.}康生在发言中以当年王明副手的身分指责王明
实际上与博古有着一样的思想,他还特别指出王明回国後也犯了错误。对
于他自己有何错误,康生说,今天看起来是少奇的对,自己当时反对少奇,
把少奇完全看成机会主义者,一是由于自己的主观,二是听国际说少奇是
机会主义。康生还说,主观主义的错误路线把白区工作弄光了。如果中央
那时是刘少奇负责,情况将是另一样\footnote{参见杨奎松:
〈毛泽东发动延安整风的台前幕後〉
,载《近代史研究》
,1998 年第 4 期,页 15;另参见《胡乔木
回忆毛泽东》
,页 198.}。

陈云、康生的发言抬出了刘少奇,把刘少奇看作是中共白区工作正确
路线的代表,这在当时和毛泽东的战略目的并无冲突。毛虽然想独占「唯
一正确」的光环,但毛早有将刘少奇扶持为中共第二号人物的意图,此时
毛也考虑到中共尚有白区工作这一块,于是毛接受了「刘少奇是白区工作
正确路线代表」的说法。

参加九月政治局会议的领导人已经就他们过去的错误作出初步检讨,
迈出了「思想觉悟」的第一步。如何迈出下一步,似乎无需毛泽东的点拨,
与会者几乎个个都无师自通,迅速而自然地就跨越了这一步:他们开始纷
纷当着毛的面对毛进行赞颂。王稼祥说:「毛主席代表了唯物辩证法,在
白区刘少奇同志是代表了唯物辩证法」。\footnote{《王稼祥选集》
(北京:人民出版社,1989 年)
,页 326.}陈云说,「毛主席是中国革命
的旗帜」。罗迈说,「毛主席——创造的马克思主义者之模范、典型」。
叶剑英说,「毛主席由实践到理论,这是我们应该学习的」。
\footnote{参见徐则浩:
《王稼祥传》
,页 361.}还有人当
面向毛提议,「多作论著,教育全党」。\footnote{参见华世俊、胡育民:
《延安整风始末》
(上海:上海人民出版社,1985 年)
,页 16.} 在这类对毛的赞颂辞中,只有
任弼时的调门较低,他说:「我党的毛主席、刘少奇同志能根据实际情形
来工作,所以犯主观主义少些。」\footnote{《胡乔木回忆毛泽东》
,页 198.} 

与会者的检讨和表态使会议出现了毛泽东所盼望的一边倒的局面,
可
是仅仅限于这一步仍是不够的。熟谙共产党斗争哲学的毛明白,若不将王
明盯住,会议散後,王明仍有可能兴风作浪,只有「擒贼先擒王」,将火
烧到王明的身上,才算真正断了国际派的後路。

如前所述,在九月政治局会议上,博古、张闻天是被批判的主角,对
王明几乎没有正面涉及。这段时间,毛泽东两次找王明交谈
\footnote{《毛泽东年谱》
,中卷,页 330.},试图说
服王明承认「错误」。毛此时尚没有明确提及王明在「苏维埃後期的错
误」,而是集中于对王明返国後「错误」的批评。毛提出王明 1937 年 12
月会议上的报告和武汉时期的工作都有错误,强调王明的错误为:政治上
犯有原则性的错误,组织上闹独立性的错误。但是,毛与王明谈话并无结
果,双方实际处于胶着的状态,直到 10 月初,一个偶然的事件触发了王明
向毛泽东作出正式回击。

1941 年 10 月 4 日,毛泽东将一份季米特洛夫给中共中央查询中国抗战
情况的电报交给王明,提议王明先作考虑,然後集体讨论如何给共产国际
回电。季米特洛夫在这封电报中提出十五个问题要中共回答,其中有中共
准备采取什么措施在德国法西斯继续进攻苏联的情况下,在中日战场打击
日军,从而使日本不可能开辟第二战场打击苏联;中共对抗日民族统一战
线的团结究竟采取什么立场等。\footnote{王明:
《中共五十年》
,页 37;另参见中共中央文献研究室编:
《任弼时传》
,页 472.}王明看到这封电报,如获至宝,觉得反
击的机会到来了。

10 月 7 日晚,毛泽东偕同任弼时、王稼祥到王明住所,共同商讨回电
季米特洛夫的问题。王明在与毛的谈话中,就季米特洛夫来电所提有关中
共统战政策等问题,提出了自己的看法。王明认为,「我党已处于孤立,
与日蒋两面战争,无同盟者,国共对立。原因何在?党的方针太左,新民
主主义论太左」。王明又说,
「我党的黄金时代是抗战之初的武汉时期,1937
年 12 月会议前 1938 年 10 月六届六中全会以後,这两头的政策皆是错误
的。」\footnote{《胡乔木回忆毛泽东》
,页 199-200.}

10 月 8 日晚,
王明在杨家岭召开的中央书记处工作会议上作长篇发言,
全面阐述他对当前党的政策的看法。王明集中谈论三个问题。第一,批评
中共(实指毛泽东)有些地方政策「过左」,「妨碍统一战线」。王明说,
「对地方实力派消灭过份,对地主搞得太过火」,「今後阶级斗争要采用
新的方式,使党不站在斗争的前线,而使广大群众出面,党居于仲裁地位,
可有回旋馀地」。王明提出,在中国与苏联都异常困难的形势下,中共不
仅应与民族资产阶级搞好关系,而且还应与国民党把关系搞好,此「既有
必要,也有可能」。第二,对毛的〈新民主主义论〉提出异议。王明认为,
「在目前统一战线时期,国共双方都要避免两面战斗,要把反帝反封建加
以区别。含混并举是不妥的」。王明指出,「新民主主义只是我们奋斗的
目标,今天主要是共同打日本」。第三,对毛 1937 年 12 月政治局会议和
长江局的指控提出反驳。王明宣称「十二月会议与六中全会的政治路线是
一致的」。长江局总的路线是对的,只是在「个别问题上有错误」,如「强
调斗争性不够」,「在客观上形成半独立自主」(指与延安的关系),但
王明紧接着又强调,「我在武汉工作时是讲独立性的」(指与国民党的关
系)。\footnote{参见《胡乔木回忆毛泽东》
,页 200;另参见《毛泽东年谱》
,中卷,页 330-31;
《王明年谱》
,页 142;
(延安整
风中的王明——兼驳王明的《中共五十年》,载《党史通讯》
)
,1984 年第 7 期。
}

王明在 10 月 8 日的发言是他最後的背水一战。1938 年共产国际在斯大
林大清洗中遭到严重摧残,
王明的恩师米夫因与布哈林有牵连,
已被处决,
王明失去了保护伞。共产国际出于现实的考虑,实际上已半抛弃了王明。
在近三年的时间里,季米特洛夫未与王明直接联络(或许有联络,但电报
被毛截留),备感凄楚的王明只能审时度势,违心地向毛低头。正当王明
独自一人承受来自毛的巨大压力时,季米特洛夫的电报犹如一剂强心针,
顿时给王明注入了活力。他抱着孤注一掷的心理,向毛作最後一搏。
王明的反击被毛泽东当场粉碎。毛泽东针对王明的指责,反驳道,王
明认为我们太左了,恰恰相反,我们认为王明的观点太右了。毛说,王明
同志在武汉时期的许多错误,我们是等待了他许久,等待他慢慢地了解,
直到现在还没有向国际报告过。前几天与他谈话指出了武汉时期有这样几
个错误:一、对形势估计,主要表现是过于乐观;二、对处理国共关系,
没坚持独立性与斗争性;三、军事战略,助长了反对洛川会议的独立自主
的山地游击战的方针;四、在组织上,长江局与中央的关系是不正常的,
常用个人名义打电报给中央与前总(指八路军总部),有些是带指示性的
电报,不得中央的同意,用中央的名义发表了许多文件,这些都是极不对
的。现在王明同志谈了他的看法,大家可以讨论\footnote{《胡乔木回忆毛泽东》
,页 200-201.}。

毛泽东的迅速反击有效阻止了王明再次聚合国际派的努力。
参加当晚
会议的任弼时、康生、张闻天、陈云、王稼祥、凯丰纷纷表态,一致批评
王明,拥护毛的发言。王稼祥、任弼时还以当事人的身份,搬出了季米特
洛夫、曼努伊尔斯基当年对王明的批评性评价。任弼时回忆季米特洛夫曾
向他说,「王明缺乏工作经验」,「王明有些滑头的样子」。
\footnote{中共中央文献研究室编:
《任弼时传》
,页 474.}张闻天插
话说,《救国时报》(王明领导的中共驻共产国际代表团在巴黎创办的中
文刊物)宣传王明为英明领袖。\footnote{中共中央文献研究室编:
《任弼时传》
,页 474.}在会议结束时,毛泽东当机立断,关于
苏维埃运动後期错误问题,停止讨论。「准备在政治局会议上讨论王明提
出的政治问题」,希望王明对武汉时期的错误及对目前政治问题,在政治
局会上作出说明。毛特别提出「王明在武汉时期政治上、组织上都有原则
错误,但不是路线的错误」\footnote{《毛泽东年谱》
,中卷,页 331.}。

据胡乔木记述,在 10 月 8 日书记处会议後,毛泽东随即写了一份「较
为详细的讲话大纲」,准备在拟定 12 日召开的政治局会议上发言。毛在这
份讲话大纲中进一步扩展了自己在 10 月 8 日会议上的讲话要点,认为王明
的首要错误是统一战线中的迁就倾向,其它错误是,在中日战争问题上,
有盲目乐观倾向;在军事问题上,只是空谈五个统一与七个统一,
\footnote{指王明在「十二月政治局会议」和「三月政治局会议」上提出的与国民党军事合作的方针,即统一指挥、统一
编制、统一武装、统一纪律、统一作战计划、统一作战行动。
}以对
抗「独立自主的游击战争」;对中央关于发展长江流域游击战争的意见置
之不理;在处理党内关系上,坚持要到武汉去,使长江局成为实际上的中
央,反对延安用书记处的名义,对延安、华北下命令;不印〈论持久战〉
小册子;开六中全会不肯回来,到了西安还想回武汉去,形成「独立自主
局面」。胡乔木说,毛的大纲也指出了王明还有一些「对的地方」,但胡
未予列出。\footnote{《胡乔木回忆毛泽东》
,页 201-202.} 

毛泽东在 10 月 8 日书记处工作会议上对王明的批驳,及会上出现的的
一边倒局势,使王明深感孤掌难呜,只得全线撤退。1941 年 10 月 12 日,王
明宣布有病,不能参加政治局会议。这样,原定的会议未开成,毛泽东准
备的「大纲」也未能宣讲(此「大纲」所列各点日後成为中共党史编纂学
对长江局评价的定论)。次日,任弼时在中央书记处工作会议报告说:今
天李富春参加医生的会诊,医生要王明休息三个月。王明提出,病休三个
月中,不参加书记处会议,只参加政治局会议(实际上自此以後,王明就
再也未参加政治局和书记处的任何会议)。王明并提出两点意见:一、关
于武汉时期的工作,「同意毛主席 10 月 8 日结论」。二、关于他对目前时
局的意见,请政治局同志到他住处去谈,以後由政治局进行讨论,他病好
後再看记录。\footnote{中共中央文献研究室编:
《任弼时传》
,页 475.} 

对于王明声称有病,不能出席政治局会议事,在中共党史编纂学中长
期被解释为「诈病」。\footnote{参见华世俊、胡育民:
《延安整风始末》
(上海:上海人民出版社,1985 年)
,页 50.}王明则在三十年後出版的《中共五十年》中说,
10 月 8 日,他的胃大出血,10 月 9 日,毛泽东派其私人秘书叶子龙将他「从
床上拖到了会上」。\footnote{经笔者查阅《毛泽东年谱》等党史资料,10 月 9 日,政治局或书记处均未召开会议。但也不能排除王明所述的
真实性。因为《毛泽东年谱》的编辑是有选择性的,例如毛泽东在 1941 年 10 月 22 日政治局会议上的讲话,就没有在
《年谱》中反映,
《年谱》中 1941 年 10 月 22 日的栏目空缺。
}从 10 月 10 日起,他就卧病不起。10 月 14 日,王明被
李富春等送进延安中央医院。王明说,毛泽东是为了「甩开」他,「强迫」
其住院治病。\footnote{王明:
《中共五十年》
,页 39、38.}

王明自动请假,使毛泽东最大的一块绊脚石已不复存在。在 10 月 13
日中央书记处工作会议上,毛对王明表现出少有的温和态度。他请任弼时
向王明转达,「对他说明,他在武汉时期的工作,路线是对的。但个别问
题上的错误是有的,我们就是这些意见,如他还有什么意见,等他病好後
随时都可以谈」。\footnote{《毛泽东年谱》,中卷,页 332.}然而就在同一天,毛一改数日前宣布的停止讨论「苏
维埃後期错误」的决定,再谈过去历史问题。毛说,苏维埃运动後期错误
的主要负责者是博古同志,张闻天算犯第二等的错误,王明在四中全会中
形式上纠正了立三路线,但後来在实际工作中仍未克服立三路线。
\footnote{《胡乔木回忆毛泽东》
,页 223.}会议
最後决定成立「清算过去历史委员会」,由毛泽东、任弼时、康生、王稼
祥、彭真五人组成,以毛泽东为首,再次引导党的高级领导人把目光转回
到过去历史问题。\footnote{《毛泽东年谱》
,中卷,页 333.} 

以毛泽东为首的「清算过去历史委员会」的工作任务并不复杂,其实
就是根据毛的意图重新改写中共党史。毛当然知道此项工作的意义所在,
根本不需他人帮忙,而是自己亲自动手将改写党史的框架建立起来。据近
年披露的史料显示,
毛泽东在这段期间一共写了两组系统批判
「左倾路线」
的文稿:一组是至今仍未完全公开的长达五万多字的「九篇文章」;另一
组为近两万字的作为九月政治局会议结论的〈关于四中全会以来中央领导
路线问题结论草案〉。

所谓「九篇文章」,是毛泽东针对原中共中央 1931 年 9 月 1932 年 5
月发出的九个文件而写的批判文章。这组文章以嘲讽怒骂的方式尖锐指责
了博古、张闻天、周恩来等当时党的领导人。初始题目是「关于和博古路
线有关的主要文件」,以後改为「关于和左倾机会主义路线有关的一些主
要文件」,再改为「关於 1931 年 9 月 1935 年 1 月期间中央路线的批判」。
\footnote{《胡乔木回忆毛泽东》
,页 213、214.}

据看过「九篇文章」的胡乔木评论,该文「用词辛辣、尖刻,甚至还
带有某些挖苦」,是毛泽东的「激愤之作」,「也是过去长期被压抑的郁闷
情绪的大宣泄,刺人的过头话不少」。\footnote{《胡乔木回忆毛泽东》
,页 213、214.}毛在「九篇文章」中提出一些重
要观点,
如认为 1931 年秋成立的临时中央是非法的,
刘少奇是白区工作
「正
确的领袖人物」,文中还多处援引刘少奇的观点。文章写成後,毛泽东只
送给当时与他关系最密切的刘少奇、任弼时看过,没有在其他领导人中间
传阅。\footnote{参见《胡乔木回忆毛泽东》
,页 213;另参见《任弼时传》
,页 477.}相信这是毛泽东在对刘少奇、任弼时进行「路线交底」。胡乔木
作为毛的秘书,「看过此文,属于例外」\footnote{《胡乔木回忆毛泽东》
,页 213.}。

1964 年春,「九篇文章」从北京中央档案馆发现。毛泽东将这组极具
攻击性的文稿批给刘少奇、周恩来、邓小平、彭真、康生、陈伯达以及陈
毅传阅,毛在批语上写「请提意见,准备修改」。1965 年 1 月 2 日,毛又将
此件批送谢富治、李井泉、陶铸传阅,
让他们对文章提出意见,
以便毛
「修
改」。毛 1965 年 1 月 2 日的批语中写道,将删去文中提到的周恩来的名字,
「因为总理一生正确比(错)误多得很多」。毛说,「此文过去没有发表,
现在也不宜发表, (几十年後)
将来
是否发表,
由将来的同志们去作决定」
\footnote{《建国以来毛泽东文稿》
,第 11 册(北京:中央文献出版社,1996 年)
,页 49-51.}。

1965 年 5 月 12 日,毛泽东在对「九篇文章」作了修改後,将原文标题
改为〈驳第三次左倾路线(关於 1931 年 9 月 1935 年 1 月期间中央路线的批
判)〉,
「在内容上也增加了一些文字」。据胡乔木透露,经毛修改 1965
年稿。仍然保留了「咄咄逼人、锋芒毕露」的特点。此时,毛泽东突然改
变了原先不准备发表此文的想法,准备重新发表该文。毛解释道,「由于
年深日久,这个不利于团结的因素——写得太尖锐,不存在了,干部不会
因为看了这篇文章怒发冲冠,不许犯错误的同志改正错误,从而破坏党的
『惩前毖後,治病救人』的政策了」。\footnote{《胡乔木回忆毛泽东》
,页 214、214-15、215.}然而不久毛泽东再次改变了主意,
打消了公布这组文章的念头——概因 1965 年春毛已开始酝酿文化大革命,
毛着实不愿公开这组文稿给刘少奇脸上贴金。1974 年 6 月,毛泽东又找出
「九篇文章」,将有关称赞刘少奇的内容尽行删去,
「打算印发中央委员,
但後来只发给部分政治局委员看过」, \footnote{《胡乔木回忆毛泽东》
,页 214、214-15、215.}当时正是毛对周恩来再度不满、
策动反周的「批林批孔」运动期间,毛似乎有意将此文作为反周的石头。
毛在临去世前一个月 1976 年 8 月,又一次对此文发生兴趣,「请人读给
他听过一遍」。\footnote{《胡乔木回忆毛泽东》
,页 214、214-15、215.}此时刘少奇、周恩来、王明、张闻天、彭德怀均已不在
人世,邓小平也被毛再次打倒,毛的「敌人」已尽情铲除,夕阳日下,毛
重温旧稿,也许只是出于对这篇尽遂性情文章的特别珍爱,却再无发表此
文的念头了。

与「九篇文章」有所不同,毛泽东写的〈结论草案〉具有较正式的辞
章形式和较强的理论色彩。《结论草案》的核心精神是,首次明确认定在
中央历史上,自「九一八」事变到遵义会议的「三年零四个月」为「左倾
路线」统治时期,改变了毛 1940 年 12 月以来 1934 年中共六届五中全会
为「苏维埃後期左的错误」起点的划分。毛的这个改动非同寻常,由此,
可以将王明与「苏维埃後期左的错误」挂上钩。由于这个改动,毛在起草
《结论草案》时,对原先只点出博古一人的名字作出修改,加上了王明的
名字。毛写道:
\begin{quote}
{\fzwkai 王明同志与博古同志领导的这条路线是在思想上、政治上、
军事上、组织上各方面都犯了严重原则错误的,集各方面错误之
大成,它是形态最完备的一系错误路线。\footnote{《胡乔木回忆毛泽东》
,页 226、229、224、224-25.}}
\end{quote}
毛泽东并从思想形态、政治形态、军事形态、组织形态等方面具体剖
析了这条「错误路线」的特征和对中共革命造成的巨大危害。特别指出,
在长征前把中央领导变成「三人团」,变为外国顾问一人专政,剥夺政治
局委员与闻军国大事的权利,甚至根本停止政治局的工作。以及「故意地
压抑刘少奇同志(他是很好的与很老的群众领袖,又是政治局委员),而
提拔了两个新党员(博古、张闻天)」,是这条「错误路线」最恶劣的表
现方面之一。\footnote{《胡乔木回忆毛泽东》
,页 226、229、224、224-25.} 

〈结论草案〉
的另一个引人注目之处是关于对中共六届四中全会的评
价。与中共党史编纂学的传统解释完全不同,毛泽东在这份〈结论草案〉
中不仅没有将六届四中全会称之为「左倾机会主义路线的起点」,相反,
毛写道:
\begin{quote}
{\fzwkai 中央政治局在收集详细材料经过详细讨论之後,一致认为四
中全会及其以後一个时期,中央领导路线虽有缺点、错误,但在
基本上是正确的。\footnote{《胡乔木回忆毛泽东》
,页 226、229、224、224-25.}}
\end{quote}
毛泽东在〈结论草案〉中列举了「四中全会的成功方面」:「反对了
李立三的错误路线与罗章龙的反党行为」,恢复了共产国际在中国党内的
信任,放弃了组织城市暴动和攻打大城市,解决富田事变的争论。以及在
粉碎第二次、第三次「围剿」中都作了工作等等。毛指出,这些都应肯定。
「这是四中全会的主要的根本的方面」\footnote{《胡乔木回忆毛泽东》
,页 226、229、224、224-25.}。

毛泽东为什么如此肯定以後遭致他强烈谴责的六届四中全会?综合
两方面的因素,可得出下列判断:

首先,中共六届四中全会派往江西的中央代表团 1931 年 4 至 10 月,
曾经全力支持毛泽东。正是在中共中央和任弼时的全力支持下,才最终把
一度被项英翻案的富田事变重新定为反革命事变,有力地维护了毛泽东的
威信。

第二,几个与毛泽东关系最密切的政治局委员,如任弼时、刘少奇是
在六届四中全会上当选的(任弼时当选为政治局委员,刘少奇当选为政治
局候补委员),陈云、康生在六届四中全会上当选为中央委员。否定六届
四中全会不可避免将触及他们。有可能对任弼时、刘少奇等与毛的合作产
生不利的影响。

至于毛泽东是否顾及到共产国际的因素并不是主要的。
六届四中全会
和五中全会都是经共产国际批准召开的,毛既可以否定六届五中全会,就
没有理由担心否定六届四中全会将遭致共产国际的不满。

然而毛泽东终究是要彻底否定六届四中全会的,
非此不能彻底剥掉王
明等国际派「合法性」的外衣。这一幕何时推出,还得取决于党内形势的
变化,尤其是政治局多数成员态度的进一步转变。现在,毛泽东的老对手
项英已死,只要任弼时能够配合,江西「肃 AB 团」的那一段历史,任何人
都不敢牵扯到毛的身上,毛反而可将「肃反扩大化」的帽子扣到王明、博
古的头上。只是眼下形势还未臻于完全成熟,毛泽东还需继续等待。

毛泽东 1941 年 9 月政治局会议上虽然没能彻底否定中共六届四中全
会,但是他的一些最重要的目标已经实现。毛已一举摧毁王明等国际派,
完全控制了中央政治局和书记处。毛还顺利地打下了重写中共党史的基
础。1941 年秋,在毛的倡议下,中央政治局宣布成立几个机构:以康生为
主任的「中央党与非党干部审查委员会」,9 月 26 日成立的以毛泽东为组
长的「中央高级学习组」和以陈云为首的复查过去被错误打击干部的委员
会,前两个机构成为毛领导整风运动的最重要的工具。

1941 年 9 月政治局会议的结果表明,共产国际对中共的控制力已基本
丧失,在中共政治生活上曾经发挥过重大作用的国际派已经土崩瓦解。王
明实际上已退出中共核心上层,从此不再对中共重大决策起任何作用。毛
泽东的下一步骤是乘胜前进,全面肃清王明等国际派在党内的影响,在全
党搞臭王明、博古,彻底改变中共的面貌。
