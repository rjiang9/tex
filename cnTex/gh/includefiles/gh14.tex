\chapter{进两步,退一步:「抢救」的落潮}
\section{「审干九条」再颁布後「抢救」为什么愈演愈烈?}

1943 年春夏之际,「审干」逐步转入「反奸」、「抢救」,延安三万
多党、政、军干部全被卷入进去,「特务」、「叛徒」、「内奸」,如滚
雪球般越滚越多,人心浮荡,个个自危,一片肃杀气氛弥漫于各机关、学
校。高压下的人们普遍感到惶恐,不知运动将往何处发展,个人的命运将
有何变化。

7 月 1 日,毛泽东在给康生的批示中,提出了「防奸工作的两条路线」问题。毛
说,正确路线是:「首长负责,自己动手,领导骨干与广大群众 相结合,一般号
召与个别指导相结合,调查研究,分清是非轻重,争取失 足者,培养干部,教育
群众」。错误路线是:「逼、供、信」。\footnote{毛泽东: 〈防奸工作的两条
路线〉(1943 年 7 月 1 日),载《文献和研究》,1984 年第 4 期。}此即
是所 谓「审干九条方针」的首次表述。毛的这段指示长期被认为是毛反对审干、
肃奸极左倾向的有力依据,可是在这之後,延安的「抢救」反而一步步深 入,走
向高潮。

在「抢救」正处于高峰之时,康生动了开杀戒的念头,他提出要以边
区政府的名义公审枪毙人犯。在这紧急关头,林伯渠立即将这一最新情况
向毛泽东汇报,毛泽东否决了康生的提议,避免了一场眼看就要发生的内
部残杀的惨剧。\footnote{《林伯渠传》编写组:
《林伯渠传》
,页 286-87.}1943 年 7 月 30 日,毛泽东指示停止「抢救失足者运动」。
\footnote{刘家栋:
《陈云在延安》
,页 114.}

1943 年 8 月 15 日,中共总学委向全党及延安各机关、学校发出一道
指示,要求有系统地进行一次关于国民党本质的教育,「决定自 8 月 16
日至 8 月 31 日这半个月中,各单位一律以主要力量来进行这个教育」,
并宣布「抢救运动」告一段落。

同日,中共中央又一次作出〈关于审查干部的决定〉,重申毛泽东关
于审查干部和肃清内奸的九条方针,这是在全党范围内第一次公布「审干
九条方针」。该决定宣称,此次审干及「进一步审查一切人员」,「不称
为肃反」,并将「逼供信」称之为「主观主义方针与方法」。

同日,毛泽东在康生提交的绥德反奸大会材料上批示,提出在「反特
务斗争中」,必须坚持「一个不杀,大部不抓」的政策。

由此看来,延安的「抢救」应该停止了,因为毛泽东已经多次发话,
并对「逼供信」提出了批评。但事实上延安的「抢救」丝毫没有降温,反
而在 8 月 15 日审干决定颁布後,愈演愈烈,又掀起新的反奸、坦白高潮。
此时,运动已易名为「自救」运动,但内容、实质与「抢救」别无二致。9
月 21、22 日,延安《解放日报》连续刊登〈延安县开展防奸活动〉和〈绥
师失足青年纷纷悔过,控诉国民党特务机关万恶罪行〉的报道,将所谓参
加了「复兴社」的十四岁小女孩刘锦楣和特务暗杀组织「石头队」的「负
责人」、十六岁的小男孩马逢臣的坦白交待经验推向社会。10 月 6 日,延
安市在边区参议会会场召开反汉奸特务大会,议期五天,\footnote{《谢觉哉日记》
,上,页 543.}「抢救」的邪火
越烧越旺。

难道毛泽东已失去对延安的控制力?难道毛泽东的话不再一言九鼎?
难道康生胆大妄为、
已不把毛泽东放在眼里——所有这些答案都是否定的,
毛泽东一分钟也没失去权力,他只是采取了某些手法,在表面上批评一下
「抢救」的「过火」行为,而实际上继续将「抢救」往更深入的方向推进。

1943 年初,
毛泽东搬入了戒备森严的神秘机构——设在枣园的中央社
会部机关,中社部并在通往枣园的要道小砭沟开办了一家小杂货铺,用以
监视一切过往的「可疑人员」。毛泽东住进枣园後,与康生的联系更加方
便,更加频繁,毛泽东每天起床的第一件事就是听取康生对审干、抢救的
汇报。\footnote{金城:
《延安交际处回忆录》
(北京:中国青年出版社,1986 年)
,页 187.} 在枣园,
他虽然多次看到由交际处呈交的关于钱来苏动态的报告—
—交际处向毛泽东不断报告,目的就在等毛的一句话,好解脱钱来苏——
报告详细反映钱来苏的焦虑、不安、惶恐以及钱来苏反复陈述白己不是汉
奸、特务的内容,但是毛泽东就是不明确表示态度,致使钱来苏长期不得
解脱。

人们可能会发出疑问,毛泽东不是已经批评「逼供信」了吗?他在和
一些负责干部的个别谈话中,甚至提出要注重调查研究,不要搞肉刑,为
什么他说的是一套,做的又是一套?

确实,毛泽东隐蔽的思想很难被一般人所发现,许多负责干部只看到
毛批评过火行为的一些词语,而忽略了他这些话後面的更深涵义,毛泽东
所强调的恰恰不是纠偏,而是将运动进一步搞深搞透。

毛泽东的这套谋略,充分反映在被许多干部视为是纠偏文件的 1943
年 8 月 15 日〈关于审查干部的决定〉中。

「八一五决定」提出「一个不杀,大部不捉」,是毛泽东鉴于内战时
期「肃 AB 团」等极左肃反政策的教训而提出的一个重要方针。毛泽东当然
知道当年他自己所作所为的真正动机是什么,只是现在毛泽东的身份、地
位已不同于当年,延安的局面更非往昔江西时期的情况可比。眼下,主客
观条件均不允许再重演
「肃 AB 团」
的一幕,
因此毛明智地宣布
「一个不杀,
大部不捉」,此所谓「过一不过二」也。但是这个新方针的基本前提仍然
是肯定有大批特务混入革命队伍,文件强调,「特务是一个世界性群众性
的问题」,指出,认识这个基本前提才有可能采取正确方针。

所谓「一个不杀,大部不提」,并非认为运动方向错了,而是着眼于
将运动向纵深方向推进。
毛泽东解释道,
一个不杀——将使特务敢于坦白;
大部不捉——保卫机关只处理小部,各机关学校自己处理大部。毛泽东还
具体规定了捕人的规模:普通嫌疑分子,占有问题人员的 80\%,留在各机
关学校接受审查;10\%的问题人员送入西北公学、行政学院反省机关;另
有 10\%的人员送入社会部、保安处的监狱系统。毛泽东规定,这三类人员
要进行交流,即普通嫌疑问题严重者将升格进入二类、一类系统。反之,
坦白、交待彻底的特务,也可降到二类、三类系统。

对于留在各单位受审人员的审查和监护,毛泽东也不厌其烦地进行具
体指导:一切有问题人员都暂时禁止外出,在延安实行通行证制度,毛泽
东并要求,「在一定时候实行戒严」\footnote{中央档案馆编:
《中共中央文件选集》
(1943 一 1944)
,第 14 册,页 89-96.}。

精明、仔细、对大、小事都要过问的毛泽东,\footnote{在师哲的回忆中,对毛泽
东的精细有很生动的描述。其中有一段写道:当胡宗南军队攻占延安前夕,毛泽东
亲 自监督师哲销毁与莫斯科来往的文电密码及记录,毛最後还用小棍翻播灰烬,
待确定已燃尽後,才放心离开。参见师 哲: 《在历史的巨人身边——师哲回忆录》
,页 201-202. 在延安时代与毛交往较多的萧三也认为毛做事很细。参见《谢 觉
哉日记》,下,页 681.  }难道不知道在审干抢救 中会发生左倾狂热?他当然知
道,而且了解得十分透彻。「八一五决定」中称, 在审查运动中, 一定会有过
左的行为发生, 一定会犯逼供信错误 (个 人的逼供信与群众的逼供信),一定
会有以非为是、以轻为重的情形发生。可是毛泽东明知故纵,偏不予制止,执意
听之任之下去。「八一五决定」提出,纠左不能太早,亦不能太迟。既然运动已
经发生偏差,为什么不立 即制止?毛泽东自有一套逻辑:「对于过左偏向,纠正
太早与纠正太迟都 不好。太早则无的放矢,妨碍运动的开展,太迟则造成错误,
损伤元气, 故以精密注意,适时纠正为原则」。\footnote{中央档案馆编: 《中
共中央文件选集》(1943 一 1944),第 14 册,页 89-96.}恰恰由于毛泽东的
这套逻辑,才使审 干、抢救的极端行为恶性发展,因为谁都不知应在何时采取纠
偏行为才谓
「适时」,而「抢救」的野马,只有毛泽东才能勒住缰绳,他若不采取明
确措施予以制止,谁都不敢,也无权力纠偏。

毛泽东执意扩大审干,精密筹划各种具体方法和措施,他提出审干的
正确路线应是
「首长亲自动手」,于是许多机关学校负责人就亲自审问
「犯人」,亲自动手打入。毛泽东提出要依靠重众力量审干,于是各单位纷纷
召开群众大会,造成恐怖的群众专政的声势。毛泽东提出要「调查研究」,
拟定有问题与没问题两种人名单,对所谓「有问题的人」要结合平时言行,
从其交待的历史资料中找出破绽,对他们进行「劝说」,「质问」,各单
位如法炮制,车轮战、攻心战,纷纷上阵。毛泽东声称「愈是大特务,转
变过来愈有用处」,表扬「延安几个月来已经争取一大批特务分子,很快
地转变过来为我党服务,便利了我们的清查工作」,于是各单位纷纷利用
坦白的「特务」进一步检举其他特务,「特务」一串一串地被揭露出来。
毛泽东别出心裁,要求「着重注意,将反革命特务分子转变为革命的锄奸
干部」\footnote{中央档案馆编:
《中共中央文件选集》
(1943 一 1944)
,第 14 册,页 89-96.} ——按照共产党的逻辑,
「反革命特务分子」与「革命的锄奸干部」
本是风马牛不相及的两个概念,毛泽东究竟是指示「以毒攻毒」,或是暗
示「只要为我所用,管他乌龟王八蛋」,语意含混,难得要领,结果是被
诬为「大特务」的原中央政治研究室的成全等人,果真被留在中央社会部,
转变为「革命的锄奸干部」。

1943 年 8 月以後,在毛泽东有关批评「逼供信」的只言片语的後面, 隐藏着深
深的玄机。他的面孔是多重的,有时,他会轻描淡写说几句「逼 供信」不好,转
眼间,他又会说「既然没问题,为什么怕审查呢」,「真 金不怕火炼嘛」,
\footnote{在「抢救」运动中,毛泽东和王世英说过类似的话,当时王世英已被人
诬陷为「特务」,参见段建国、贾岷岫著, 罗青长审核: 《王世英传奇》,页
193.}毛泽东的「注意正确的审干方针」只是一句空话,他所 关心的是如何彻底查
整全党的干部。他的目的只有一个: 以暴力震慑全党, 造成党内的肃杀气氛,
以彻底根绝一切个性化的独立思想,使全党完全臣 服于唯一的、至高无上的权威
之下——毛泽东的威权之下。应该说,毛泽 东达到了他的目标, 几十年後, 当年
经历过审查的干部还在说, 对他们 「教 育」最大、使他们得到「锻炼」、真正
触及了灵魂的是审干抢救运动,而 不是前一阶段的整风学习。
 
\section{中央主要领导干部对「抢救」的反应}
由毛泽东、康生主导的延安「抢救」和审干、反奸,是以中央书记处、
中央总学委的名义推行的,毛泽东已将中央政治局完全控制于掌中,当他
需要时,才主持召开政治局扩大会议,此时的中央政治局实际上已名存实
亡,在延安的中央政治局委员大多已处于被批判的地位,被分割在各高级
学习组进行整风学习,在毛的高压下,中共大多数领导干部都难以表达不
同意见。

作为整风头号目标的王明,自 1941 年 10 月住进医院以後,他的政治 局委员、
书记处书记和中央统战部部长等职务虽未免去, 但已是形同虚设。1943 年春之
後,毛泽东作出决定,由他亲自掌管重庆办事处,由任弼时负 责驻西安办事处,
在延安的中央统战部几乎已无事可做,王明真正陷入了 四面楚歌的境地。1943 年
11 月,中央总学委、中央办公厅召开揭发王明 错误大会,王明之妻孟庆树登台为
其夫辩护,会场气氛一度对毛泽东十分 不利。毛泽东大为光火,将这次批判大会
斥之为「低级趣味」,下令终止 这种允许被批判者登台辩解的斗争大会。
\footnote{王明: 《中共五十年》,页 148.}王明没有任何地方可以渲泻他的
不满和「怨屈」,他怕苏联人去看他,从而招惹毛泽东、康生的忌恨,可 是又控
制不住想见苏联人,只能在前来探病的苏联医生面前「痛哭流涕」\footnote{弗
拉基米洛夫: 《延安日记》,页 108、164. 据弗拉基米洛夫说,他是在得到毛
泽东的许可後,才派奥洛夫医生 于 1943 年 10 月 28 日前去为王明治病的,就
是在这一天,王明在苏联医生面前哭了出来。}。

博古对整风、审干、抢救极度厌恶,但他也没有任何可以与之倾诉的
对象。他以工作为由,找到苏联驻延安代表,痛骂毛泽东。博古深知康生
情报机关的厉害,与苏联代表讲话时,不时出门观察门外动静,确定没有
人偷听,才敢进屋与苏联代表倾谈\footnote{弗拉基米洛夫:
《延安日记》
,页 137.}。

在延安的几位德高望重的中共元老林伯渠、徐特立、吴玉章、谢觉哉
均非整风目标。四老皆与毛泽东有历史旧谊,徐特立、谢觉哉还是毛泽东
长沙时期的师友,林伯渠、吴玉章早在广州、武汉国共合作时期即与毛泽
东共事,同为国民党中央执行委员。在瑞金时期林伯渠担任国民经济部部
长和财政部部长,与时任中华苏维埃共和国中央执委会主席的毛泽东相处
融洽。整风转入审干、抢救後,林伯渠等诸老日见愈来愈多的同志被打成
「特务」、「叛徒」,国统区中共组织被诬为「红旗党」,均感到震惊,
但是诸老性格各异,对党内斗争的体会各自深浅不同,其反应也是存有差
别的。

林伯渠是诸老中唯一担负实际领导责任的。他曾为钱来苏一案多次呈
文毛泽东。作为老政治家的林伯渠,深知此类运动能
全面展开,非毛泽东批准而绝不可能。因此,他的态度极为谨慎。林伯渠
曾劝慰对运动感到怀疑的同志说,运动高潮阶段,出现「逼供信」是不足
为奇的,但运动後期一定会复查核实。他并表示,大批青年知识分子都经
过中共驻重庆、西安办事处审查才介绍进延安的,他从未听说过整个大後
方党组织已变质,对此问题,他心里有数。」\footnote{《林伯渠传》编写组:
《林伯渠传》
,页 138.}

徐特立长期从事教育工作,性格率真,他曾当面质问负责自然科学院
审干抢救运动的陈伯村,凭什么证据,将一批批师生抓走。\footnote{《徐特立在延安》
,页 45、118、139.}徐特立全然不
管他实际上已被闲置的处境,愤然道,我是院长,我有责任,为什么不许
我管!\footnote{《徐特立在延安》
,页 45、118、139.} 徐特立还亲自前往窑洞看望被关押的师生,
当别人劝他应予以注意
时,他再次表示:我是院长,我就是要保护人才\footnote{《徐特立在延安》
,页 45、118、139.}。

谢觉哉此时担任边区参议会副议长兼党团书记,1943 年 7 月「抢救」
乍起,他在初期也是从好的方面去理解。谢觉哉在 7 月 31 日的日记中写
道,「我对某些失足青年怜多于恨,处在反动环境下……生死判诸俄倾,
革命与反革命又其模糊,于是乎就失足了。」谢觉哉继续说:「这次反特
务斗争,给我们教育不少……没有这次斗争要我们青年党员知道阶级斗争
不易,就是老年党员也一样。」\footnote{《谢觉哉日记》
,上,页 521、603、694.} 但是很快,谢觉哉就发现「抢救」出了大
问题, 「反奸斗争被逼死的人,
他说
无法审查了,
但其中未必有主要特务。」\footnote{《谢觉哉日记》
,上,页 521、603、694.}

谢觉哉对「抢救」的疑问与其经历过的党内残酷斗争有关。1932 年在
湘鄂西苏区,他曾亲眼目睹红军内部自相残杀的惨景,当时谢觉哉也被列
入有待处置的肃反名单,只是幸而被国民党清剿部队俘虏,才侥幸躲过那
场灾难
\footnote{《谢觉哉日记》
,上,页 521、603、694.}。于是尽其
所能保护自己所在单位的干部。但因他在 1937-1938 年曾负责中共驻兰
州办事处,眼下,中共甘肃工委已被康生打成「红旗党」,谢觉哉竟也遭
到责难。康生白恃握有上方宝剑,对享有盛望的谢觉哉丝毫不看在眼里,
公然指责「谢老是老右倾」。在抢救高潮中,康生直扑谢觉哉办公的窑洞,
一进门就盛气凌人地指责谢觉哉,「据 XXX 交待(笔者注:指张克勤),
他的父亲是个老特务。看来兰州地下党全是特务,是个『红旗党』,你这
个兰办的党代表可真是麻木不仁啊」。面对康生的责难,谢觉哉明确表示
不同意他的看法,竟被康生指责为「庇护特务组织」。谢觉哉一气之下。
「干脆不去开会,不参加学习,呆在家里睡觉」。\footnote{《谢觉哉传》编写组:
《谢觉哉传》
(北京:人民出版社,1984 年)
,页 292-93.}面对抢救野火蔓延,谢
觉哉只能自我安慰:
「不可能没有吃冤枉的个人,
只求没有吃冤枉的阶级。」\footnote{《谢觉哉日记》
,上,页 708.}


在几老中间,稍微特殊的是吴玉章,此时吴玉章挂名延安大学校长,
但该校实权由副校长周扬掌握,吴玉章只是做些「新文字」
(汉字拉丁化)
的研究和推行工作,并不具体过问延安大学的审干、肃奸、抢救工作。或
许是因为在莫斯科曾与王明共过事,也可能是因为缺少在三十年代苏区生
活的经历,吴玉章对来势凶猛的运动似乎感到有些紧张。吴玉章在「抢救」
运动期间曾拄着拐杖、流着眼泪劝说被诬为「特务」的中直机关的青年向
党坦白交代。他还通过写自传表态支持「抢救」,吴玉章写道:
\begin{quote}
	\fzwkai 在整风中人人写思想自传,并且号召坦白运动。这里就发
现了国民党派了不少特务到我们党内来,到我们边区来,到我们
军队中,专门作破坏工作,这是整风初期所未料及的。我党以宽
大政策,号召这些被国民党特务分子陷害了的青年改过自新,已
经有不少的特务分子响应了党的号召,改过自新了,且愿为反对
特务尽力。至于死心塌地、甘为反革命尽力的少数人,则已逮捕
起来。这又是反共分子而料所不及的,真所谓「作伪心劳日拙」。
\footnote{吴玉章:
〈我的思想自传〉
(1943 年)
,载《吴玉章文集》
,下,页 1338.}
\end{quote}

尽管林伯渠、徐特立、谢觉哉等诸老都对「抢救」表示了怀疑与
不满,但是,他们并没有就此向毛泽东进言,林伯渠相信,一阵风以後,
高潮过了,头脑发热的人会清醒下来。\footnote{《林伯渠传》编写组:
《林伯渠传》
,页 137.}他们要等毛泽东自己去纠偏,
而不愿去冒犯忤的风险。

身为八路军总司令,但毫无实权的朱德,对毛泽东、康生的行径
心知肚明,知道讲话没用,只能暗暗焦急,而无可奈何。

陈云此时仍是政治局委员和中组部部长,作为干部审查和管理的 最高机关中组部
的部长,陈云本应直接参与领导审干、抢救运动,但是 毛泽东没有让他与闻审干
和「抢救」运动,公开的理由是「毛主席关心 陈云同志的身体,让他搬到枣园去
休养」。然而真实的情况是毛泽东、 康生对陈云领导下的中组部很不满意, 康
生曾指责中组部 「坏人那么多, 你们组织部都是怎么搞的呀?」认为中组部「在
审查问题上右了,太宽 了,使得特务钻到了我们党内」\footnote{刘家栋: 《陈
云在延安》,页 30、112、114.}。 让陈云「休养」的另一个原因是陈云 对开
展「抢救」运动的态度很不积极。据陈云当年秘书回忆,当时陈云 「根本就不同
意搞这场『抢救』运动」,他认为是夸大了敌情。他也不 相信所谓「红旗党」的
说法,认为这不符合事实。陈云更认为对许多青 年知识分子和老干部实行「抢救」
违背了中央关于如何整风的规定。正 是因为陈云的这种态度,毛泽东就不要他过
问审干、反奸一类事,而是 让陈云「到他身边去治疗休养」。从 1943 年 3 月
陈云住进枣园到 1944 年 3 月,他离开枣园,调往西北财经办事处,恰是审干、
反奸、抢救从 开场到落幕的一年,在这一年间,陈云实际是被「靠边站」了,他
没有
参与有关决策,
「许多事情都不知道」。\footnote{刘家栋:
《陈云在延安》
,页 114.}中组部部长一职则由彭真代理,
当陈云调往西北财经办事处後,彭真就被正式任命为中组部部长。

林彪于 1943 年 7 月与周恩来等一行从重庆返回延安,受到毛泽东
的特别关照,毛嘱林彪休息,林彪只是挂名担任中央党校副校长,并不
具体过问党校的具体工作(此时延安除整风、审干外,没有任何紧急工
作)
。林彪在延安对康生一直保持距离,对审干、抢救持沉默态度,完
全置身于运动之外。

身为中央军委参谋长的叶剑英,在审干、抢救中,曾向中央负责
人反映军委直属机关抢救中出现的严重问题,
他明确表示延安不可能有
这么多特务,不能这样搞运动。但是叶剑英本人也受到康生的怀疑。康
生以叶剑英长期在国统区工作,社会关系广泛,不时在毛泽东面前进谗
言,并对叶剑英在延安的亲属进行「抢救」迫害。叶剑英前妻危拱之,
被打成「河南红旗党」的特务,从 1943 年秋至 1945 年春被长期关押,
「身心遭受严重损伤」
,精神一度失常。\footnote{任质斌:
〈纪念党的好女儿——危拱之同志〉
,载《怀念危拱之》
(郑州:河南人民出版社,1986 年)
,页 20.}叶剑英虽未隔离审查,但两次
被剥夺参加讨论内战时期中共路线的政治局扩大会议。
据苏联驻延安观
察员透露,叶剑英对康生深恶痛绝。
中共几位重要将领刘伯承、聂荣臻、陈毅此时均被召回延安,正
要对各自在内战时期及抗战初期的「错误」进行反省,他们虽然均对审
干、抢救中的极端行为不满,但是,他们的地位和身份都使他们难以开
口。

除了少数几个积极参与康生「抢救」的重要干部,大多数领导人 都对抢救、审干
的过火行为表示不满。陈云、王若飞等人均曾在私下对 「抢救」有所疑问,但是
他们都不曾在毛泽东面前表示。在当时的肃杀 气氛下,这些怀疑和不满都处在分
散状态,没有人敢于在重要会议上将 问题正式提出,更有甚者,一些重要干部,
包括中央委员们都已中断来 往,大家只有在公众场合才能见面,互相交谈都极为
谨慎小心\footnote{弗拉基米洛夫: 《延安日记》,页 186-87.}。

对「抢救」正式向毛泽东、康生表示怀疑的中共高层领导干部仅
有周恩来、任弼时、张闻天、高岗。
 
1943 年 7 月 16 日,周恩来返回延安,准备参加中央核心层的路线 检讨。周恩
来一返回,就发现由他直接领导的国统区中共地下党已被诬 为国民党特务组织
「红旗党」,给周恩来造成巨大压力。周恩来本人甚 至也受到康生的怀疑,认为
周等「在白区天天与国民党接触,靠不住」。
\footnote{《聂荣臻回忆录》
(中)
(北京:解放军出版社,1984 年)
,页 562.}
周恩来一方面为许多被康生机关及各单位关押的原部下写证明材料,
另一方面,周在与李维汉等谈话时,明确表示不存在所谓「红旗党」,
国统区中共地下党的情况是清楚的。周恩来此时在党内的地位十分软
弱,且是带罪之身。但他仍直接向毛泽东进言,表示了自己对运动的看
法。

除了周恩来,敢于向毛泽东表达怀疑的还有任弼时和张闻天。任
弼时此时处于权力核心,但他为人较为正直,对毛的一些做法颇不以为
然。
毛虽感觉到任弼时有些碍于碍脚,
但为了利用他作为老干部的象征,
分化打击王明、周恩来等,故对任弼时仍予以容忍。在 1943 年秋冬之
际,任弼时两次向毛泽东提出「抢救」的严重弊端,要求予以纠正。

和处于权力核心的任弼时相比,张闻天早已成为失势人物,且正
处在被批判斗争的地位。然而,张闻天却直接向康生表示他对「抢救」
成果的怀疑,他明确告诉康生,社会部所编辑的《防奸经验》全是假的。
\footnote{刘英:
《在历史的激流中——刘英回忆录》
,页 127-28.}
和那些明哲保身的其他高干相比,张闻天全不计较个人得失,显示出
他身上仍保有一些书生本色。

高岗在整风、 「抢救」中原是一个「积极分子」,但随着「抢救」不断深入,
他也感到似乎出了问题。据师哲透露, 高岗曾向毛泽东反映, 抢救的「作法过
激」。\footnote{师哲: 《峰与谷——师哲回忆录》,页 157.} 

在次一级的负责干部中,也有人通过不同的方式,向毛泽东和中
共中央表示对「抢救」的怀疑和反对,在这些人中,最具胆识的是原八
路军驻二战区办事处主任王世英和社会部治安科科长陈龙。

1942 年整风运动开始後,王世英被调回延安,先在王家坪中央军 委学分会工作,
後调入中央党校参加整风学习。整风转入审干後,王世 英参加了党校的审干小组。
但是很快就对运动产生了怀疑,一些过去受 王世英领导、在白区从事秘密工作的
同志,被人揭发成为「特务」;王 世英在经手调查党校「特务」的案件时,也发
现指控与事实不符。对此, 王世英在支部会上公开表示了对运动的怀疑,并写出
了〈关于请求中央 纠正抢救失足者运动过左问题的报告〉,上书毛泽东、刘少奇、
康生。王世英在这份报告里,明确提出运动发展已经过「左」,要求中央予以
纠正,并以自己的党籍和脑袋作担保,为已被打成「特务」或「特嫌」的钱来苏、
白天(即以後成为名作家的魏巍)等六人申诉。在这六人中, 由王世英亲自介绍
前来延安的钱来苏受到长期怀疑,一直未能解脱;另 两人也在车轮战下供认自己
是「特务」。王世英上书的举动引起康生的 强烈反弹, 康生连夜给王世英覆信,
指责王是 「主观主义」、 「好人观点」。在中央学委会上,康生当面责骂王
是「大自由主义者,想逞英雄」,威 胁王世英「有几个脑袋」?\footnote{段建
国、贾岷岫著,罗青长审核: 《王世英传奇》,页 191-92、192、193.}不久,
果然出现了针对王世英的行动,在 中央党校千人干部大会上,有人公开指认王世
英和孔原是「大特务」\footnote{段建国、贾岷岫著,罗青长审核: 《王世英传
奇》,页 191-92、192、193.} 。王世英虽然受到 「抢救」的波及, 但是毛
泽东与他曾有过多次个别接触, 对王世英在山西开展的统战和情报工作较为满意,
因此,王世英并没有 因上书反对「抢救」而遭致较大的不幸。以後他在〈自传〉
中提及此事 时说: 「问题虽然提出来了(指有人诬指王为「特务」一事),始
终没有 向我开火,说明中央是很关心我和爱护我的,而且说明也是很了解我 的。」
\footnote{段建国、贾岷岫著,罗青长审核: 《王世英传奇》,页 191-92、192、
193.} 

在王世英为反对「抢救」上书之际,领导「抢救」的社会部内也 有工作人员对运
动表示了怀疑。治安科长陈龙当面向顶头上司康生陈述 他对运动情况的不理解。
依照规定,杜会部治安科每周要写一份简报, 分送毛泽东、刘少奇、朱德、任弼
时、康生等五至七人,陈龙和社会部 工作人员甘露通过这份手写的材料,曲折向
毛等表示对运动的异议。经 陈龙等汇总的材料上有详细的统计数目,具体反映各
单位运动进展情 况:奸细、特务的比例,自杀身亡人数,被关押人数等。材料的
最後结 论是:延安各单位 50\%以上的干部已被抢救。\footnote{陈龙等的上报材
料估计反映的是运动初期——1943 年 7 至 8 月的情况,因为在此之後,几乎所有
外来知识分子干部都程度不同地被抢救。} 陈龙等整理的资料依正 常程序上报後,
中央总学委原计划开持续 七天的延安党、政、军、学校 参加的全市规模的「抢救」
大会,结果开到第三天 就没再继续下去。\footnote{修来荣: 《陈龙传》,页
148-49 485 特洛夫的绝密电报,这份电报涉及到一系列重要的问题,客观上促成
了 毛对「抢救」的刹车。}当 然,不召开全市抢救大会,并不表 明运动就降温了,
陈龙等 毕竟不能真 正影响毛泽东的决策,以後,各机关、学校 在内部继续开抢
救大会, 挖 出来的「特务」、「内奸」比以前更多。

在延安的中共高层领导干部对「抢救」极端行为的不满议论,通
过种种渠道传到毛泽东那里,然而毛泽东何尝不知道这些人的态度,他
所关心的并非是他们的不满——毛泽东所要的是另一种效果,这就是,
即使中共高层领导干部腹有怨言,但绝大多数人已不敢在他面前陈述。
毛泽东借助审干、反奸、抢救达到了他多年来一直孜孜追求的目标,从
精神上完全控制住昔日这批敢于与他斗争的同僚。

\section{1943 年 12 月 22 日季米特洛夫来电与「抢救」的中止}

周恩来、任弼时等为扭转抢救、审干中的极端行为,直言相劝毛
泽东究竟有无作用?毛泽东是否立即采纳周恩来、任弼时等的意见,下
令纠偏,停止运动?与人们一般的推测相反,毛泽东并没有立即部署纠
偏,对于刚愎自用的毛泽东,只有当他自己意识到必须转弯时,他才会
采取行动。所谓「适时纠正」的恰当时机,只有他才能决定,勿需别人
多嘴。

毛泽东一点也不认为抢救、反奸有什么过错,他不是多次批示「一 个不杀,大部
不捉」吗?他不是提出反对「逼供信」吗?如此,继续运 动又有何害?无非是过
左一些,无非是受一点委屈,可是又没要你们的 命,多坐几天班房又有什么关系
呢?如果不对广大干部真正有所触动, 「两条心」、「半条心」,能转变为「一
条心」吗?

当然,对于任弼时、周恩来等的意见,毛泽东还是会加以周全考
虑的,因为毛心里明白,延安不可能有那么多特务,毛总要想出一个办
法,来收抬眼下这个局面。恰在这时,毛泽东收到一份来自莫斯科季米
季米特洛夫电报全文如下:
\begin{quote}
	\fzwkai 
\noindent	
	1943年 12 月 22 日
\\毛泽东(亲启)

一、关于令郎。我已安排他在军政学院学习,他毕业後当能在
马克思列宁主义和现代军事方面获得扎实的学识。这个小伙子很能
干,我相信您会把他培养成一个可靠的好帮手。他向您致以热烈的
敬意。

二、关于政治问题。不言而喻,在共产国际解散之後,它过去
的任何领导人都不得干预各国共产党的内部事务。但是从私人友情
考虑,我又不能不告诉您我对中国共产党党内状况的担忧。您知道,
从 1935 年起,我就不得不经常密切过问中国的事务。我认为,从反
抗外国侵略者的斗争中退缩的方针,以及明显偏离民族统一战线的
政策,在政治上都是错误的,在中国人民进行民族战争期问,采取
这样的方针,有把党孤立于人民群众之外的危险,有导致内战加剧
的危险。这只能有利外国侵略者及其在国民党内的代理人。我认为,
发动反对周恩来和王明的运动,指控他们执行了共产国际推荐的民
族统一战线,说他们把党引向分裂,这在政治上是错误的。不应该
把周恩来和王明这样的人排除在党之外,而应该把他们保留在党内,
千方百计利用他们为党工作。另外一件使我担心的事是,一部分党
的干部对苏联抱有不健康的情绪。我对康生所起的作用也心存疑虑。
清除党内敌对分子和把党团结起来的党内正确措施,被康生及其机
构扭曲得面目全非,这样做只能散布互相猜疑的情绪,引起普通党
员群众的无比愤怒,帮助敌人瓦解党。早在今年 8 月,我们就从重
庆获得完全可靠的消息说,国民党决定派遣奸细混入延安挑动您同
王明和党内其他活动家争吵,挑起敌对情绪以反对所有在莫斯科居
留和学习过的人。关于国民党的这一诡计,我已及时预先通知了您。
国民党秘而不宣的打算是,从内部瓦解共产党,从而轻易把它摧毁。
我毫不怀疑,康生的所作所为正在为这些奸细助长声势。请原谅我
这种同志式的坦率。我对您怀有深深的敬意,坚信您作为全党公认
的领袖,定能洞察事物的真相。仅仅由于这一点,我才如此坦率地
同您谈问题。请按我给您发送这封信的方式给我一封回信。紧紧与
您握手。

\bigskip\mbox{}\large\hfill 季〔米特洛夫〕
\footnote{原载《共产国际与中国革命(文件资料集)。页 295-96(莫斯科:
1986)》,引自《国外中国近代史研究》,第 13 辑,郑厚安译(北京:中国
社会科学出版社,1989 年),页 2-3.} \end{quote}

 季米特洛夫来电是一个严重事件, 1943 年 5 月共产国际解散以後, 自 毛泽东
 已彻底放开了手脚,事实上,当毛决定向国际派摊牌之时,他就没 有把莫斯科太
 多放在眼中。但是问题还有另外一面:共产国际虽解散了, 苏共和苏联并没解散,
 现在莫斯科已完全知悉延安党内高层斗争的最新动 态,斯大林通过季米特洛夫,
 以间接的方式对毛泽东发出警告,并且特别 关注王明、周恩来的政治命运,似乎
 也影射到毛泽东的个人品质问题。季 米特洛夫的来电特别提到康生,直指康生行
 为可疑,此说亦对毛泽东构成 沉重打击。

 接到季米特洛夫来电後,毛泽东立即精密部署,\footnote{接到季米特洛夫 1943
 年 12 月 22 日来电後,毛一时情绪激动,他在 1944 年 1 月 2 日通过苏联驻
 延安观察员给 季米特洛夫发出一份覆电。毛声明中共没有削弱对日本的斗争,与
 国民党合作的方针也没有改变。针对季氏对周恩来、 王明的关心,毛答复道:
 「我们与周恩来的关系是好的,我们毫无把他开除出党的打算。周已经取得了相
 当大的进步。」至于王明,毛掩饰不住心中的愤恨,在电文中说「王明一直从
 事各种反党活动」,「在我看来,王明是不可靠的」。毛举 出两个例子予以说
 明:一、王明过去被国民党逮捕过,在狱中承认了自己的党员身份,後来才被释
 放出来(在弗拉基 米洛夫的《延安日记》中也提到作者本人强烈感受到毛对王明
 的痛恨,在 1943 年 11 月 29 日的日记中,弗拉基米洛夫 写道,针对王明的新
 指控是「国民党同谋,反革命」,证据之一是王明曾被国民党逮捕,又给放了出
 来。参见《延安日 记》,页 190、185-86);二、王明与米夫的关系可疑。毛
 对康生则表现出完全信赖的态度,他告诉季米特洛夫「康生是 一个值得信赖的人」
 。一天以後,毛又後悔日前发出的电报可能会造成远方的误解,于是找到弗拉基
 米洛夫,询问昨天 的电报是否发出,他告诉苏联观察员,前电可能不妥。紧接着,
 毛开展对苏联人的热情公关,据弗拉基米洛夫记载,1 月 4 日,毛泽东夫妇单独
 邀弗氏同观京剧,毛向弗氏大谈他如何尊敬苏联,尊敬斯大林,尊敬那些过去在
 苏联学习过 的中国同志,以及如何感激季米特洛夫,参见《延安日记》,页
 199-200. 1 月 6 日,毛、刘、周邀请弗氏等苏联人畅 叙友情。1 月 7 日,毛
 单独访问弗氏,再一次谈他如何深深地尊重斯大林和季米特洛夫,参见《延安日
 记》,页 203.  在谈话中,毛完全改变了原先对王明的强烈敌对态度,其态度
 之友善使弗氏大吃一惊,毛请弗拉基米洛夫再给季米特 洛夫发一电报,并告诉弗
 氏,团结的方针同样适用于王明。参见〈弗拉基米洛夫转毛泽东给季米特洛夫的
 电报〉(1944 年 1 月 3 日)〈弗拉基米洛夫转毛泽克给季米特洛夫电及情况
 说明〉, ,引自杨奎松: 〈毛泽东发动延安整风的台前幕後〉, 载《近代史
 研究》,1998 第 4 期,页 51-54. 另参见《延安日记》,页 190、185-86、
 199-200、202-205.  }除了频频向苏联驻 延安代表详剖心迹,强调整风的重要和
 他的光明正大,又派任弼时、周恩 来与苏联代表谈话,用任、周等的嘴,澄清毛
 整人的「流言」。毛泽东同时 加紧对王明的「诱」、「压」,迫使王明承认错
 误,让莫斯科无言以对。

毛泽东出台的措施可谓周密完善:莫斯科要求停止党内斗争,毛偏在
此时召开上层会议,逼使所有同僚检讨、反省,用周恩来、王明等人的检
讨堵住莫斯科的嘴,给莫斯科造成既成事实;莫斯科指责康生的反奸肃特
是执行敌人的分化破坏阴谋,纯属胡说八道,延安的整肃全在毛的一手指
挥下进行;莫斯科讨厌康生,正说明康生对毛的忠诚不贰,毛全然不顾莫
斯科的警告,照样倚重康生。
 
然而,在季米特洛夫来电後继续抢救、反奸的极端行为,似乎已显得
不妥。莫斯科已明确提出反对意见,此时的苏德战场形势已明显有利于苏
联,而中共的未来将有赖于斯大林的支持,对莫斯科的意见毕竟不能完全
置之不理;党内怨言继续蔓延终将损害毛泽东的个人威信,况且,审干、
反奸、抢救所要达到的震慑人心的目的已基本实现,现在应是调整政策的
「适时」时候了。

正是在上述背景下,1943 年 12 月 22 日中央书记处召开工作会议, 讨论听取康
生作的反特务斗争的汇报,任弼时在发言中提出,那种认为百 分之八十的新知识
分子是特务分子的看法应于否定,新知识分子中的百分 之八十至九十是好的,现
在应该进行甄别。毛泽东接受了任弼时的意见, 同意进行甄别工作。
\footnote{《胡乔木回忆毛泽东》,页 278-80.}在这次会议之後,延安的「抢救」
开始逐渐落潮,但 是,毛泽东精密掌握落潮的速度,不使运动骤然停下,避免广
大干部对运 动的「合理性」产生怀疑。1944 年初,延安各单位纷纷接待绥德县
「坦白 运动先进典型报告团」, 该团由绥德师范师生组成, 他们住在社会部所
属的 交际处租用的旅店,每天分头到各机关、学校做「现身说法」式的报告。其
中一个十二、三岁的女学生,描述自己怎样受国民党特务机关派遣,专 门施用
「美人计」引诱革命干部……尽管毛泽东已开始着手准备「纠偏」, 但是却放
任「抢救」、坦白的闹剧继续演下去。


对钱来苏一案的处理,也反映出毛泽东欲维护「抢救」的复杂心态。
自「抢救」运动开始,一直被软禁在交际处的钱来苏心情极为抑郁,多次
表示後悔当初投奔延安。
林伯渠等人欲救无力,
只能等毛泽东的最後发话,
1944 年 2 月 8 日,
毛泽东在交际处处长金城呈交的有关钱来苏情况的报告
上批示:
\begin{quote}
	\fzwkai 金城同志:
钱拯(即钱来苏,引者注)应优待他,他可能不是汉奸,他的
子婿是否特务,也还是疑问,如不是,应平反的。
\footnote{金城:
《延安交际处回忆录》
(北京:中国青年出版社,1986 年)
,页 186.}
\end{quote}

在这个批示中,毛泽东虽然提出应予钱来苏优待等,但没有用明确的
语言肯定钱来苏及子婿不是汉奸、特务,毛泽东的模棱两可,为保留「抢
救」成果预埋了伏笔。

\section{甄别:在毛泽东「道歉」的背後}

1944 年春夏之际,审干、抢救运动进入到甄别阶段,各机关、学校
原有的审干小组一变为「甄别委员会」
,仍由原先领导审干、抢救的班子负
责对干部的甄别工作。
 
 
所谓「甄别」,有异于「平反」。若干结论有不实之处,予以改正,谓 之「甄
别」。「平反」则是推翻原有错误结论,给蒙冤的对象恢复名誉。延安的审干、
抢救的纠偏工作,名曰「甄别」,不称「平反」,其含义即在此。

甄别绝非一风吹,而是将受审坦白的人员划分成六类。据 1994 年出
版的《胡乔木回忆毛泽东》一书透露,1943 至 1944 一年内,延安清出的
「特务」共一万五千人, \footnote{《胡乔木回忆毛泽东》
,页 280.}现在就是要对这一万五千人作出具体的划分:

「第一类是职业特务。他们是受一定的特务机关或特务人员的主使, 对我们进行
过或进行着特务工作(长期埋伏,也是一种潜伏工作),确有真 凭实据的」。
「但这类职业特务是极少数, 仅占全体坦白分子百分之十左右, 其中又有自觉被
迫首要胁从之别」。

「第二类是变节分子。其中有的破坏过党的组织, 捉过人, 杀过人的; 有的自
首过写过反共文件,但未做过其它坏事的;有的被敌人短促突击, 接受了敌人的
任务, 但回来既未实行也未报告的; 有的是内战时做过坏事, 抗战後中立或改
过的等等」。「这类人在坦白分子中也是少数」。

「第三类是党派问题。他们加入过国民党、三青团或其它党派。在加
入我党後并未向党报告,但还不是特务分子,这类人亦占颇大数目」。

「第四类是被特务利用和蒙蔽的分子。
有的是在敌人红旗政策下不自
觉地被特务利用和蒙蔽的分子,有的因半条心或幼稚无知,作了特务的工
具」。

「第五类是党内错误。如假造历史,虚报党龄,与坏人来往,泄露秘
密、包庇亲友、政治错误及贪污腐化等等,在坦白运动中被怀疑而误认为
特务」。

「第六类是在审干时完全弄错或被特务诬害的」,「这类人虽然是少
数,但确实是有的,甚至在逮捕的人中也还有的」\footnote{参见〈中央关于坦白分子的六种分析给各地的指示〉
(1944 年 1 月 24 日)
,载《中共党史教学参考资料》
,第 17
册,页 387;另参见《刘少奇年谱》
,上卷,页 435;王素园:
〈陕甘宁边区「抢救」始末〉
。载《中共党史资料》
,第 37
辑,页 225.}。

从以上对六类被审人员的划分依据看,被审查人员或多或少都有问
题,完全搞错的只占一小部分,且放在最後一类,以示审干、抢救的成绩
巨大,缺点是次要的。在上述划分标准中,中共中央仍坚持原先对国民党
所谓「红旗政策」的判断。显而易见周恩来对「红旗党」的辩诬根本没被
毛泽东等接受,中共在国统区尤其在西南地区的地下组织,在政治上仍不
被充分信任。

对于这六类人员如何处理,中共中央也做了规定:

对坦白了的特务分子和变节分子,如果证据确凿,采取一个不杀,团
结抗日的政策;如果没有真凭实据,不要加以深追,以免造成对立僵局,
有碍争取;或中敌人诬陷同志的奸计。

对有真凭实据的暗藏的破坏分子,应继续执行宽大政策。即以宽大为
主,镇压为辅;感化改造为主,惩罚为辅,给改过自新者以将功赎罪的出
路。

对一时审查不清的重大嫌疑分子,不要急于求得解决,以免造成逼供
信。可以有意识地放松一个时期,或暂时按他讲的作一个结论,然後继续
进行调查研究和秘密侦查。

对有党派问题的,被欺骗蒙蔽的,或仅属于党内错误这三种人,在分
清是非後,均应平反,取消特务帽子,按其情况,作出适当结论。对完全
弄错或被特务诬陷的,一经查清,立即平反\footnote{参见王素园:
〈陕甘宁边区「抢救」始未〉
,载《中共党史资料》
,第 37 辑,页 225-26、226.}。

上述这个规定, 有许多似是而非、 自相矛盾之处, 既然没有真凭实据, 为何不
立即解脱,何以谈上「争取」,还要争取什么?所谓「被特务诬陷」, 大量的
检举揭发全是在逼供信下发生的,这些干部都是被迫指咬旁人的, 又何以能称之
为 「特务」?更有甚者, 对「一时审查不清的重大嫌疑分子」, 还布置「继
续进行调查研究和秘密侦查」,只是在表面上「有意识地放松一 个时期,或暂时
按他讲的做一结论」。

有关甄别政策中所隐含的深意,在毛泽东对延安受审干部的「道歉」
中也充分地展现出来。

从 1944 年春夏至 1945 年春,由甄别所引发的延安广大干部对审干、 抢救的强
烈不满处于半公开的状态,在这种情况下,毛泽东先後在行政学 院、中央党校、
边区政府等场合,向延安干部「脱帽」、「道歉」。毛泽东绝 口不提「抢救」为
一错误运动,只是说一两句:运动搞过火了,使一些同 志受了委屈云云。1944 年
元旦,军委三局局长王铮带领一批原受审查、刚 被解脱还没做结论的干部给毛泽
东拜年(军委三局承担延安与各根据地的 电讯往来业务, 工作极其繁重, 故最
先解脱), 在毛住所前黑压压站了一片, 目的在于向毛泽东讨一个说法。毛泽
东似真似假地说, 本意为同志们洗澡, 灰猛氧放多了一些,伤了同志们娇嫩的皮
肤。毛泽东虽然向大家敬一个礼 表示「道歉」,但言下之意又似乎在暗责受审干
部斤斤计较,对接受党的审 查耿耿于怀。

毛泽东的这种暧昧的态度,当然影响到各单位的甄别复查工作。自甄 别展开後,
经过三个月的复查,延安仅甄别了八百人,占坦白人数的四分 之一。中直机关、
边区政府、中央社会部、边区保安处、中央党校、延安 大学、陕甘宁晋绥联防司
令部等七个单位,经过八个月的甄别,在 487 人 中被确定为「特务」的有六十四
人, 「叛徒」四十一人,合占总甄别人数的 22\%。其中康生直接掌管的中央社会
部甄别二十七人,定为「职业特务」(当时的术语, 「特务」分「职业与非职业」
两类)的有六人, 「叛徒」二 人,两者占甄别人数的 30\%。由周兴任处长的边
区保安处,甄别了九十六 人,其中定为「特务」的有三十六人, 「叛徒」二人,
占 40\%。\footnote{参见王素园:
〈陕甘宁边区「抢救」始未〉
,载《中共党史资料》
,第 37 辑,页 225-26、226.} 在甄别 复查中,将「特务」、「叛徒」的比例定得如此之高,无非是
企图证明,开 展审干、 「坦白」、「抢救」是完全正确的。

中央党校作为「抢救」的重点单位直到 1944 年 9 月才全部转入甄别 阶段。一部
第六支部书记朱瑞与薄一波、倪志亮等组成一个小组。协助对 集中较多问题人物
的特别支部进行甄别工作。朱瑞等同情危拱之的遭遇, 认为危拱之在「抢救」中
虽然有自杀行为,并提出退党要求,但这都是精 神错乱所致, 「结论是没问题」
。然而朱瑞等对危拱之的结论却受到党校一 部负责人的批评,认为朱瑞等「代危
抗辩,没有原则立场」,朱瑞等为此作 了大量的工作, 最後才解决了危拱之的
结论问题。\footnote{郑建英: 《朱瑞传》,页 294.} 从这件事可以看出,
「抢 救」受害者的甄别工作往往会出现波折,一旦被「抢救」,要想完全洗清罪
名,并非容易。

甄别、复查进展缓慢,激起延安广大干部的不满,为了平息、舒缓干 部中的不满
情绪,更重要的是,进入 1945 年後,国内、国际形势急剧变 化,客观上要求尽
速解决大量积压的审干、抢救遗留问题。在这种背景下, 甄别、复查工作的进度
有所加快,到了 1945 年春,延安各单位的甄别工 作已基本结束,对 2,475 人
作出了组织结论。\footnote{参见王素园: 〈陕甘宁边区「抢救」始末〉,载
《中共党史资料》,第 37 辑,页 228. 1943 至 1945 年,延安有三 万党员及
非党员干部,受抢救冲击的大多为抗战後投奔延安的青年知识分子干部,也有相当
数量的老干部,主要为国 统区中共地下组织的领导人,以及从苏联返回的原留苏
干部,据胡乔木提供的数字,被抢救的干部达一万五千人。}这个数字也许只是当
时被 关押进几个重要反省机关的「重犯」被甄别的数目,因为根据胡乔木透露 的
数目,延安「特务」的总数为一万五千人。

即使受审人员被作了结论,也绝非意味着万事大吉。凡受审人员均按
四种情况分别给予不同等级的结论:

问题已澄清,完全可以做结论的;

对有关被查重点疑问问题做部分结论的;

在结论中留有待查尾巴,即仍有疑点,有待再查的;

无法查证,不予结论的。

在总计 2475 受审人员的结论中,有党派政治问题的约占 30\%,其中
叛徒、特务、自首三类人员各占 10\%;党内错误问题,约占 40\%,完全
弄错的约占 26\%,保留疑问不予结论的约 4\%。\footnote{参见王素园:
〈陕甘宁边区「抢救」始未〉
,载《中共党史资料》
,第 37 辑,页 228、221.}对于这一部分人的处理
方法,谁都不敢作主。直到毛泽东最後发话,大意是现在东北快解放了,
需要大批干部,让他们到前线自己去做结论吧,是共产党人,一定留在共
产党内,是国民党人让他跑到国民党去,怕什么呢。毛泽东虽然是这般说
了,然而延安不仅没有释放任何一个像王实味这样已「定性」的人,那些
未做结论的干部,也没有按照党的组织程序分配至各单位,而是仍然受到
社会部的监控。这些干部虽然没有跑到国民党去,但他们在政治上还是继
续受到怀疑与歧视。他们档案中的「疑点」和「尾巴」
,在 1949 年後仍然
被长期揪住不放,给当事人带来无穷的灾难,使他们的大好年华全被葬送
在连绵不断的审干、肃反等运动中。

彭而宁——钱来苏之子,背着沉重的「特嫌」黑锅,在 1949 年後每
一次审干运动中受审,直到 1980 年康生被揭露和清算後,才获彻底平反。
一位匿名女干部,当年曾被诬为「日特」兼「国特」的「双料特务」,
八十年代沉痛地回忆说,一生前後被审查长达十四年,前七年是我一踏进
革命的门就受审查,还是美好的青年时期,当时只有十九岁,後七年正当
壮年,是可以很好工作的时期,都丧失在康生的反动血统论和主观主义的
逼供信下了\footnote{参见王素园:
〈陕甘宁边区「抢救」始未〉
,载《中共党史资料》
,第 37 辑,页 228、221.}。

张克勤——当年康生精心培养出的一个坦白典型,
康生在抢救高潮时
还以张克勤为例,自夸自己已将反革命特务分子转变为革命服务。到了
1945 年甄别时期,康生还不放过他,坚决拒绝为张克勤做结论,康生要将
张克勤作为证明抢救正确性的「成果」继续保持下去。1945 年 11 月,张
克勤随社会部部分干部向东北转移,经历了严峻的考验,由于得到社会部
第三室主任陈龙的关心和照顾,曾一度被安排在北安市公安局担任股长,
但其「问题」一直无法解决。1949 年 11 月,又是在陈龙的关照下,张克
勤随陈龙从哈尔滨去北京,经中组部介绍去西北局澄清其历史问题。直到
1950 年 4 月,
「经中央有关部门批准,组织上才作了历史上没有问题的结
论」
,张克勤的党籍得到了恢复,此时恰是康生在政治上失意,自我赋闲的
阶段。张克勤以後虽曾官至中共兰州大学党委书记,但在各种运动中均被
波及。1986 年,时任甘肃省政协常委的张克勤无限感伤地回忆道:
「今年
是我参加革命五十周年,五十年中一半时间是在挨整」。「1943 年『抢救』
开始就戴上『特务』帽子。抗战胜利後,戴着『帽子』调到东北」,「1959
年又打成『右倾机会主义分子』,『文革』中又被康生点名,关了五年半监
狱」。

延安还有一位叫蔡子伟的干部,
曾任边区中学校长, 在《谢觉哉日记》
中,还有他在 1938 年 9 月活动的记载,以後此人即从延安公众生活中消
失。蔡子伟被长期关押,详情外界不知,此人在八十年代曾任全国政协委
员。

延安最後一批人的甄别,是在 1945 年 8 月日本投降後进行的。这批
人全是边区保安处关押的重犯,总数约五六百人。由于当时中共中央要派
大批干部前往东北,催促社会部和保安处抓紧甄别工作,1945 年 11 月 9
日,中社部负责干部陈刚和陈龙率二百多干部步行前往东北。陈刚,四川
人,即富田事变中被扣的中央提款委员刘作抚。他在从江西返回上海後长
期领导中央交通局,1932 年与何叔衡之女何实山结婚,1935 年刘作抚和
孔原秘密前往苏联,
何实山稍迟也抵苏。
1937 年底,
刘作抚夫妇回到延安,
1938 年春参与组建「敌区工作委员会」
,主办了八期秘密工作干部训练班。
延安时代刘作抚早已易名为陈刚,在中社部主管人事,1945 年参加了中共
七大,1948 年 12 月被任命为中社部副部长,从 1956 年起,陈刚任中共四
川省委书记,1963 年升任西南局书记处书记。1945 年被陈刚带往东北的
干部,其中有一半人是被「抢救」而未作甄别的,这批人多在建国
初才得到甄别。

最後,对那些留在延安参加甄别的犯人,保安处决定,由他们本人
甄别自己,给自己写出结论,再交保安处三科审阅後,本人签字定论,到
1946 年上半年,被关押的大部分人基本甄别完毕。

和那些已作结论或虽然带着
「帽子」
仍然被派往东北的那批前
「犯人」
相比,被继续关押在保安处的一百多个人的命运就太不幸了,这批人中有
王实味等,他们将被作为抢救审干的牺牲品送上祭坛。1947 年春,国民党
军队进攻延安,保卫部门押着这批「犯人」向山西临县转移,经康生批准,
于黄河边全部被处决\footnote{参见仲侃:
《康生评传》
,页 95.}。

这批被杀的人,除了王实味,都没有留下名字(当然,原保安处会有
这批人的详细档案)。与此同时,在山西晋绥根据地贺龙辖区,也相继处
决一批受审人员,被誉为「爱国五青年」的蔺克义,就是被冤杀者之一。

蔺克义被捕前为晋绥《抗战日报》社出版发行部主任, 1936 年他在西
安师范读书时即参加了中共地下党,「在兰州、西安等地从事地下秘密工
作和抗日救亡工作中,一直表现很好。在与国民党反动当局进行斗争中,
立场坚定,勇敢顽强」。他在 1939 年 9 月到延安,先後在中央青委、中央
出版发行部等单位工作, 1940 年冬被派至晋西北。延安「抢救」展开後,
有人被逼供咬出蔺克义是「特务」,检举材料由中央社会部转到晋绥《抗
战日报》,蔺克义以「特嫌」被单独看管,最後被转至晋绥公安总局关押
审查。「1947 年胡宗南进攻延安,一位负责人指示,要求在历史悬案(指
整风中关押起来的)中,罪行比较严重的处死一批」,蔺克义便成了这个
「负责人」的刀下鬼,时年仅二十九岁,其冤案直到十一届三中全会後才
获平反\footnote{参见王素园:
〈陕甘宁边区「抢救」始未〉
,载《中共党史资料》
,第 37 辑,页 223.}。

另据师哲披露,
在 1947 年山西被处决的人中间,
还有四名外国人。
1944
年初,有四名外国人从晋察冀边区经晋西北押送到延安,其中三名是俄罗
斯人,一名是南斯拉夫人,他们本是假道中共根据地设法去南洋或澳洲谋
生的。这四名洋人被康生交边区保安处长期关押。直到 1947 年初,随其他
「犯人」向山西永坪转移。康生在转往山西参加土改、途经永坪时,下令
将这四名外国人秘密处决,事後把尸体全部塞进一口枯井,以後被国民党
胡宗南部发现,造成很大轰动,彭德怀、周恩来、陆定一均表示了强烈不
满(毛泽东呢,转战陕北时,周恩来、陆定一一直和毛泽东在一起,他不
可能不知道此事),于是保安处处长周兴代康生受过,「只好自己承担责
任,受批评,挨斗争」\footnote{师哲:
《峰与谷——师哲回忆录》
,页 217、216.}。

在被杀、或被释放解脱的人之外,还有另一类人,他们活着被抓进社
会部或保安处,却再也没见他们出来。在这些神秘失踪的人中间,有一个
叫王遵极的姑娘, 1939 年奔赴延安时,
年仅十九岁。据师哲称,王遵极「长
相漂亮,举止文雅」,因其是大汉奸王克敏的侄女,一来延安就被关押,
经反复审查还是没发现问题,经办此事的师哲建议「在一定条件」下释放
她,却遭到康生及其妻曹轶欧的坚决反对。师哲称,「其中原委,始终令
人不解」,王遵极以後下落不明。\footnote{师哲:
《峰与谷——师哲回忆录》
,页 217、216.}(另据仲侃《康生评传》称,王遵极
从 1939 年至 1946 年在延安被长期关押,暗示她在 1946 年 获释,但未交代此
人以後的行踪。参见该书,页78. )

从 1942 年揭幕的审干、坦白、反奸、抢救运动,到 1947 年王实味、蔺
克义等被秘密处决,终于完全落幕。1945 年 3 月,蒋南翔给刘少奇写了一
份(关于抢救运动的意见书),对于这场灾难进行了较为公允的批评(蒋
南翔批评抢救运动「得不偿失」)。然而刘少奇本人也与这场运动有千丝
万缕的联系,他的主要部属彭真更是运动的主要领导人之一,因此刘少奇
不敢、也不愿对这场由毛泽东亲自主持,康生幕前指挥的运动说些什么。
不仅如此,蒋南翔的〈意见书〉还被认为是「错误」的,蒋本人也受到了
党内批评。从此,「抢救」的历史被彻底掩埋,凡经历过这场风暴的人们
都知道应对此三缄其口,人们从书本、报刊、报告中只知道「伟大的整风
运动」,一直到毛泽东、康生离世後的八十年代初,有关「抢救」的内幕
才陆续被披露出来,此时已距当年近四十年。
