\chapter{毛泽东对王明的重大胜利} 
\section{毛泽东迂回反击王明}

自 1937 年
11 月底王明返国到 1938 年春夏之间, 王明和周恩来等在统 一战线问题上形成
的共识,在相当大的程度上扭转了中共核心层原由毛泽 东主控的气氛,迫使毛泽
东不得不作出拥护中央新方针的姿态。十二月政 治局会议後,毛泽东被迫降低以
往突出强调「独立自主」的调门,转而发 表若干支持统一战线的言论。1937 年
12 月 24 日, 毛致电边区各军政首长, 要求在与国民党「共同负责,共同领导,
互相帮助,互相发展的口号下」, 扩大和巩固统一战线,「避免讥笑与讽刺」友
党、友军。毛甚至提出,中 共应「帮助政府进行征兵动员」,「没收汉奸财产及
处理汉奸,必须取得 政府的同意,最好是交给他们处理」。\footnote{〈毛泽东、
肃劲光、谭政致边区各军政首长电〉(1937 年 12 月 24 日),载中央档案馆
编: 《中共中央文件选集》(1936-1938),第 11 册,页 408-409.} 1938 年
1 至 3 月,毛泽东在接见 美国合众国际社记者王公度和在延安大会上几次发表演
讲,都没有再宣传 自己过去所坚持的主张。毛泽东在 1938 年初春的这类言论,
以後都没收入 《毛泽东选集》。
 
如果将毛泽东上述言论视为他已改变了原有的观点,转而接受了王明
等人的主张,
那就大错特错了。
毛泽东实际上一天也没有放弃自己的意见,
他只是在与己不利的形势下,被迫在公开场合调整了自己的态度。在私下
里,毛泽东却在精心地准备向其对手发起反击。

1938 年 3 月政治局会议後,毛泽东加紧酝酿反击王明等的「炮弹」,
毛泽东的「炮弹」就是他在 5 月写成的〈抗日游击战争的战略问题〉,和
5 月 26 日至 6 月 3 日在延安作的〈论持久战〉的演讲。

〈抗日游击战争的战略问题〉有完整的论文结构。毛泽东针对抗战以
来中共党内以王明、周恩来为代表的强调运动战的观点,详细论述了游击
战争在抗战中的重大战略作用和对共产党建立抗日根据地的意义,完善了
他在洛川会议上及以後在游击战问题上的主张。
\footnote{毛泽东的〈抗日游击战争的战略问题〉
,原是为延安解放社编印的《抗日游击战争的一般问题》一书所写的第七
章。最早刊载于 1938 年 5 月 30 日出版的《解放》周刊第 14 期。1952 年收入《毛泽东选集》第 2 卷时,毛作了文字上
的修改。}

〈论持久战〉是毛泽东的一篇演讲大纲,在这篇演讲中,毛不指名地
批评周恩来等人在抗战初期「怀疑」、「轻视」游击战的战略地位。周恩
来过去曾批评过孤立强调游击战的看法是「机械论」的观点,现在也被毛
单独拎出来,被指责为是「不赞成八路军的战略方针」。毛在演讲中,以
不容置疑的口气,将党内的不同意见一概视之为旁门左道。

毛泽东发表上述两篇论文正逢其时。1938 年春,日军已开始进攻武汉 外围, 长
江局在王明、 周恩来的领导下, 正配合国民党展开轰轰烈烈的 「保 卫大武汉」
的活动。但是,毛泽东根本不相信「东方马德里」能守住。他 认为, 国民党丢失
武汉是迟早要发生的事情, 此将有力证明毛泽东有关 「国 民党片面抗战必然失
败」的论断,也是对鼓吹「保卫东方马德里」的王明, 周恩来的沉重打击。正因
为毛的文章暗藏玄机,王明、博古、项英、凯丰 一致决定,《新华日报》不予转
载毛泽东的〈论持久战〉\footnote{王明:
《中共五十年》
(北京:现代史料编刊社,1981 年)
,页 192.}。

更为重要的是,毛泽东有关只有开展游击战,建立根据地才能壮大中 共的意见,
已日益被事实证明是有远见的。1937 年 12 月政治局会议後, 毛泽东虽处于少数,
但他对派遣八路军东进毫不放松。1938 年 3 月 24 日, 毛泽东与刘少奇联名致
电朱德、彭德怀、朱瑞等,责成他们「必须立即组 织以八路军名义出现的游击兵
团」,为八路军主力转移至华北各地区事先 作好准备。同日,毛泽东、张闻天、
刘少奇又致电中共北方局代表朱瑞, 令其 「以最快速度」, 建立 「完全在党领
导下的有战斗力的若干游击兵团」, 「创造冀晋豫边区」。\footnote{参见中央
档案馆编: 《中共中央文件选集》(1936-1938),第 11 册,页 475-80.}毛
泽东还采取措施,严防王明等对八路军的干预, 在发给军方将领的电报中,毛泽
东总是将自己的名字放在第一位,显示他 对巩固自己在军中个人影响的极度重视。
尽管王明在 1938 年春正处于他政 治上的鼎盛阶段, 但在毛的隔离下, 王明在
军队中没有担任任何领导职务。

在毛泽东的全力催促下,1938 年初,朱德、彭德怀、刘伯承、邓小平、 贺龙、
聂荣臻等率八路军陆续深入华北各地, 相继建立了共产党的根据地, 仅一年时间,
八路军就从 1937 年 9 月的不足三万人,发展到 1938 年秋的 二十五万人。
\footnote{《彭德怀自述》,页 229.}中共武装发展壮大的事实,使朱德、彭德
怀等八路军领导 人信服了毛泽东,因而放弃了过去的观点,转而接受了毛泽东的
意见。

八路军高级军事将领回到「正确路线」,使毛泽东大大舒缓了一口气,
但还有一个更大的障碍横亘在毛的前面,这就是如何调整与莫斯科的关系
问题。

在毛泽东为夺取中共最高领袖而展开的斗争中,如何处理与莫斯科的
关系一直是最为棘手的问题。随着王明的返国,共产国际对中共的影响明
显加强,王明「拉大旗,做虎皮」,处处以莫斯科代言人自居,对毛的活
动造成严重的掣肘。毛泽东深知如果得不到莫斯科的支持——至少是表面
的支持,要想执掌中共最高领导权,几乎是不可能的。为此,毛暂且压住
对斯大林的不满,竭力显出顺从的姿态,同时又施展种种手段,以图嬴得
莫斯科的好感与支持。三月政治局会议决定派遣任弼时赴苏联,客观上为
毛「借洋师助剿」王明提供了便利的条件。

具有讽刺意味的是,提议派任弼时赴苏汇报的竟是王明。任是王明的 老熟人,
1931 年初六届四中全会上, 在 两人同时进入政治局, 王明返国後, 任也和政
治局其他同事一样,对王明传达的共产国际指示表示拥护,于是 王明就认为任是
自己可以联合的盟友。殊不知,上层风云可以瞬息万变, 到了 1938 年春, 已有
迹象显出任已明显偏向毛泽东一边, 而王明竟浑然不
知。于是王明无意中就做了一件令毛拍手称快的事,毛实在有太多的理由
为王明的这项提议感到高兴。
 
任弼时是受到共产国际信任的中共元老之一,
1921 至 1923 年旅苏三年,在莫斯科「东方劳动者大学」(东方大学)学
习期间,曾出席过远东各国共产党及民族革命团体第一次代表大会和共产
国际第三次代表大会。1931 年中共六届四中全会上,
在
任弼时又被共产国
际东方部部长米夫指定为中央政治局委员。任弼时与毛泽东也有较深的历
史渊源,他早年曾参加由毛组织的湖南「俄罗斯研究会」,并由毛推荐,
进入上海「外国语学社」(中共发起组织的掩护机构)学习俄语,旋被派
往苏俄留学。1931 年春,任弼时被六届四中全会後的中共中央派往江西苏
区,一度全力支持毛,打击了毛的对手项英,与毛的关系又亲近了一层。
1937 年夏至 1938 年春,任弼时在一度摇摆後,较早在毛泽东与王明之间
作了选择,明确支持毛的各项主张,并在 1938 年 3 月政治局会议上,公开
站在毛泽东一边,反对王明前往武汉工作。由于任弼时既为莫斯科信任,
又与毛泽东接近,遂被毛视为是派往莫斯科充当高级说客的最佳人选。

几乎在任弼时前往苏联的同时,1938 年 4 月下旬,毛泽东也向即将被 派往苏联
伏龙芝军事学院学习的刘亚楼交待了任务。毛为了考察刘亚楼及 提高刘的「觉
悟」,已让刘在自己的身边帮助整理文件三个月。毛向刘亚 楼布置道, 有一件事
情让你去做: 把我党历史上的几次左倾错误及其危害, 把我党当前推行的抗日民
族统一战线政策直接向共产国际季米特洛夫汇 报,并多带几套〈实践论〉、〈矛
盾论〉和〈中国革命战争的战略问题〉当面交给季米特洛夫, 并请他转呈斯大林。
刘亚楼当时只是一名年轻干部, 毛且用心如此之深,更遑论对任弼时了。
\footnote{刘亚楼赴苏後,曾参加了 1939 年共产国际讨论李德问题的委员会,
「并认真稳妥地完成了毛泽东交给的任务」。参见杨万青、齐春元: 《刘亚褛将
军传》(北京:中共党史出版社,1995 年),页 175-76.  } 

1938 年 3 月 5 日,任弼时从延安出发,在西安搭乘苏联运输卡车,于
兰州搭乘苏联飞机经迪化抵达莫斯科後,于 1938 年 4 月 14 日代表中共中
央向共产国际主席团提交了〈中国抗日战争的形势与中国共产党的工作与
任务〉报告大纲。任弼时的这份大纲巧妙地揉和了毛泽东和王明的观点,
既迎合了斯大林,又不失时机,为毛泽东在共产国际打下了楔子,「挂上
号」。

任弼时在「大纲」中高度评价了王明返国後传达的季米特洛夫有关发 展中国统一
战线的指示。他声称, 中共党内原先对统一战线的认识存在 「某 些不足够的地
方」,对国民党有「深刻成见」,与国民党政府和军队有「某 些摩擦」, 「在十
二月政治局会议以後乃渐减少」。任弼时说:中共已「认 定」王明提出的「抗日
高于一切」,「一切服从抗日」、「民主、民生、 皆在其次」的主张,在发布了
中共中央十二月宣言以後, 「国共两党关系, 在基本上有了进步」
\footnote{1937 年 12 月 25 日,由王明起草,经中共长江局集体讨论通过,以
中共中央的名义在武汉发出〈中共中央对时 局的宣言〉,毛泽东当时并未加以反
对,以後又在中共六届六中全会政治报告〈论新阶段〉中加以肯定。但以後,此宣
言被指责是王明等背着中央发出,集中体现了「王明右倾投降主义」的错误主张,
受到长期批判。参见史锋: 《反对王 明投降主义路线的斗争》(上海:上海人
民出版社,1976 年)页 34.}。

「大纲」对王明、周恩来等提出的「保卫武汉、河南与陕西」的口号, 也给予了
积极的评价。任弼时强调,这是目前中共「最重要任务中」的「最 中心的一环」。
他还表示,中共中央将「纠正『左』的关门主义的工作方 式」,「决定于最近半
年内,召集党的第七次代表大会」。

然而,任弼时此次赴苏的目的,绝不是为王明等做「宣传员」,他之 所以在给共
产国际的汇报大纲中反映王明等的观点,乃是王明的主张已被 中共十二月政治局
会议和三月政治局会议所接受,已成为现阶段中共的主 导性方针。尽管如此,任
弼时仍不失时机地要将毛泽东的看法反映出来。「大纲」指出,「蒋介石和国民
党内一部分人,不愿国共两党以平等地位 合作,而企图在合作中削弱共产党」,
因此,中共将「以最大努力,扩大 八路军和新四军」, 「巩固共产党在八路军、
新四军中的绝对领导」, 「阻 止外界恶劣影响的侵入」。\footnote{任弼时给
共产国际的上述两份报告,因对王明及「十二月政治局会议」多有赞誉之词,因此
一直未予公布。直 到 1986 年, 才被收入由童小鹏等编辑的 《中共中央抗日民
族统一战线文件选编》。参见中央统战部、 中央档案馆编: 《中 共中央抗日民
族统一战线文件选编》(下)(北京:档案出版社,1986 年),页 104-105、
110-111、113;122.  }

任弼时的「大纲」送交共产国际後,在将近一个月的时间里,没有得 到任何反应。
任弼时当机立断,迅速调整策略,在送给共产国际的菜肴中, 再添加一把王明的
佐料。1938 年 5 月 17 日,任弼时又向共产国际执委会提交一份关于 4 月 14
日报告大纲的「补充说明」。任弼时在这份「补充说明」中,提高了对王 明的热
度,进一步陈述了王明的观点。「补充说明」突出强调了王明返国 後, 在对中共
中央完善统一战线策略方面所作出的贡献。任弼时明确指出, 在王明返国前,
之所以造成国共摩擦, 除了国民党方面的因素外, 中共 「过 份强调了独立自主、
民主与民生的要求」,也是「重要原因」。任弼时表 示。令後中共的迫切任务就
是将广大群众组织到统一战线的各种群众组织 中去。\footnote{任弼时给共产国
际的上述两份报告,因对王明及「十二月政治局会议」多有赞誉之词,因此一直未
予公布。直 到 1986 年, 才被收入由童小鹏等编辑的 《中共中央抗日民族统一
战线文件选编》。参见中央统战部、 中央档案馆编: 《中 共中央抗日民族统一
战线文件选编》(下)(北京:档案出版社,1986 年),页 104-105、
110-111、113;122.   }任弼时在这份「补充说明」中,虽然对王 明及其思想观点作出进 一步的
肯定,但这只是策略手段,任弼时的目的是要尽量 消除共产国际对 毛泽东的怀疑,
力争共产国际尽早批准 4 月 14 日提交的报告 大纲。正是由 于任弼时 1938 年
春给共产国际的上述两份报告对王明及 「十二 月政治局会 议」多有赞誉之词,
所以几十年来一直未予公开。

任弼时的策略立竿见影,中和王明与毛泽东观点 的 4 月 14
日报告大纲和 5 月 17 日的「补充说明」,终于获得了共产国际 的肯定。1938
年 6 月 11 日,共产国际执委会主席团通过了〈关于中共代 表团报告的决议案〉,
对于毛泽东,这个决议案中最有价值的一段话就是 共产国际承认「中国共产党的
政治路线是正确的」\footnote{〈共产国际执委会主 席团关于中共代表报告的决
议案〉(1938 年 6 月),载《中共中央抗日民族统一战线文件选 编》(下)
,页 863.  }。

毛泽东当然清楚,这段话中的「政治路线」实指 1937 年 12 月政治局
会议以来的中共路线,但毛却可以对之「移花接木」,因为十二月政治局
会议後,毛的中共首席地位并没有改变,对中共政治路线的褒扬,完全可
以解释成是对毛的路线的肯定。

事实上,毛泽东非常了解共产国际在他与王明的争执中,是明显偏袒
王明一方的。共产国际在 6 月 11 日的决议案中,要求中共「诚实」、「诚
恳」、「积极」、「用一切办法」帮助国民党,甚至提出了中共应协助国
民党,「实行征兵制」,「建立新军」,「发展国防工业」等一系列与中
共毫不相干的建议。然而,毛泽东仍然可以接受这个建议,因为对于不合
口味的莫斯科的指令,毛早就练就了一套灵活应付的本领,或将其搁置一
边,或避重就轻,总之毛不会让莫斯科束缚住自己的手脚。

在获得初步的成功後,任弼时在莫斯科展开了第二步行动,现在任弼 时已毫无必
要再向共产国际大捧王明了,任转身一变,开始为改变王明对 共产国际的「错误
影响」四处奔走。他满怀热情地充当起毛泽东在莫斯科 柳克斯(LUX)大厦(共产
国际办公所在地及驻共产国际各国共产党代表 团驻地)的说客。据当时担任任
弼时俄文翻译的师哲回忆,任弼时不仅 亲自到各国共产党驻苏代表团去宣讲毛泽
东对中国革命的贡献,还把在莫 斯科的一批中共党员干部分别派到各国代表团去
介绍「毛泽东的革命理 论」\footnote{1978 年 11 月,师哲的回忆,载《中共党
史人物传》,第 8 卷(西安:陕西人民出版社,1983 年),页 46.}。

师哲回忆道,任弼时在向共产国际提交了汇报大纲後,曾写过一份有
关中国情况的报告,专门介绍毛泽东的贡献。其中「断定说,只有『毛泽
东才是中国共产党的领袖』」。师哲的回忆没有引证任何文献资料,恐是
师哲根据任弼时当时的政治态度,把任弼时 5 月 17 日提交的「补充说明」
误以为是任弼时向共产国际举荐毛泽东。因为迄今为止,只见到任弼时向
共产国际提交的 4 月 14 日「报告大纲」和 5 月 17 日「补充说明」,而未
见师哲所言的任弼时这份举荐毛泽东的补充报告。如果确实有这份报告,
毛泽东在延安整风中一定会将其在党的领导层中公开。

任弼时在莫斯科的活动在促使共产国际加深对毛泽东的印象方面起到 了重要的作
用,但是共产国际显然未能如毛泽东、任弼时之愿,明确承认 和支持毛泽东为中
共最高领袖。相反,莫斯科却对毛泽东与王明的分歧有 可能造成对中共的损害,
表现出强烈的忧虑。在 6 月 11 日共产国际执委会主席团有关中共代表团的决
议案中,莫斯科敦促中共领导层特别要警惕: 「日本侦探及国民党的反共分子用
一切可能的阴谋诡计」,「在中共领导 同志中」,「造成分歧和纷乱的企图,
来破坏中共领导的集体工作」。\footnote{〈共产国际执委会主席团关于中共代
表报告的决议案〉(1938 年 6 月),载《中共中央抗日民族统一战线文件选
编》(下),页 863.} 因 此,任弼时仍有必要继续留在莫斯科,为毛泽东继续
作改善形象的公关工 作。不久,任弼时未竟成功的使命,竟意外地由同时在莫斯
科的另一人圆 满完成,他就是原属王明集团、後跳槽至毛泽东营垒的王稼祥。

\section{关于季米特洛夫支持毛泽东为中共领袖的「口信」}

毛泽东在为争取中共最高领袖而进行的持续斗争中,在不同的历史阶
段,凭着他个人的政见主张,娴熟运用各种谋略,吸引、争取了许多昔日
属于对立营垒的党内高层人物,使他们成为自己的同盟者。王稼祥就是较
早被毛泽东争取过来的原王明集团的重要成员。1938 年 8 月,王稼祥从莫
斯科返回延安,带回共产国际总书记季米特洛夫关于「承认」毛泽东为中
共领袖的重要口信:「在(中共)领导机关中要在毛泽东为首的领导下解
决,
领导机关要有亲密团结的空气」\footnote{王稼祥:
〈国际指示报告〉
(1938 年 9 月)
,载《文献和研究》
(1986 年汇编本)
,页 70-71.}。王稼祥传达的莫斯科这一重要口信,
在 1938 年充满强烈亲苏气氛的中共党内所发生的巨大效力,
非局外人所能
想象,它简直就是一封莫斯科对毛泽东中共领袖地位的承认书。从此尘埃
落定,毛泽东虽未立即成为中共中央总书记,但已成为事实上的中共最高
领导人。至于王明,一旦遭莫斯科冷遇,则完全丧失了政治上的回旋馀地,
开始迅速走下坡路,最终被毛泽东一脚踢进「历史的垃圾堆」。

毛泽东之所以能够获得季米特洛夫的支持,是与王稼祥在莫斯科开展 的积极活动
分不开的。王稼祥是为医治内战期间所受的枪伤,于 1937 年 6 月下旬在上海秘
密搭乘苏联轮船前往苏联的。\footnote{北京大学国际政治系已故向青教授提出
的王稼祥是由新疆赴苏,于 1937 年春抵达的说法是错误的。参见向青: 《共产
国际与中国革命关系论文集》,页 389;另参见郑育之: 〈王稼祥在上海养伤的
日子里〉,载《回忆王稼祥》,页 78-81. 郑育之系作家周文之妻,夫妇均为中
共地下党员,1937 年 3 至 6 月,王稼祥在沪等候苏联轮船的三个月里,一 直秘
密住在周家。}王稼祥在治病之外,是否 另有特殊 使命于事隔五十年後才真相大
白。1985 年,时任中共中央总书记 的胡耀邦在纪念 王稼祥的文章中称,王稼祥
赴苏系受「中央派遣」,「向 共产国际领导人介绍中 国革命情况, 包括他个人
对中国党的领导的看法」。\footnote{胡耀邦: 〈深 切地纪念王稼祥同志〉
,载《回忆王稼祥》,页 2.} 三十年代曾在共产国际工 作的师哲也说,王稼祥
是「身负重任」来到莫斯 科的。\footnote{师哲: 〈忠 心耿耿,光明磊落——回
忆王稼祥同志〉,载《回忆王稼祥》,页 83.}这里涉及 到一个关键问题, 王
稼祥赴苏向共产国际陈述他 「个人」对中国党领导的意见, 究竟是谁授权的?
在王稼祥启程赴苏的 1936 年 12 月初,在中共中央所在地保安 的政治局委员,
有毛泽东、张闻天、张国焘、 周恩来和博古。张国焘甫抵保安, 因「另立中央」
享受到批评,已心灰意 冷, \footnote{张国焘: 《我的回忆》,第 3 册,页
329.}周恩来、博古等正忙于和西北军、东北军交涉。在毛泽东和 张闻天 两人中,
张闻天授意王稼祥去莫斯科陈述王个人对中国党领导的看法,可 能性极小。答案
只能是一个,是毛泽东授意王稼祥去共产国际开展要求改 变中共 领导的活动,也
唯有毛泽东才会这样做。
 
王稼祥这次在苏联居留有一年时间。王稼祥抵苏後, 王明已准备返国, 从 1937
年 11 月起,王稼祥就接替了原由王明担任的中共驻共产国际代表 团团长的职务,
直至 1938 年 3 月任弼时抵莫斯科, 才转由任弼时接任该职。据王稼祥自述,
1938 年 7 月,在王稼祥返回延安的前夕,共产国际总书记 季米特洛夫曾与他及
任弼时进行了一次重要谈话。这次谈话并没有任何文 字记录,因此无从考证季氏
与王稼祥、任弼时会谈的具体时间和地点。据 王稼祥说,季米特洛夫谈了以下一
段话:
\begin{quote}
	\fzwkai 「应该支持毛泽东为中国共产党的领导人」,「他是实际斗争
	锻炼出来的领袖」,「其他人如王明,不要再争当领导人了」。
	\footnote{王稼祥(遗作):〈回忆毛泽东同志与王明机会主义路线的斗争〉
,载《人民日报》,1979 年 12 月 27 日。} 
\end{quote}

关于季米特洛夫对王稼祥讲的这段话,前苏联中国问题专家季托夫对
它的真实性予以了否定。季托夫在〈抗日战争初期中共领导内部的两条路
线斗争
(1937-1939)〉一文中声称,
王稼祥传达的季米特洛夫的
「指示」,
是毛泽东和王稼祥联手搞的「阴谋诡计」。季托夫说:
\begin{quote}
	\fzwkai 共产国际根本没有(决定毛泽东为中共领袖)那个意思。王稼
祥是在 1937 年初作为毛泽东密使被派往莫斯科的。为了完成毛泽东
的委托,王稼祥本人同共产国际个别工作人员(指季米特洛夫——
引者注)进行了交谈。曾谈到似乎中共中央认为必须选毛泽东当党
的总书记。但是共产国际执委会并没有提出什么建议,认为这个问
题应由中共第七次代表大会决定\footnote{ A·季托夫:
〈抗日战争初期中共领导内部的两条路线斗争(1937-1939)〉,原载苏联《远东问题》1981 年第 3
期,转引自《共产国际与中国革命——苏联学者论文选译》,页 356-57.}。
\end{quote}

在没有进一步史料证实季托夫的
「伪造说」
以前,
笔者倾向于接受
「季
米特洛夫曾向王稼祥表示支持毛泽东为中共领袖」这一说法。虽然「口信」
一事确实存有不少疑点,例如,在共产国际 6 月 11 日通过的〈关于中共代
表团报告的决议案〉中并没有涉及对中共领导人个别评价的内容,为什么
到了 7 月,季米特洛夫竟会在如此敏感的重大问题上,向王稼祥个人作出
明确的表态?为什么共产国际竟会没有这次会见的正式文字记录?种种迹
象表明,季米特洛夫向王稼祥讲述的这番话,在很大程度上是他个人的看
法,而季氏的意见似乎得到斯大林的默认。因为,如果不经斯大林,季氏
似无胆量在如此重大问题上自作主张。1938 年,苏联正处于「肃反」大风
暴中,王明的後台与恩师米夫已被加之以「人民公敌」的罪名遭到处决。
季氏利用这一机会,或为显示其「革命的原则性」,有意对王明表示轻慢;
或受「阶级斗争」之习惯思维之影响,对王明表示某种政治上的不信任,
都是十分可能的。季氏虽与王明私交其笃,王明唯一女儿在其返延安前即
托付给季氏抚养,但在 1938 年大恐怖的血雨腥风中,饱受寄人篱下之苦、
且常遭斯大林轻侮的季氏于惊恐中钦羡毛泽东所创造的革命功业,亦在情
理之中。
斯大林在 1938 年属意毛泽东也不奇怪。
斯大林知道毛是中共实际
的最高领导人,一年前派王明回中国只是为了「帮助」毛,而无赶毛下台
的意思。斯大林因米夫而对王明表示冷淡,但他却无惩治王明的念头,因
为从各方面资料看,斯大林对王明关怀备至几十年可谓不变。

王稼祥带回延安的是季米特洛夫的「口信」,但是,北京大学研究共
产国际与中国革命关系的向青教授,在未提供任何历史资料的情况下却断
言,王稼祥回国带来了「共产国际从组织上支持毛泽东同志为中共领袖的
重要文件」\footnote{参见向青:
《共产国际与中国革命关系论文集》
,页 391-92.}。向青教授的上述叙述是完全错误的。王稼祥本人及其遗孀
朱仲丽也只是说,王从莫斯科带回的是季米特洛夫的「口信」或「意见」。
如果共产国际果真有这份「从组织上支持毛泽东同志为中共领袖的重要文
件」,毛泽东在其最需要莫斯科支持的 1938 年,难道会将其束之高阁?事
实是,
当王稼祥于 1938 年 8 月返回延安後,
毛泽东就立即决定召开中央会
议,让王稼祥向政治局委员郑重传达共产国际指示(季氏为共产国际总书
记,即是共产国际的化身)。毛泽东如此急不可待,如果确有共产国际的
书面意见,毛一定会在一定的范围内正式公布,绝不会秘而不宣。至今,
在中国出版的各种文件集中,都没有这份「文件」。可见,1938 年 7 月,
共产国际没有颁布过承认毛泽东为中共领袖的正式文件。1938 年 8 月,王
稼祥返回延安时确实带回了一份共产国际文件,这份文件不是向青教授所
说的那个子虚乌有的东西,而是 1938 年 6 月 11 日共产国际执委会主席团
通过的〈关于中共代表团报告的决议案〉。

王稼祥返回延安所带回的季米特洛夫「口信」,对毛泽东具有决定性
的意义,毛泽东终于得到莫斯科的承认。现在他的地位已得到加强,下一
步就是要对王明等发起全面反击,来巩固自己的中共领袖地位。
 
\section{两面策略:中共六届六中全会与毛泽东的〈论新阶段〉} 中共六届六中
全会是中国共产党历史上一次极为重要的会议。1945 年 6 月 10 日,毛泽东在
中共七大作关于选举中央委员问题的讲话时,把遵义 会议与六届六中全会相提并
请, 称之为党的历史上 「两个重要关键的会议」, 并强调: 「六中全会是决
定中国之命运的」\footnote{毛泽东在中共七大选举候补中央委员大会上的讲话
(1945 年 6 月 10 日),载《文献和研究》(1986 年汇编本), 页
20-21.}。六中全会的召开与王稼祥 回国有密切的关系。1938 年 7 月上旬,王
稼祥乘苏联军用飞机经新疆迪化 飞抵兰州,经陆路于 8 月下旬返回延安。王稼祥
一回延安,立即向毛泽东 转达季米特洛夫的「口信」。现在毛泽东认为到召开党
的重要会议的时候 了。

然而,召开什么样的会议,却令毛泽东颇伤脑筋。本来 1937 年 12 月
政治局会议和 1938 年 3 月政治局会议都已通过决议,
在近期召开党的第七
次全国代表大会,现在王稼祥已带回支持毛泽东为中共领袖的季米特洛夫
口信,眼前正是国共合作比较顺利的时期,参加七大的代表前来延安并不
特别困难,召开七大的时机已经成熟。

但是,毛泽东并不想在这个时候召开中共七大,因为此时召开七大, 还不足以一
举消除王明及其势力。毛泽东十分清楚,莫斯科并没有把王明 搞下台的意思,他
只能在莫斯科划定的范围内活动。王稼祥带回的季米特 洛夫口信,虽然提及中共
领导机关要以毛为首解决问题,但同时又强调中 共领导层应加强团结, 「不应花
很久时间去争论过去十年内战中的问题」, 对于总结十年经验,「要特别慎重」。
共产国际尤其告诫中共领导层,要 谨防日寇挑拨中共主要负责人之间关系的阴谋。
\footnote{王稼祥: 〈国际指示报告〉(1938 年 9 月),载《文献和研究》
(1986 年汇编本),页 70-71.}这就是说,利用共产国 际的招牌,孤立、打击
王明的时机尚未到来。在王明影响 仍然十分强大的 1938 年秋,如果冒然召开中
共七大,很难保证毛泽东能获得全党 的一致拥 戴,从而一劳永逸地掌握中共最高
权力。正是鉴于这种考虑,毛泽东决定不开党的七大,而是召开党的六届六中全
会。

为了「开好」六届六中全会,毛泽东做了精心的准备。在 8 至 9 月间, 毛抓紧
起草在中央全会上的政治报告——这是毛参加中共十七年後,第一 次在党中央全会
上做政治报告。毛还向长江局发电,通知王明、周恩来、 博古、项英等来延安听
取共产国际指示的传达。毛深知控制信息的重要, 在王明等回返延安以前,向他
们严密封锁季米特洛夫「口信」的内容 \footnote{参见朱仲丽(王稼祥遗孀)
《黎明与晚霞》: (北京:解放军出版社,1986 年),页 287-88.}。

1938 年 9 月 10 日,毛泽东笑容满面,站在延安各界欢迎王明、周恩 来、博古、
徐特立队伍的最前列。就在不久前,也还是毛泽东,却在少数 人面前, 用十分尖
刻的语言, 不指名地讽刺、 挖苦王明是 「涂胭脂抹了粉」, 「送上门人家也
不要的女人」(暗指王明等曲意讨好国民党)\footnote{朱仲丽: 《黎明与晚霞》
,页 286;另参见萧劲光: 〈服从真理,坚持真理——回忆王稼祥同志〉,载《回
忆王稼祥》, 页 12.}。但是在欢迎 王明等的队伍里,毛泽东却显得热情诚恳,俨
然在迎接久别重逢的亲密战 友。

1938 年 9 月 14 日,中央政治局召开会议,由王稼祥传达共产国际关 于任弼时
报告的决议案和季米特洛夫的「口信」。9 月 26 日,政治局再次 开会,确定六
届六中全会的议程。在这次会议上,毛泽东运用刚刚获得的 政治优势,迅速对中
共组织机构进行了重大调整。会议决定撤消长江局, 另成立南方局以代之,由周
恩来任书记,王明的长江局书记一职被无形中 止;将东南分局升格为东南局,由
项英任书记,东南局直属延安领导;继 续保留北方局, 并新设立中原局, 由刘
少奇一身兼任这两个中央局的书记\footnote{参见《周恩来年谱》页 419}。

1938 年 9 月 29 日,扩大的中共六届六中全会在延安正式开幕,这次
会议具有比较隆重的外观形式,出席会议的政治局委员共十二人,三名政
治局委员缺席:任弼时在莫斯科,邓发在迪化,凯丰在武汉留守中共代表
团。原政治局委员张国焘已脱离中共,投奔国民党。大会主席团由十二名
政治局委员组成,秘书长由李富春担任。全会前後共开了四十天,是至那
时为止中共中央所召开的历时最长的一次会议。在会上作报告、发言的代
表的十七人,几乎占了与会者的三分之一。

六届六中全会的灵魂人物是毛泽东。会议之初,由王稼祥作了传达共
产国际指示的报告。10 月 12 日至 14 日,毛泽东代表政治局向会议作了题
为〈论新阶段〉的政治报告。这个报告以後未全文收入毛泽东主编的《六
大以来》,也未被全文收入《毛泽东选集》。
 
毛泽东为什么不愿意将〈论新阶段〉全文收入《毛选》?\footnote{收入《毛选》
第 2 卷的部分内容易名为〈中国共产党在民族战争中的地位〉,且这一部分也作
了大量删改,有关 「统一战线中独立性不能超过统一性,而是服从统一性」等有
关内容已被尽行删去。}毛为什么意 欲人们忘掉这个如此重要的文件?一言以蔽之,
〈论新阶段〉中包含了大 量与王明相一致的观点,而依照中共党史编纂学的解释,
毛泽东正是在六 届六中全会上「彻底批判了王明的右倾投降主义路线」。
 
毛泽东在六届六中全会期间的言论和行为,集中展现了他变幻多端、
前後矛盾、
出尔反尔的政治性格。
毛为了政治上的需要和巩固个人的权力,
可以言不由衷、信誓旦旦接过其政敌的所有政治主张;他也可以一瞬间完
全变脸,说出前後判若两人、完全相反的另一套语言。

毛泽东在〈论新阶段〉的政治报告中,使用了与王明几乎类同的语言,
毛指出「抗战的发动与坚持,离开国民党是不可想象的」,他称赞国民党
「有孙中山先生蒋介石先生前後两个伟大领袖」\footnote{毛泽东:
〈论新阶段〉(1938 年 10 月 12-14 日) ,载中央档案馆编:
《中共中央文件选集》(1936-1938),第 11
册,页 595;560、595-96、606;629.},并高度评价抗战以来在
「民族领袖与最高统帅蒋委员长的统一领导之下」,中国已「形成了一个
空前的抗日大团结」,毛强调在抗战和抗日民族统一战线的组成中,「国
民党居于领导与基干的地位」,他似乎忘记了他本人自洛川会议以来一贯
宣传的「片面抗战必然失败论」,而谈起国民党的「光明前途」。毛批评
「至今仍有不少的人对于国民党存在着一种不正确的观察,他们对于国民
党的前途是怀疑的」,毛郑重号召全党必须「全体一致诚心诚意拥护蒋委
员长」\footnote{毛泽东:
〈论新阶段〉
(1938 年 10 月 12-14 日)
,载中央档案馆编:
《中共中央文件选集》
(1936-1938)
,第 11
册,页 595;560、595-96、606;629.}。

毛泽东为了充分显示中共和他本人对改善国共关系的诚意,向国民党
提议,共产党员在保留中共党籍的条件下,公开参加国民党。毛并表示,
中共将主动向当局提交加入国民党的共产党员的名单。毛向国民党保证,
中共不在国民党军队中组织党支部,
也不在国民党党员中征收共产党员。
\footnote{毛泽东:
〈论新阶段〉
(1938 年 10 月 12-14 日)
,载中央档案馆编:
《中共中央文件选集》
(1936-1938)
,第 11
册,页 595;560、595-96、606;629.}

毛泽东的最惊人之笔是他在六中全会开幕的当天——1938 年 9 月 29 日,写给蒋介
石的一封亲笔信。在这封信中,毛泽东表示对蒋介石「钦佩 无既」,声称对「先
生盛德」,「凡在国人,无不崇仰」。\footnote{毛泽东 1938 年 9 月 29 日致
蒋介石亲笔函,载台湾中央研究院近代史研究所编: 《抗战建国史研讨会论文集》
, 下册(台北:1985 年)页 694-95.}毛的这封信由 周恩来于 10 月 4 日在武
汉当面交给了蒋介石, 但从未收入 《毛泽东选集》、 《毛泽东书信选集》和任
何中共中央文件集,直到 1990 年 3 月出版《周恩 来年谱(1898-1949)》时,
才提及此事,但仍没有全文公布该信的内容。\footnote{据《周恩来年谱(1898-
1949)》一书披露,周恩来于 1938 年 10 月 4 日见蒋介石时,除了转交毛泽东
信件外, 还送交了王明给蒋介石的信件。参见《周恩来年谱》,页 420.}

毛泽东自抗战前夕就强调中共应在抗日民族统一战线中保持高度的独
立自主,七七事变後,毛更是坚决维护自己的这一政治主张。1937 年 12
月政治局会议後,毛在这个问题上的态度虽有稍许软化,但其初衷始终未
变,为什么到了六届六中全会,毛竟会来个一百八十度大转变?毛难道真
正转变了思想?答案是否定的。毛泽东所有这类言论都是违心之言,这是
他在当时复杂、微妙的形势下,为应付斯大林和中共党内的不同意见,为
麻痹蒋介石而精心设计的谋略手段。

毛泽东的〈论新阶段〉是送给斯大林的一份礼物。毛清楚知道斯大林
高度重视中国抗日民族统一战线,1938 年 6 月 11 日共产国际关于任弼时
报告的决议案,对此问题又再一次予以重申。在莫斯科事实上已承认毛为
中共领袖的形势下,完全有必要向斯大林作出服从的姿态。毛甚至还可以
走得更远一些,以显示自己对国共合作的诚意,提出中共加入国民党的建
议,
正好堵住斯大林的嘴,
看谁还再能批评毛泽东对国民党缺乏合作诚意!
对毛泽东而言,从一个极端跳到另一个极端,毫不困难,只要能牢牢控制
中共最高领导权,对国民党说什么好话都无所谓。

毛泽东的〈论新阶段〉也是说给蒋介石听的,其目的是为了麻痹蒋介
石,但是毛这次却碰上了真正的对手。和毛绝不信任国民党一样,蒋介石
也丝毫不相信共产党。蒋介石一眼就看穿了毛泽东的意图。蒋介石在接到
了周恩来转交的毛亲笔信的当天,在日记中写道:

「毛泽东这封亲笔手书的措词,开口『两党长期合作』,闭口是『中 华民族统一
团结』,完全不是共党素来口吻,反使我产生疑虑」。「于是 我知道这是中共企
图第二次大规模渗透本党的阴谋。我们依据民国十三年 到十六年的惨痛经验,是
不能再上当了」。\footnote{蒋介石: 《苏俄在中国》(台北:黎明文化事业
股份有限公司,1982 年)。页 71;另见《蒋总统秘录》,第 12 册, 页 74.}
 
1938 年 12 月 6 日,蒋介石的见周恩来,对 10 月 4 日周转交的毛泽东
亲笔信中的四点建议作出答复。毛在信中提议:一、停止两党的斗争。二、
共产党可以加入国民党,或令其一部分先加入,如情形良好,再全部加入。
三、中共取消一切青年组织,其全体分子一律加入三青团。四、以上参加
者,均保持其共产党党籍。蒋介石提出,跨党不赞成,共产党既行三民主
义,最好与国民党合并为一个组织。如果办不到,可否以一部分加入国民
党而不跨党。蒋介石的建议被周恩来拒绝,周表示中共除了跨党外,不可
能加入国民党\footnote{参见《周恩来年谱》页 427、420.}。

毛泽东以虚虚实实、半真半假的态度试探蒋介石,蒋介石则以不硬不
软的态度回应毛泽东,其实两人心中都有谱,国共两党不可能再回到 1924
-1927 年党内合作的年代。毛泽东本来就没有对此当真,提出中共加入国
民党的建议原本就是作给人看的。姜太公钓鱼,愿者上钩,蒋介石不愿上
钩,毛泽东也就顺手将其搁置一边了。

在六届六中全会召开的数十天内,毛泽东四面八方,席不暇暖,为应
付内外形势,费尽了心机。一方面,毛要虚应故事,敷衍莫斯科和国民党;
另一方面,他更急于利用来之不易的季米特洛夫的「口信」,加速巩固自
己在党内的地位。毛深知来自莫斯科的「支持」很不牢靠,陈独秀、瞿秋
白、李立三,难道不都是莫斯科立在先,最後还不是都被莫斯科所废吗?
眼下,共产国际虽然承认了毛的领袖地位。但是,说不定转眼间莫斯科就
会改变主意。因此,当务之急就是赶快造势,形成毛是中共唯一领袖的既
成事实,并使之不可动摇,即使莫斯科日後反悔,也将对毛无可奈何!

毛泽东必须首先亮出自己的观点,以集合同志,把王明孤立起来。然 而,在 1938
年毛泽东要做到这一点,却非易事。不久前毛泽东刚做的报告 〈论新阶段〉还在
与会者的耳畔回响,由于毛泽东自己的语言已被淹没在 王明理论的汪洋大海中,
大多数与会者尚不能分辨毛与王明的差别。加之, 王明、博古等都在延安,如果
当着他们的面,亮出自己的观点,又与〈论 新阶段〉自相矛盾,一时似乎又拉不
下脸面。毛泽东终于想出一条妙计。9 月 30 日,毛以转交致蒋介石亲笔信为由,
先将周恩来支去武汉。10 月初, 周恩来、凯丰自武汉来电,提议王明等速来武汉,
出席国民参政会一届二 次会议\footnote{参见《周恩来年谱》页 427、420.}。毛
顺水推舟,又让王明、博古于 10 月下旬去重庆(国民党中枢 机构此时已从武汉
迁至重庆)。王明、周恩来、博古、凯丰不在延安使毛 泽东大畅所欲,这样,毛
已不再需要遮遮掩掩,他要直抒胸臆,将自己的 真实观点在党中央全会上和盘托
出。

1938 年 11 月 5 日至 6 日,在王明、周恩来、博古等缺席的情况下,
毛泽东在六届六中全会的闭幕式发表了不指名抨击王明、周恩来的重要演
说。六中全会会议期间,因季米特洛夫「口信」,毛泽东的威望迅速上升,
许多烦导人,包括王明都发表了支持毛为中共领袖,赞颂毛的言论。毛泽
东利用会议期间这种有利于自己的气氛,将批评矛头骤然转向王明等人。
他似乎已完全忘掉自己曾在同一场合,刚刚在〈论新阶段〉报告中谈过加
强统一战线的意见,现在他却直言不讳地宣称,不应提出「一切经过统一
战线」的口号\footnote{《毛泽东年谱》
,中卷,页 94.}。

毛泽东在不到一个月前,在同样的党的高级干部面前,亲口说过拥护
蒋委员长一类话,现在却改口抨击对国民党的「投降主义」。毛的翻手为
云,覆手为雨,与会的党的高级干部竟没
有一个人提出异议!至此,毛终于看清自己在中共党内的主宰地位已经基
本确立,「势」既形成,其「威」立时显现,即使党内仍有不同意见,在
毛的权势下,也只得噤口。

毛泽东在 11 月 5 日至 6 日的闭幕式上还拉上刘少奇,
毛称赞刘少奇道,
还是少奇讲得对,所谓「一切经过统一战线」实际是「一切经过阎锡山」、
「一切经过国民党」。毛指责道,这是忘记了党的独立自主方针的右倾投
降主义。毛提出对国民党应采取「先斩後奏」、「先奏後斩」、「不斩不
奏」、「只斩不奏」的灵活策略,最终目标是争取壮大中共武装,为未来
的胜利奠定基础——毛泽东终于讲出了被压抑很久的心里话!

毛泽东利用王明等缺席,将自己的真实观点公开表达出来,这是毛泽
东取得的对王明的重大胜利。但是,王明的观点毕竟没有在六届六中全会
上受到正式批判,〈中共中央扩大的六中全会政治决议案〉仍然包含了王
明大量的观点。

1938 年 11 月 6 日,中共中央通过了由王明起草的六中全会政治决议 案。该决
议案批准了「一切为着抗日民族统一战线,抗日民族统一战线高 于一切」的口号。
同时宣布,国共合作最好的组织形式就是共产党员加入 国民党和三青团,中共愿
向国民党当局交出中共党员名单。决议案再次重 申,「不在国民党中及国民党军
队中建立共产党的秘密组织」。决议案没 有充分反映毛泽东在闭幕式讲话中有关
反对「右倾投降主义」的内容,而 是根据〈论新阶段〉报告的精神,并且吸取了
10 月 20 日王明在六中全会 上报告的精神,提出各级党组织应防止统一战线中的
「左」、「右」两种 倾向, 保证党在政治上和组织上的独立性, 强调不要给党
内同志乱加 「左」、 「右」的帽子\footnote{〈中共扩大的六中全会政治决议案〉
(1938 年 11 月 6 日), 载中央档案馆编:  《中共中央文件选集》(1936-
1938), 第 11 册,页 755、753-54、758.}。决议案还正式宣布,完全同意政
治局自五中全 会至六中 全 会的「政治路线和具体工作」,\footnote{毛泽东在
〈论新阶段〉政治报告中, 称「遵义会议纠正了在五次『围剿』斗争中所犯的左
倾机会主义性质的严 重的原则错误」,但又指出「当时的这种错误并非党的总
路线的错误」。}毛泽东对于 这个结论虽然很不情愿,但在 当时也只能违心接受。

毛泽东在六届六中全会大大加强了自己的地位,但是,六届六中全会
政治决议案显示,毛仍然受到党内不同意见的牵制,毛泽东还需找到能够
在政治上制敌于死命的武器。毛终于找到了这个武器,它就是「马克思主
义的中国化」。

\section{毛泽东的「新话」:「马克思主义的中国化」}

「马克思主义的中国化」是毛泽东经长期酝酿,为彻底打倒王明和党
内的留苏势力,铲除中共党内根深蒂固的对斯大林的崇拜,最终确立自己
在中共党内的「导师」地位,而在中共六届六中全会上提出的一个具有重
大战略意义的口号。毛通过「马克思主义的中国化」口号,为自己所有「异
端」观点寻找到解释的依据,它集中体现了毛本人对中国共产主义运动的
基本观点和态度,构成了渐趋成型的毛主义的理论核心。
 
1938 年 10 月,毛泽东在六届六中全会的讲台上,第一次提出「马克
思主义的中国化」的概念,毛指出:
\begin{quote}
	\fzwkai 我们这个大民族数千年的历史,有它的发展法则,有它的民族
特点,有它的许多珍贵品。......共产党员是国际主义的马克思主义
者,但马克思主义必须通过民族形式才能实现。......离开中国特点
来谈马克思主义,只是抽象的空洞的马克思主义。因此,马克思主
义的中国化,使之在其每一表现中带着中国的特性,即是说,按照
中国的特点去应用它,成为全党亟持了解并亟待解决的问题。洋八
股必须废止,空洞抽象的调头必须少唱,教条主义必须休息,而代
之以新鲜活泼的,为中国老百姓所喜闻乐见的中国作风与中国气派。
\footnote{毛泽东: 〈论新阶段〉(1938
年 10 月 12-14 日),载中央档案馆编: 《中共中央文件选集》(1936-1938),
第 11 册;页 658-59.}
\end{quote}


毛泽东的上述言论,逻辑严密,言简意赅,在民族主义高涨的抗战阶
段,完全契合中共干部党员的心态,极具正当性和鼓动性。对于参加六中
全会的许多中共领导干部来说,他们似乎并不完全了解毛的这番讲话实际
上已标志中共战略思想正发生重大改变。他们也没有觉察到毛的有关「马
克思主义的中国化」
的讲话掩藏着不久後将向党内留苏派势力开刀的预兆。
时下,他们都乐意听到毛这种令人耳目一新,充满民族自尊感的话语。

毛泽东的
「马克思主义的中国化」
的新概念并非一时心血来潮的产物,
而是他积蓄已久看法的总结与升华。毛的「马克思主义的中国化」,就其
主要精神而言,即在于他吸取、应用马列阶级斗争、暴力革命的思想和苏
共党的组织结构形式,将其与中国历史的重大遗产——农民造反,「马上
打天下」的传统融汇统一,使之转化为由共产党领导的、以推翻国民党统
治为基本目标的现代农民大革命。作为中共摹本的俄式革命理论及经验,
虽在毛将中国传统遗产转化为现代农民革命战争的过程中发挥了重要作
用,但俄式理论及其经验与毛的观念和行动又常有不合之处。「马克思主
义的中国化」的口号为中国共产主义运动注入了民族主义的活力,它不仅
为毛所有的观点提供了合理性的解释,也给毛创造了自由活动的广阔的空
间,它更有助于改变「中共乃外来观念之产物」这一在当时颇为流行的观
念,而大益于中共在中国社会的生根。

毛泽东从不讳言自己负有解救中国人民、再造中国的历史使命,他也 从未怀疑过
自己具有别人无法企及的智能和能力。这种强烈的 「舍我其谁」的自信力与坚
强的个人意志力一旦结合。确实使毛泽东产生了一种「能强 迫历史朝他的理想迈
进」的力量。\footnote{参见白修德著,马清槐、方生译: 《探索历史》(北
京:生活·读书·新知三联书店,1987 年),页 177.} 1935 年後,毛逐渐控制了
中共的实 权, 进而迫切需要创造一个在理论上能为自己自圆其说的解释系统。
同时, 熟知中国传统的毛泽东也深知, 欲成为党的最高领袖, 仅手握兵符还不
够, 还需要成为能为广大追随者提供精神资源的「导师」。换言之,有「君」之
实,而未有「师」之名,是令毛泽东深感缺憾的。于是,怎样创建「师」之理论
体系,如何使之达到「君师合一」,就成为长期困扰毛思绪的一大 难题。提出
「马克思主义的中国化」,不仅表明毛泽东的解释系统已初建 成功,而且标志着
毛为确定自己的「导师」(教主)地位而进行的努力已 取得了重大的进展。

毛泽东在 1938 年 10 月正式提出「马克思主义的中国化」,主要出自 他长期的
思考,但是若干资料显示,毛之所以能提出这个命题,也和他受 到梁漱溟的启发
有关。梁漱溟与毛泽东相识于五四时期的北京,两人均未 出国留洋, 也未受过国
内正规大学的训练, 都是自修型知识分子出身。1938 年 1 月,这两位在政治思
想和个人性格方面迥然不同的老友,在延安的窑 洞却有过六次十分深入的交谈
\footnote{梁漱溟: 〈我努力的是什幺〉(1941 年),载《我的努力与反省》,
页 144-45;138、147-48;154.}。

梁漱溟是一个新旧杂揉的「中国文化至上论」者,一生汲汲于融汇民
主、科学与中国固有文明,以济世的情怀,长期徘徊于学术与政治之间。
毛泽东则并非是一单纯的「中国文化至上论」者,此时的毛已接受了大量
列宁、斯大林的思想,又是中国共产主义运动和中国共产党的实际领袖,
毛虽对中国历史和文化情有独锺,但是他的这种态度却完全服从于现实功
利的需要。

梁氏直言不讳地向毛泽东表明,他对中国共产主义运动的必要性存在 怀疑。他认
为,中国的共产革命是从外来引发的,而非内部自发。由于中 共不了解中国社会
有其特殊构造,与欧洲中古、近代社会均非同物,而拿 外国办法到中国来用,才
造成中共过去十年劳而无功,仅靠中共军队维持 党的生命。梁氏认为,抗战後,
因中共放弃对内斗争,倡导团结抗日,适 合人心要求,中共才「声光出于各党之
上」。梁氏希望今後中共要「认识 老中国,建设新中国」,「不要再靠军事维持
一条党命」\footnote{梁漱溟: 〈我努力的是什幺〉(1941 年),载《我的努
力与反省》,页 144-45;138、147-48;154.}。

毛泽东对梁氏所言,「笑言相谢」,称赞道,「很可感」。梁氏所论 虽从根本上
否认了中共存在的必要性,但梁氏看法的某些方面却与毛不谋 而合。毛在内心中
早就对中共党内的全盘俄化持有异议,且认为正是靠着 军队,才挽救了共产党。
至于梁氏关于中国社会特殊性的看法,毛也未贸 然否认,而是予以「相当承认」,
只是不同意梁氏太重中国特殊性而忽略 中国之与世界各国所共有之一般性,即各
社会均存在的阶级、阶级压迫和 阶级斗争。\footnote{梁漱溟: 〈我努力的是什
么〉(1941 年),载《我的努力与反省》,页 144-45;138、147-48;154.}
毛并表示赞成梁氏《乡村建设理论》一书中不搞上层表面文章 的「宪政运动」,
而从改造社会的基层入手,从农村入手的主张\footnote{参见梁漱溟:
《再忆初访延安》,载《我的努力与反省》,页 317-19.}。

毛泽东出于意识形态的限制,
不可能对梁漱溟的观点全然赞成。但是 两人说话投机,在思想和观念上有许多共
鸣却是不争的事实。毛和梁都具 有强烈的中国历史文化意识,在精神和气质上都
具有浓厚的中国色彩,但 是,毛毕竟不是一个单纯的思想型人物。1938 年 1 月,
梁漱溟在延安见到 的毛泽东,待人接物谦恭有礼,至于毛所具有的金刚怒目的另
一面,梁漱 溟则要在十五年後才能真正领教,当然这是後话了。

在毛泽东提出 「马克思主义的中国化」的 1938 年,真正能够一眼看穿 毛之动机
的人,在中共 党内唯有王明。1938 年 10 月 20 日,王明应部分与 会代表的要
求,在前一阶段 作过〈共产党员参政员在国民参政会中的工作 报告〉後,再一次
在六中全会上做 〈目前抗战形势与如何坚持持久战争争 取最後胜利〉的长篇发言。
王明在发言中 表示拥护毛的意见,但是他又对 「马克思主义的中国化」在实行中
可能出现的偏 差忧心忡忡。王明提出, 在使马克思主义理论中国化的过程中,应
注意五个方面 的问题:一、首先 应学习马列主义;二、不能庸俗化和牵强附会;
三、不能以孔 子的折衷论 和烦琐哲学代替唯物辨证法;四、不能以中国旧文化旧
学说来曲解马 列主 义;五、不能在「民族化」的误解之下,来忽视国际经验的研
究和应用 \footnote{王明: 〈目前抗战形势与如何坚持持久战争取最後胜利〉
(1938 年 10 月 20 日);载《王明言论选辑》,页 637-38. }。王明的上述
看法显然出自其亲莫斯科的立场,表达了中共党内留苏势力对 毛泽东有可能利用
这个口号背离正统马列的警戒和担心。

然而若从另一个角度来观察「马克思主义的中国化」的口号,确可发
现,这个口号对中共的作用是双重的。毛泽东提出这个概念,一方面固然
大大有益于中共的发展壮大;但是在另一方面,伴随这个口号,中国传统
中的某些消极因素也被注入到中共体内,给党自身带来了长期的困扰。中
共成立之初,原本就缺少理论的准备,党内一直存在着轻视理论的根深蒂
固的传统。1927 年後,中共长期战斗在落後的农村边远地区,其阶级构成
发生重大变化,农民党员的比重在党内占了压倒优势,早已养成农村军事
性格。「马克思主义的中国化」的提出,为中国传统因素大规模浸润中共
正式打开了大门,使原来就深受农民造反传统影响的中共,更加显现出农
民化的色彩。「马克思主义的中国化」的这种双重作用,在中共以後的历
史和 1949 年後中国的历史中都得到充分的证实。

但是在 1938 年秋冬的延安,毛泽东抓住「中国化」这面旗帜,已使自
己处于完全主动的地位。莫斯科的「承认」大大提高了毛泽东的威望,毛
泽东「奉天承运」,一手牢牢掌握中共军队,一手挥舞「马克思主义的中
国化」的大旗,名正言顺,师出有名。相形之下,王明等已陷入无以自拔
的窘境。
