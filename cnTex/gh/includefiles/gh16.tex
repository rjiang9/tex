\begin{preface}
	
这本书从酝酿到写作经历了一个漫长的过程,
八十年代,
我产生了写这本书的念
头,但促使我对延安整风这一历史事件萌发兴趣则是在更久远的年代。

我是在 1961 年的南京读小学的,那是一个政治意识畸形发展的年代。从 1963
年初开始,我对母亲订阅的《参考消息》发生了兴趣,经常躲着她偷偷阅读。我也从
那时起,养成了每天读江苏省党报《新华日报》的习惯。可是我对那时的社会状况并
不清楚——应该说,除了雷锋、革命先烈、越南、红军长征的故事,那时我的头脑中
并没有任何其它东西,但是到了 1963 年下半年後,情况发生了变化,我愈来愈注意
《参考消息》和报纸上刊载的有关中苏两党论战的报道。1964 年春夏之间,我从《人
民日报》上看到苏共中央书记苏斯洛夫在苏共二月全会上作的「反华报告」,第一次
看到对斯大林、
莫洛托夫制造三十年代大恐怖罪行的揭露,
以及对毛泽东、
对大跃进、
人民公社的批评——这对于我是一个极大的震动(这份报告给我留下极为深刻的印
象,以後我长期保留这份《人民日报》)。我开始思考苏斯洛夫报告中所论及的一些
词汇:毛泽东是「左倾冒险主义」、「新托洛茨基主义」、「唯意志论」等等(七十
年代,我从内部读物才知道,苏斯洛夫是一个顽固的教条主义者。近年出版的俄罗斯
资料透露,1964 年苏共党内的革新势力利用与中共的论战,削弱了斯大林主义者在
苏联的阵地,一度遏制了保守势力复辟的势头,正是在这样的背景下,保守的苏斯洛
夫才在苏共中央二月全会上作了这个报告)。对于这些话,当时我似懂非懂。我联想
自己的日常生活,几年前那些饥饿的日子,我随母亲去南京郊外的农场去探望因「右
派」问题而被下放劳动的父亲,{\bf 1963 年夏,我已被南京市外语学校录取,却因政审
不通过而被拒之门外。}不久甚至连小学也讲起「阶级路线」,我因出身问题越来越感
到压力。在这种情况下,我迎来了 1966 年,也就是在这个时候,在学校的号召下,
我通读了《毛选》一至四卷,我多次阅读收人《毛选》中的〈关于若于历史问题的决
议〉以及毛的〈改造我们的学习〉、〈反对党八股〉,于是我知道了「整风运动」这
个词。

紧接着文革爆发,我从每天读的《新华日报》上发现,1966 年 5 月初北京召开
的欢迎阿尔巴尼亚党政代表团的群众大会上不见了彭真的名字,
接下来我就读小学的
一些干部子弟(我的小学邻近南京军区後勤部家属大院和《新华日报》社家属区),
手拎红白相间的体操棒在操场上殴打一位「成份不好」的三十多岁姓余的美术教师,
校长兼支部书记则装着什么也没看见。

南京 1966 年 8 月下旬的红色恐怖给我留下了终身难忘的印象。我的家庭受到冲
击,有一天,我无意中听到父母的谈话,父亲说,这一次可能躲不过去了,再不跑,
可能会被活活打死。父亲终于离家出逃。躲在山东农村老家那些纯朴的乡亲中避难,
不久,我家附近到处贴满了父亲单位捉拿他的「通缉令」。

在文革的血雨腥风中,我看到了多少景象!曾几何时,那些在文革初期指挥揪斗
「死老虎」的当权派自己很快也被拉下了马, 「周扬」、 「彭罗陆杨」、
「刘 邓陶」像走马灯似地被「扫入历史的垃圾堆」,真是「一顶顶皇冠落地」!
从那时起, 我就无师自通地学会了报纸上的「排名学」。1967 年初,在南京大学
的操场上,我 亲眼看见江苏省委第一书记江渭清被批斗, 就在半年前, 我们小
学的校长还是满口 「江 政委」唤个不停。不久我又去了省委办公大褛,那里正举
办所谓「修正主义老爷腐朽 生活」的展览,那宽大的带卫生间和休息室的书记办
公室,那嵌在舞厅天花板壁槽内 的柔和灯光,无一不使我头脑翻江倒海。

我的家庭背景使我不能参加这场革命, 我在家庭中受的教育以及我从各种书籍中
所获得的精神营养也使我不会去欣赏那些在革命名义下所干的种种凌虐人的暴行。
在 文革前, 我家有一个区文化馆图书室的借书证, 因此我读过不少中外文学、
历史读物。 至今我还记得,在恐怖的 1966 年 8 月,我如何从母亲的手中夺下她
正准备烧掉的那 套杨绛翻译、勒萨日著的《吉尔·布拉斯》等十几本书籍。在焚书
烈火中被抢救下来 的《吉尔·布拉斯》、范文澜的《中国通史简编》、普希金诗选、
《唐诗三百首》等 给了我许多温暖,让我在黑暗的隧道中看到远处一簇光。

在文化大革命的狂风暴雨中,希望之光是黯淡和飘忽的。1967 年初,我在家附
近的长江路南北货商店墙上看到一张写有「特大喜讯」的大字报,上面赫然写着叶剑
英元帅最近的一次讲话,他说,我们伟大领袖身体非常健康,医生说,毛主席可以活
到一百五十岁。

看到这张大字报,
我头脑轰地一响,
虽然有所怀疑,
但当时的直觉是,
这一下,
我这一辈子都注定要生活在毛泽东时代了。
我马上去找我的好友贺军——他
目前住在美国的波士顿,告诉他这个消息,我们一致认为,毛主席不可能活到一百五
十岁,因为这违反科学常识。

从这时起,我在心里消悄地对毛泽东有了疑问,我知道在中国,一切都凭他一个
人说了算,其他人,即使刘少奇,虽然〈历史决议〉对他评价极高,虽然在文革前
到 处都能看到毛、 刘并列的领袖标准像, 虽然刘少奇夫妇访问东南亚是何等地
热烈和风 光, 但是如果毛泽东不喜欢, 刘少奇马上就被打倒。 我又看到自己身
边发生的一些事, 离我家不远一个小巷的破矮平房里,住着与我同校但不同班的
一对姐弟和他们的父 母,他们的父亲是「阶级敌人」,他们的妈妈是一位普通的
劳动妇女。因为不能忍受 歧视和侮辱,这位母亲竟失去控制,将毛主席的画像撕
碎并呼喊「反动口号」,结果 在 1970 年南京的「一打三反」运动中被枪毙。召
开公判大会那一天,我的中学将所 有学生拉到路边,观看行刑车队通过,美其名
曰「接受教育」,这姐弟两人也被安排 在人群中,亲眼目睹他们的母亲被五花大
绑押赴刑场。车队通过後,学校革委会副主 任要求全校各班立即分组讨论,于是
所有同学都表态拥护「镇压反革命」——所有这 一切都让我对毛产生了看法。 我知
道这些看法绝不能和任何人讲, 甚至不能和自己的 父母讲,只能深埋在心中。

在那令人窒息的岁月里,没有希望,没有绿色,除了从小在一起长大的贺军,差
不多也没有任何可以与之交心的朋友 (即使我们之间的谈话也小心翼翼, 绝不敢
议论 毛泽东),但是,我的心中仍存有一线微弱的光。我的家附近有南京某中学
留守处, 这个中学已被勒令搬至农村, 所有被封存的图书都堆放在留守处的大仓
库里, 由一姜 姓老先生看管(老人是山东人,年轻时被国民党拉去当兵,被解放
军俘虏後成为「解 放」 战士) 至令我仍感激这位老先生, 。 是他允许我每周
进一次仓库借一旅行袋的书, 下周依时交换。正是在那里,我翻检到 1958 年
《文艺报》的〈再批判〉专辑,因而 我第一次读了王实味的〈野百合花〉和丁玲
的〈三八节有感〉。在那几年,我从这个 仓库借去大量的中外文学和历史书籍,
至今还记得, 孟德斯鸠的 《一个波斯人的信札》 、 罗曼·罗兰的《约翰·克利斯
朵夫》、惠特曼的《草叶集》、叶圣陶的《倪焕之》、 老舍的《骆驼祥子》,就
是在那个时候读的。1971 年後南京图书馆局部恢复开放, 我又在每个休息日去那
里读《鲁迅全集》,将包括鲁迅译著在内的旧版《鲁迅全集》 全部通读了一遍。
正是这些作品支撑起我的人文主义的信念。

七十年代中期,国内的政治局势更加险恶,我的一位熟人的弟弟,因愤恨江青的
专横,在 1975 年从其家中的阁楼上跳下自杀身亡。我家也每天受到居民小组老太
太的监视,只要家里来一外人,她就站在门口探头探脑。1976 年夏天的一个晚上,
我与好友贺军坐在长江路人行道的路边,我背诵了鲁迅的话:
「地下火在运行,岩浆
在奔突......」(1995 年 8 月底,我与贺军在纽约第五大道的街心花园坐了半天,我
们共同回忆起往昔岁月,我们都谈到 1976 年夏在长江路边的那次谈话)。

在文革期间,我读了许多毛泽东的内部讲话和有关「两条路线斗争」的资料,这
些资料真真假假,
其中不少充斥着大量的歪曲和谎言,
然而它们还是激起了我强烈的
兴趣。结合文革中所发生、暴露出的一切,以及自己的生活感受,我愈来愈有一种想
探究中共革命历史的愿望,
在这个过程中,
我注意到了延安整风运动——这虽然是距
那时以前几十年的往事,但我还是隐约感到,眼下一切似乎都与它有联系。在大字报
和各种文革材料中,
我难道不是经常读到毛和其他
「中央首长」
的讲话吗: 「XXX
什么
最坏,在宁都会议上,他想枪毙我」,「刘少奇在抗战期间勾结王明反对毛主席的独
立自主方针」,什么「XXX 在延安审干中查出是自首分子,因此对他控制使用」,
还有「王明化名马马维奇在苏联恶毒攻击伟大领袖毛主席」等等。

在那些年里,我虽然是「生在新社会,长在红旗下」,却不知填了多少表格,从
小学、中学到工作单位,每一次都要在「政治面貌厂「社会关系」栏内填写老一套的
内容。
看着周围的人,
大家也一样要填表。
我工作单位的人事干事是从老解放区来的,
她说,这是党的审干的传统,是从延安整风开始推广的,那么延安整风运动又是怎么
一回事呢?带着这些疑问,1978 年秋,我以历史专业作为自己的第一选择:考入了
南京大学历史系。

1979 年後的中国大学教育开始发生一系列重大变化,我经历了那几年由思想解
放运动而带来的震撼并引发了更多的思考。
在课堂上,
我再次听老师讲延安整风运动,
我也陆续看到一些谈论「抢救」运动的材料,然而所有这些都在维持一个基本解释:
延安整风运动是一场伟大的马克思主义的教育运动。1979 年我还读到周扬那篇有名
的文章〈三次伟大的思想解放运动〉,周扬将延安整风与五四运动、七十年代末的思
想解放运动相提并论,谓之为「思想解放运动」。在大学读书的那几年,我知道,虽
然毛泽东晚年的错误已被批评,
但毛的极左的一套仍很深蒂固,
它已渗透到当代人思
想意识的深处,成为某种习惯性思维,表现在中国现代史、中共党史研究领域,就是
官学盛行,
为圣人避讳或研究为某种权威论述作注脚几乎成为一种流行的风尚。
当然
我十分理解前辈学者的矛盾和苦衷,
他们或被过去的极左搞怕了,
或是因为年轻时受
到《联共党史》、《中国共产党的三十年》的思想训练太深,以至根本无法跳出官学
的窠臼。

然而,我难以忘怀过去岁月留下的精神记忆,刘知几云,治史要具史才、史学、
史识,其最重要之处就是秉笔直书,「在齐太史简,在晋董狐笔」。至令我还清楚记
得 1979 年在课堂上听老师讲授司马迁〈报任安书〉时内心所引起的激动,我也时时
忆及范文澜先生对史学後进的教诲:板梁甘坐十年冷,文章不写一字空。所有这些都
促使我跳出僵硬教条的束缚,努力发挥出自己的主体意识,让思想真正自由起来。从
那时起,我萌生一个愿望,将来要写一本真实反映延安整风的史书,为此我开始搜集
资料。

由于延安整风在主流话语中是一个特殊的符号,有关史料的开放一直非常有限。
这给研究者带来极大的困难。
但在八十年代以後。
官方也陆续披露了某些与延安整风
运动相关的历史资料,除了少量档案、文件集外,也出版了不少回忆资料,这给研究
者既带来了便利,同时也带来了新的问题,这就是如何分析、辨别、解释这些材料。
应该说,
我在中国大陆长期的生活体验以及我对有关史料的广泛涉猎。
加强了我阅读
资料的敏感性,我逐渐能够判断在那些话语後面所隐蔽的东西。

经过对多年搜集、
积累资料的反复研究和体会,
我头脑中的延安整风的轮廓逐渐
清晰起来,我开始发现散乱在各种零碎资料之间的有机联系。1991 年 8 月中旬我开
始动笔,到 1992 年底,我已完成初稿的三分之二。

从 1993 年始,我的写作速度慢了下来,我感到自己需要对所论述的问题作进一
步的思考,同时需要更广泛地搜集、阅读各种资料。

1995 年夏至 1996 年秋,我有机会去设于美国首都华盛顿的约翰斯·霍普金斯大
学高级国际问题研究院作访问学者。
我在美国的研究题目与延安整风无关,
但我仍利
用在华盛顿的便利,在国会图书馆工作了一个月。然而根遗憾,国会图书馆中文部虽
然收藏十分丰富,但是几乎找不到我所需要的有关延安整风的材料。1996 年 10 月我
返国後,又重新开始写作,到了 1998 年夏,全书已经完成。我又用半年时间对书稿
作了三次修改补充,1999 年初交稿後,在编辑校对阶段,我接触到若干新资料,对
书中的个别内容再次做了充实,于 1999 年春夏之交,全书最後定稿。

我写这本书在思想上一直以求真求实为依归,
在写作过程中,
始终遵循据事言理
的治学方法。我以为,重要的是,首先应将延安整风的来龙去脉叙述清楚,这个问题
之所以重要,
乃是因为数十年意识形态的解释学早将当年那场事件搞得云环雾绕,
面
目不清。为此我作了大量的工作,对各种重要的和非重要的资料进行点滴归拢,爬梳
鉴别,
再对之反复研究体会,
使之融汇贯通。
这方面的工作用去我最多的时间和精力。

我不反对对延安整风这一重大现象进行严谨的理论分析,
且认为,
这个工作极为
重要,但是我又担心过度解释会妨碍读者自己的判断。陈寅恪先生言,「大处着眼,
小处着手」,「滴水观沧海」,因此在本书中,我从实证研究的角度,以分析性论述
的方式展开,这也与我个人比较重视历史的个案研究有关。

在写作此书的七年里,我一直怀有深深的遗憾,这就是,我无法得到更重要的原
始资料。众所周知,有关延安整风期间的中共中央政治局、书记处、中社部、中组部
的档案文献,除少量披露外,绝大部分迄今仍未公开。1992 年,我看到一位负责人
在中央档案馆的讲话,他说,鉴于苏东巨变深刻的历史教训,应该加强对档案工作重
要性的认识。他指出,党的档案资料的保管,关系到中国社会主义的前途和命运。我
可以理解这位负责人的观点,
但是站在学术研究的角度,
却为不能阅读和利用这些珍
贵史料而感到遗憾。

由于这是一本私人写作,
十多年来我从自己不多的工资里挤出钱购买了大量的书
籍资料,我从没有以此选题申请国家、省级或大学的任何社科研究项目的资助,所以
我的另一个遗憾是,
我无法对一些当年参加过延安整风运动的人士进行口述采访,
如
果我做了这样的工作,一定会对本书的内容有所充实。
我还有一个遗憾是我没有机会去莫斯科搜寻有关资料。
九十年代後,
莫斯科开禁
历史档案,
涉及四十年代苏共与中共交往的文献记录也已开放。
中国历史学会的沈志
华博士近年来为搜集这些史料作了大量工作,
他并已将其中某些材料转送北京研究者
(沈博士告诉我,苏共与中共在延安整风期间交往的史料很少),因沈博士去美国,
一时联系不上,这也使我深感遗憾。

伏案几载,
每天神游于当年的历史景像之中,
自然会对延安整风运动及其相关的
史事与人物产生种种体会,
这方面的体会与感受的绝大部分已化为书中的叙述,
但是
还有几点需在此予以说明:

一、予生也晚,未能躬逢中共草创革命的年代。吾细读历史,站在二十世纪全局
观二十年代後中国共产革命之风起云涌,
心中自对中共革命抱持一种深切的同情和理
解。
吾将其看成是二十世纪中国民族解放和社会改造运动的产物,
认为在历史上自有
其重大正面价值和意义。

二、从中共革命夺权、推翻国民党统治的角度观之,延安整风运动对于中共革命
成功助力巨大。但是,延安整风运动中的某些概念、范式以後又对中国的发展和进步
产生若于消极作用,极左思想、权谋政治汇溪成流,终至酿成建国後思想领域一系列
过左的政治运动直至文革惨祸,真所谓「成也萧何,败也箫何」!所幸中共十一届三
中全会後,国家已逐步走出过去那种怀疑一切、无情斗争的极左道路,但旧习惯思维
的清理仍需长期努力。吾期盼旧时极左的「以我划线」、权谋政治永不再来,国家从
此能步入民主、法治的轨道,如此,则国家幸甚,民族幸甚!

三、本书涵盖面颇宽,涉及中国现代史上许多著名人物,对于本书论及的所有人
物,我只将其看成历史人物,不存任何既定的好恶偏见,主观上力求客观公允,「不
虚美,不隐恶」。当然任何研究都不可能完全排除作者的价值关怀,陈衡哲先生曾说
过,
「若仅缕述某人某国于某年征服某地......那有什么意思」,说的也是研究者的价
值关怀问题,
只是这种价值关怀不应妨害到叙述的中立和客观。
如果说本书的叙述中
有什么价值倾向的话,
那就是我至今还深以为然的五四的新价值:
民主、
自由、
独立、
社会正义和人道主义。

在写作此书的几年里,
我得到了一些朋友宝贵的支持和鼓励,
在本书即将出版之
际,我谨向他们表示真挚的感谢。

上海师范大学的许纪霖教授多年来一直关心我的研究的进展,
他还为本书的出版
提出许多好的建议。
在与许教授的交往中,
他的深厚的学养和对二十世纪中国历史的
卓越见解总是使我深获教益。

我衷心感谢香港中文大学中国文化研究所的金观涛教授和刘青峰教授。
他们对本
书的出版提供了热情的帮助,
在本书定稿过程中,
他们提出一些富有启迪性的建议和
意见,对于本书臻于学术规范化的要求,有重要的作用。

我也向我的同事和好友,
南京大学历史系颜世安教授和现旅居美国的贺军先生表
达我的感激,他们的友谊和支持,对于我一直是一个激励。
我曾与美国哈佛大学东亚系孔斐力教授(Phi1ip
Kuhn)和美国约翰斯·霍普
金斯大学高级国际问题研究院「华盛顿——南京办公室」主任甘安哲博士(Anthony
Kane)有过多次关于三十至四十年代中共党史问题的愉快讨论,他们的支持和鼓励
对于我的写作是一种推动。

在写作此书的几年里,我始终得到我过去的学生甘思德(Scott
Kennedy)和唐
山(Jcff Zuckerberg)、林志涛(Felex Lin)的关心和帮助,我的研究生郭洵澈对我
帮助尤大,他不仅帮我用电脑输入文字,还与我分享了讨论的乐趣,在此我向他们表
示深切的感谢。

我也向本书所引用文字的作者、
编者表示我的谢意,
我虽然在引述文字时都做有
详细的注释,但没有他们提供的资料基础,我要完成这本书也是不可能的。
我要向本书的责任编辑郑会欣博士表达我深深的谢意,
郑博士自己有大量的研究
任务,
但他还是拨冗为本书做了许多琐细的工作,
他的慷慨支持对本书的出版有重要
的帮助。

1998 年夏秋之际,我有机会前往香港中文大学中国文化研究所作访问研究,在
「大学服务中心」得到熊景明女士的热情接待和帮助,在这个收藏丰富的史料中心,
我为本书补充了若干新的资料,在此向熊景明女士和「大学服务中心」表示深切的谢
意。

南京大学历史系资料室的老师们多年来在图书资料方面给了我许多帮助,
对他们
的友好、善意和敬业精神,我深表感激。
最後,我要深深地感谢我的妻子刘韶洪和儿子高欣,我的妻子在每天工作之馀,
承担了大量的家务,
使我可以专心致志进行研究,
她还为书稿作了一部分的电脑输入
工作。为了写作这本书,许多年我不能和妻子、孩子一同出外游玩,也不能与孩子经
常讨论他的功课,没有他们的支持、帮助和理解,我要完成此书是完全不可能的。

{\bigskip\mbox{}\large\hfill  高华\qquad\qquad\qquad\\
\bigskip\mbox{}\large\hfill 
1999 年 6 月于南京大学}

\end{preface}
\begin{thebibliography}{999}
\bibitem{} 中央青运史研究室、中央档案馆编:《中共中央青年运动文件选编》(1921 年 7 月——1949 年 9 月)(北京:中国青年出版社,1988 年)。
\bibitem{} 中央统战部、中央档案馆编:《中共中央抗日民族统一战线文件选编》(北京:档案出版社,
1984——1986 年)。
\bibitem{} 中央档案馆编:《中共中央文件选集》(内部本),第 1—14 册(北京:中共中央党校出版社,
1982—1987 年)。
\bibitem{} 中央档案馆编:
\bibitem{} 《中共中央文件选集》,第 1—15 册(北京:中共中央党校出版社,1989—1992
年)。
\bibitem{} 中国社会科学院新闻研究所编:《中国共产党新闻工作文件汇编》(北京:新华出版社,1980
年)。
\bibitem{} 陕西省档案馆、陕西省社会科学院合编:
\bibitem{} 《陕甘宁边区政府文件选编》,第 1——14 辑(北京:
档案出版社,1986——1991 年)。
\bibitem{} 《丁玲集外文选》(北京:人民文学出版社,1983 年)。
山东大学编写组编:《中国革命史论文辑要》(北京:中共党史资料出版社,1987 年)。
\bibitem{} 中共上海市委组织部、中共上海市委党史资料征集委员会、中共上海市委党史研究室、上海市
档案馆编:《中国共产党上海市组织史资料》(1920.8——1987.10)(上海:上海人民出版社,
1991 年)
\bibitem{} 中共上海市委党史资料征集委员会、中共上海市委党史研究室、中共上海市委宣传部党史资料
征集委员会合编:
《上海革命文化大事记(1919——1937)》
(上海:上海书店出版社,1995 年)。
\bibitem{} 中共山西省委党史研究室编:《彭真生平大事年表》(北京:中共党史出版社,1995 年)
\bibitem{} 中共山东省委党史资料征集研究委员会编:
《山东抗日根据地》
\bibitem{} (北京:中共党史资料出版社,
1989 年)
\bibitem{} 中共中央文献研究室编:《毛泽东年谱》(1893——1949)》(北京:中央文献出版社,1993
年)
\bibitem{} 中共中央文献研究室编:《朱德年谱》(北京:人民出版社,1986 年)
\bibitem{} 中共中央文献研究室编:《周恩来年谱》(1998——1949)》(北京:中央文献出版社、人民
出版社,1990 年)
\bibitem{} 中共中央文献研究室编:《周恩来年谱(1949——1976)》(北京:中央文献出版社,1997
年)
\bibitem{} 中共中央文献研究室褊:《刘少奇年谱(1898——1969)》(北京:中央文献出版社,1996
年)
\bibitem{} 中共中央文献研究室、中共湖南省委《毛泽东早期文稿》编辑组编:《毛泽东早期文稿》(北
京:人民出版社、湖南出版社,1990 年)
\bibitem{} 中共中央文献研究室编:《毛泽东哲学批注集》(北京:中央文献出版社,1988 年)
\bibitem{} 中共中央文献研究室编:《文献和研究》(1984 年汇编本)(北京:人民出版社,1986 年)
\bibitem{} 中共中央文献研究室编:《文献和研究》(1985 年汇编本)(北京:人民出版社,1986 年)
\bibitem{} 中共中央文献研究室编:《文献和研究》(1986 年汇编本)(北京:人民出版社,1988 年)
\bibitem{} 中共中央文献研究室编:《文献和研究》(19 印年汇编本)(北京:档案出版社,1991 年)
\bibitem{} 中共中央文献研究室、中国人民解放军军事科学院编:《毛泽东军事文集》,第 1 一 6 卷(北
京:军事科学出版社、中央文献出版社,1993 年)
\bibitem{} 中共中央书记处编:《六大以来——党内秘密文件》(北京:人民出版社,1980 年)
\bibitem{} 中共中央党史研究室编:《中共党史大事年表》(北京:人民出版社,1987 年)
\bibitem{} 中共中央编译局国际共运史研究室编:《国际共运史研究资料增刊》(卢森堡专辑)(北京:
人民出版社,1981 年)
\bibitem{} 中共中央党史资料征集委员会、中共中央党史研究室编:《中共党史资料》,第 1 一 4 辑(北
京:中共中央党校出版社,1982 年)。
\bibitem{} 中共中央党史资料征集委员会、中共中央党史研究室编:《中共党史资料》,第 5 一 8 辑(北
京:中共党史资料出版社,1983 年)
\bibitem{} 中共中央党史资料征集委员会编:《中共党史资料》,第 9 一 28 辑(北京:中共党史资料出
版社,1984—1988 年)。
\bibitem{} 中共中央党史研究室编:
《中共党史资料》,第 29 一 36 辑(北京:中共党史资料出版社,1989
—1990 年)。
\bibitem{} 中共中央党史研究室编:《中共党史资料》,第 37 一 60 辑(北京:中共党史出版社,1991
—1996 年)
\bibitem{} 中共中央党史研究室、中央档案馆编:《中共党史资料》,第 61 一 67 辑(北京:中共党史出
版社,1997—1998 年)。
\bibitem{} 中共中央党史资料征集委员会、中央档案馆褊:《遵义会议文献》(北京:人民出版社,1985
年)
\bibitem{} 中共天津市委组织部、中共天津党史资料征集委员会、天津市档案馆编:《中国共产党天津组
\bibitem{} 织史资椰(1920——1987(北京:中国城市出版社,1991 年)
\bibitem{} 中共四川省委党史工作委员会《吴玉章传》编写组编:《吴玉章文集》(重庆:重庆出版杜,
1987 年)
\bibitem{} 中央档案馆编:《秋收起义》(资料选辑)(北京:中共中央党校出版社,1982 年)
\bibitem{} 中央档案馆编:《中共党史报告选编》(北京:中共中央党校出版社,1982 年)
\bibitem{} 中国人民解放军政治学院编:《中共党史参考资料》,1——11 册(北京:中国人民解放军政
治学院印行,1979 年)
\bibitem{} 中国人民解放军政治学院编:《中共党史教学参考资料》,12——14 册(北京:中国人民解
放军政治学院印行,1985 年)
\bibitem{} 中国人民解放军国防大学党史党建政工教研室褊:《中共党史教学参考资料》,15——17 册
(北京:中国人民解放军国防大学印行,1989 年)。
\bibitem{} 中国人民解放军军事科学院编:
《毛泽东军事文选》
(北京:
中国人民解放军战士出版社,
1981
年)。
\bibitem{} 中国杜会科学院新闻研究所编:《新闻研究资料》,第 17、18 辑(北京:中国社会科学出版
社,1983 年)。
\bibitem{} 中国社会科学院新闻研究所中国报刊史研究室编:《延安文萃》(北京:北京出版社,1984
年)
\bibitem{} 中国现代史学会编:《中国现代史论文摘编》(郑州:河南人民出版社,1984 年)
\bibitem{} 《中国现代革命史资料丛刊——米夫关于中国革命言论》(北京:人民出版社,1986 年)
\bibitem{} 《王明言论选辑》(北京:人民出版社,1982 年)
\bibitem{} 王洪模主编:《中共党史文摘年刊》(1989 年)(北京:中共党史出版社,1992 年)
\bibitem{} 《中国共产党组织史资料汇编——领导机构沿革和成员名录》
(北京:红旗出版社,
王建英编:
1983 年)
\bibitem{} 王朝美主编:《中共党史文摘年刊》(1990 年)(北京:中共党史出版社,1994 年)
\bibitem{} 王焰主编:《彭德怀年谱》(北京:人民出版社,1998 年)
\bibitem{} 《王稼祥选集》(北京:人民出版社,1989 年)
\bibitem{} 《毛泽东选集》,第 1 卷(无出版地点,苏中出版社,1945 年)
\bibitem{} 毛泽东:《毛泽东选集》,1 一 4 卷(北京:人民出版社,1951——1960 年)
\bibitem{} 《毛泽东选集》,第 5 卷(北京:人民出版社,1977 年)
\bibitem{} 《毛泽东文集》,第 1 一 5 卷(北京:人民出版社,1993——1996 年)
\bibitem{} 《毛泽东文选·毛泽东思想万岁》(无印行单位、地点,1967 年)
\bibitem{} 《毛泽东在七大的报告和讲话集》(北京:中央文献出版杜,1995 年)
\bibitem{} 《毛泽东书信选集》(北京:人民出版社,1984 年)
\bibitem{} 《毛泽东论党的历史》(南京:南京大学印行,无出版日期)
\bibitem{} 《毛泽东思想万岁》(无印行单位、地点,1967 年)
\bibitem{} 《毛泽东思想万岁》(据日本小仓编集企画版重印,无印行单位、地点,1967 年)
\bibitem{} 《毛泽东思想万岁》(第三辑)(无印行单位、地点,1967 年)
\bibitem{} 《毛泽东思想万岁》(第四辑)(无印行单位、地点,1967 年)
\bibitem{} 《毛泽东思想万岁》(无印行单位、地点,1969 年)
\bibitem{} 《毛泽东新闻工作文选》(北京:新华出版社,1983 年)
\bibitem{} 《毛泽东论文艺》(北京:人民文学出版社,1983 年)
\bibitem{} 《毛泽东诗词选》(北京:人民文学出版社,1986 年)
\bibitem{} 太行革命根据地史总编委会编:《党的建设》,太行革命根据地史料丛书之二(太原:山西人
民出版社,1989 年)
\bibitem{} 太行革命根据地史总编委会编:《公安保卫工作》,太行革命根据地史料丛书之九(太原:山
西人民出版社,1989 年)
\bibitem{} 本庄比佐子编:《王明选集》,1 一 5 卷(东京:汲古书院,1971——1975 年)
\bibitem{} 《生活全国总书目》(1935)(上海:上海生活书店印行,1935 年)
\bibitem{} 《朱德选集》(北京:人民出版社,1983 年)
\bibitem{} 《任弼时选集》(北京:人民出版社,1987 年)
\bibitem{} 共青团中央青运史研究室、共青团陕西省委青运史研究室编:
\bibitem{} 《安吴古堡的钟声——安吴青训
班史料集》(北京:中央党史资料出版社,1987 年)
\bibitem{} 竹内实编:《毛泽东集》,第 1—10 卷(东京:北望社;1970—1972 年)
\bibitem{} 江西省档案馆编:《井冈山革命根据地史料选编》(南昌:江西人民出版社,1986 年)
\bibitem{} 江西省档案馆、中共江西省委党校党史教研室编:《中央革命根据地史料选编》(南昌:江西
人民出版社,1983 年)
\bibitem{} 《江青同志讲话选编》(北京:人民出版社,1968 年)
\bibitem{} 江苏省社科院《恽逸群文集》编选组编:《恽逸群文集》(南京:江苏人民出版社,1986 年)
\bibitem{} 《李伯钊文集》编辑委员会编:《李伯钊文集》(北京:解放军出版社,1989 年)
\bibitem{} 李烈主编:《贺龙年谱》(北京:人民出版社,1996 年)
\bibitem{} 《李维汉选集》编辑组编:《李维汉选集》(北京:人民出版社,1987 年)
\bibitem{} 《李富春选集》编辑组编:《李富春选集》(北京:中国计划出版社,1992 年)。
\bibitem{} 《抗战初期的中共中央长江局》(武汉:湖北人民出版社,1991 年)。
\bibitem{} 《延安自然科学院史料》编辑委员会编:《延安自然科学院史料》(北京:中共党史资料出版
社、北京工业学院出版社,1986 年)。
\bibitem{} 延安整风运动编写组编:《延安整风运动纪事》(北京:求实出版社。1982 年)。
\bibitem{} 吴相湘主编:《中国现代史料丛书》,第 3 辑(台北:文星书店,1962 年)。
\bibitem{} 《何干之文集》(北京:中国人民大学出版社,1989 年)。
\bibitem{} 《建国以来毛泽东文稿》,1——13 册(北京:中央文献出版社,1987—1998 年)。
\bibitem{} 《建国以来刘少奇文稿》,第 1 册(北京:中央文献出版社,1998 年)。
\bibitem{} 《周恩来选集》(北京:人民出版社,1980 年)。
\bibitem{} 《周恩来书信选集》(北京:中央文献出版社,1988 年)。
\bibitem{} 周国全、郭德宏编:《王明年谱》(合肥:安徽人民出版杜,1991 年)。
\bibitem{} 《周扬文集》(北京:人民文学出版社,1984 年)。
\bibitem{} 武衡主编:《徐特立文存》(广州:广东教育出版社,1995 年)。
\bibitem{} 《胡乔木文集》,1 一 3 卷(北京:人民出版社,1992——1994 年)。
\bibitem{} 《革命史资料》,第 3 辑(北京:文史资料出版社,1981 年)。
\bibitem{} 《革命史资料》,第 15 辑(北京:中国文史出版社,1986 年)。
\bibitem{} 《革命史资料》,第 17 辑(北京:中国文史出版社,1987 年)。
\bibitem{} 《南方局党史资料——党的建设》(重庆:重庆出版社,1990 年)。
\bibitem{} 淮南抗日根据地编审委员会编:
《淮南抗日根据地》
(北京:中共党史资料出版社,1986 年)。
\bibitem{} 《陈绍禹(王明)救国言论选集》(汉口:中国出版社,1938 年)。
\bibitem{} 《陈云文选》(1926——1949)(北京:人民出版社;1984 年)。
\bibitem{} 《张闻天文集》,第 1 一 4 集,(北京:中共党史出版社,1993——1995 年)。
\bibitem{} 《张闻天选集》(北京:人民出版社;1985 年)。
\bibitem{} 《张闻天选集》传记组编、张培森主编:《张闻天在 1935——1958》(年谱)(北京:中共
党史出版社,1997 年)。
\bibitem{} 《习仲勋文选》(北京:中央文献出版社,1995 年)。
\bibitem{} 《陆定一文集》(北京:人民出版社,1992 年)。
\bibitem{} 《陶铸文集》(北京:人民出版社;1987 年)。
\bibitem{} 《雪峰文集》(北京:人民文学出版社,1985 年)。
\bibitem{} 「国防部新闻局」编印:《共匪反动文件汇编》(文化教育)(无出版时期、地点)。
\bibitem{} 「国防部新闻局」编印:《共匪反动文件汇编》(社会运动)(无出版时期、地点)。
\bibitem{} 《董必武年谱》编辑组编:《董必武年谱》(北京:中央文献出版社。1991 年)
\bibitem{} 华东水利学院《革联》编:
《毛主席论教育》
(南京:南京大学八二七兵团教改办公室材料俎,
1967 年)。
\bibitem{} 彭真:
《关于晋察冀边区党的工作和具体政策报告》
(北京:中共中央党校出版社,1981 年)。
\bibitem{} 《彭真文选》(1941—1990)(北京:人民出版社,1991 年)。
\bibitem{} 彭德怀传记编写组编:《彭德怀军事文选》(北京:中央文献出版社,1988 年)。
\bibitem{} 解放社编:《整风文献》(订正本)(上海:新华书店,1950 年)。
\bibitem{} 《晋绥革命根据地大事记》(太原:山西人民出版社,1989 年)
\bibitem{} 《晋察冀抗日根据地》史料丛书编审委员会、中央档案馆编:《晋察冀抗日根据地》(北京:
中共党史资料出版社,1989 年
\bibitem{} 《叶剑英选集》(北京:人民出版社,1996 年)
\bibitem{} 《新四军和华中抗日根据地史料选》,第 7 辑(上海:上海人民出版社,1984 年)
\bibitem{} 《新闻工作文集》(北京:解放军报社,1979 年)
\bibitem{} 《广东文史资料》,第 33 辑(广州:广东人民出版社,1981 年)
(上海:华东师范大学出版社,
\bibitem{} 翟作君、邬正洪主编:《中国革命史研究菅萃(1911—1949)》
1986 年)
\bibitem{} 福建省档案馆编:《福建事变档案资料》(福州:福建人民出版社,1984 年)
\bibitem{} 蔡尚思主编、姜义华编:《中国现代思想史资料简编》,第 3、1 卷(杭州:浙江人民出版社,
1983 年)
\bibitem{} 廖盖隆主编:《中共党史文摘年刊》(1984 年)(北京:中共党史资料出版社,1987 年)
\bibitem{} 廖盖隆主编:《中共党史文摘年刊》(1985 年)(北京:中共党史资料出版社,1987 年)
\bibitem{} 廖盖隆主编:《中共党史文摘年刊》(1986 年)(北京:中共党史资料出版社,1988 年)
\bibitem{} 廖盖隆主编:《中共党史文摘年刊》(1987 年)(北京:中共党史资料出版社,1990 年)
\bibitem{} 《邓小平文选》(1938——1965)(北京:人民出版社,1989 年)
\bibitem{} 邓力群:《延安整风以後》(北京:当代中国出版社,1998 年)
\bibitem{} 《刘少奇选集》(北京:人民出版社,1981——1986 年)
\bibitem{} 刘增杰、赵明等编:《抗日战争时期延安及抗日民主根据地文学运动史料》,上、中、下(太
原:山西人民出版社,1983 年)
\bibitem{} 刘树发主编:《陈毅年谱》(北京:人民出版社,1995 年)
\bibitem{} 韩劲草主编:《安子文组织工作文选》(北京:中共中央党校出版社,1988 年)
\bibitem{} 边区总学委编:《整顿三风二十二个文件》(延安:1942 年)
\bibitem{} 《薄一波文选》(北京:人民出版社,1992 年)。
\bibitem{} 《苏维埃中国》第 1 集(莫斯科:外国工人出版社,1933 年),中国现代史资料编辑委员会
1957 年翻印。
\bibitem{} 《党史资料丛刊》,1980 年第 2 辑(上海:上海人民出版社,1980 年)。
\bibitem{} 《党史资料丛刊》,1981 年第 1、3 辑(上海:上海人民出版社,1981 年)。
\bibitem{} 孔永松、林太乙、戴金生:《中央革命根据地史要》(南昌:江西人民出版社,1985 年)
\bibitem{} 于光远:《文革中的我》(上海:上海远东出版社,1995 年)。
\bibitem{} 于光远:〈初识陈伯达〉,载《读书》,1998 年第 6 期。
\bibitem{} 中共中央党史研究室:《中国共产党历史(上卷)若干问题说明》(北京:中共党史出版社,
1991 年)。
\bibitem{} 中共中央党史研究室、《中共党史大事年表》编写组编:《中共党史大事年表说明》(北京:
中央党校出版社,1983 年)。
\bibitem{} 中共中央文献研究室编:
\bibitem{} 《毛泽东传(1893——1949)》
(北京:中央文献出版社,1996 年)。
\bibitem{} 中共中央文献研究室编:《刘少奇传》(北京:中央文献出版社,1998 年)。
\bibitem{} 中共中央文献研究室编:《周恩来传(1898——1949)》(北京:中央文献出版社、人民出版
社,1990 年)。
\bibitem{} 中共中央文献研究室编:《任弼时传》(北京:中央文献出版社、人民出版社,1994 年)
\bibitem{} 中共中央文献研究室编:《朱德传》(北京:人民出版社、中央文献出版社,1993 年)
\bibitem{} 中共中央文献研究室:《关于建国以来党的若干历史问题的决议注释本》(修订)(北京:人
民出版社,1985 年)
\bibitem{} 《中共党史人物传》,第 1 卷(西安:陕西人民出版社,1980 年)。
\bibitem{} 《中共党史人物传》,第 8 卷(西安:陕西人民出版社,1983 年)。
\bibitem{} 《中共党史人物传》,第 25 卷(西安:陕西人民出版社,1983 年)。
\bibitem{} 《中共党史人物传》,第 33 卷(西安:陕西人民出版社,1987 年)。
\bibitem{} 《中共党史人物传》,第 34 卷(西安:陕西人民出版社,1987 年)。
\bibitem{} 中共上海市委党史研究室、金立人、李华、李小苏著:《王明「左」倾冒险主义在上海》(上 海:上海远东出版社,1994 年)。
\bibitem{} 中共江苏省委党史工作办公室:《中共江苏地方史》(1919——1949)(南京:江苏人民出版 社,1996 年)。
\bibitem{} 中共河南省党史工作委员会编:《怀念危拱之》(郑州:河南人民出版社,1986 年)。
\bibitem{} 《中共现形》(无出版地点,爱国出版社,1946 年)。
\bibitem{} 中央研究院近代史研究所编:《抗战建国史研讨会论文集》(台北:1985 年)。
\bibitem{} 中央档案馆党史资料研究室:〈延安整风中的王明——兼驳王明的《中共五十年》〉,载《党 史通讯》,1984 年第 7 期。
\bibitem{} 《中国新民主主义时期新闻事业史》(杭州:杭州大学新闻系翻印,1962 年)
\bibitem{} 中华人民共和国卫生部《苏井观传》编写组:《苏井观传》(北京:北京医科大学、中国协和 医科大学联合出版社,1991 年)
\bibitem{} 《毛泽东的哲学活动——回忆与评述》(北京:中共中央党校科研办公室,1985 年)
\bibitem{} 《毛泽东自述》(北京:人民出版社,1993 年)
\bibitem{} 尹均生主编:《中外名记者眼中的延安解放区》(武汉:华中师范大学出版社,1995 年)
\bibitem{} 尹骈:《潘汉年传》(北京:中国人民公安大学出版社,1996 年)
\bibitem{} 尹骈:《潘汉年的情报生涯》(北京:人民出版社,1996 年)
\bibitem{} 王力:《现场历史——文化大革命纪事》(香港:牛津大学出版社,1993 年)
\bibitem{} 王必胜:《邓拓评传》(北京:群众出版社,1986 年)
\bibitem{} 王仲清主编:《党校教育历史概述》(1927——1947)(北京:中共中央党校出版社,1992 年)
\bibitem{} 王克之编:《延安内幕》(上海:经纬书店,1946 年)
\bibitem{} 王良:〈罗烽、白朗蒙冤散记〉,载《新文学史料》,1995 年第 2 期。
\bibitem{} 王明:《中共五十年》(北京:现代史料编刊社,1981 年)
\bibitem{} 王素园:(陕甘宁边区「抢救运动」始末),载中共中央党史研究室编:《中共党史资料》, (北京:中共党史出版社,1991 年)
\bibitem{} 王培元:《抗战时期的延安鲁艺》(桂林:广西师范大学,1999 年)
\bibitem{} 王首道:《王首道回忆录》(北京:解放军出版社,1988 年)
\bibitem{} 王首道:《怀念集》(长沙:湖南人民出版社,1983 年)
\bibitem{} 王秀鑫:〈延安『抢救运动』述评〉,载《党的文献》,1990 年第 3 期。
\bibitem{} 王秀鑫:〈对 1949 年 12 月 22 日季米特洛夫给毛泽东信的评述〉,载黄修荣主编:《苏联、 共产国际与中国革命的关系新探》(北京:中共党史出版社,1995 年)
\bibitem{} 王超北口述、师宁编写:《来自秘密战线的报告》(天津:百花文艺出版社。1997 年)
\bibitem{} 王琳:《狂飙诗人柯仲平传》(北京:中国文联出版社,1992 年)
\bibitem{} 王蒙、袁鹰主编:《忆周扬》(呼和浩特:内蒙古人民出版社,1998 年)。
\bibitem{} 王辅一:《项英传》(北京:中共党史出版社,1995 年)。
\bibitem{} 王实味等著,沉默编:《野百合花》(广州:花城出版社,1992 年)。
\bibitem{} 王德芬:〈萧军在延安〉,载《新文学史料》,1987 年第 4 期。
\bibitem{} 王德芬:〈安息吧,萧军老伴!〉,载《新文学史料》,1989 年第 2 期。
\bibitem{} (苏)瓦·崔可夫著,万成才译: 《在华使命——一个军事顾问的笔记》 (北京:新华出版社, 1980 年)。
\bibitem{} 公安部档案馆编注: 《在蒋介石身边八年——侍从室高级幕僚唐纵日记》 (北京:群众出版社, 1991 年)。
\bibitem{} 方长明:〈试述共产国际与我党对闽变的策略〉,载《党史资料与研究》(福建),1983 年 第 3 期。
\bibitem{} 《文化灵苗播种人——姜椿芳》(北京:中国文史出版社,1990 年)。
\bibitem{} 甘棠寿、王致中、郭维仪主编: 《陕甘宁革命根据地史研究》 (西安:三秦出版社,1988 年)。
\bibitem{} 史锋:《反对王明投降主义路线的斗争》(上海:上海人民出版社,1976 年)。
\bibitem{} (俄)尼·费德林著,周爱琦译: 《我所接触的中苏领导人》 (北京:新华出版社,1995 年)。
\bibitem{} (美)尼姆·威尔斯著,陶宜、徐复译:《续西行漫记》(北京:生活·读书·新知三联书店, 1991 年)。
\bibitem{} (苏)弗拉基米洛夫著,周新译:《延安日记》(台北:联经出版事业公司,1976 年)。
\bibitem{} (美)白修德著,马清槐、方生证:《探索历史》(北京:生活·读书·新知三联书店,1987 年)。
\bibitem{} 古屋奎二:《蒋总统秘录》(台北:中央日报社,1977 年)。
\bibitem{} (美)D·包瑞德著,万高潮、卫大匡等译:《美军观察组在延安》,(北京:解放军出版社, 1990 年)。
\bibitem{} 本书编辑组编:《回忆潘汉年》(南京:江苏人民出版社,1985 年)。
\bibitem{} 石志夫、周文琪编:《李德与中国革命》(北京:中共党史资料出版社,1987 年)。
\bibitem{} 《回忆王稼祥》(北京:人民出版社,1985 年)。
\bibitem{} 《回忆邓子恢》编辑委员会编:《回忆邓子恢》(北京:人民出版社。1996 年)。
\bibitem{} 《回忆雪峰》(北京:中国文史出版社,1986 年)。
\bibitem{} 仲侃:《康生评传》(北京:红旗出版社,1982 年)。
\bibitem{} 向青:《共产国际与中国革命关系论文集》(上海:上海人民出版社,1985 年)。
\bibitem{} 向青:烘产国际和中国革命关系史稿》(北京:北京大学出版社,1988 年)。
\bibitem{} 《李先念传》编写组编,朱玉主编:《李先念传》(1909——1949)》(北京:中央文献出版 社,1999 年)。
\bibitem{} 李新:《回望流年——李新回忆录续编》(北京:北京图书馆出版社,1998 年)。
\bibitem{} 朱仲丽:《黎明与晚霞》(北京:解放军出版社,1986 年)。
\bibitem{} 朱镜明:《李达传记》(武汉:湖北人民出版社,1986 年)。
\bibitem{} 司马璐:《斗争十八年》(全本)(香港:亚洲出版社,1952 年)。
\bibitem{} 司马璐:《斗争十八年》(节本)(香港:自联出版社,1967 年)。
\bibitem{} 艾克恩编:《延安文艺回忆录》(北京:中国社会科学出版社,1992 年)。
\bibitem{} (美)安娜·路易斯·斯特朗著,陈裕年译: 《安娜·路易斯·斯特朗回忆录》 (北京:生活·读 书·新知三联书店,1982 年)。
\bibitem{} 《成仿吾传》编写组编:《成仿吾传》(北京:中共中央党校出版社,1988 年)。
\bibitem{} 江华:《追忆与思考——江华回忆录》(杭州:浙江人民出版社,1991 年)。
\bibitem{} 伍修权:《我的历程(1908—1949)》(北京:解放军出版社,1984 年)。
\bibitem{} 伍修权:《回忆与怀念》(北京:中共中央党校出版社,1991 年)。
\bibitem{} 伊·爱泼斯坦:《突破封锁访延安——1944 年的通讯和家书》(北京:人民日报出版社,1995 年)。
\bibitem{} (美)伊罗生著,刘海生译:《中国革命的悲剧》,东亚丛书第 11 种(香港:和记印刷有限 公司,1973 年)。
\bibitem{} 曲士培:《抗日战争时期解放区高等教育》(北京:北京大学出版社,1985 年)。
\bibitem{} 《匡亚明纪念文集》编委会褊:《匡亚明纪念文集》(南京:南京大学出版社,1997 年)。
\bibitem{} (俄)安·麦·列多夫斯基:〈在中国的外交生涯(1942——1952)〉,载俄《近代史和现代 史》杂志,1996 年第 6 期。
\bibitem{} (俄)安·麦·列多夫斯基:(外交官笔记(1942——1952)〉,载俄《近代史和现代史》杂 志,1996 年第 6 期。
\bibitem{} (苏)托洛茨基著,王家华、张海滨译: 《论列宁》 (北京:生活·读书·新知三联书店,1980 年)。
\bibitem{} (苏)米·伊·巴斯马诺夫著,苗为振、陈永秀、阎志民译:《三十至七十年代的托洛茨基主 义》(北京:人民出版社,1983 年)。
\bibitem{} 吕正操:《吕正操回忆录》(北京:解放军出版社,1988 年)
\bibitem{} 《延安中央党校的整风学习》,第 1 集(北京:中共中央党校出版社,1988 年)
\bibitem{} 《延安中央党校的整风学习》,第 2 集(北京:中共中央党校出版社,1989 年)
\bibitem{} 《延安中央研究院回忆录》(北京:中国社会科学出版社、湖南人民出版社,1984 年)
\bibitem{} 《抗战初期的八路军驻南京办事处》(南京:南京大学出版社;1987 年)
\bibitem{} 李一氓:《模糊的荧屏——李一氓回忆录》(北京:人民出版社,1992 年)
\bibitem{} 李天民:《林彪评传》(香港:明报月刊社,1978 年)
\bibitem{} 李天民著,邓锦辉等译:《刘少奇传》(长沙:湖南人民出版社,1989 年)
\bibitem{} 李志民:《革命熔炉》(北京:中共党史资料出版社,1986 年)
\bibitem{} 李志英:《博古传》(北京:当代中国出版社;1994 年)
\bibitem{} 李南央:〈六十年恩怨情仇:记我的父亲李锐和母亲范元甄〉,载《开放》(香港),1997 年 12 月号。
\bibitem{} 李雪峰:《李雪峰回忆录(上)——太行十年》(北京:中共党史出版社,1998 年)
\bibitem{} 李逸民:《李逸民回忆录》(长沙:湖南人民出版社,1986 年)
\bibitem{} 李维民、潘天嘉:《罗荣桓在山东》(北京:人民出版社,1986 年)
\bibitem{} 李维汉:《回忆与研究》(北京:中共党史资料出版社,1986 年)
\bibitem{} 李辉:《胡风集团冤案始末》北京:人民日报出版社,1989 年)
\bibitem{} 李辉:《李辉文集·往事苍老》(广州:花城出版社,1998 年)
\bibitem{} 李锐:《怀念廿篇》(北京:生活·读书·新知三联书店,1987 年)
\bibitem{} 李锐:《庐山会议实录》(北京:春秋出版社,湖南教育出版社,1989 年)。
\bibitem{} 李锐:《毛泽东早年读书生活》(沈阳:辽宁人民出版社,1992 年)。
\bibitem{} 李锐:《李锐往事琐忆》(南京:江苏人民出版社,1995 年)。
\bibitem{} 李锐:《直言》(北京:今日中国出版社,1998 年)。
\bibitem{} 李樵:《徐以新传》(北京:世界知识出版社,1996 年)。
\bibitem{} 吴介民主编:《延安马列学院回忆录》(北京:中国社会科学出版社,1991 年)。
\bibitem{} 志平编:《历史不容歪曲——评〈项英传〉(南京:无出版单位,1997 年)。
\bibitem{} 沙汀:《沙河自传》(太原:北岳文艺出版社 1998 年)。
\bibitem{} 余伯流、夏道汉:《井冈山革命根据地研究》(南昌:江西人民出版社,1987 年)。
\bibitem{} 杜重石:《风雨岁月》(香港:天地图书有限公司,1993 年)。
\bibitem{} 宋任穷:《宋任穷回忆录〉(北京:解放军出版社,1994 年)。
\bibitem{} 宋金寿:〈关于王实味问题〉,载中共中央党史研究室编:《党史通讯》,1984 年第 8 期。
\bibitem{} 宋晓梦:〈李锐与延安「轻骑队」〉,载广州《岭南文化时报》,1998 年 9 月 10 日。
\bibitem{} 宋晓梦:《李锐其人》(郑州:河南人民出版社,1999 年)。
\bibitem{} 言行:〈高长虹传略〉,载《新文学史料》,1990 年第 4 期
\bibitem{} 言行:〈高长虹晚年的「萎缩」〉,载《新文学史料》,1996 年第 4 期。
\bibitem{} 何满子口述,吴仲华整理:《跋涉者——何满子自传》(北京:北京大学出版社,1999 年)。
\bibitem{} 沈志华:《毛泽东、斯大林与韩战——中苏最高机密档案》(香港:天地图书有限公司,1998 年)。
\bibitem{} 金东平:《延安见闻录》(重庆:民族书店,1945 年)
\bibitem{} 金城:《延安交际处回忆录》(北京:中国青年出版社,1986 年)
\bibitem{} 金观涛、刘青峰著: 《开放中的变迁——再论中国社会超稳定结构》
\bibitem{} (香港:中文大学出版杜, 1993 年)
\bibitem{} 范青:《陈昌浩传》(北京:中共党史出版社,1993 年)
\bibitem{} 《林渠传》编写组编:《林渠传》(北京:红旗出版社,1986 年)
\bibitem{} 周良沛:《丁玲传》(北京:北京十月文艺出版社,1993 年)
\bibitem{} 周国全、郭德宏等:《王明评传》(合肥:安徽人民出版社,1989 年)
\bibitem{} 周森:《马海德》(北京:生活·读书·新知三联书店,1982 年)
\bibitem{} 〈周扬关于现代文学的一次谈话〉,载《新文学史料》,1990 年第 1 期。
\bibitem{} 周维仁:《贾拓夫传》(北京:中共党史出版社,1993 年)
\bibitem{} 周鲸文:《风暴十年》(香港:时代批评社,1962 年)
\bibitem{} 《革命回忆录》增刊(1)(北京:人民出版社,1981 年)
\bibitem{} 季羡林:《牛棚杂忆》(北京:中共中央党史出版社,1998 年)
\bibitem{} 茅盾:《我走过的道路》(北京:人民文学出版社,1997 年)
\bibitem{} 武衡:《无悔——武衡回忆录》(北京:专利文献出版社,1997 年)
\bibitem{} 珏石:〈周恩来与抗战初期的长江局〉,载《中共党史研究》,1998 年第 2 期。
\bibitem{} 逄先知(1950——1966 年曾负责管理毛泽东图书):〈关于党的文献编辑工作的几个问题〉, 载《文献和研究》,1987 年第 3 期。
\bibitem{} 胡风:《胡风晚年作品选》(南宁:漓江出版社,1987 年)
\bibitem{} 胡风:〈再返重庆〉(之二),载《新文学史料》,1989 年第 1 期。
\bibitem{} 胡风:《胡风回忆录》(北京:人民文学出版社,1997 年)
\bibitem{} 胡乔木:《胡乔木回忆毛泽东》(北京:人民出版社,1994 年)
\bibitem{} 姚艮:《一个朝圣者的囚徒经历》(北京:群众出版社,1995 年)。
\bibitem{} 施巨流:《王明问题研究》(成都:无出版单位,1998 年)
\bibitem{} 侯外庐:《韧的追求》(北京:生活·读书·新知三联书店,1985 年)。
\bibitem{} 段建国、贾岷岫著,罗青长审核:《王世英传奇》(太原:山西人民出版社,1992 年)
\bibitem{} 《星火燎原》,第 6 集(北京:人民文学出版社,1962 年)
\bibitem{} 《陕甘宁边区法制史话》(诉讼狱政篇)(北京:法律出版社,1987 年)
\bibitem{} 修来荣:《陈龙传》(北京:群众出版社,1995 年)
\bibitem{} 韦君宜:《思痛录》(北京:十月文艺出版社,1998 年)
\bibitem{} 马洪武主编:《中国革命根据地史研究》(南京:南京大学出版社,1992 年)
\bibitem{} 马员生:《旅苏纪事》(北京:群众出版社,1987 年)
\bibitem{} 马齐彬、陈绍畴:〈刘少奇与华北抗日根据地的创立〉,载《文献和研究》,1986 年第 5 期。
\bibitem{} (苏)马克西姆·马克西莫维奇·高尔基著,朱希渝译:《不合时宜的思想——关于革命与文 化的思考》(南京:江苏人民出版社,1998 年)
\bibitem{} 师哲:《在历史巨人身边——师哲回忆录》(北京:中央文献出版社,1991 年)
\bibitem{} 师哲:《峰与谷——师哲回忆录》(北京:红旗出版社,1992 年)。
\bibitem{} 夏衍:《懒寻旧梦录》(北京:生活·读书·新知三联书店,1985 年)
\bibitem{} 唐天然:〈有关延安文艺运动的「党务广播」稿〉,载《新文学史料》,1991 年第 2 期。
\bibitem{} 唐有章口述,刘普庆整理:《革命与流放》(长沙:湖南人民出版社。1988 年)。
\bibitem{} 唐纯良:《李立三传》(哈尔滨:黑龙江人民出版社,1984 年)。
\bibitem{} 唐瑜编: 《零落成泥香如故——忆念潘汉年、董慧》 (北京:生活·读书·新知三联书店,1984 年)。
\bibitem{} 〈秦邦宪与《无锡评论》〉,载《江苏出版史志》,1991 年第 3 期。
\bibitem{} 徐中远:《毛泽东读评五部古典小说》(北京:华文出版社,1997 年)。
\bibitem{} 徐正明、许俊基等译:《共产国际与中国革命——苏联学者论文选译》(成都:四川人民出版 社,1987 年)。
\bibitem{} 徐则浩:《王稼祥传》(北京:当代中国出版社,1996 年)。
\bibitem{} 《徐特立在延安》(西安:陕西教育出版社,1991 年)。
\bibitem{} 《徐复观杂文续集》(台北:时报文化出版有限公司,1986 年)。
\bibitem{} 徐懋庸:《徐懋庸回忆录》(北京:人民文学出版社,1982 年)。
\bibitem{} 孙志远:《感谢苦难:彦涵传》(北京:人民文学出版社,1997 年)。
\bibitem{} 孙春山:《无悔人生·杨献珍》(济南:山东昼报出版社,1997 年)。
\bibitem{} 孙新元、尚德全编: 《延安岁月——延安时期革命美术回忆录》 (西安:陕西人民美术出版社, 1985 年)。
\bibitem{} 高华:〈毛泽东与 1937 年的刘、洛之争〉;载《南京大学学报(哲社版)》,1993 年第 3 期。
\bibitem{} 〈在道与势之间——毛泽东为发动延安整风运动所作的准备〉,载《中国社会科学季刊》 高华: (香港),1993 年秋季号。
\bibitem{} 高陶:《天涯萍踪——记萧三》(北京:中国青年出版社,1991 年)。
\bibitem{} (美)莫里斯·迈斯纳著;中共中央文献研究室《国外研究毛泽东思想资料选辑》编辑组编译: 《毛泽东与马克思主义、乌托邦主义》(北京:中央文献出版社,1991 年)。
\bibitem{} (苏)索波列夫等著,吴道弘等译:《共产国际史纲》(北京:人民出版社,1985 年)
\bibitem{} (俄)格·阿·阿尔巴托夫著,徐葵等译:《苏联政治内幕:知情者的见证》(北京:新华出 版社,1998 年)
\bibitem{} (美)埃德加·斯诺著,奚博铨译:《红色中华散记》(南京:江苏人民出版社,1992 年)
\bibitem{} 梁漱溟:《我的努力与反省》(南宁:漓江出版社,1987 年)
\bibitem{} 曹仲彬、戴茂林:《王明传》(长春:吉林文史出版社,1991 年)
\bibitem{} 曹伯一: 《江西苏维埃之建立及其崩溃(1931——1934)》 (台北:国立政治大学东亚研究所, 1969 年)
\bibitem{} 陈丕显:《苏中解放区十年》(上海:上海人民出版社,1988 年)
\bibitem{} 陈永发:〈延安的整风、审干与肃反〉,载《抗战建国史研讨会论文集》(1937——1945), 下册(台北:中央研究院近代史研究所,1985 年)
\bibitem{} 陈永发:《延安的阴影》(台北:中央研究院近代史研究所,1990 年)。
\bibitem{} 陈明:〈丁玲在延安——她不是主张暴露黑暗派的代表人物〉,载《新文学史料》,1993 年 第 2 期。
\bibitem{} 陈晋:《毛泽东的文化性格》(北京:中国青年出版社,1991 年)
\bibitem{} 陈晋:《毛泽东与文艺传统》(北京:中央文献出版社,1992 年)
\bibitem{} 陈荷夫编:《张友渔回忆录》(北京:北京大学出版社,1990 年)
\bibitem{} 陈恭怀:(陈企霞传略),载《新文学史料》,1989 年第 3 期。
\bibitem{} 陈野苹、韩劲草主编:《安子文传略》(太原:山西人民出版社,1985 年)
\bibitem{} 陈绍畴:《刘少奇在白区》(北京:中共党史出版社,1992 年)
\bibitem{} 陈绍畴主编:《刘少奇研究述评》(北京:中央文献出版社,1997 年)
\bibitem{} 陈复生(陈湖生): 《三次被开除党籍的人——一个老红军的自述》 (北京:文化艺术出版社, 1992 年)。
\bibitem{} 陈碧兰:《我的回忆》(香港:十月书屋,1994 年)。
\bibitem{} 陈翰笙:《四个时代的我》(北京:中国文史出版社,1988 年)。
\bibitem{} 《国外中国近代史研究》,第 13 辑(北京:中国社会科学出版社,1989 年)。
\bibitem{} 许纪霖:《智者的尊严——知识分子与近代文化》(上海:学林出版社,1991 年)。
\bibitem{} 黄火青:《一个平凡共产党员的经历》(北京:人民出版社,1995 年)。
\bibitem{} 黄平:《往事回忆》(北京:人民出版社,1981 年)。
\bibitem{} 黄克诚:《黄克诚回忆录》(北京:解放军出版社,1989 年)。
\bibitem{} 黄克诚:《黄克诚自述》(北京:人民出版社,1994 年)。
\bibitem{} 黄炎培:《八十年来》(北京:文史资料出版社,1982 年)。
\bibitem{} 黄樾:《延安四怪》(北京:中国青年出版社,1998 年)。
\bibitem{} 黄药眠:《动荡:我所经历的半个世纪》(上海:上海文艺出版社。1987 年)。
\bibitem{} 盛岳:《莫斯科中山大学与中国革命》(北京:现代史料编刊社,1980 年)。
\bibitem{} 张劲夫:《怀念集》(北京:中共中央党校出版社,1994 年)。
\bibitem{} 张重天:《共和国第一冤案》(北京:华艺出版社,1989 年)。
\bibitem{} 张国焘:《我的回忆》(北京:现代史料编刊社,1980 年)。
\bibitem{} 张贻玖:《毛泽东和诗》(北京:春秋出版社,1987 年)。
\bibitem{} 张贻玖:《毛泽东读史》(北京:中国友谊出版公司,1991 年)。
\bibitem{} 张毓茂:《萧军传》(重庆:重庆出版社,1992 年)。
\bibitem{} 张毓茂:〈我所知道的萧军先生〉,载《新文学史料》,1989 年第 2 期。
\bibitem{} 张静如、唐曼珍主编:《中共党史学史》(北京:中国人民大学出版社,1990 年)。
\bibitem{} 张积玉、王钜春编:《马克思主义理论家翻译家张仲实》(西安:陕西人民教育出版社,1991 年)
\bibitem{} 郭化若:《郭化若回忆录》(北京:军事科学出版社,1995 年)
\bibitem{} 郭华伦:《中共史论》(台北:国际关系研究所、国立政治大学东亚研究所,1971 年)
\bibitem{} 郭晨: 《巾帼列传——红一方面军三十位长征女红军生平事迹》 (北京: 农村读物出版社, 1986 年)
\bibitem{} 郭影秋:《往事漫忆》(北京:中国人民大学出版社,1986 年)
\bibitem{} 毕兴、贺安华:《阎红彦传略》(成都:四川人民出版社,1987 年)。
\bibitem{} (美)斯图尔特·施拉姆著,中共中央文献研究室《国外研究毛泽东思想资料选辑》编辑组编 译:《毛泽东》(北京:红旗出版社,1987 年)
\bibitem{} (美)斯图尔特·施拉姆著,中共中央文献研究室《国外研究毛泽东思想资料选辑》编辑组编译:《毛泽东的思想》(北京:中央文献出版社,1990 年)
\bibitem{} 陶坊资:〈回忆父亲〉;陶瀛孙、陶乃煌:〈〉陶晶孙小传〉,载《新文学史料》,1992 年 第 4 期。
\bibitem{} 华世俊、胡育民:《延安整风始末》(上海:上海人民出版社,1985 年)
\bibitem{} 彭德怀:《彭德怀自述》(北京:人民出版社,1981 年)
\bibitem{} 程中原:《张闻天论稿》(南京:河海大学出版社,1990 年)
\bibitem{} 程中原:《张闻天传》(北京:当代中国出版社,1993 年)
\bibitem{} (苏)奥·鲍里索夫等:《苏中关系(1945——1980)》(北京:生活·读书·新知三联书店, 1982 年)
\bibitem{} (德)奥托·布劳恩(李德): 《中国纪事(1932——1939)》 (北京:现代史料编刊社,1980 年)
\bibitem{} 强晓初等:《延安整风回忆录》(哈尔滨:黑龙江人民出版社,1958 年)。
\bibitem{} 强晓初、李力安、姬也力等主编:《马明方传略》(西安:陕西人民出版社,1990 年)。
\bibitem{} 鲁平:《生活在延安》(西安:新华社,1938 年)。
\bibitem{} 惠浴宇口述,俞黑子记录整理:《朋友人》(南京:江苏人民出版社,1995 年)。
\bibitem{} 开诚:《李克农——中共隐蔽战线的卓越领导人》(北京:中国友谊出版公司,1996 年)。
\bibitem{} 童小鹏:《风雨四十年》,第一部(北京:中央文献出版社,1994 年)。
\bibitem{} 童小鹏:《风雨四十年》,第二部(北京:中央文献出版社,1996 年)。
\bibitem{} 扬帆:《扬帆自述》(北京:群众出版社,1989 年)。
\bibitem{} 温济泽:《王实味冤案平反纪实》(北京:群众出版社,1993 年)。
\bibitem{} 温济泽:《第一个平反的「右派」:温济泽自述》(北京:中国青年出版社,1999 年)。
\bibitem{} (西班牙)费南德·克劳丁著,方光明、商亚南等译:《共产主义运动——从共产国际到共产 党情报局》(福州:福建人民出版社,1983 年)。
\bibitem{} 费云东、余贵华:《中共秘书工作简史》(1921——1949)(沈阳:辽宁人民出版社,1992 年)。
\bibitem{} 费云东主编:《中共保密工作简史(1921——1949)》(北京:金城出版社,1994 年)。
\bibitem{} 董边、镡德山、曾自编:《毛泽东和他的秘书田家英》(北京:中央文献出版社,1989 年)。
\bibitem{} 董边、坛德山、曾自编:《毛泽东和他的秘书田家英》(增订本)(北京:中央文献出版社, 1996 年)。
\bibitem{} 杨子烈:《张国焘夫人回忆录》(原名《往事如烟》)(香港:自联出版社,1970 年)。
\bibitem{} 杨中美:《遵义会议与延安整风》(香港:奔马出版社,1989 年)。
\bibitem{} 杨立:《带刺的红玫瑰——古大存沉冤录》(广州:中共广东省委党史研究室,1997 年)。
\bibitem{} 杨尚昆:〈在全军党史资料征集工作座谈会上的讲话〉(1984 年 7 月 9 日),载中共中央党 史研究室编:《党史通讯》1984 年第 11 期。
\bibitem{} 杨尚昆等著:《我所知道的胡乔木》(北京:当代中国出版物,1997 年)。
\bibitem{} 杨放之:〈《解放日报》改版与延安整风〉,载《新闻研究资料》,第 18 辑。
\bibitem{} 杨奎松:〈毛泽东发动延安整风的台前幕後〉,载《近代史研究》,1998 年第 4 期。
\bibitem{} 杨复沛、吴一虹主编:《从延安到中南海——中共中央部分机要人员的忆》(北京:北京出版 社,1994 年)。
\bibitem{} 杨万青、齐春元:《刘亚褛将军传》(北京:中共党史出版社,1995 年)。
\bibitem{} 《当代中国人物传记》丛书编辑部编:《徐向前传》(北京:当代中国出版社,1992 年)。
\bibitem{} 《当代中国人物传记》丛书编辑部编:《刘伯承传》(北京:当代中国出版社,1992 年)。
\bibitem{} 《当代中国人物传记》丛书编辑部编:《聂荣臻传》(北京:当代中国出版社,1994 年)。
\bibitem{} 《当代中国人物传记》丛书编辑部编:《贺龙传》(北京:当代中国出版社,1993 年)。
\bibitem{} 《当代中国人物传记》丛书编辑部编:《罗瑞卿传》(北京:当代中国出版社,1996 年)。
\bibitem{} 《当代中国人物传记》丛书编辑部编:《叶剑英传》(北京:当代中国出版社,1995 年)
\bibitem{} 《当代中国人物传记》丛书编辑部编:《陈毅传》(北京:当代中国出版社,1991 年)
\bibitem{} 《当代中国人物传记》丛书编辑部编:《罗荣桓传》(北京:当代中国出版社,1991 年)
\bibitem{} (闽西「肃清社会民主党」历史冤案已平反昭雪),载中共中央党史研究室编: 《党史通讯》, 1986 年第 5 期。
\bibitem{} 《闽浙赣革命根据地史稿》编写组编:《闽浙赣革命根据地史稿》(南昌:江西人民出版社, 1984 年)
\bibitem{} (苏)M·C·贾比才等著,张静译:《中国革命与苏联顾问(1920——1935 年)》(北京: 中国社会科学出版社,1981 年)
\bibitem{} 《新四军第五师抗日战争史稿》(武汉:湖北人民出版社,1983 年)
\bibitem{} 雷云峰等编:《陕甘宁边区大事记述》(西安:三秦出版社,1990 年)。
\bibitem{} (俄)爱德华·拉津斯基著,李惠生等译:《斯大林秘闻——原苏联秘密档案最新披露》(北 京:新华出版社;1997 年)
\bibitem{} 赵生晖:《中国共产党组织史纲要》(合肥:安徽人民出版社,1987 年)
\bibitem{} 赵来群:〈毛泽东与王观澜〉,载《党的文献》,1996 年第 6 期。
\bibitem{} (日)对马忠行著,大洪泽:《托洛茨基主义》(哈尔滨:黑龙江人民出版社,1984 年)
\bibitem{} 熊向晖:《地下十二年与周恩来》(北京:中共中央党校出版社,1991 年)
\bibitem{} 郑建英:《朱瑞传》(北京:中央文献出版社,1994 年)
\bibitem{} 郑异凡:《天鹅之歌》(沈阳·辽宁教育出版社,1996 年)
\bibitem{} 裴淑英:〈关于《六大以来》一书的若干情况〉,载《党的文献》,1989 年第 1 期。
\bibitem{} 黎辛:〈丁玲和延安《解放日报》文艺栏〉,载《新文学史料》,1994 年第 4 期。
\bibitem{} 黎辛:〈《野百合花》·延安整风·《再批判》〉,载《新文学史料》,1995 年第 4 期。
\bibitem{} 廖盖隆:〈徐向前元帅生前的肺腑之言〉,载《炎黄春秋》,1993 年第 1 期。
\bibitem{} 刘中海、郑惠、程中原编:《回忆胡乔木》(北京:当代中国出版社,1994 年)
\bibitem{} 刘英:《在历史的激流中——刘英回忆录》(北京:中共党史出版社,1992 年)
\bibitem{} 刘茂林、叶桂生:《吕振羽评传》(北京:社会科学文献出版社,1990 年)
\bibitem{} 刘家栋:《陈云在延安》(北京:中央文献出版社,1995 年)
\bibitem{} 欧阳植梁、陈芳国主编:《武汉抗战史》(武汉:湖北人民出版社,1995 年)
\bibitem{} 德里特里·安东诺维奇·沃尔科戈诺夫:《胜利与悲剧——斯大林政治肖像》(北京:世界知 识出版社,1990 年)
\bibitem{} 蔡孝干:《江西苏区·红军西窜回忆》(台北:中共研究杂志社,1970 年)
\bibitem{} 《缅怀刘少奇》(北京:中央文献出版社,1988 年)
\bibitem{} 蒋介石:《苏俄在中国》(台北:黎明文化事业股份有限公司,1982 年)
\bibitem{} 蒋伯英:《闽西革命根据地史》(福州:福建人民出版社,1988 年)。
\bibitem{} 蒋南翔:〈关于抢救运动的意见书〉,载《中共党史研究》,1988 年第 4 期。
\bibitem{} 蒋祖林(丁玲之子):〈胭脂河畔〉,载《新文学史料》,1993 年第 4 期。
\bibitem{} 邓力群:〈回忆延安整风〉,载《党的文献》,1998 年第 2 期。
\bibitem{} 邓力群:《我为少奇同志说些话》(北京:当代中国出版社,1998 年)。
\bibitem{} 邓小平:〈七军工作报告〉(1931 年 4 月 29 日),载中共中央文献研究室、中央档案馆编: 《党的文献》,1989 年第 3 期。
\bibitem{} 《邓子恢传》编辑委员会编:《邓子恢传》(北京:人民出版社,1996 年)。
\bibitem{} 《斯大林反对托洛茨基主义和布哈林主义的斗争》 (北京:人民出版社,1978 年)。
\bibitem{} 邓言实编: 卢弘:《李伯钊传》(北京:解放军出版杜,1989 年)。
\bibitem{} 萧克、陈毅等:《回忆中央苏区》(南昌:江西人民出版社,1981 年)。
\bibitem{} 萧克:《萧克回忆录》(北京:解放军出版社,1997 年)。
\bibitem{} 〈萧克谈中央苏区初期的肃反运动〉,载中国革命博物馆编:《党史研究资料》,1982 年第 5 期。
\bibitem{} 苏平著:《蔡畅传》(北京:中国妇女出版社,1990 年)。
\bibitem{} 薛里:〈周恩来与党的隐蔽战线〉,载《中共党史研究》,1998 年第 1 期。
\bibitem{} 薛暮桥:《薛暮桥回忆录》(天津:天津人民出版社,1996 年)。
\bibitem{} 穆欣:《林枫传略》(哈尔滨:黑龙江人民出版社,1990 年)。
\bibitem{} 穆欣:《劫後长忆》(香港:新民出版社,1997 年)。
\bibitem{} 穆欣:〈秦城监狱里的 6813 号〉,载《中华儿女》(国内版),1998 年第 10 期。
\bibitem{} 《忆钱瑛》(北京:解放军出版社,1986 年)。
\bibitem{} 薄一波:《领袖·元帅·战友》(增订本)(北京:中共中央党校出版社,1992 年)。
\bibitem{} 薄一波:《七十年奋斗与思考》,上卷《战争岁月》(北京:中共党史出版社,1996 年)。
\bibitem{} 戴向青:《中央革命根据地研究》(南昌:江西人民出版社,1985 年)。
\bibitem{} 戴向青、余伯流、夏道汉、陈衍森:《中央革命根据地史稿》(上海:上海人民出版社,1986 年)。
\bibitem{} 戴向青、罗惠兰:《A8 团与富田事变始末》(郑州:河南人民出版社;1994 年)。
\bibitem{} 戴晴:《梁漱溟、王实味、储安平》(南京:江苏文艺出版社,1989 年)。
\bibitem{} (英)戴维·麦克莱伦著,余其铨、赵常林等译:《马克思以後的马克思主义》(北京:中国 社会科学出版社,1986 年)。
\bibitem{} 谢燕:《张琴秋的一生》(北京:中国纺织出版社,1995 年)。
\bibitem{} 谢觉哉:《谢觉哉日记》(北京:人民出版社,1984 年)。
\bibitem{} 《谢觉哉传》编写组编:《谢觉哉传》(北京:人民出版社,1984 年)。
\bibitem{} 联共(布)中央特设委员会编、联共(布)中央审定:《联共(布)党史简明教程》(莫斯科: 外国文书籍出版局印行,1951 年)。
\bibitem{} (苏)谢·列·齐赫文斯基,程骅之等译:《我的一生与中国(30—90 年代)》,(北京: 社会科学文献出版社,1994 年)。
\bibitem{} 聂荣臻:《聂荣臻回忆录》(北京:解放军出版社,1984 年)。
\bibitem{} 韩厉观、陈立平:《华克之传奇》(南京:江苏人民出版社,1991 年)。
\bibitem{} 韩辛茹:《新华日报史 1938——1947》,上(北京:中国展望出版社,1987 年)。
\bibitem{} 沈阳军区《赖传珠日记》整理编辑领导小组编:《赖传珠日记》(北京:人民出版社,1989 年)。
\bibitem{} 《谭震林传》编纂委员会编:《谭震林传》(杭州:浙江人民出版社,1992 年)。
\bibitem{} 《怀念林伯渠同志》(长沙:湖南人民出版社。1986 年)。
\bibitem{} 《怀念刘少奇同志》(长沙:湖南人民出版社,1980 年)。
\bibitem{} 严慰冰:《魂归江南》(上海:上海文艺出版社,1987 年)。
\bibitem{} (苏)罗·亚·麦德维杰夫著,赵洵、林英译:《让历史来审判——斯大林主义的起源及其後 果》(北京:人民出版社,1983 年)。
\bibitem{} (苏)罗·亚·麦德维杰夫著,李援朝、黄元等译:《让历史来审判》(续篇)(长春:吉林 人民出版社,1983 年)。
\bibitem{} (苏)罗·亚·麦德维杰夫著,彭卓吾等译:《斯大林和斯大林主义》(北京:中国杜会科学 出版社,1989 年)。
\bibitem{} 罗贵波:《革命回忆录》(北京:中国档案出版社,1997 年)。
\bibitem{} 罗点点:《非凡的年代》(上海:上海文艺出版社,1987 年)。
\bibitem{} 〈关于延安对文化人的工作的经验介绍〉(1943 年 4 月 22 日党务广播),载《新文学史料》, 1991 年第 2 期。
\bibitem{} 顾行、成美:《邓拓传》(太原:山西教育出版社,1991 年)。
\bibitem{} 龚士其主编:《杨献珍传》(北京:中共党史出版社,1996 年)。
\bibitem{} 龚希光:〈朱德与华北初期的「运动游击战」问题〉,载《党的文献》,1996 年第 6 期。
\bibitem{} 龚明德:〈《太阳照在桑干河上》版本变迁〉,载《新文学史料》,1991 年第 1 期。
\bibitem{} 龚育之、逄先知、石仲泉:《毛泽东的读书生活》(北京:生活·读书·新知三联书店,1986 年)
\bibitem{} 龚楚:《我与红军》(香港:香港南风出版社,1954 年)
\bibitem{} 龚楚:《龚楚将军回忆录》(香港:明报月刊社,1978 年)
\bibitem{} 《二十一世纪》(香港:香港中文大学中国文化研究所)。
\bibitem{} 《中共党史研究》(北京:中共中央党史研究室)。
\bibitem{} 《中国社会科学季刊》(香港)(香港:香港社会科学出版社、中国社会科学研究所)。
\bibitem{} 《中国现代史》(北京:中国人民大学书报资料中心)。
\bibitem{} 《文献和研究》(北京:中共中央文献研究室)
\bibitem{} 《共产党人》,1939—1941 年。
\bibitem{} 《百年潮》(北京:中国中共党史学会)
\bibitem{} 《炎黄春秋》(北京:中华炎黄文化研究会)
\bibitem{} 《近代史研究》(北京:中国社会科学院近代史研究所亚互代史研究)编辑部)
\bibitem{} 《解放日报》,1941—1945 年。
\bibitem{} 《解放周刊》,1937—1941 年。
\bibitem{} 《新华日报》,1938—1945 年。
\bibitem{} 《新文学史料》(北京:人民文学出版社)
\bibitem{} 《新华文摘》(北京:人民出版社)
\bibitem{} 《当代中国史研究》(北京:当代中国史研究所)
\bibitem{} 《党的文献》(北京:中共中央文献研究室、中央档案馆)
\bibitem{} 《党史通讯》(北京:中共中央党史研究室)
\bibitem{} 《党史研究》(北京:中共中央党校)
\bibitem{} 《党史研究资料》(北京:中国革命博物馆党史研究室)
\end{thebibliography}
