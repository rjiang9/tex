\chapter{王明返国前後中共核心层的争论与力量重
组}

\section{毛泽东和周恩来等在处理国共关系及八路军军事战略方针上的分
歧}
1937 年 5 至 6 月,
毛泽东利用刘少奇向张闻天发起试探性的攻击遭到
挫折,这表明在通往中共领袖的道路上,毛泽东还有待克服重重障碍。白
区工作会议结束後不久,卢沟桥事变爆发,全国形势发生剧变,毛泽东迅
速搁置对十年内战期间党的历史经验的讨论,全力应付眼前的局面。

抗战之初,中共获得了合法地位,实现了第二次国共合作,毛泽东最
关心的问题有两个:这就是如何处理国共关系,使中共的实力,尤其是中
共军队的实力借抗战得到发展;
以及如何统一在新形势下党内高层的认识,
进一步强化自己在党的核心层中的地位。然而事态的发展并非总是朝着有
利于毛的方向发展。从 1937 年 8 月下旬开始,围绕国共合作方针及八路
军军事战略方针,毛泽东与周恩来等产生了意见分歧,这种分歧在洛川会
议上首次表现出来,继而又表现为毛泽东与周恩来及与以朱德、彭德怀为
首的八路军总部的分歧。

1937 年 8 月 22 至 25 日,
中共为确定在抗战期间的政治路线和军事方
针,在陕北洛川召开了政治局扩大会议,参加会议的有党和军队领导人共
二十三人。数十年来,在中共党史编纂学中,这次会议都被解释为「毛泽
东思想取得了伟大的胜利」,然而历史事实是,毛的意见在洛川会议上并
没有得到党内高层的一致拥护,
周恩来等的主张却获得与会者的普遍共呜。

洛川会议的中心议题是:如何评价国民党的抗战及确定中共对国共合
作的原则;中共究竟应以「山地游击战」、抑或是「运动游击战」作为军
事战略方针。

分歧的一方为毛泽东、张闻天,另一方为周恩来、博古、朱德、张国
焘、彭德怀等。

毛泽东认为,国民党反动本质并未因抗战而改变,因此国民党的抗战
必然失败。毛认为蒋介石进行的只是一场半心半意、单方面的局部战争,
这场战争无疑会归于失败,国民党迟早要投降日本。或者,如果国民党军
队一部分继续作战,就会遭到日本毁灭性的打击,这样中共就要在全国起
领导作用。因此中共不能再迎合国民党而必须保持自己在政治上和军事上
的独立和自主,一旦情况允许或必要,就起来反对它。\footnote{毛泽东在洛川会议上的讲话至今尚未全文披露,现在可见的毛在洛川会议上的讲话是收人《毛泽东选集》第二
卷的〈为动员一切力量争取抗战胜利而斗争〉
,此文是毛在 1937 年 8 月为中央宣传部写的宣传鼓动提纲。该提纲的主
要内容是〈中国共产党抗日救国十大纲领〉
,据《毛泽东年谱》和《张闻天传》称,1937 年 8 月 9 日,毛泽东对原在〈论
反对日本帝国主义进攻的方针、办法与前途〉中提出的「八大纲领」加上了补充意见,扩充为十条。参见《毛泽东年
谱》
,中卷,页 12;程中原:
《张闻天传》
,页 386. 但是李德却说〈十大纲领〉系王明在莫斯科起草、经过共产国际执
委会书记处批准而颁布的。本处引用的毛在洛川会议上的讲话,出自李德:
《中国纪事》
,页 288-89. 李德声称,有关
洛川会议上的争论情况,是参加会议的博古告诉他的。毛泽东起草的宣传大纲除了包含〈十大纲领〉的内容外,还加
上了由他撰写的「甲、乙、丁」部分,这一部分构成了毛的〈为动员一切力量争取抗战胜利而斗争〉的导言和结语。
毛在导言和结语中融入了他对〈十大纲领〉的补充,这就是对国民党「单纯政府抗战的方针」的批评。另参见中央档
案馆编:
《中共中央文件选集》
(1936 一 1938)
,第 11 册,第 330 页的注释;张国焘:
《我的回忆》
,第 3 册,页 387. 
} 

毛泽东的上述主张,除了得到张闻天一人的明确支持外,周恩来等多
数与会者均表示异议。

周恩来在发言中提出,中共和八路军的独立只能是相对的,中共不应
公开对抗南京的命令,为了巩固自己的地位,中共应积极抗战,信守向国
民党许下的精诚合作、共同争取全胜的诺言。

周恩来对毛泽东有关国民党必定会投降日本的论断也持异议。周恩来 认为,「蒋
介石既已开始抗战,就决不会中途妥协」,鉴于蒋介石秉性倔 强以及国内外形势,
没有什么可不放心的\footnote{参见张国焘: 《我的回忆》,第 3 册,页 389;
另参见 A·季托夫: 〈抗日战争初期中共领导内部的两条路线斗争 〉, 1981 年
第 3 期,转引自《共产国际与中国革命——苏联学者论文选译》(1937-1939)。载
苏联《远东问题》,页 350.}。

在洛川会议上,引起争议的另一个问题是中共在抗战阶段究竟应制定 何种军事战
略方针。早在 8 月 1 日,毛与张闻天致电周恩来等,提出中共 军队应「在整个
战略方针下执行独立自主的分散作战的游击战争,而不是 阵地战,也不是集中作
战」。 \footnote{《毛泽东年谱》,中卷,页 8.}8 月 4 日,毛与张闻天又致
电正在山西云阳 镇的周恩来、朱德等,再次提出中共军队「应执行侧面的游击
战」,毛并 强调,「对此方针游移是必败之道」。\footnote{《毛泽东年谱》,
中卷,页 9.}8 月 5 日,毛、张再致电周、 朱、 博古、彭德怀、任弼时等,提
到「红军担负以独立自主的游击运动战,钳 制 敌人大部,消灭敌人的一部」,但
是,毛紧接着又强调,红军只宜作侧 面战,而 不是「独当一面」。
\footnote{《毛泽东年谱》,中卷,页 9.}8 月 9 日,毛在延安干部会议上发表
讲话, 指出「红军应 当实行独立自主的指挥与分散的游击战争。......防人之心
不 可无,应有戒心」。 8 月 10 日,毛在致彭雪枫的电文中更是具体指导道,
在与国民党方面交涉时, 「要有谦逊的态度」,「不可隐瞒红军若干不应 该隐瞒
的缺点」,「例如只会打 游击战,不会打阵地战;只会打山地战, 不会打平原战;
只宜于在总的战略下进 行独立自主的指挥,不宜于以战役 战术上的集中指挥去束
缚」。 \footnote{转引自龚希光: 〈朱德与华北初期的「运动游击战」问题〉。
载《党的文献》,1996 年第 6 期;另参见《毛泽东年 谱》,中卷,页 12.}
毛泽东为共产 党和红军的前途深谋远虑, 极为担心中共军队将在对日作战中吃大
亏。他知道党 内许多高级干部已被 爱国主义冲昏了头脑,可是毛又不能把话说得
太直接,于是 只能一而再, 再而三,不断地向他的同事和部属反复解释,百般劝
说。毛泽东在 8 月 22 日的发言中强调,中共必须把国内战争时期的正规军和运
动战转变为游击 军和游击战,八路军的任务是分散兵力,用来发动群众,建立根
据地。至 于和日 军作战,「打得赢就打,打不赢就跑」\footnote{《毛泽东年谱》
,中卷,页 15;另参见张国焘: 《我的回忆》,第 3 册,页 387.}。

周恩来不同意毛有关八路军避开日军进攻只进行游击战的主张,他提 议应以「运
动游击战」作为军事战略方针。周恩来说,「还是运动游击战 好」。
\footnote{参见《周恩来年谱》,页 378;另参见〈周恩来在中共中央政治局会
议上的发言记录〉(1937 年 8 月 22 日),转 引自中共中央文献研究室编:
《周恩来传(1898 一 1949)(北京:中央文献出版社、人民出版社,1990 年)
》,页 371.}周恩来认为,如果避开日军,那会给党的声誉带来损害,似乎中共
没有全力抗日。周表示「即使八路军在这种运动战中,蒙受相当的损失, 也是值
得的,因为这可以在全国人民面前,证明我们努力抗战」 \footnote{参见张国焘:
《我的回忆》,第 3 册,页 390;另 A·季托夫: 〈抗日战争初期中共领导内部
的两条路线斗争(1937 —1939),原载苏联《远东问题》〉,1981 年第 3 期,
转引自《共产国际与中国革命——苏联学者论文选译》,页 350.}。

朱德、彭德怀支持周恩来的意见。朱德认为在保持中共军事独立性的 同时,在一
些重大战略问题上应服从南京军委会,这样做对于中共有切实 的好处,八路军可
以从南京得到军饷和装备。朱、彭还提出,中共军队应 与国民党军队真诚合作,
八路军应避免阵地战,但是应进行把运动战和游 击战相结合的战争,即「运动游
击战」。\footnote{奥托·布劳恩(李德):《中国纪事(1932-1939)》,页 290.} 
 
毛泽东处于少数地位,不得不暂时退却。在张闻天的调和下,洛川会 议用妥协的
办法,尽量弥合毛泽东与周恩来等的分歧。在对国民党抗战的 评价问题上,张闻
天声称,国民党所进行的战争虽然隐藏着失败的极大危 险,但在另一方面,若能
唤起千百万群众支持统一战线,最後的胜利则是 肯定的。 \footnote{奥托·布劳
恩(李德):《中国纪事(1932 一 1939)》,页 290;另参见程中原:   《张
闻天传》,页 390.}张闻天的这种折衷性的表述,被与会者一致接受。

洛川会议对于军事战略方针虽然没有达成统一的认识,但是也形成了某种折衷
性的意见,这就是,八路军先在山西与国民党军队进行一定程 度的并肩作战,当
前线不守时,八路军再分散到整个华北地区,依毛泽东 的意见开展活动。
\footnote{在张闻天的调和下,洛川会议通过的〈中央关于目前形势与党的任务的
决定〉没有提及八路军应以「独立自主 的山地游击战」作为自己的军事战略方针,
这反映了在洛川会议上围绕军事战略问题而发生的争论尚未最後解决。参 见中央
档案馆编: 《中共中央文件选集》(1936 一 1938),第 11 册,页 325-26.} 

在洛川会议上,毛与周的分歧以双方的互相让步而得到暂时的解决。
周恩来在洛川会议上,
虽然对毛在国共关系等问题上的消极态度有所保留,
但是毛的大部分意见与周恩来并无冲突,周也同意在两党关系上要进一步
冲破国民党的限制,要坚持中共对红军的领导。然而,周恩来在洛川会议
上的言论却引起毛泽东的严重不安,毛十分担心周的主张将会对红军将领
产生影响,但是会议期间的一些重要组织措施,又减缓了毛泽东的忧虑。
1937 年 8 月 23 日,新改组的中央军委成立,毛泽东正式担任了书记一职
(实际上称「主席」),朱德、周恩来任副书记(「副主席」),毛已名
正言顺地成为党对军队的最高领导。洛川会议并决定由周恩来担任负责与
国民党谈判、领导国统区中共组织的长江沿岸委员会书记,周将主要在国
统区工作。这项安排也有利于毛,从而为毛泽东加紧在八路军贯彻自己的
主张创造了有利的条件。

洛川会议一结束,朱德、彭德怀率领的八路军第一一五师、第一二〇
师、第一二九师迅速开进山西。事态的发展完全不出毛泽东之所料,由朱、
彭、任弼时组成的前方中央军委军分会(亦称「华北军分会」)果然受到
周恩来的影响,提出了「运动游击战」的口号(「运动游击战」又称「游
击运动战」)
 
洛川会议後,周恩来于 8 月 29 日抵达西安,原准备和博古、彭德怀
去南京,与国民党继续谈判,并筹组中共长江沿岸委员会。8 月 30 日、31
日,毛泽东两次急电周,要周勿去南京转赴太原,与阎锡山会谈有关红军
入晋事宜。张国焘在《我的回忆》中说,周恩来因不满意洛川会议,故意
在山西停留而不去南京,因而遭到毛的多次埋怨,此说有误,盖因张国焘
当时并不完全了解毛、周之间的秘密联络。 \footnote{参见张国焘:
《我的回忆》
,第 3 册,页 409.}在百废待举的非常时刻,毛
似乎只图发挥周的外交与组织才干,而疏忽了周恩来留晋可能将对八路军
军事战略方针产生的复杂影响。
 
1937 年 9 月 7 日,周恩来与阎锡山在代县会谈,随後又转赴大同会见
傅作义,就八路军入晋後的活动区域、指挥关系、作战原则与阎、傅达成
协议,双方商定八路军将以游击运动战作为作战原则。周并主动提出将派
八路军一一五师配合阎军,布防平型关一带,在侧翼待机歼敌。彭德怀在
与阎锡山会见时也表达了相同的意见。9 月 13 日,周恩来将和阎锡山谈判
情况电告毛泽东、张闻天,要求派八路军两个师迅速集中徕源、灵丘、阜
平地区,依靠太行山发展游击运动战\footnote{《周恩来年谱》
,页 381.}。
 
此时周恩来提出的「游击运动战」已从原先的「运动游击战」後进一 步,显示了
周对毛的让步。尽管周恩来的立场已有明显软化,但是,周恩 来所表现出的与国
民党合作的热情仍使毛泽东极为忧虑,毛泽东没有回复 周要求派八路军两个师援
晋的电报。 \footnote{根据对《周恩来年谱》的分析,在 1937 年 9 月中下旬毛
泽东对周恩来的电报采取两种方法处理,凡周要求加派 八路军增援国民党军的电
报,毛一般不予回复;周建议组织游击战争,向山地转移的电报,毛均立即覆电表
示同意。}自 9 月中旬至下旬,毛接连电示周、 彭等,反复强调中共应保存力量,
「坚持依傍山地与不打硬仗的原则」, \footnote{〈1937 年 9 月 12 日毛泽东
致彭德怀〉,载中央档案馆编: 《中共中央文件选集》(1936-1938),第 11
册,页 337.} 避免与日军发生正面冲突,在军事上。保持高度行动自由, 「用游
击战斗 配合友军作战」, \footnote{〈1937 年 9 月 16 日毛泽东致林彪等〉
,载中央档案馆褊: 《中共中央文件选集》(1936-1938),第 11 册,页
338.}尽速向敌後挺进,创建共产党根据地。
 
1937 年 9 月 17 日,毛泽东致电朱德、彭德怀和八路军各师正副师长,
再次强调:
\begin{quote}
	\fzwkai 红军此时是支队性质,不起决定作用。但如部署得当,能起在
	华北(主要在山西)支持进击战争的决定作用\footnote{〈1937 年 9 月 17 日毛泽东致朱德、彭德怀等〉
,载《毛泽东军事文选》
(北京:战士出版社,1981 年)
,页 83.}。
\end{quote}

 
9 月 21 日,朱德、任弼时、邓小平、左权等率八路军总部到达太原。
当晚,任弼时、邓小平等与中共北方局书记刘少奇及彭真等人开会讨论八
路军行动方针问题。就在同一天,毛泽东再电彭德怀,以极为强烈的语气,
分析了华北抗战的形势,指出不管阎锡山与日军「决战胜败如何,太原与
整个华北都是危如累卵」。毛批评党内有个别同志被暂时的情况所迷惑,
没有深刻认识「这种客观必然趋势」而把红军主力全部用于支持友军的正
规抗战。毛警告道,如果照此办理,「势必红军也同阎锡山相似,陷入于
被动的应付的挨打的被敌各个击破的境遇中」。\footnote{参见中央档案馆编《中共中央文件选集》
(1936 一 1938)
,第 11 册,页 339.}毛责成彭「对个别同志不
妥的观点给予深刻的解释,使战略方针归于一致」,并敦促彭要「从远处
大处着想」。毛泽东在这封电报中反复告诫道:

\begin{quote}
	\fzwkai 今日红军在决战问题上不起任何决定作用,而有一种自己
的拿手好戏,在这种拿手好戏中一定能起作用,这就是具正独
立自主的山地游击战争(不是运动战)。

要以创造根据地发动产众为主,就要分散兵力,而不是以集中
打仗为主,......集中打仗在目前是毫无结果可言的。\footnote{〈关于独立自主山地游击战原则的指示〉
(1937 年 9 月 21 日毛泽东致彭德怀),载中央档案馆编:
《中共中央文 件选集》
(1936 一 1938),第 11 册,页 339-40.} 
\end{quote}

远在延安窑洞的毛泽东对于前方的周恩来和红军诸将领能否执行自己
的指示毫无把握,
于是只能依靠电报不断陈述自己的一贯主张。9 月 25 日,
毛泽东急电周恩来和北方局负责人刘少奇、杨尚昆、朱瑞等,重申「整个
华北工作,应以游击战争为唯一方向」。 \footnote{〈关于整个华北工作应以游击战争为唯一方向的楷示〉
(193 年 9 月 25 日毛泽东致周恩来、刘少奇、杨尚昆等),
载中央档案馆编:
《中共中央文件选集》
(1936-1938)
,第 11 册,页 353.}同日,毛泽东又致电朱德、任
弼时、周恩来,提醒他们勿将红军实力暴露出来,以免遭受不测:
\begin{quote}
	\fzwkai 目前红军不宜过早暴露,尤不宜过早派遣战术支队,......暴露
红军目标,引起敌人注意,那是不利的。

请暂时把我军兵力一概隐蔽并养精蓄锐,待必要条件具备时实
行\footnote{参见中央档案馆编《中共中央文件选集》
(1936 一 1938),第 11 册,页 351.}。
\end{quote}

 
毛泽东在这封电报中所提的「尤不宜过早派遣战术支队」,实际上即
是要求朱、彭勿派八路军配合国民党军作战。
 
尽管毛泽东飞檄传书,三番五次急电周恩来、彭德怀等,劝说他们务
必克服求战心理,爱护、保存共产党历经千辛万苦才保存下来的那股血脉
——不到三万的红军兵力,\footnote{长久以来,国共双方都激烈指责对方消极抗战,强调自己独力支撑抗战,取得打败日帝的胜利。国民党方面宣
称,1937 年 9 月 26 日毛泽东曾对八路军发出下列指示:
「中日战争是本党发展的绝好机会,我们共产党的基本政策是
七分发展,两分应付,一分抗日。」参见古屋奎二:
《蒋总统秘录》,第 11 分册(台北:中央日报社,1977 年),页 117. 
前苏联方面的论著中也有类似的表述,前苏共中央对外联络部高级官员罗满宁以奥·鲍里索夫的笔名出版的《苏中关
系(1945 一 1980)》披露,毛在抗战初期要求中共和八路军「用一分力量和日本斗,用二分力量来和国民党斗,用七
分力量来发展自己」,参见《苏中关系(1945 一 1980)》
(北京:生活·读书新知三联书店,1982 年),页 100. 对台湾
和苏联方面的这类言论,大陆方面既不承认,也未正式否认。1976 年 4 月,
「四人帮」在上海的写作组为了影射周恩来,
化名「史锋」出版了《反封王明投降主义路线的斗争》的小册于,在删去了毛泽东最鲜明的几段话後,首次不加引号
地公布了毛泽东在 1937 年 9 月 21 日电报的精神。
参见史锋:
《反封王明投降主义路线的斗争》:
(上海人民出版社, 1976 年),页 24. 
由于史锋的小册子第一次披露了毛的电报,很快引起苏联方面的重视,
他们认为毛的电报证明了苏联方面
六十年代以来对毛消极抗战的批评。参见 A·季托夫:
〈抗日战争初期中共领导内部的两条路线斗争(1937 一 1939)〉,
原载苏联《远东问题》,1981 年第 3 期,转引自《共产国际与中国
革命——苏联学者论文选译》,页 351.} 但是周恩来、朱德、彭德怀仍然坚持要配合
国民党抗战。在周恩来的积极策划下,1937 年 9 月 23 日,进驻五台山的
八路军总部命令八路军一一五师在右翼配合阎军作战,一二〇 师从左翼驰
援驻守雁门关的阎军。同日,华北军分会常委朱德、彭德怀、任弼时将有
关军力部署电告毛泽东\footnote{中共中央文献研究室编:
《任弼时传》,页 407-408.}。
 
1937 年 9 月 25 日,一一五师在林彪指挥下,在晋东北的平型关附近
伏击日军板垣师团第 21 旅团。
歼敌千馀人,
取得八路军出征後的第一个重
大胜利。显然,一一五师首战平型关是贯彻了周恩来等关于八路军配合友
军作战及「运动游击战」军事战略方针,而和毛有关避免与日军正面作战
的「独立自主的山地游击战」的方针大相径庭。但是由于平型关战斗大大
提高了共产党和八路军的威望,赢得国内外舆论的高度评价,毛泽东也表
示了高兴。尽管华北军分会 9 月 23 日给毛的电报和周恩来于 9 月 24 日向
毛泽东、张闻天发出的八路军参与平型关战斗的军力布置的电报,都未得
到毛泽东的覆电。
\footnote{毛泽东在 1937 年 9 月 24 日没有就周恩来报告的有关八路军参与平型关战斗的军力布置的电报作出答复,但是
在同一天毛给周恩来、朱德的电报中却强调山西地方党目前应将工作重点放在五台山脉,立即组织地方支队和群众组
织,一切工作应在敌占太原的设想下作布置的出发点。显然,当时毛与周恩来的思路是完全不同的。参见《周恩来年
谱》
,页 383;
《毛泽东年谱》
,中卷,页 23.}
 
在中共党史编纂学中,平型关战斗长期被描述为贯彻毛泽东军事思想
而获胜利的一个成功范例。1945 年春在延安召开的「华北座谈会」上,彭
德怀被指控在抗战初期违背毛泽东军事战略方针而遭受严厉指责,朱德、
任弼时等也被迫对此作了「自我批评」\footnote{龚希光:
〈朱德与华北初期的「运动游击战」问题〉
,载《党的文献》
,1996 年第 6 期;另参见《任弼时传》
,第
410.},林彪却丝毫未受牵连。只是到了七
十年代初,林彪事件爆发,官方才把林彪与彭德怀并列,指责彭德怀、林
彪是「王明右倾投降主义路线的走卒」。即使如此,七十年代的中共党史
编纂学仍未公开批评平型关战斗,
因为在批判了彭德怀领导的 1940 年
「百
团大战」後,再否定平型关战斗,将有损毛泽东和中共抗战的形象。
 
平型关战斗的胜利及其广泛的政治影响,突出了周恩来、朱德、彭德
怀坚持「运动游击战」的成效,一度也使毛泽东对原坚持的观点产生了稍
许动摇,毛在固守原有的游击战主张的同时,对运动战不再绝对排斥。平
型关战斗後的第四天,1937 年 9 月 29 日,毛泽东致电周恩来、朱德、彭
德怀、任弼时:「阎(锡山)必要求我军与他配合来打一仗,为了给晋军
以更好的影响,如果在确实有利的条件下,当然是可以参加的」。
\footnote{中共中央文献研究室编:
《朱德传》
(北京:人民出版社、中央文献出版社,1993 年),页 413. 本文所引用的毛
泽东这份电文在《毛泽东年谱》中卷被略去。参见《毛泽东年谱》,中卷,页 25-26. } 
自平型关战斗获胜後直至 10 月中旬,毛泽东对周恩来有关建议八路
军配合国民党作战的电报转而采取较为积极的态度。10 月 4 日,
毛致电朱、
彭、任弼时等,指示「对于国民党交给我们指挥之部队。应采取爱护协助
态度,不使他们担任最危险的任务,不使他们给养物资缺乏」。\footnote{《毛泽东年谱》
,中卷;页 27.}次日,毛
覆电同意周在 10 月 4 日提出的有关调王震旅归还贺龙师建制,以加强阎
军抵御日军进攻忻口的建议。10 月 14 日,毛覆电批准周在 10 月 12 日有
关调张宗逊旅主力及刘伯承师先头团截击日军後方,配合晋军中路作战的
建议。\footnote{周恩来年谱》
,页 385-86.}直至 10 月 25 日,毛在同英国记者贝特兰的谈话中还说,「现在
八路军采用的战法,我们名之为独立自主的游击战与运动战」
\footnote{《毛泽东年谱》
,中卷;页 34.}。

毛泽东对运动战态度的松动,
大大鼓舞了周恩来、
朱德、
彭德怀。
1937 年 10 月 8 日,华北军分会在一份文件中正式提出八路军以「运动游击战」
作为作战的战略方针。\footnote{参见中共中央文献研究室编:
《朱德年谱》
(北京:人民出版社,1986 年)
,页 173;另参见《彭德怀自述》
,页
222-23.}这份文件还含蓄地批评了那种认为抗战必然导致失
败的观点是「宿命论」,主张八路军在保卫太原的战斗中应配合、支持友
军作战。\footnote{龚希光:
〈朱德与华北抗战初期的「运动游击战」问题〉
,载《党的文献》
,1996 年第 6 期。}10 月下旬,在周恩来的积极组织和毛泽东的默认下,八路军三
个师全部出动,配合国民党发起忻口会战,重创日军,取得重大战果。
 
毛泽东对八路军采取「运动战」作战方式的默许和宽容,随着山西抗
战形势的变化而很快结束。1937 年 10 月中旬後,山西形势日趋恶化,毛
泽东又迅速退回到原来的立场,重谈游击战,毛甚至更进一步,开始批判
起「右倾投降主义的倾向」。1937 年 10 月 13 日,毛泽东、张闻天把给上
海地下党负责人刘晓、潘汉年的电报同时转抄周恩来。毛、张在这封电报
中,敲山震虎,激烈指责「同国民党和平共居」,「只知同国民党统一,
处处迁就他的要求,而不知同他的错误政策作斗争」的「投降主义倾向」。
10 月 17 日,毛与张闻天联名,致电朱德、彭德怀、任弼时并告周恩来:
「军分会 10 月 8 日指示文件有原则错误,望停止传达。」
\footnote{《毛泽东年谱》
,中卷,页 31.}同日,毛、张
又以中央书记处的名义发出指示,首次提出「投降主义的危险,开始成为
党内的主要危险」的论断。 \footnote{参见中央档案馆编:
《中共中央文件选集》
(1936-1938)
,第 11 册,页 365、372.}毛泽东、张闻天虽未挑明所指何人,但 10 月
13 日的电文和中央书记处 10 月 17 日指示所内含的对周恩来警告之涵义,
则是不言而喻的。

周恩来对来自毛泽东、张闻天方面的讯息迅速作出反应,周在接到中
共中央「坚持同阎锡山合作,但是在原则问题上不让步」的来电後,于 10
月 21 日给延安覆电,
声称自己
「十多天均本此方针,
在党内军内既反右倾,
又反『左倾』」\footnote{《周恩来年谱》
,页 387.}。

周恩来的辩解在山西抗战日趋恶化的形势下显得十分软弱,11 月 18
日,日军侵占太原,国民党军纷纷撤退,八路军配会国民党军作战的条件
已不复存在,客观形势迫使八路军开始分散兵力,逐步从运动游击战转到
游击战的作战形式。这使毛泽东更加确信自己原先对国民党抗战的分析和
所坚持的「独立自主的山地游击战」的主张是完全正确的。12 月 12 日,
毛在政治局会议的讲话中旧事重提,他针对 10 月 8 日华北军分会训令中
对抗战「宿命论」的批评,指出,公开批评中央是失败主义是不对的。
\footnote{龚希光:
〈朱德与华北抗战初期的「运动游击战」问题〉
,载《党的文献》
,1996 年第 6 期。}尽
管华北军分会所讲的「宿命论」主要是针对以刘少奇为首的北方局,但由
于毛在 9 月 21 日电报中实际上也是持这种观点,因而毛将这种批评看成
是针对自己的。使毛难以容忍的是,八路军在 1937 年 11 月至 1938 年 2
月,仍然进行了几次集中作战。1938 年 2 月,彭德怀指挥的八路军为配合
阎锡山「反攻太原」和在晋东南为粉碎日军九路进攻,继续采用了集中作
战的运动战形式。周恩来则走得更远,1938 年 6 月 15 日,周恩来从汉口
给毛发电。提议八路军集结较大兵力于一些较大城市附近,以调动日军和
打击日军增援部队。毛虽然并不完全反对在有利条件下八路军也可以打运
动战。但他的主旨与周恩来等并不一致。因此毛对这份电报的反应是,避
开周的具体要求,
告诫周在具体作战方面
「需全依敌我当前实际条件而定,
不因人家议论而自乱步骤」。\footnote{《毛泽东年谱》
,中卷,页 78.}周恩来等对配合国民党作战所持的积极态度,
加深了毛泽东对周恩来、彭德怀的不满,成为数年後周、彭在整风运动中
挨整的重要原因之一。
 
在 1937 年 8、9、10 三个月,毛与周等围绕共产党军队是否配合国民
党军作战及中共应采取何种军事战略方针而产生的分歧,对毛、周关系,
尤其对毛、彭关系,投下了沉重的阴影。在中共核心层,除了张闻天与毛
泽东站在一边,只有刘少奇旗帜鲜明地支持毛。
\section{ 毛泽东的理论攻势与刘少奇对毛的支持}
1937 年 11 月对于毛泽东是一个忧多于喜的时刻。在毛泽东一再敦促
和反复劝说下,周恩来等虽然在军事战略方针问题上部分地接受了毛泽东
的意见,但在如何处理与国民党的关系等问题上仍固守与国民党积极合作
的方针。毛与周恩来等的分歧尚未最後解决,又有一个更棘手的难题困扰
着毛,这就是远在莫斯科的王明即将返回延安,毛已预感到在党的核心层
中一场风暴即将来临。

为了防止周恩来与即将返国的王明在统一战线问题上结成同盟,同时
也是为了争取党内更多高级干部接受自己的政治主张,毛泽东决定主动出
击,向反对派发起进攻。

1937 年 11 月 12 日,距王明飞抵迪化(乌鲁木齐)的前两天,毛泽东
在延安党的活动分子会议上作了一个极为重要的报告——〈上海太原失陷
以後抗日战争的形势和任务〉。在这个报告中,毛继续发展他在洛川会议
上对国民党抗战方针的批判,毛认为,上海太原失陷已证实了他关于国民
党很快会失败的预言。毛声称,国民党的「片面抗战」虽然也带有「革命
性」,但却是「一定失败的」。目前抗战已进入一个「青黄不接的时期」,
其依据就是日军进攻不断获胜。

毛泽东这个报告最引起党的干部惊骇的是他对党内「阶级投降主义」
的指责。毛不惜使用最激烈的词语,将党的核心层内与自己相左的主张斥
之为「右倾机会主义」,甚至耸人听闻地把它称之为「阶级投降主义」,
并将其说成是「民族投降主义的後备军」,是民族投降主义的客观同谋者。
毛正式宣布,右倾投降主义已成为中共党内的主要危险。

毛泽东还少有的对中共军队进行了批评。作为党在军队方面的最高领
导人,毛泽东难以抑制他对彭德怀等的强烈不满,毛开始谴责起八路军中
的「新军阀主义的倾向」,声称其表现为八路军中有人以接受国民党委任
为荣耀」。随後毛话锋一转,又表扬八路军执行了「独立自主的山地游击
战」的战略方针。尽管毛明知八路军进行的是「运动游击战」,他本人曾
三番五次、苦口婆心敦促彭德怀等改弦易辙。毛采取打一下、再拉一把的
策略,在对彭德怀等猛击一掌後,仍竭尽全力争取彭德怀等接受自己的主
张。

正当毛泽东孤军作战时,刘少奇给予了他宝贵的支持。
刘少奇未参加洛川会议。1937 年 7 月 28 日,刘少奇以北方局书记的
身份抵达太原,主持刚迁到此地的北方局的工作。在这个阶段,刘少奇把
主要精力放在动员、群众抗战和支持薄一波联络阎锡山、建立山西新军的
方面,而与朱德、彭德怀领导的八路军较少发生直接联系。

刘少奇虽没有卷入到洛川会议上的争论,但是他在两个重大问题上的
观点却与毛泽东十分接近。

刘少奇支持毛泽东对国民党的政治判断,并进一步主张对国民党进行
左、中、右三派的划分。1937 年 9 月 20 日,刘少奇在与周恩来联名给毛
泽东、张闻天等的电报中(此份电报基本反映的是刘少奇的观点,在《周
恩来年谱》
中未提及此电报)提出中共在山西统一战线中的策略应是 ,
「巩固左派,联合中派,孤立右派」,而区分左、中、右派的标准则在于他们
是否「能听取我们的意见」。\footnote{《刘少奇年谱》
,上卷,页 189.}刘少奇的这个观点在当时十分震撼,
涉及到党对统一战线策略的最重要方面,以至于毛泽东在刘少奇的来
电上批示,「如此类电报须绝对保守秘密」\footnote{参见中共中央文献研究室编:
《文献和研究》
(1986 年汇编本)
(北京:人民出版社,1988 年)
;页 192、193.}。抗战之初,中共领导人一般
都认为,评价国民党的标准只是看其对抗战的态度,虽然张闻天在洛川会
议的补充报告中提过国民党内存在「左、中、右」三派分野的问题,但张
划分国民党「左、中、右」三派的标准仍是对抗战的态度,张并将蒋介石
封为「中派」。\footnote{程中原:
《张闻天传》
,页 389.}然而刘少奇的看法符合毛泽东的思路,启发了他稍後详细
论证这个问题。在对国民党及其领导抗战的评价方面,刘少奇和毛泽东一
样,是属于「悲观派」,而和周恩来等「乐观派」有明显的区别。

刘少奇对毛泽东有关开展游击战的主张也同样给予了坚决支持。抗战
爆发後,刘少奇是中共党内最早提出进行游击战的少数人之一。1937 年 8
月 3 日,
刘少奇致电张闻天,
汇报他已下达在平、
津发动游击战的指示。\footnote{刘少奇:
〈为发给各地指示信给中央的报告〉
(1937 年 8 月 3 日)
,载《文献和研究》
(1986 年汇编本)
,页 191;
另参见《刘少奇年谱》
,上卷,页 183.} 9
月 28 日,在接到毛泽东关于「整个华北工作,应以游击战争为唯一方向」
电报的三天後,刘少奇致电张闻天,通报他已下令平、津党组织「动员大
批干部同志与抗日游击分子下乡,发展领导游击」。\footnote{〈刘少奇致洛甫电〉
(1937 年 9 月 25 日)。载《文献和研究》
(1986 年汇编本),页 1942 另参见《刘少奇年谱》,
上卷,页 190.}太原失守前的 10 月
16 日,刘少奇又发表了日後引起党内高层争议的〈抗日游击战争中各种基
本政策问题〉一文。

对于发展中共武装问题,刘少奇与毛泽东一样都极度重视。1937 年 9
月 21 日,
在太原举行的八路军负责人与北方局领导人联席会议上,
刘少奇
提出要「扩大八路军到拥有数十万人枪」的意见。\footnote{《刘少奇年谱》
,上卷,页 190.}据杨尚昆和当时任北方
局军委书记的朱瑞在 1945 年 3 月延安举行的华北座谈会上的发言,在八
路军是否要配合阎锡山军队打仗的问题上,华北局(即是北方局——笔者
注)负责人主张八路军应分散打游击,发动群众,因为太原反正是保不住
的。\footnote{龚希光:
〈朱德与华北抗战初期的「运动游击战」问题〉
,载《党的文献》
,1996 年第 6 期。}11 月 17 日,刘少奇和北方局副书记杨尚昆致电毛泽东、张闻天,提
出「扩大红军要成为华北全党及红军全体指战员第一位重要工作」,「必
须计划在三月内扩大到十万,半年内扩大到二十万」。\footnote{〈刘少奇、杨尚昆致毛泽东、洛甫电〉
(1937 年 11 月 17 日)
,载《文献和研究》
(1986 年汇编本)
,页 198;另
参见《刘少奇年谱》
,上卷,页 199.}而在这之前的 11
月 1 日,刘少奇向延安报告,四个月内北方局已在山西、河北建立起十几
支中共领导的较大的游击队,人数达六、七千人,\footnote{参见马齐彬等:
〈刘少奇与华北抗日根据地的创立〉
,载《文献和研究》
(1986 年汇编本)
,页 291;但是在《刘
少奇年谱》中,这个数字被笼统为「数千人」
,页 196.} 随後华北地区中共地方
武装迅速发展,几乎遍布华北所有地区,人数达到数万。

在毛泽东暂时处于少数地位、周恩来的看法在党内占上风的时刻,刘
少奇的态度对毛是一个很大的支持。然而在对形势的估计及党的任务等问
题上,刘少奇的意见仍与毛有一定的距离。例如,刘少奇在 1937 年 8 月 3
日给张闻天的电报中虽提出开展游击战的建议,但对游击战的认识与毛并
不完全一致。刘少奇认为进行游击战的目的是「响应抗日军的武装斗争」,
配合「武装暴动,收复平津」。 \footnote{《刘少奇年谱》
,上卷,页 186、191.}刘少奇在动员平津党员下乡打游击的同
时,还指示「改组完全公开的同乡会并加入汉奸团体」。\footnote{《刘少奇年谱》
,上卷,页 186、191.}刘少奇提出此问
题虽然在主观上是为了共产党的事业,但刘的这个建议毕竟太显眼,极易
引起误解,造成刘少奇缺乏原则性、思想太「右」的印象。刘少奇在〈抗
日游击战争中的各种基本政策〉一文中虽然强调「游击战争是今後华北人
民抗日的主要斗争形式」,但在 11 月 17 日给毛泽东、张闻天的电报中又
提出「争取华北游击战争胜利,重复转变为正规战」, \footnote{载《文献和研究》
(1986 年汇编本),页 191、197. 在
《刘少奇年谱》中,对刘这句话予以删节,见《刘少奇年 谱》,上卷,页 199.}而与毛的主张不
尽一致。尽管刘与毛的认识存在若干差异,然而在毛泽东急需党内高层积
极支持的时刻,这些差异丝毫不影响毛泽东对刘少奇的重视。

尽管毛泽东已获得刘少奇的重要支持,但是党内上层的态势并没有朝
着有利于毛的方向发展,随着王明抵达延安日期的日益临近,毛的忧虑愈
益强烈。
毛凭着自己对党内斗争历史的深切了解和对眼前党内争论的判断,
确信中共上层内部的关系将随着王明返国而发生新的组合,一批与自己意
见不合的同志将会聚集在王明的周围。

毛泽东对王明素无好感,
愤恨王明依仗莫斯科的支持而获得党内高位。
毛也嫉恨王明垄断了与斯大林的联系,「挟天子以令诸侯」。毛更难以容
忍王明以中共唯一理论家自居,独霸了中共意识形态的解释权。毛泽东虽
还不知王明返国将携带莫斯科什么新指示,但从一年前共产国际对中共的
干预就可判断,莫斯科和王明主要将关注中共对国民党的统一战线方针,
而这些都是和自己的主张不尽相同的。
 
在这微妙的时刻,毛泽东最不放心的是周恩来、朱德以及博古等人。
 
周恩来、博古、张闻天在 1931 至 1935 年基本上是国际路线的拥护者
和执行者。在这三人中,毛可以得到张闻天的全力支持。对于在军中毫无
基础、其声望和权威都因刘少奇的挑战而遭到削弱的张闻天而言,要维护
现有地位的唯一途径,只能是继续与毛合作。对于毛泽东而言,博古的危
害也相对较弱,博古自遵义会议被赶下台後,声望已大大下降,即使博古
再与王明合流,若无周恩来的支持,也掀不起风浪。周恩来是核心层中最
重要角色,周在党内军内拥有雄厚的基础,周和朱德、刘伯承等一贯在国
际派和毛之间摇摆,
虽然在更多的情况下,
周恩来等对毛多作妥协和让步,
但抗战以来,周的看法与毛存有不小的分歧,极有可能在王明与毛泽东之
间偏向王明。
 
1937 年 11 月,是毛泽东最为艰难的日子。他只能抓紧时间,主动出
击,在王明将回国而未回国之际,把自己的理论石块先甩出去,同时严密
控制与共产国际的电讯联系,严禁任何人插手,以求对莫斯科指示灵活处
理,「为我所用」。一切该做的毛泽东都做了,下一步就是迎接从「昆仑
山下来的神仙」——王明\footnote{何松:
〈当王明回到延安时〉
,载鲁平:
《生活在延安》
(西安:新华社,1938 年)
,页 57.}。


\section{让步与等待:1937 年 12 月政治局会议}

1937 年 11 月 29 日,王明、康生、陈云等在苏联顾问的陪同下乘苏联
大型军用飞机抵达延安,在机场受到毛泽东、张闻天、周恩来、张国焘等
领导人和千馀名延安干部战士的热烈欢迎,毛泽东在机场举行的欢迎大会
上发表讲话,将王明等称作是「马克思给我们送来了天兵天将」。\footnote{刘家栋(陈云在延安时期的秘书)《陈云在延安》
:
(北京:中央文献出版社,1995 年)
,页 1.}「天兵
天将」既已下凡,当然要传达「天王」的「圣旨」。12 月 9 至 14 日,中
共中央政治局在延安举行会议,由王明传达共产国际指示,并进而检讨抗
战以来党的路线,此次会议王明的主张在党的核心层中占据了上风,史称
「十二月政治局会议」。

在中共党史编纂学中,对「十二月政治局会议」长期持否定态度,对 会议主要内
容也多予以回避。 在官方党史著述中, 一般将这次会议列入 「毛 泽东反对王明
右倾投降路线」或「中共六届六中全会」的背景部分,其主 要论点:一是全盘否
定王明在会议上所作的报告,指斥其为「系统的投降 主义主张」;第二,绝口不
提王明的报告获政治局一致通过,以及会议所 通过的一系列决定。1987 年中共中
央党史研究室编写的《中共党史大事年 表》,在对王明报告的评价上首次发生变
化,在继续指责王明「右倾投降 主义」的同时,开始承认王明的报告「在坚持联
合国民党抗战问题上,发 表了一些正确意见」 \footnote{见中共中央党史研究室:
《中共党史大事年表》
(北京:人民出版社,1987 年)
,页 128.}。官方权威的党史研究部门局部修
改对王明报告的评价,主要是出于现实政治的需要,它意图表明抗战之初,中
共就怀有与国 民党合作抗日的真诚愿望。

十二月政治局会议是 1934 年 1 月中共六届五中全会召开以来第一次 有绝大多数
政治局委员参加的会议,也是 1931 年後中共的国内领导机构 与派驻莫斯科的代
表团实现汇合後召开的第一次会议。这次政治局会议的 召开,是中共中央事先议
定的,并非出于王明的压力。1937 年 11 月初, 毛泽东就已知王明即将返国的信
息,毛估计王明返国後必定要传达共产国 际的指示,召开政治局会议一事不可避
免,尽管毛内心十分不悦,但仍向 外地的一些政治局委员发出电报,通知他们返
回延安参加会议\footnote{〈毛泽东致周恩来、朱德、彭德怀、任弼时
的电报〉(1937 年 11 月 5 日),载《周恩来传(1898-1949),页 》391.}。1937 年 
11 月 5 日,毛发电报给周恩来,催促周来延安开会。 11 月 15 日,毛在 给周
恩来等的电报中,再次提及周回延安开会事。\footnote{参见〈毛泽东致周恩来等
电〉(11 月 15 日)。载中央档案馆编: 《中共中央文件选集》(1936-
1938),第 11 册, 页 394. }远在南昌的项英,如果 不是较早接到开会的通知,
是 来不及赶到延安准时参加会议的。

十二月政治局会议是一次严肃的党的核心层的会议,它改变了遵义会
议後政治局开会一般多邀请重要军事干部参加、常以政治局扩大会议形式
开会的惯例。出席这次政治局会议的政治局委员、候补委员共十二人,他
们是毛泽东、王明、张闻天、周恩来、博古、康生、陈云、彭德怀、刘少
奇、项英、张国焘、凯丰。林伯渠不是中央委员和政治局委员,但作为中
共元老,也出席了这次会议。

四名政治局委员、候补委员缺席:

朱德、任弼时在山西八路军总部,

邓发在新疆迪化(乌鲁木齐)主持八路军办事处。

王稼祥因病在莫斯科治疗。

十二月政治局会议的主持人是在党内负总责的张闻天,王明是十二月
政治局会议的主角。在 12 月 9 日会议的第一天,王明作了〈如何继续全
国抗战与争取抗战胜利呢〉的报告。第二天王明又作了有关中共驻共产国
际代表团工作的报告。王明在会议上传达了共产国际的指示,强调中共必
须加速转变内战时期的策略,建立广泛的抗日民族统一战线。在论及抗战
以来中共政治方针时,王明不点名地批评了毛泽东,公开点名指责了刘少
奇。

王明在报告中批评洛川会议没有突出「抗日高于一切」、「一切服从
抗日」
的原则。
他认为不恰当地强调
「片面抗战路线」 和「全面抗战路线」,
将抗日与民主、民生问题并列,都是不对的。王明说,群众运动要取得合
法地位,应去国民党政府备案,在抗日条件下,不怕国民党的限制。王明
举洛川会议制定的〈抗日救国十大纲领〉将「抗日的民族团结」放至第十
条作为证据,指责洛川会议对国共合作的重要性认识不足,过份强调了独
立自主。王明还说,洛川会议虽主张发动群众,却没有找到发动群众的具
体方法,即没有提出「一切经过统一战线」的口号。

王明批评 1937 年 9 月 25 日〈中共中央关于共产党参加政府问题的决
定(草案)〉对国民党的进步认识不足。王明认为,国民党由不抗日到抗
日,由剿共到联共是根本转变。王明强调中共参加政府的条件是看政府是
否抗日,只要国民党抗日,中共就可以参加政府。王明还认为,把复兴社
看成是法西斯也是不对的,因为法西斯的主要特征是对外侵略,而复兴社
分子仍抗日。

王明在报告中表示,
他不同意毛泽东在 1937 年 11 月 12 日所作的
〈上
海太原失陷以後抗日战争的形势和任务〉中提出的某些论断。王明认为,
所谓卢沟桥事变前党的主要危险是「左倾」、之後是「右倾」的分析,夸
大了右倾的危险,是一种机械论。王明认为:在报告大纲第十九条中的提
法——即「是共产党领导资产阶级,还是资产阶级领导无产阶级?是国民
党吸引共产党,还是共产党吸引国民党」,也是不对的。因为历史上没有
无产阶级领导资产阶级的事情,应提「共同领导」。王明表示不应空喊领
导权,空喊只会吓走同盟者。王明还反对在国民党和其它政治派别中划分
左、中、右,认为只可划分抗日或降日派。王明也不同意中共对章乃器的
批评,他指出章氏提出的「少号召、多建议」口号不无可取之处,中共应
与国民党采取商量建议的方式,而不宜在政治上号召要这样办,那样办。

如果说王明对毛泽东还多少有所顾忌,未敢直接点名,只是就毛撰写 的 1937 年
11 月 12 日大纲进行了批评,那么,他对刘少奇就没有这么客 气了。王明点名批
评刘少奇在〈抗日游击战争中的各种基本政策〉一文中 对国民党提出的各项要求
「过高」、「过多」,而没有反映「抗日高于一 切」的中心问题。王明认为,中
共目前应与国民党「求同而非立异」 \footnote{王明在 1937 年 12 月政治局会
议上的讲话至今仍没有完全公开。《六大以来》收有他在 12 月 9 日会议上所作
报 告提纲〈如何进行全国抗战和争取抗战胜利呢?〉,但是王明在会上另有口头
报告。这个口头报告即〈王明在中共中央 政治局会议上的发言记录〉(1937 年
12 月 9 日),近年来在少数权威性著作中披露了若干内容,详见《周恩来年谱》
, 页 393;中共中央文献研究室编: 《周恩来传(1898-1949)》,页 392;另
参见中共中央文献研究室编: 《毛泽东传(1893-1949)》,页 506-507.  }。

王明在「十二月政治局会议」上充当了斯大林代言人的角色,他的报 告基本上是
「宣达圣旨」,完全反映的是斯大林、季米特洛夫对中共当前 任务的观点。1937
年 11 月初,在王明返国前夕,斯大林、季米特洛夫在 莫斯科召见了王明、康生、
王稼祥、邓发。斯大林出于对苏联安全的考虑, 指望中国拖住日本,使日本身陷
中国战场的泥沼而无力进攻苏联。\footnote{参见王稼祥: 〈我的履历〉(1968
年 5 月),转引自徐则浩: 〈王稼祥对六届六中全会的贡献〉,载《文献和研
究》(1986 年汇编本),页 435.}11 月 14 日,季米特洛夫在共产国际执委会
书记处会议上谈道,中共应遵循「一 切服从统一战线」、「一切通过统一战线」,
不要过份强调独立自主。 \footnote{3《毛泽东传(1893-1949)》,页 505}斯 大
林认为,中共力量薄弱,无法充当统一战线的核心,蒋介石则可充当这
个角色,中共不要刺激、惹恼蒋介石,而要全力加强与国民党的合作。对
于毛泽东,斯大林既不熟悉,也不放心,\footnote{瓦·崔可夫(1940 年任苏联驻华使馆武官和蒋介石的苏联总军事顾问)《在华使命——一个军事顾问的笔记》
:
(北京:新华出版社,1980 年)
,页 34-36.}且十分怀疑毛泽东是否能够忠实
贯彻莫斯科的战略意图,因而派其门徒王明返回中国,监督中共执行这个
联合国民党的新方针。对于王明所肩负的使命,共产国际总书记季米特洛
夫曾给予清楚的阐释。1937 年 8 月 10 日,季氏在共产国际执委会书记处
讨论中国形势的会议上,对中共能否转变政策信心不足。他认为,由于中
共过去领导红军为建立苏维埃而斗争,现在同时还是这些人却要执行另一
种政策,对于中共这将是十分困难的。因此「需要能在国际形势中辩明方
向、有朝气的新人去帮助中共中央」。
\footnote{〈季米特洛夫在共产国际执委会书记处讨论中国问题会议上的发言〉
(1937 年 8 月 10 日),载《中共党史研究》,
1988 年第 3 期。}
在斯大林和季米特洛夫的眼里,
王明正是这样一个合适的人选。

王明自恃有斯大林作靠山,在十二月政治局会议上踌躇满志,毛泽东
为欢迎他回国而刻意作出的友好姿态麻痹了王明\footnote{据 1937 年 11 月 29 日随王明飞抵延安的王明的警卫员李光灿回忆(陈光灿原为西路军战士,1937 年 11 月中旬
由八路军驻迪化办事指派为王明的警卫员,在迪化登机护送王明等抵延安),
王明回延安後,毛泽东等在陕北公学主持
召开欢迎大会,毛在讲话时「很热烈,很兴奋......好象喝了点酒」。
参见曹仲彬、戴茂林:
《王明传》(长春:吉林文史
出版社,1991 年),页 287. },无形中膨胀了他的自我
中心意识,使王明陷入了错误的判断。王明在报告中无视毛的权威,将他
个人自 1934 年以来与毛修好的努力毁于一旦。王明以为刘少奇没有实力,
以批刘来影射毛,也造成严重的後果,促使毛泽东与刘少奇在反对王明的
基础上进一步加紧联合。

所幸,这一切对于王明还是未来的事,现在则形势大好,前途一片光
明,几乎所有的政治局委员都一致拥护王明的报告,连毛泽东也被迫予以
附和。

王明的报告得到周恩来的支持。周在 12 月 11 日的发言中,对毛泽东
抗战以来的言论进行了不点名的批评,周恩来说:四个月来未能推动抗日
统一战线更大的发展,主要原因是由于以前「片面抗战必然失败论」。不
应把片面抗战、全面抗战对立起来,硬要讲片面抗战必然失败,以後全面
抗战必然胜利,这不符合辩证法\footnote{参见珏石:
〈周恩来与抗战初期的长江局〉
,载《中共党史研究》
,1988 年第 2 期。}。

周恩来认为,以山西情况为例,由于没有实行「抗日高于一切」的原
则,而把独立自主提得太高,所以党内、军内和各地都有不利于抗战、不
利于统一战线的思想、言论及行动。
\footnote{《周恩来年谱》
,页 393.}
周提出,把独立自主发展到各方
面会妨害统一战线,应公开指出并纠正统一战线中的错误,使友党更加信
任和佩服我们\footnote{参见珏石:
〈周恩来与抗战初期的长江局〉
,载《中共党史研究》
,1988 年第 2 期。}。

周恩来的发言反映了与会大多数政治局委员的观点,形势明显对毛泽
东不利。他强忍心中的不快,为了避免自己陷入孤立,被迫对王明、周恩
来作出妥协姿态。毛在会议的发言中承认存在着王明所批评的「抗战发动
後对国民党的转变估计不足」的情况\footnote{《毛泽东传(1893-1949)》,页 507.}。 毛表示同意王明提出的「国共两
党共同负责、共同领导」的主张,但是,毛并没有完全放弃自己的立场,
仍坚持认为在国共两党之间存在着「谁吸引谁」的问题\footnote{中共迄今仍未公布毛在 1937 年 12 月政治局会议上讲话的全文。
毛泽东在这次会议上发言的少量片断散见于近
年出版的某些官方权威论著中,如中共中央文献研究室编:
《毛泽东传(1893-1949)》。此处资料来源于马齐彬(前中
央文献研究室副主任):〈抗战初期的王明投降主义路线错误〉,
载《党史资料丛刊》,1981 年第 1 辑(上海:上海人民
出版社,1981 年);《毛泽东传(1893-1949)》,页 508.  }。

十二月政治局会议上所发生的一切,证实了毛泽东早些日子对党内形
势的预测:随着王明返国,将有一批政治局委员重新聚合在王明的周围。
为了应付这种局面,
毛泽东实际上早已有所准备, 12 月上旬就对自己在
从统一战线方面的立场作了局部调整。1937 年 12 月 6 日,即在王明到达延
安後的第八天,毛泽东就与周恩来、彭德怀,联名致电八路军总部,要求
坚决执行统一战线方针,加强部队的统战教育。在这份电文中,毛尤其强
调八路军应与阎锡山及地方行政机构多方沟通,避免发生摩擦,指示八路
军停止自行征粮、征布,所需物品改为「向政府借拨」。\footnote{参见〈毛泽东、周恩来、彭德怀致朱德、任弼时、邓小平等电〉
(1937 年 12 月 6 日)
,载中央档案馆编:
《中共
中央文件选集》
(1936 一 1938)
,第 11 册,页 400-401.} 尽管毛泽东在统
一战线问题上的强硬态度已有所软化,但仍未能减缓王明等在会议上对自
己的进攻。

在十二月政治局会议上,毛泽东暂时处于下风,王明虽然获得普遍响
应,但王明获得的成果也仅此而已。毛泽东、王明一时势均力敌,谁也不
具特别优势。会议宣布改组中央书记处,决定不设总书记一职,由书记处
实行集体领导,事实上形成了毛泽东、王明分享中共最高权力的格局。

十二月政治局会议公布了有十六人组成的政治局委员、
候补委员名单。
 

他们分别是:毛泽东、王明、张闻天、周恩来、博古、朱德、张国焘、王
稼祥、任弼时、彭德怀、项英、刘少奇、康生、陈云、邓发、凯丰
\footnote{参见王建英编:
《中国共产党组织史资料汇编——领导机构沿革和成员名录》
(北京:红旗出版社,1983 年),页 296.}。
 
据张国焘回忆,
这份名单是得到斯大林批准,
由王明在会议上宣布的。
由于王明「事先没和任何人商量」就拿出名单,毛泽东「似感不安」。但
是毛对这份名单并未表示反对意见,其原因可能是这份新名单与原有政治
局成员的构成基本一致,王明并没有将新人塞入这份名单。在新公布的政
治局委员中,绝大多数都是六届四中全会和五中全会产生的政治局委员,
关向应原为六届五中全会的政治局候补委员,
此次未进入政治局。
彭德怀、
张浩于 1936 年 1 月进入政治局,但是不久就不再通知张浩参加政治局会
议。张浩以共产国际代表身份,动员张国焘北上的使命完成後,他的政治
局委员一职也就停止了。在王明拿出这份名单宣布後,毛泽东并没有完全
放下心来,他要试探一下王明。毛在会上「极力推崇王明为中共中央领
袖,......力主将王明名字列入第一名」。王明则竭力表明,他提这份名单
决无「夺帅印」的意思。\footnote{参见张国焘:
《我的回忆》
,第 3 册,页 424-25.}经这番试探,毛泽东已知王明确无取而代之的意
思,也就不再坚持了。

对于毛泽东而言,十二月政治局会议是不愉快的,但在不愉快中,也 有一两件令
人快慰之事,这就是会议决定终止党中央「负总责」之人的设 置, 剥夺了张闻天
实际担任的总书记一职\footnote{张闻天于 1938 年 4 月 12 日在《新华日报》
上发表声明,说明自己从 1937 年 12 月後即不再担任中共中央总负 责人之职。
这个声明实际上是由王明起草,以张闻天的名义在武汉发表的,此亦说明王明回国
後张闻天政治地位已下 降。然而在以後清算王明在武汉「闹独立性」时,却没有
就此事特别指责王明,盖因剥夺张闻天「总负责」的名义亦 符合毛泽东的心愿。},
并把周恩来逐出了中央书记处\footnote{有关周恩来在 1937 年 12 月政治局会议
上被免除中央书记处书记一职事,可从政治局在 1937 年 12 月 13 日通过 的准
备召集党的第七次全国代表大会的决议中得到证实。该决议宣布,成立由二十五人
组成的中共七大准备委员会, 周恩来虽被列名在内,但是在准备委员会下设立的
组织秘书处成员只有中央书记处的五名书记,周被排除在外。参见 中央档案馆编:
《中共中央文件选集》(1936-1938),第 11 册,页 406-407.  }。

自三十年代中期始,中央书记处长期承担政治局常务委员会的功能,
1934 年 1 月六届五中全会成立的中央书记处由博古、周恩来、张闻天、项
英四人组成,是党的最高权力机构,由博古在书记处负总责。遵义会议改
组了中央书记处,毛泽东、王稼祥进入书记处,项英因留在江西,事实上
停止了书记的职权,新的书记处有张闻天、毛泽东、周恩来、博古、王稼
祥五位成员,由张闻天担任党的总负责人,即实际上的总书记一职。这种
人事安排一直持续到 1937 年 12 月政治局会议的召开。

十二月政治局会议决定不设党的总负责人,张闻天由总书记变为书记
之一,地位明显下降,在毛泽东暂时处于守势时,毛宁愿此位空缺。会议
还免去了周恩来长期担任的中央书记处书记一职。新成立的书记处由毛泽
东、王明、张闻天、陈云、康生组成。从苏联返国的王明等三人,在书记
处占了五分之三的比重,从表面上看,王明似乎占了上风。原中央书记博
古、王稼祥也被免去了书记的职务。

周恩来离开中央书记处标志着周在党的核心层中的影响已明显减弱。
不知是王明的政治敏感较差,还是他想和过去的路线撇清关系,王明眼看
着周恩来、博古、王稼祥这三位过去路线的代表人物离开书记处竟毫无反
应。
周恩来地位的下降显然符合毛泽东的心意,
毛反感周恩来向王明靠拢,
而无援周之意,毛将乐意看到周恩来与王明的关系出现裂缝。

十二月政治局会议对张闻天的打击十分沉重。张闻天本来就对王明返
国抱有警惕,
担心王明将取代自己成为党的总书记,
如今不设总书记一职,
使王明、张闻天皆有所失。目前王明的地位上升,张闻天对王明的不满也
急剧加深,毛泽东将坐山观虎斗,看彼等四分五裂。互相内讧,从这个意
义上说,十二月政治局会议未尝不是一件「好事」。

十二月政治局会议达成的毛泽东、王明共治的局面,还体现在会议决
定成立的中共七大准备委员会等有关人事安排上。
该委员会共有二十五人,
除了十六名政治局委员,还包括了数名党的元老和重要的军政干部,由毛
泽东担任委员会主席,王明担任书记。在准备委员会之下,另成立由书记
处五名书记组成的秘书处,周恩来、博古、项英等皆被排除在外
\footnote{参见〈中央政治局关于准备召集党第七次全国代表大会的决议〉
(1937 年 12 月 13 日通过)
,载中央档案馆编:
《中共中央文件选集》
(1936-1938)
,第 11 册,页 405、406.}。

毛泽东在十二月政治局会议上遭到遵义会议以来最大的挫折,毛泽东
的权威受到王明等的沉重打击,政治局几乎一致支持王明的主张,使王明
在党的核心层中的影响急剧扩大。毛泽东将被迫与王明进行合作,形成了
对毛极不情愿的毛、王体制。在这种体制内,毛泽东暂时不占优势,而王
明的地位则相对稳固。
 
在毛泽东暂时处下风的时刻,毛对自己的前途并没有任何的悲观。毛
牢牢掌握着军队和与莫斯科的电讯来往。毛清楚知道,王明得势的原因是
他的莫斯科代言人角色,以及周恩来等对王明的支持。毛相信,将王明与
周恩来等结合在一起的只是政见上的一致而非宗派上的结合,因此,一旦
形势转变,周恩来等将会改变看法,转而支持自己的主张,彭德怀等也会
迅速转变过来,周与王明的结合也就会随之结束。此次会议决定王明将率
代表团前往武汉与蒋介石会谈,并成立中共中央长江局,此项决定也有利
于毛泽东。王明远离延安,也就避免了在延安出现「二主并立」、分庭抗
礼的尴尬局面。对于张闻天等人,毛泽东更有充分的信心,毛断定张闻天
因利害关系绝不会与王明「重结二度梅」,毛且作好准备,等着看他们互
相火并。至于王明,则有很多理由为十二月政治局会议的结果而高兴。首
先,他的报告被与会者一致接受,政治局的同事们都对他表示了热诚的欢
迎;其二,政治局对中共驻共产国际代表团的工作给予了高度评价,称赞
代表团「在关于抗日民族统一战线新的政策的确定与发展上给了中央以极
大的帮助」,\footnote{参见中央档案馆编:
《中共中央文件选集》
(1936-1938)
,第 11 册,页 402. 此决议如同十二月政治局会议所通
过的其它决议一样,在中共党史编纂学中长期被隐瞒,直到八十年代後才陆续予以公开。}
中共中央表彰了代表团,也就是对王明本人的表彰;其三,
王明的地位在会议上得到确定,成为党的第二号人物,由于自己显示了政
治领袖的水平并具有雄厚的国际背景,争得了在政治上「帮助」毛泽东的
资格,跟随自己从莫斯科返国的康生等也都成了书记处成员;其四,会议
决定成立以王明为首的中共代表团与国民党谈判,这将使王明成为国内活
跃的领袖人物。

在胜利的喜悦中,王明没有看见笼罩在他头上的乌云正慢慢聚集,在
十二月政治局会议上,由于王明不能保证争得苏联军援,对他不满的气氛
已经悄悄出现。

早在江西时期,中共就迫切希望从苏联获得军援,但苏联一向口惠而
实不至,令中共军政领导人大为失望。1933 年,中共听信了共产国际军事
代表团顾问弗雷德(驻上海)和李德的许诺,在瑞金花费大量人力修建了
机场,
准备迎接苏联军用飞机的降落,
结果是望断秋水,
空欢喜一场。\footnote{奥托·布劳恩(李德):
《中国纪事(1932-1939)》,页 86.} 1936
年 8 月 25 日,陕北局势危急,红军财政、粮食已达最後极限,毛与张闻
天、周恩来、博古联名,急电王明,希望王明向苏方请求给红军提供飞机、
大炮,\footnote{《周恩来年谱》
,页 318.} 以实现红军占领甘肃西部、宁夏、绥远一带的计划,此项求援也
因各种条件的制约,最後也未能落实。所有这些都削弱了王明在中共核心
层中的地位。

抗战爆发後,
苏联对国民党政府开始了大规模的军事援助,
但恪于
「中 苏互不侵犯条约」的限制,对中共基本没有提供军火方面的援助。令中共
领袖们普遍反感的是,当源源不断的苏联军火沿阿拉木图——迪化(乌鲁
木齐)——兰州公路,向重庆运去的时候,苏联军用飞机只给延安送来一
些中文版的斯大林、列宁著作和少量的高射机枪、药品、大型无线电台等。
于是领导人抱怨——「书籍给了无产阶级,军火给了资产阶级」。

十二月政治局会议上,中共领袖最关心的问题之一是苏联能否给中共
实际的援助。王明为斯大林的对华政策作了解释,表示苏联不大可能给延
安军援,王明的答复令所有人失望。 \footnote{参见张国焘:
《我的回忆》
,第 3 册,页 418、420.}对苏联不满的情绪已经存在,只要
稍加鼓动,十分容易将矛头转向王明,作为中共驻共产国际的代表,既然
是被斯大林派回来的,可是在帮助国内的斗争中又作出什么具体贡献呢?
苏联不给中共军火,难道与王明的无能没有关系吗?

王明在十二月政治局会议上取得的胜利使他对自己的前途充满自信,
王明只看到政治局委员们支持他的一面,而没看到这种政治支持的脆弱一
面。王明与多数政治局委员只存在一般的工作关系,并无历史渊源、个人
友谊作这种关系的基础。他忘记了,这种建立在政见一致基础上的政治上
的结合,经常会因形势变化、人际关系等因素而处于波动中。少年得志的
王明只是一个深受俄化教育的共产党新贵,对这种世故哲学似乎理解得不
深。十二月政治局会议後,中共核心层的内部关系十分微妙,尽管王明影
响上升,但支持王明的力量缺少稳定性,大多数政治局委员并没有把「宝」
押在王明一边,而是谨言慎行。在毛泽东、王明之间犹疑摇摆。
周恩来、朱德、彭德怀对王明的态度是友好的,但他们对毛泽东也很
尊敬,且一切都是光明正大,和王明没有任何私下交易。

康生和陈云是王明昔日在莫斯科的老同事,康生虽然多年来与王明密
切合作,但甫抵延安,却小心翼翼,实际上是在默默估算毛泽东与王明各
自的实力阵容,为自己要走的下一步棋在暗中准备。陈云虽与王明共事较
久,但是只限于工作关系,与王明谈不上志同道合。没有任何证据可显示
康生、陈云会在政治上继续支持王明。

政治局候补委员邓发,这位中共在江西时期的捷尔仁斯基,早在遵义
会议後就已权势大减。邓发因狂热肃反,在党内、军内积怨甚多,加之邓
发与周恩来关系密切,1936 年 4 月,就被毛泽东、张闻天打发去了苏联,
他的国家政治保卫局一摊子也早由毛泽东的老部下王首道接替。目前邓发
担任了中共驻新疆办事处代表,已远离权力核心,邓发甚至连十二月政治
局会议也未能参加。邓发对王明不可能有任何实际的帮助。

博古是王明昔日的亲密战友,多年来与毛泽东对垒,对王明返国自然 是由衷的高
兴,但王明在十二月政治局会议上讲的一些话却使博古很不满 意。王明为了显示
自己一贯正确和立场公正,对博古主持的 1934 年 1 月 的中共六届五中全会提出
了尖锐的批评。 \footnote{王明在 1937 年 12 月 10 日下午的会议上作关于中
共驻共产国际代表团工作报告,王明在会议快结束时说: 「我 们现在估计党中央
的路线一般的是正确的,要估计到较大的错误便是五中全会的决议。」参见周国全、
郭德宏等: 《王 明评传》,页 303.  }王明此举只 能使博古对自己徒 增不满,
从而影响两人的关系。

王明真正的支持者只有项英一人。项英多年来就对毛泽东的个人品质 存有严重的
怀疑,长征前夕,项英曾预感到毛将利用党的危急形势,夺取 最高领导权。
\footnote{奥托·布劳恩(李德):
《中国纪事(1932-1939)》,页 117-18.}1937 年
12 月,项英从南方赶赴延安,与分手三年多的战 友们重逢,并参加了十二月政治
局会议。在项英的眼中,王明无疑是值得 信赖的党的领袖,更是抗衡、制约毛泽
东的最佳人选,因而,项英真诚地 拥护王明。十二月政治局会议决定成立由项英
任书记中共东南分局,在党 的关系上,直属长江局领导,新四军则受延安和长江
局双重领导,项英对 这项安排由衷地满意。

十二月政治局会议在毛泽东与王明之间投下了长长的阴影,王明在严
重冒犯了毛泽东之後,也许并不知道自己已被毛视为必欲除之而後快的党
内头号敌人。正当王明自呜得意,陶醉在眼前的胜利时,毛泽东则在妥协、
退却烟幕的掩护下,为打败王明卧薪尝胆、积蓄力量。


\section{毛泽东与武汉「第二政治局」的对立}

1937 年 12 月 18 日, 仅距十二月政治局会议闭幕四天, 王明、 周恩来、 博
古等率领中共代表团就抵达武汉。12 月 23 日,中共代表团与中共长江 局举行会
议,决定将两个组织合并,对外称中共代表团,对内为长江局。

中共长江局是陕北以外中共最大的组织机构,在长江局集中了当时十 六名政治局
委员中的五人, 他们分别是王明、 周恩来、 博古、 项英以及 1938 年 4 月调
入的凯丰。在长江局担负领导工作的还有董必武、林伯渠、吴玉 章三位中共元老
(林伯渠不久调任中共驻西安办事处代表),以及叶剑英、 邓颖超、李克农、吴
克坚、廖焕星(王明在莫斯科期间的秘书)等。

长江局是 1927 年国共分裂後,中共在国民党统治区成立的最重要的
机构。长江局担负着领导南中国中共地下组织和新四军的工作,承担与国
民党谈判以及联络社会各界的繁重事务。
长江局还公开出版中共机关报
《新
华日报》和党刊《群众》。\footnote{ 1937 年 12 月 28 日,中共中央书记处致电共产国际,汇报「十二月政治局会议」的召开经过及人事安排事项,
并要求共产国际派曾在巴黎办中共报纸《救国时报》的吴玉章、萧三、廖焕星回国办《新华日报》
。}由于长江局工作范围极其广泛,对中共全局性
的路线、方针发挥着重大影响,其领导人在中共党内声名显赫,所以长江
局又有「第二政治局」的称呼。

中共长江局的核心灵魂是王明和周恩来。在十二月政治局会议上,王
明得到周恩来的有力支持,会议之後,王明与周恩来在党内的影响明显增
强。王明、周恩来抵汉後,立即展开紧张的工作,全面落实、贯彻「十二
月政治局会议」关于加强国共合作、巩固与国民党的统一战线的方针,从
而与毛泽东的矛盾进一步扩大。

王明、
周恩来强调中共应尽力维护同国民党的友好关系,
与国民党
「开
诚合作」,对国民党的意见「一般宜采取赞助的立场」。特别在成立地方
政权问题上,中共应事先征得国民政府的同意,促成在「国民政府基础上
建立统一的国防政府」。1938 年 1 月 28 日,王明、周恩来、博古、董必
武、叶剑英致电延安中央书记处,对晋察冀边区临时行政委员会不向国民
政府备案而自行成立提出批评,认为此举「对全国统战工作将发生不良影
响」。\footnote{《周恩来年谱》
,页 398、402. 几年後此封电报被当作「王明投降主义」的罪证受到严厉批判,建国後在略去周
恩来等名字後,反复受到批判。 } 

在军事问题上,王明、周恩来主张中共领导的武装游击队应取得合法
地位,他们并且认为中共军队应积极配合国民党军作战,应抽调八路军主
力参加对日作战。1938 年 1 月 11 日,周恩来、叶剑英致电毛泽东、朱德、
彭德怀,建议调刘伯承师或林彪师,待日军进攻郑州紧急时,渡黄河参加
陇海路西段的战斗\footnote{《周恩来年谱》
,页 399.}。

王明、周恩来在中共参加政府问题上的态度也和毛泽东存在差别。抗
战爆发後,毛泽东、张闻天对中共参加政府事一直持谨慎、保守的态度。
1937 年 9 月 25 日,
中共中央作出
〈关于共产党员参加政府问题的决定
(草
案)〉,提出在国民党未改变其一党专政实质前,中共参加政府只会模糊
其阶级性质。毛泽东、张闻天的这种立场对王明也产生了影响。1938 年 2
月 2 日,王明在武汉以毛泽东名义发表〈毛泽东先生与延安新中华报记者
其光先生的谈话〉,再次重申中共不参加政府的立场,但是王明并不反对
个别中共领导人以个人身份参加政府。1938 年 1 月,国民政府军事委员会
改组,蒋介石邀请周恩来担任军委会政治部副部长,周恩来因知道中共中
央有关禁令,一再推辞,但蒋介石仍坚持原有意见。1 月 11 日,王明、周
恩来、博古、董必武、叶剑英致电延安中央书记处,告知以上情况,并请
延安考虑具体意见,
实际上是要求延安收回不许共产党员参加政府的成命。
毛泽东等对王明、周恩来等的来电,拒不答复,显示毛坚持原有立场的态
度。王明、周恩来、博古等在未得到延安覆电的情况下,于 1 月 21 日再
电中央书记处,陈述周恩来担任此职,将有利于扩大共产党的影响,如果
屡推不就,会使蒋介石、陈诚认为共产党无意相助。对这份说理透彻,言
辞恳切的电报,毛泽东照样不覆电。在延安拒不表态的情况下,长江局决
定,周恩来代表中共担任国民党政府军委会政治部副部长,1938 年 2 月 6
日,军委会政治部成立,周恩来出任副部长。而中共中央直到 1938 年 2
月末至 3 月初召开的政治局会议上才正式批准周恩来担任此职。
\footnote{《周恩来年谱》,页 399、401、406 }

王明、周恩来等在长江局的联合行动,构成了对毛泽东权威的严重挑
 战,长江局不仅对中共全局性的方针起着重要影响,对中共组织问题也曾
一度拥有决定权。王明、周恩来联手,打破了毛泽东自遵义会议後一手独
揽政治局的局面,
迫使毛泽东在一些重大问题上不得不按长江局的意见办。
1938 年 2 月末,在王明、周恩来等压力下,毛泽东被迫同意召开政治局会
议,就是一突出事例。

1938 年 2 月 7 日,王明、周恩来、博古、董必武、叶剑英联名致电延
安中央书记处,提议 2 月 20 日前後召开政治局会议,以研究国共关系中
出现的「许多新的严重问题」\footnote{《周恩来年谱》
,页 401-403.}。

王明、周恩来等在这封电报中提到的国共关系中的「新的严重问题」,
是指 1938 年 1 月後,在国民党内出现的反共舆论。1 月 23 日,代表国民
党出席国共两党关系委员会的康泽、刘健群当着王明、周恩来的面,指责
八路军在华北「游而不击」。随後,在武汉的国民党报纸出现攻击中共的
言论。1 月 17 日《新华日报》营业部及印刷厂被一批身份不明的人捣毁。
上述情况引起王明、
周恩来的警惕,
迅速向延安紧急通报武汉的最新动向,
提议召集政治局会议,商讨有关对策。

然而,对长江局的这份电报毛泽东仍是不予回答。毛泽东在无力正面
反击不同意见时,惯于用这种「留中不发」的方式来表明自己的立场。
在迟迟未接到延安覆电的情况下,2 月 15 日,王明、周恩来、博古又
一次致电中央书记处并任弼时、凯丰、朱德、彭德怀,提议在即将召开的
政治局会议上讨论两个议题:抗战形势和中共七大准备工作问题
\footnote{《周恩来年谱》
,页 404.}。

王明、周恩来、博古似乎估计到毛泽东仍将会对彼等建议采取消极态
度,索性将电报发给书记处和书记处以外的政治局委员,试图用政治局集
体的力量,迫使毛泽东同意长江局的建议。

1938 年 2 月 23 日,王明、周恩来、博古再次致电中央书记处,提议
举行政治局会议的日期为两天,「由王明、恩来将长江局会议讨论结果,
向政治局建议,会完後立即返汉」。\footnote{参见珏石:
〈周恩来与抗战初期的长江局〉,载《中共党史研究》
,1988 年第 2 期。}面对王明、周恩来的持续压力,毛泽
东被迫退却,最後,不得不接受了召开政治局会议的建议。
 
王明、周恩来等在召开政治局会议一事上的态度,引致毛泽东的强烈 不满。毛认
为,这是王明、周恩来造成既成事实,逼其就范。几年後,毛 回忆 1938 春发生
的这次事件时说,「三月会议,长江局先打一个电报, 规定议事日程,决定某某
要人回长江局工作,这种态度我很不满意」\footnote{参见珏石: 〈周恩来与抗
战初期的长江局〉,载《中共党史研究》,1988 年第 2 期。}。 只 是这些都
是後话,当时毛泽东虽然极不情愿,也只能忍受。
 
在王明、周恩来的一再要求下,1938 年 2 月 27 至 3 月 1 日,中央政
治局在延安举行了会议;史称「三月政治局会议」,毛泽东在这次会议上
又一次受挫,被迫再次对王明、周恩来等让步。
 
三月政治局会议批准了王明、周恩来提出的加强与国民党建立统一战 线的方针。
在参加会议的八名政治局委员中,毛泽东、张闻天、任弼时对 王明的报告提了一
些保留意见,王明、周恩来、凯丰、张国焘的意见则基 本一致,康生的态度模棱
两可。周恩来在支持王明意见的方面发挥了突出 的作用。周提出应向蒋介石建议,
确立「以运动战为主,包括阵地战,以 游击战为辅」的军事战略方针,以及「组
织新的军队」的意见 \footnote{长江局时期,王明、周恩来等对运动战、阵地战
的强调乃是作为一种建议向国民党提出的。并非像中共党史编纂学有意曲解的那
样,是企图把此方针强加给八路军、新四军,事实上毛泽东也提出过类似意见。
1938 年 1 月 13 日, 毛曾说,「为什么说游击战为辅呢?因为游击战不能最後
消灭敌人。所以, 现在要以运动战为主, 阵地战、 游击战为辅。」 参见《毛
泽东年谱》,中卷,页 45;另参见龚希光: 〈朱德与华北初期的「运动游击战」
问题〉,载《党的文献》,1996 年第 6 期。 }。
 
三月政治局会议在王明、周恩来等的努力下,把加紧筹备中共七大列
入党在近期工作的主要任务。「在最近时期内」召开七大,是「十二月政
治局会议」通过的一项重要决定,王明、周恩来对落实这一决定十分重视。
三月政治局会议决定,
发布为召集党的七大告全党同志书和告全国同胞书;
向地方党组织发布进行七大工作准备的指示;
成立大会报告的准备委员会;
责成政治局及中央同志起草七大报告的政治提纲和专题论文。在毛泽东暂
处守势的形势下,召开七大无疑会进一步削弱毛的影响力。令毛稍许宽慰
的是,三月政治局会议没有明确中共七大政治报告的起草人和报告人。
 
三月政治局会议对毛泽东的又一打击是,会议否决了毛提出的留王明 在延安工作
的意见,同意王明返回武汉继续主持长江局的工作。在 3 月 1 日的会议上,毛提
出,「在今天的形势下,王明不能再到武汉去」。\footnote{参见廖心文:〈抗
日战争初期长江中央局的组织变动情况——兼谈王明是怎样当上书记的〉,载中共
中央文献研 究室编: 《文献和研究》(1987 年汇编本)(北京:档案出版社,
1991 年),页 285.} 毛泽 东公开的理由是,应考虑王明的安全和延安中央书记
处工作的需要。但是 实际上,经过近三个月时间的观察,毛已完全领教了王明、
周恩来结合的 厉害。把王明调回延安是一箭双雕,不仅可以切断王明与周恩来的
联系, 还可以使王明在延安无所作为, 无事可做。 对于毛的这项提议的真实用
意, 王明似有警觉,他在 3 月 1 日的会议上,明确表示自己希望继续在武汉工
作。\footnote{1941 年 10 月 8 日,王明在中央书记处会议的发言中谈到他当时
的心情是「不愿留在延安工作」,参见周国全、 郭德宏等: 《王明评传》,页
340.} 会议最後对毛泽东的建议进行表决,以五票反对,三票赞成,作出最 後决
定,调政治局候补委员凯丰去长江局工作,王明在汉一月後返延\footnote{参见廖
心文:
〈抗日战争初期长江中央局的组织变动情况——兼谈王明是怎样当上书记的〉,载中共中央文献研 究室编:
《文献和研究》(1987 年汇编本)(北京:档
案出版社,1991 年),页 285.}。面 对多数人的意见,毛只得暂时搁置自己的
计划。

三月政治局会议的召开加强了王明、周恩来在党内的影响,尤其在中 共政策制定
方面的影响。会议一致同意,由王明代为起草会议的总结\footnote{参见廖心文:
〈抗日战争初期长江中央局的组织变动情况——兼谈王明是怎样当上书记的〉,载
中共中央文献研 究室编: 《文献和研究》(1987 年汇编本)(北京:档案出
版社,1991 年),页 285.},会 议还决定,由王明代表中共中央起草致预定在
三月下旬召开的国民党临时 全国代表大会的建议书,由周恩来起草对国民党的军
事建议书。中央书记 处并最终批准周恩来出任国民政府军事委员会政治部副部长。

对于王明、周恩来在中共党内地位的上升,毛泽东无时不刻想予以限 制。3 月 21
日长江局将王明起草的中共致国民党临时全国代表大会的建议 书电告延安,但几
天都未得到毛等的答复,于是王明等在 3 月 24 日将建 议书送交国民党方面。一
天後,延安中央书记处来电,批评此信有错误, 要求另送一份由中央书记处起草
的贺信。王明、周恩来等长江局全体领导 成员共同署名,于 4 月 1 日给延安覆
电,指出来电太迟,且大会已于 4 月 1 日结束,要求延安不要公布这份文件,
「否则对党内党外都会发生重大 的不良政治影响。」\footnote{参见珏石: 〈周
恩来与抗战初期的长江局〉,载《中共党史研究》,1988 年第 2 期。}

1938 年
6 至 7 月,延安 与长江局的冲突又起,中央 书记处来电,表示不同意王明、周
恩来、博古等联合 发表的〈我们对于保 卫武汉与第三期抗战底意见〉。长江局覆
电延安,坚持原议,双方互不相让,电报往来不绝,几成「电报战」。
\footnote{《周恩来年谱》,页 414、416.}在延安发往长江局的电报中, 毛泽
东故意隐身在後, 多以书记处或在延安的政治局委员联合署名的形式发电, 在署
名排列顺序上,特意将自己的名字放在最後一位。尽管毛是这一切的
指使者,
但是在长江局的压力下,
毛被迫淡化自己的党内首席角色的色彩。
\footnote{ 1938 年上半年,凡发往北方局和八路军的电报,毛泽东的名字都放在第一位,而在发往长江局的电报排名顺序
上,毛的名字一般放在最後一位。}
 
毛泽东在三月政治局会议上的受挫及王明、周恩来的成功,使得长江 局的影响开
始超出华南、华中的范围,向八路军和华北地区扩散。1938 年 春,彭德怀在八路
军传达十二月政治局会议精神。彭德怀检查过去党「对 国民党的基本转变认识不
够」,认为「机械地划分某一时期以左倾或右倾 为主的阶段论是不正确的」。
\footnote{参见彭德怀: 〈目前抗战形势与争取抗战胜利的方针——中央政治局 12
月会议总结与精神〉(1938 年春)载 , 《六 大以来——党内秘密文件》,上,
页 916、919-20.}几年後,毛泽东抓住彭德怀这些话,要彭德 怀承认在抗战初期
执行了 「王明右倾投降主义路线」, 并把彭德怀传达
「十二月政治局会议」的报告,作为党内反面材料,收入《六大以来》一书。
其实,彭德怀只是根据张闻天交给他的一份经由中央书记处草拟的传达大
纲,即〈中央政治局十二月会议的总结与精神〉,在八路军五四三旅团级
干部大会上,照本宣读了一遍\footnote{《彭德怀自述》
,页 226.}。

依照毛泽东的逻辑,在这个阶段,彭德怀跟王明跑了。但是事实上,
军队和地方的领导人根本不知道,在中央精神之外,还另有一个「毛泽东
的正确路线」。1938 年 1 月 7 至 8 日,毛泽民在八路军驻兰州办事处传达
十二月政治局会议精神。\footnote{《谢觉哉日记》
,上卷(北京:人民出版社,1984 年)
,页 204.}1938 年 4 月,中共北方局冀察晋边区分局书记
彭真在边区第一次党代表大会上作〈关于全国抗战形势和争取抗战胜利方
针〉的报告,强调「一切工作都充分把握住,一切经过统一战线,一切服
从统一战线」的原则。1938 年 5 月,邓小平也批评了晋中地区的向地主借
粮运动,指出这是「破坏农村统一战线」的「左」倾错误。在 1938 年 7
月之前,彭德怀、杨尚昆、彭真、邓小平在华北部分地区,还展开了反对
党内「左倾关门主义」的斗争。

在毛泽东和长江局对抗的日子里,张闻天仍然坚定地站在毛的一边。
毛还发现了另外几个新盟友,一位是毛在江西时期的老熟人任弼时,另两
位是与王明一道刚从苏联返国的康生和陈云。至于毛泽东志同道合的盟友
则非刘少奇莫属。刘少奇虽然没出席三月政治局会议,但其观点与毛几乎
完全一致。三月政治局会议一结束,毛泽东就将刘少奇紧急召回延安,以
加强自己在政治局中的力量。毛泽东为了让刘少奇放心,1938 年 3 月 24
日,中央书记处作出北方局领导人员分工的新决定,明确规定胡服(刘少
奇)在延期间,仍然担负华北党的领导工作,所有情况须直接向胡服报告。
\footnote{参见中央档案馆编《中共中央文件选集》
(1936 一 1938)
,第 11 册,页 477.}

1938 年 7 月 10 日,刘少奇写信给彭真,提醒彭真立即修正政策。刘少奇 在信
中告诫道:「国民党与阎锡山至今不承认统一战线,因此在文件和宣 传上以少说
或者不说统一战线为好」\footnote{参见郭华伦: 《中共史论》,第 3 册,页
365.}。对于政治局同事们的表现,毛泽东 不动声色,一一看在眼里,在清冷的延
安窑洞里,毛在紧张地作着各项准 备,等待着反击日子的到来。

