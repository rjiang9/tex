\documentclass[12pt,a4paper]{article}
\usepackage{fontspec,xunicode,xltxtra,lipsum}
\usepackage{titlesec}
\usepackage[top=1in,bottom=1in,left=1.25in,right=1.25in]{geometry}
\usepackage{xcolor}
\usepackage{xunicode}
\usepackage{xltxtra}
\usepackage{polyglossia}

\usepackage{graphicx} 		% 插入图片
\usepackage{indentfirst}             									% 段首缩进
\XeTeXlinebreaklocale "zh"    	           							% 中文断行
\XeTeXlinebreakskip = 0pt plus 1pt minus 0.1pt			% 中文断行
\renewcommand{\baselinestretch}{1.25}

\defaultfontfeatures{Mapping=tex-text}

% \setdefaultlanguage{english}  		% 设置默认语言 english | chinese

% 设置字体
\setsansfont{Calibri}
\setmainfont[BoldFont=SimHei]{SimHei}

\titleformat{\section}{\Large\song}{\thesection}{1em}{}

% 新定义字体
\newfontfamily\shu{FZShuTi}                           % 方正舒体
\newfontfamily\nming{MingLiU_HKSCS} 		% 新细明体
\newfontfamily\lishu{LiSu} 								% 隶书
\newfontfamily\hei{SimHei} 								% 黑体
\newfontfamily\yahei{Microsoft YaHei} 			% 微软雅黑
\newfontfamily\henghei{Microsoft JhengHei}  % 微軟正黑體
\newfontfamily\song{SimSun}  			% 宋体
\newfontfamily\nsong{NSimSun}  		% 新宋体
\newfontfamily\yao{FZYaoTi}                % 方正姚体
\newfontfamily\ming{MingLiU} 			% 细明体
\newfontfamily\youyuan{YouYuan}		% 幼圆
\newfontfamily\kai{KaiTi}         			% 楷体
\newfontfamily\cai{STCaiyun}   			% 华文彩云



\begin{document}

%%%% 段落首行缩进两个字 %%%%
\makeatletter
\let\@afterindentfalse\@afterindenttrue
\@afterindenttrue
\makeatother
\setlength{\parindent}{2em}  %中文缩进两个汉字位


\title{\hei XeTeX使用小结}
\author{\song 江长里}
\date{\kai 2009年6月21日}

\maketitle

\section{简介}
(默认字体) 以前使用CJK进行中文的排版,需要自己生成字体库,近日,出现了XeTeX,\yao (姚体)可以比较好的解决中文字体问题,不需要额外
生成LaTeX字体库,\kai (楷体)直接使用计算机 \henghei (微軟正黑體)系统里的字体。

\section{字体列表}
本文使用了大量本机自带的字体。
%%%%%%%%%%%%%%%%%%%%%%%%
\begin{enumerate}
	\item 《霍元甲》:  \lipsum[1]
	\item   《妙法莲华经》: \lipsum[1]
	\item    寒山拾得:  \lipsum[1]
\end{enumerate}




%%%%%%%%%%%%%%%%%%%%%%%%%%%%
\begin{table}[htbp]
\caption{字体列表 office station}

\centering
\begin{tabular}{|l|c|r|}
\hline
\hei 字体 & \hei 命令 & \hei 字体效果 \\
\hline

\kai 宋体 & \verb+\song+ & \song 宋体 \\
\kai 新宋体 & \verb+\nsong+ & \nsong 新宋体 \\
\kai 楷体 & \verb+\kai+ & \kai 楷体 \\
\kai 黑体 & \verb+\hei+ & \hei 黑体 \\
\kai 隶书 & \verb+\lishu+ & \lishu 隶书 \\
\kai 方正舒体 & \verb+\shu+ & \shu  方正舒体 \\
\kai 方正姚体 & \verb+\yao+ & \yao 方正姚体 \\
\kai 幼圆 & \verb+\youyuan+ & \youyuan 幼圆 \\
\kai 微软雅黑 & \verb+\yahei+ & \yahei 微软雅黑 \\
\kai 微軟正黑體 & \verb+\henghei+ & \henghei 微軟正黑體 \\
\kai 細明體 & \verb+\ming+ & \ming 細明體 \\
\kai 新细明体 & \verb+\nming+ & \nming 新细明体 \\
\kai 华文彩云 & \verb+\cai+ & \cai 华文彩云 \\


\hline
\end{tabular}
\end{table}





图片插入效果图:
    \begin{center}
		\includegraphics[width=1\textwidth]{building.jpg}
	\end{center}


\end{document}

