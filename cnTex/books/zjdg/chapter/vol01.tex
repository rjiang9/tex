% -*- coding: utf-8 -*-

\newpage
\chapter{御錄宗鏡大綱卷一}

\section{自序}
伏以真源湛寂,覺海澄清。絕名相之端,無能所之跡。最初不覺,忽起動心,成業識之由,為覺明之咎。因明起照,見分俄興,隨照立塵,相分安布。如鏡現像,頓起根身,次則隨相而世界成差,後則因智而憎愛不等。從此遺真失性,執相循名,積滯著之情塵,結相續之識浪。鎖真覺於夢夜,沈迷三界之中,瞽智眼於昏衢,匍匐九居之內。遂乃縻業係之苦,喪解脫之門。於無身中受身,向無趣中立趣。約依處則分二十五有,論正報則具十二類生,皆從情想根由,遂致依止差別。向不遷境上,虛受輪回;於無脫法中,自生係縛。如春蠶作繭,似秋蛾赴燈。以二見妄想之絲,纏苦聚之業質;用無明貪愛之翼,撲生死之火輪。用穀饗言音,論四生妍鬼;以妄想心鏡,現三有形儀。然後違順想風,動搖覺海,貪癡愛水,資潤苦芽,一向徇塵,罔知反本。發狂亂之知見,翳於自心;立幻化之色聲,認為他法。從此一微涉境,漸成戛漢之高峰;滴水興波,終起吞舟之巨浪。爾後將欲反初複本,約根習鈍。不同於一真如界中,開三乘五性。或見空而證果,或了緣而入真,或三祇重煉,漸具行門;或一念圓修,頓成佛道。斯則克證有異,一性非殊,因成凡聖之名,以分真俗之相。若欲窮微洞本,究肯通宗,則根本性離,畢竟寂滅,絕昇沈之異,無縛脫之殊,既無在世之人,亦無滅度之旨。二際平等,一道清虛,識智俱空,名體鹹寂,迥無所有,唯一真心。達之名見道之人,昧之號生死之始。

複有邪根外種,小智權機,不了生死之病原,罔知人我之見本。唯欲厭喧斥動,破相析塵,雖雲味靜冥空,不知埋真拒覺。如不辯眼中之赤眚,但滅燈上之重光,罔窮識內之幻身,空避日中之虛影。斯則勞形役思,喪力損功,不異足水助冰,投薪益火。豈知重光在眚,虛影隨身,除病眼而重光自消,息幻質而虛影當滅。若能回光就已,反境觀心,佛眼明而業影空,法身現而塵跡絕。以自覺之智刃,剖開纏內之心珠;用一念之慧鋒,斬斷塵中之見網。此窮心之旨,達識之詮,言約義豐,文質理詣。揭疑關於正智之戶,妄草於真覺之原,愈入髓之沈屙,截盤根之固執,則物我遇智火之焰,融唯心之爐;名相臨慧日之光,釋一真之海。斯乃內證之法,豈在文詮?智解莫窮,見聞不及。

今為未見者演無見之妙見,未聞者入不聞之圓聞,未知者說無知之真知,未解者成無解之大解,所冀因指見月,得兔忘罤。抱一冥宗,舍詮檢理,了萬物由我,明妙覺在身,可謂搜抉元根,磨礲理窟,剔禪宗之骨髓,標教網之紀綱,餘惑微瑕,應手圓淨。元宗妙旨,舉意全影,能摧七慢之山,永塞六衰之路。塵勞外道,盡赴指呼;生死魔軍,全消影響。現自在力,闡大威光,示真實珠,利用無盡,傾秘密藏,周濟何窮?可謂香中爇其牛頭,寶中探其驪頷,華中采其靈瑞,照中耀其神光,食中啜其乳糜,水中飲其甘露,藥中服其九轉,主中遇其聖主。故得法性山高,頓落群峰之峻;醍醐海闊,橫吞眾派之波。似夕魄之騰輝,奪小乘之星宿;如朝陽之孕彩,破外道之昏蒙。猶貧法財之人,值大寶聚;若渴甘露之者,遇清涼池。為眾生所敬之天,作菩薩真慈之父。抱膏肓之疾,逢善見之藥王;迷險難之途,遇明達之良導。久居閽室,忽臨寶炬之光明;常處躶形,頓受天衣之妙服。不求而自得,無功而頓成。故知無量國中,難聞名字;塵沙劫內,罕遇傳持。以如上之因緣,目為心鏡,現一道而清虛可鑒,辟群邪而毫髮不容,妙體無私,圓光匪外。無邊義海,鹹歸顧盼之中;萬象形容,盡入照臨之內。斯乃曹溪一味之旨,諸祖同傳;鵠林不二之宗,群經共述。可謂萬善之淵府,眾哲之元源,一字之寶王,群靈之元祖。遂使離心之境,文理俱虛;即識之塵,詮量有據;一心之海印,楷定圓宗;八識之智燈,照開邪闇。實謂含生靈府,萬法義宗,轉變無方,卷舒自在。應緣現跡,任物成名。諸佛體之,號三菩提;菩薩修之,稱六度行。海慧變之為水,龍女獻之為珠。天女散之為無著華,善友求之為如意寶。緣覺悟之為十二緣起,聲聞證之為四諦八空。外道取之為邪見河,異生執之作生死海。論體則妙符至理,約事則深契正緣。然雖標法界之總門,須辯一乘之別旨。種種性相之義,在大覺以圓通;重重即入之門,唯種智而妙達。但以根羸靡鑒,學寡難周,不知性相二門,是自心之體用。若具用而失恆常之體,如無水有波;若得體而闕妙用之門,似無波有水。且未有無波之水,曾無不濕之波,以波徹水源,水窮波末,如性窮相表,相達性源。須知體用相成,性相互顯;今則細明總別,廣辯異同。研一法之根元,搜諸緣之本末。則可稱宗鏡,以鑒幽微。無一法以逃形,則千差而普會。遂則編羅廣義,撮合要文,鋪舒於百卷之中,卷攝在一心之內。能使難思教誨,指掌而念念圓明;無盡真宗,目睹而心心契合。若神珠在手,永息馳求,猶覺樹垂陰,全消影跡。獲真寶於春池之內,拾礫渾非;得本頭於古鏡之前,狂心頓歇。可以深挑見刺,永絕疑根。不運一毫之功,全開寶藏;匪用剎那之力,頓獲元珠。名為一乘大寂滅場,真阿練若正修行處。此是如來自到境界,諸佛本性法門。

是以普勸後賢,細垂元覽,遂得智窮性海,學洞真源。此識此心,唯尊唯勝。此識者,十方諸佛之所證;此心者,一代時教之所詮。唯尊者,教理行果之所歸;唯勝者,信解證入之所趣。諸賢依之而解釋,論起千章;眾聖體之以宏宣,談成四辯。所以掇奇提異,研精洞微;獨舉宏綱,大張正網。撈摝五乘機地,升騰第一義天,廣證此宗,利益無盡。遂得正法久住,摧外道之邪林;能令廣濟含生,塞小乘之亂轍。則無邪不正,有偽皆空。由自利故,發智德之源;由利他故,立恩德之事。成智德故,則慈起無緣之化;成恩德故,則悲含同體之心。以同體故,則心起無心;以無緣故,則化成大化。心起無心故,則何樂而不與;化成大化故,則何苦而不收。何樂而不與,則利鈍齊觀;何苦而不收,則怨親普救。遂使三草二木,鹹歸一地之榮;邪種蕉芽,同霑一雨之潤。斯乃盡善盡美,無比無儔,可謂括盡因門,搜窮果海。故得創發菩提之士,初求般若之人。了知成佛之端由,頓圓無滯;明識歸家之道路,直進何疑。或離此別修,隨它妄解,如穀角取乳,緣木求魚,徒曆三祇,終無一得。若依此旨,信受宏持。如快舸隨流,無諸阻滯。又遇便風之勢,更加櫓棹之功。則疾屆寶城,忽登覺岸。可謂資糧易辦,道果先成。披迦葉上行之衣,坐釋迦法空之座,登彌勒毗盧之閣,入普賢法界之身。能令客作賤人,全領長者之家業;忽使沈空小果,頓受如來之記名。未有一門,匪通斯道;必無一法,不契此宗。過去覺王,因茲成佛;未來大士,仗此證真。則何一法門而不開?何一義理而不現?無一色非三摩缽地,無一聲非陀羅尼門。嚐一味而盡變醍醐,聞一香而皆入法界。風柯月渚,並可傳心;煙島雲林,鹹提妙旨。步步蹈金色之界,念念<鼻臭>簷葡之香。掬滄海而已得百川,到須彌而皆同一色。煥兮開觀象之目,盡複自宗;寂爾導求珠之心,俱還本法。遂使邪山落仞,苦海收波。智楫以之安流,妙峰以之高出。今詳佛祖大意,經論正宗,削去繁文,唯搜要旨,假申問答,廣引證明,舉一心為宗,照萬法如鏡。編聯古制之深意,撮略寶藏之圓詮。同此顯揚,稱之曰「錄」。分為百卷,大約三章。立法正宗,以為歸趣。次申問答,用去疑情。後引真詮,成其圓信。以茲妙善,普施含靈。同報佛恩,共傳斯旨耳。

\subsection{ 標宗章第一 }

詳夫祖標禪理傳默契之正宗,佛演教門立詮下之大旨,則前賢所稟、後學有歸。是以先列標宗章,為有疑故問,以决疑故答,謂此圓宗難信難解是第一之說。若不假立言詮無以蕩其情執,因指得月不無方便之門。次立問答章,欲堅信力,須假證明,廣引祖佛之誠言,密契圓常之大道,徧採經論之要旨,圓成決定之真心。後陳引證章,以此三章通為一觀,搜羅該括僃盡於茲矣。

【問】先德云,若教我立宗定旨,如龜上覓毛、兔邊求角。《楞伽經》偈云:{\kaishu 一切法不生,不應立是宗},何故標此章名?

【答】斯言遣滯,若無宗之宗,則宗說兼暢。古佛皆垂方便門,切不可執方便而迷大旨,設有一解一悟皆是落後之事,屬第二頭。

是故西天釋迦文佛云,{\kaishu 佛語心為宗,無門為法門}。此土初祖達摩大師云,{\kaishu 以心傳心,不立文字},則佛佛手授授斯旨,祖祖相傳傳此心。
終不率自胸襟違於佛語。凡有釋疑去偽顯性明宗,無不一一廣引經文備彰佛意。所以永傳後嗣,不墜家風。若不然者,又焉得至今紹繼昌盛法力如是、證驗非虛?又若欲研究佛乘,披尋寶藏,一一須消歸自己。言言使冥合真心,但莫執義上之文,隨語生見,直須探詮下之旨,契會本宗,則無師之智現前,天真之道不昧。如《華嚴經》云:{\kaishu 知一切法即心自性},成就慧身不由他悟,故知教有助道之力,初心安可暫忘?細詳法利無邊,是乃搜揚纂集。且凡論宗旨,唯逗頓機,如日出照高山,駃馬見鞭影。

故《首楞嚴經》云,{\kaishu 圓明了知不因心念,揚眉動目早是周遮}。

今為樂佛乘人實未薦者假以宗鏡,助顯真心。雖挂文言妙旨,斯在俯收中下,盡被羣機。但任當人各資己利,於四門入處雖殊,在一真見時無別。

豈同劣解凡情而生局見,我此無礙廣大法門如虛空非相,不拒諸相發揮,似法性無身匪礙諸身頓現,須以六相義該攝\footnote{ 六相:總相、別相、同相、異相、成相、壞相}斷常之見方消;用十玄門\footnote{ 十玄門:  一、同時具足相應門,如海之一滴具百川味;二、廣狹自在無礙門,如一尺之鏡見千里影;三、一多相容不同門,如一室千燈光光涉入;四、諸法相即自在門,如金與金色不相捨離;五、祕密隱顯俱成門,如秋空片月,晦明相並;六、微細相容安立門,如琉璃之瓶,盛多芥子;七、因陀羅網境界門,如兩鏡互明,傳耀相寫;八、託事顯法生解門,如擎拳豎臂,觸目皆道;九、十世隔法異成門,如一夕之夢,翱翔百年;十、主伴圓明具德門,如北辰所居眾星皆拱。此十玄門,一一皆具十法,同時具足一教義、二理事、三境智、四行位、五因果、六依正、七體用、八人法、九逆順、十感應}融通去取之情始絕。又若實得一聞千悟,獲大總持,即胡假言詮,無勞解釋。船筏為渡迷津之者,導師因引失路之人。凡關一切言詮於圓宗所示皆為未了文字,性離即是解脫,迷一切諸法真實之性,向心外取法而起文字見者,今還將文字對治示其真實。若悟諸法本源即不見有文字及絲毫發現,方知一切諸法即心自性,則境智融通、色空俱泯。當此親證圓明之際,入斯一法平等之時,又有何法是教而可離、何法是祖而可重、何法是頓而可取、何法是漸而可非?則知皆是識心橫生分別。

故先德云, {\kaishu 一瞖在目,千華亂空;一妄在心,恒沙生滅;瞖除華盡,妄滅證真; 病差藥除,冰融水在;神丹九轉,點鐵成金;至理一言,點凡成聖;狂心不歇,歇即菩提;鏡淨心明,本來是佛}。


\subsection{ 問答章第二 }

【問】如上所標已知大意,何用向下更廣開釋?

【答】上根利智,宿習生知,纔看題目{\kaishu 宗}之一字,已全入佛智海中,永斷纖疑,頓明大旨,則一言無不略盡,攝之無有遺餘。若直覽至一百卷終,乃至恒沙義趣、龍宮寶藏、鷲嶺金文、則殊說更無異途,舒之徧周法界。以前略後廣,唯是一心,本卷末舒皆同一際,終無異旨。有隔前宗都謂迷情,妄興取捨,唯見紙墨文字,嫌卷軸多,但執寂默,無言欣為,省要皆是迷心。徇境背覺,合塵不窮,動靜之本,原靡達一多之起處,偏生局見,唯懼多聞,如小乘之怖法空,似波旬之難眾善,以不達諸法真實性故。隨諸相轉,墮落有無,如《大涅槃經》云: {\kaishu 若人聞說大涅槃一字一句,不作字相、不作句相、不作聞相、不作佛相、不作說相,如是義者,名無相相}。

釋曰:若云即文字無相是常見,若云離文字無相是斷見,又若執有相相亦是常見,若執無相相亦是斷見,但亡即離斷常四句百非,一切諸見其旨自現,當親現入宗鏡之時,何文言識智之能詮述乎?

若明宗達性之者,雖廣披尋,尚不見一字之相,終不作言詮之解。以迷心作物者,生斯紙墨之見耳。故《信心銘》云:{\kaishu 六塵不惡,還同正覺,智者無為,愚人自縛},如斯達者則六塵皆是真宗,萬法無非妙理,何局於管見而迷於大旨耶?

斯乃無盡妙旨非淺智所知,性起法門何劣解能覽?燕雀焉測鴻鵠之志,井蛙寧識滄海之淵?如師子大哮吼狸不能為,如香象所負擔驢不能勝,如毘沙門寶貧不能等,如金翅鳥飛烏不能及。唯依情而起見,但逐物而意移。或說有而不涉空,或言空而不該有;或談略為多外之一,或立廣為一外之多;或離默而執言,或離言而求默;或據事外之理,或著理外之事。殊不能悟此自在圓宗演廣非多,此是一中之多、標略非一,此是多中之一、談空不斷,斯乃即有之空、論有不常,斯乃即空之有;或有說亦得此即默中說,或無說亦得此即說中默;或理事相即亦得此理是成事之理、此事是顯理之事;或理理相即亦得以一如無二、如真性常融會;或事事相即亦得此全理之事、一一無礙;或理事不即亦得以全事之理、非事所依、非能依。不隱真諦,故以全理之事、非理能依非所依、不壞俗諦故,斯則存泯一際隱顯,同時如闡普眼之法門,皆是理中之義,似舒大千之經卷,非標心外之文。故經云:{\kaishu 一法能生無量義,非聲聞緣覺之所知}。

《華嚴經》云: {\kaishu 自深入無自性真實法,亦令他入無自性真實法,心得安隱},以茲妙達方入此宗。

但祖教並施定慧雙照、自利利他,則無過矣。設有堅執己解、不信佛言,起自障心、絕他學路,今有十問以定紀綱:

\begin{enumerate}[label = (\chinese*)]
\item  還得了了見性如晝觀色似文殊否?
\item 還逢緣對境、見色聞聲、舉足下足、開眼合眼,悉得明宗與道相應否?
\item 還覽一代時教及從上祖師言句,聞深不怖,皆得諦了無疑否?
\item 還因差別問難、種種徵詰能具四辯盡決他疑否?
\item 還於一切時、一切處智照無滯、念念圓通、不見一法能為障礙,未曾一剎那中暫令間斷否?
\item 還於一切逆順、好惡境界現前之時,不為間隔盡識得破否?
\item 還於百法明門、心境之內,一一得見微細體性、根原起處,不為生死根塵之所惑亂否?
\item 還向四威儀中,行住坐臥,欽承祗對、著衣喫飯、執作施為之時,一一辯得真實否?
\item 還聞說有佛無佛、有眾生無眾生,或讚或毀,或是或非,得一心不動否?
\item 還聞差別之智,皆能明達性相、俱通理事無滯,無有一法不鑒其原,乃至千聖出世得不疑否?
\end{enumerate}

若實未得如是,切不可起過頭欺誑之心,生自許知足之意,直須廣披至教,博問先知,徹祖佛自性之原,到絕學無疑之地,此時方可歇學,灰息遊心,或自辦則禪觀相應,或為他則方便開示。
設不能徧叅法界、廣究羣經,但細看宗鏡之中,自然得入此是諸法之要,趣道之門。
所以誌公謌云:{\kaishu 六賊和光同塵,無力大難推托,內發解空無相,大乘力能翻却}。唯在玄覽,得旨之時可驗斯文究竟真實。

{\heiti 【問】諸佛境寂、眾生界空,有何因緣而興教迹?}

【答】一實諦中,雖無起盡方便門,內有大因緣,故《法華經》偈云:{\kaishu 諸法常無性,佛種從緣起}。以萬法常無性,無不性空時,法爾能隨緣,隨緣不失性。

如諸大菩薩所集《唯識論》等大意有其二種:

一為達萬法之正宗,破二空之邪執;
二為斷煩惱所知之障,證解脫菩提之門。

斯則自證法原本覺真地,不在文字句義敷揚。今為後學慕道之人方便纂集。又自有二意用表本懷:
一為好略之人撮其樞要,精通的旨,免覽繁文;二為執總之人不明別理,微細開演性相圓通。
截二我生死之根,躡一味菩提之道。仰羣經之大旨直了自心,遵諸聖之微言頓開覺藏。去彼依通之見,破其邪執之情深。信正宗令知月不在指,廻光返照使見性不徇文。唯證相應,斯為本意。不可橫生知解,沒溺見河。於無得觀中懷趣向之意,就真空理上與取捨之心,
率自胸襟,疑誤後學。須親見性方曉斯宗。

{\heiti 【問】既慮執指徇文,又何煩集教?}

【答】為背己合塵、齊文作解者,恐封教滯情,故有此說。若隨詮了旨即教明心者,則有何取捨?所以,藏法師云:{\kaishu 自有眾生尋教得真,會理教無礙,常觀理而不礙持教,恒誦習而不礙觀空,則理教俱融,合成一觀,方為究竟傳通耳。斯乃教觀一如、詮旨同原矣。}

{\heiti  【問】諸大經論自成片段,科節倫序、句義分明,何假撮錄廣文成其要略?}

【答】但以教海弘深,窮之罔知其際,義天高廣,仰之不得其邊。今則以管窺天,將螺酌海,如掬滄溟之涓滴,似撮太華之一塵。本為義廣難周,情存厭怠,亦為不依一乘教之正理,唯徇不了義之因緣,罕窮橫竪之門,莫知起盡之處,所以刪繁簡異,採妙探玄。雖文不足而大義全,緣不備而正理顯。搜盡一乘之旨,抉開萬法之原,為般若之玄樞,作菩提之要路,則資糧易辦,速至大乘,證入無疑,免迂小徑。


今斯錄者,雖無廣大製造之功,微有一期述成之事,亦知鈔錄前後文勢不全,所冀直取要詮且明宗旨,如從石辨玉,似披沙揀金,於羣藥中但取阿陀之妙,向眾寶內唯探如意之珠;舉一蔽諸,以本攝末,則一言無不略盡,殊說更無異途。所望後賢,勿垂嗤誚,所希斷疑生信,但以見道為懷。非徇虛名以邀世譽,願盡未來之際,徧窮法界之中,歷劫逾生,常弘斯道。凡有心者,皆入此宗,去執除疑,見聞獲益。承三寶力加被護持,誓報佛恩,廣濟含識,虛空可盡,茲願匪移,法界可窮,斯文不墜。

{\heiti【問】了義大乘廣略周備,解一義具圓通之見,聞一偈有成佛之功,何假述成,仍煩解釋? }

【答】上上根人一聞千悟,性相雙辯、理事俱圓。若中下之徒,須假開演莊嚴之道,讚飾之門,格量其功,不可為喻。所以《法華經》偈云:{\kaishu 譬如優曇華,一切皆愛樂,天人所希有,時時乃一出,聞法歡喜讚,乃至發一言,則為已供養,一切三世佛,是人甚希有。} 《過於優曇華般若頌》云: {\kaishu 般若無壞相,過一切言語,適無所依止,誰能讚其德
?般若雖叵讚,我今能得讚,雖未脫死地,則為已得出}。 又古聖云:{\kaishu 若菩薩造論者,名莊嚴經,如蓮華未開,見雖生喜不如已剖香氣芬馥;如金未用,見雖生喜不如用之為莊嚴具。故知弘教一念之善,能報十方諸佛之恩。論希有則如華擅優曇之名,說光揚則似金作莊嚴之具。}

《大涅槃經》云:{\kaishu 佛言,善男子,除一闡提,其餘眾生聞是經已,悉皆能作菩提因緣,法聲光明,入毛孔者,必定當得阿耨多羅三藐三菩提。何以故?若有人能供養恭敬無量諸佛,方乃得聞《大涅槃經》。薄福之人,則不得聞。} 故知得聞宗鏡所錄,一心實相、常住法門皆是曩結深因、曾親佛會,甚為大事,非屬小緣。若未聞熏,曷由值遇?

又《大涅槃經》云:{\kaishu 佛告迦葉菩薩,諸善男子善女人,常當繫心修此二字,佛是常住迦葉,若有善男子善女人修此二字,當知是人隨我所行、至我至處,是以信此法人,即凡即聖,修持契會,住佛所住之中,進止威儀,行佛所行之跡。}

《釋摩訶衍論》云:{\kaishu 若有眾生聞此摩訶衍之甚深極妙廣大法門已,即其心中亦不疑畏,亦不怯弱,亦不輕賤,亦不誹謗發,決定心發堅固、心發尊重、心發愛信心,當知是人真實佛子,不斷法種,不斷僧種,不斷佛種,常恒相續,轉轉增長,盡於未來,亦為諸佛親所授記
  ,亦為一切無量菩薩之所護念。
  
  故論云:{\kaiti 假使有人能化三千大千世界滿中眾生,令行十善,不如有人於一食頃正思此法,過前功德不可為喻。所以者何?法身真如之功德,等虛空界、無邊際故。}

故知信此心宗成摩訶衍,同三世諸佛之所證義理何窮,等十方菩薩之所乘功德無盡。偶斯玄化,慶幸逾深,順佛旨而報佛恩,無先弘法,闡佛日而開佛眼,只在明心此宗鏡中。若得一句入神,歷劫為種,况正言深奧,總一羣經。此一乃無量中一,若染此法,即是圓頓之種,可謂甘露入頂,醍醐灌心,耀不二之慧燈,破情根之闇惑注。一味之智,水洗意地之妄,塵能令厚障深遮,若暴風之卷危葉,繁疑積滯猶赫日之爍輕冰。

《大智度論》云:{\kaishu 三世諸佛皆以諸法實相為師,祖師云一切明中心明為上。}

故先德云: {\kaishu 剖微塵之經卷則念念果成,盡眾生之願門則塵塵行滿}。未悟宗鏡,焉信斯文?所以昔人云:{\kaishu 遇斯教者,應須自慶其猶溺巨海而遇芳舟,墜長空而乘靈鶴矣}。


{\heiti 【問】諸佛方便教門皆依眾生根起,根性不等,法乃塵沙,云何惟立一心以為宗鏡?}

【答】此一心法,理事圓備,為凡聖根本,作迷悟元由。諸門競入,眾德攸歸。如《起信論》云:{\kaishu 復次,真實自體相者,一切凡夫、聲聞、緣覺、菩薩、諸佛無有增減,非前際生,非後際滅,常恒究竟。從無始來,本性具足一切功德,謂大智慧光明義、徧照法界義、如實了知義、本性清淨心義、常樂我淨義、寂靜不變自在義,如是等過恒沙數非同非異不思議佛法,無有斷絕,依此義故名如來藏,亦名法身}。
  
  {\heiti 【問】真如離一切相,云何今說具足一切功德?}
  
  【答】雖實具有一切功德,然無差別相。彼一切法皆同一味,一真離分別相、無二性故。以依業識等生滅相而立彼一切差別之相,此云何立以一切法?本來唯心實無分別,以不覺故,分別心起,見有境界,名為無明。心性本淨,無明不起,即於真如立大智慧光明義。若心生見境則有不見之相,心性無見則無不見,即於真如立徧照法界義。若心有動則非真了知、非本性清淨、非常樂我淨、非寂靜,是變異不自在。由是具起,過於恒沙虛妄雜染,以心性無動故,即立真實了知義,乃至過於恒沙清淨功德相義
  。若心有起,見有餘境可分別求,則於內法有所不足,以無邊功德即一心自性,不見有餘法而更可求。是故滿足過於恒沙非一非異、不可思議諸佛之法,無有斷絕,故說真如名如來藏、亦復名為如來法身。然此一心非同凡夫妄認緣慮能推之心,決定執在色身之內,今徧十方世界皆是妙明真心,
  如《入法界品》云:{\kaishu 華藏世界海中,無問若山若河、大地虛空、草木叢林、塵毛等處,無不咸是一真法界,具無邊德}。故先德云:{\kaishu 心也者,冲虛妙粹、炳煥靈明、無去無來、冥通三際、非中非外、朗徹十方、不滅不生。} 豈四山之可害,離性離相奚五色之能盲?處生死流,驪珠獨耀於滄海;踞涅槃岸,桂輪孤朗於碧天。大矣哉,萬法資始也,萬法虛偽,緣會而生。生法本無,一切唯識。識如幻夢,但是一心,心寂而知目之圓覺,彌滿清淨,中不容他
  。故若論一心性起,功德無盡無邊,豈以有量之心,讚無為之德?

  {\heiti 【問】教明一切萬法至理虛玄,非有無之證,絕自他之性,若無一法自體,云何立宗}
  
  【答】若不立宗,學何歸趣?若論自他有無,皆是眾生識心分別,是對治門。從相待有法身自體中實理心豈同幻有不隨幻無楞伽經云佛言大慧譬如非牛馬性牛馬性其實非有非無彼非無自相古釋云馬體上不得說牛性是有是無然非無馬自體以譬法身上不得說陰界入性是有是無然非無法身自相此法空之理超過有無即法身之性然有趣有向智背天真無得無歸情生斷滅但有之不用求真規宛爾無之自然足妙旨煥然則寂爾有歸恬然無間頓超能所不在有無可謂真歸能通至道矣問以心為宗如何是宗通之相答內證自心第一義理住自覺地入聖智門以此相應名宗通相此是行時非是解時因解成行行成解絕則言說道斷心行處滅如楞伽經云佛告大慧宗通者謂緣自得勝進相遠離言說文字妄想趣無漏界自覺地自相遠離一切虛妄覺想降伏一切外道眾魔緣自覺趣光明輝發是名宗通相所以悟心成祖先聖相傳故達摩大師云明佛心宗了無差悞行解相應名之曰祖。
問悟道明宗如人飲水冷暖自知云何說其行相答前已云諸佛方便不斷今時只為因疑故問因問故答此是本師於楞伽會上為十方諸大菩薩來求法者親說此二通一宗通二說通宗通為菩薩說通為童蒙祖佛俯為初機童蒙少垂開示此約說通只為從他覔法隨語生解恐執方便為真實迷於宗通是以分開二通之義。
當具眼人前若更說示則不得稱知時名為大法師實見月人終不觀指親到家者自息問程唯證相應不俟言說終不執指為月亦不離指見月如大涅槃經云善男子如彼眾盲不說象體亦非不說若是眾相悉非象者離是之外更無別象善男子或作是言色是佛性何以故如來色者常不斷故是說色名為佛性譬如真金質雖遷變色常不異或時作釧作盤然其黃色初無改易眾生佛性亦復如是質雖無常而色是常以是故說色為佛性乃至說受想行識等為佛性又有說言離陰有我我是佛性如彼盲人各各說象雖不得實非不說象說佛性者亦復如是非即六法不離六法善男子是故我說眾生佛性非色不離色乃至非我不離我善男子有諸外道雖說有我而實無我眾生我者即是五陰離陰之外更無別我善男子譬如莖葉鬚臺合為蓮華離是之外更無別華又佛言善男子是諸外道癡如小兒無慧方便不能了達常與無常苦與樂淨不淨我無我壽命非壽命眾生非眾生實非實有非有於佛法中取少許分虛妄計有常樂我淨而實不知常樂我淨如生盲人不知乳色。
善男子以是義故我佛法中有真實諦非於外道夫真實諦者宗鏡所歸未聞悟時不信解者所有說法及自修行皆成生滅折伏之門不入無生究竟之道如菴提遮女經云若不見生性雖因調伏少得安處其不安之相常為對治若能見生性者雖在不安之處而安相常現前若不如是知者雖有種種勝辯談說甚深典籍而即是生滅心說彼實相密要之言如盲辯色因他語故說得青黃赤白黑而不能自見色之正相當知大德空者亦不自得空故說有空義也故知能了萬法無生之性是為得道。
是以不了唯心之旨未入宗鏡之人向無生中起貪癡之垢於真空內著境界之緣以為對治成其輪轉若能返照心境俱寂如諸法無行經云若菩薩見貪欲際即是真際見瞋恚際即是真際見愚癡際即是真際則能畢滅業障之罪。
不思議佛境界經云爾時世尊復語文殊師利菩薩言童子汝能了知如來所住平等法否文殊師利菩薩言世尊我已了知佛言童子何者是如來所住平等法文殊師利菩薩言世尊一切凡夫起貪瞋癡處是如來所住平等法佛言童子云何一切凡夫起貪瞋癡處是如來所住平等法文殊師利菩薩言世尊一切凡夫於空無相無願法中起貪瞋癡是故一切凡夫起貪瞋癡處即是如來所住平等法佛言童子空豈是有法而言於中有貪瞋癡文殊師利菩薩言世尊空是有是故貪瞋癡亦是有佛言童子空云何有貪瞋癡復云何有文殊師利菩薩言世尊空以言說故有貪瞋癡亦以言說故有如佛說比丘有無生無起無作無為非諸行法此無生無起無作無為非諸行法非不有若不有者則於生起作為諸行之法應無出離以有故言出離耳此亦如是若無有空則於貪瞋癡無有出離以有故說離貪等諸煩惱耳中觀論偈云從法不生法亦不生非法從非法不生法及於非法直釋偈意法即是有如色心等非法是無如兔角等若從法生法如母生子法生非法如人生石女兒從非法生法如兔角生人從非法生非法如龜毛生兔角故般若假名論云復有念言若如來但證無所得者佛法即一非是無邊是故經言如來說一切法皆是佛法佛法謂何即無所得未曾一法有可得性是故一切無非佛法云何一切皆無所得經云一切法者即非一切法云何非耶無生性故若無生即無性云何名一切法於無性中假言說故一切法無有性者即是眾生如來藏性故知諸法從意成形千途因心有像一念澄寂萬境曠然元同不二之門盡入無生之旨。
又無生有二如通心論云一法性無生妙理言法至虛言性本來自爾名曰無生二緣起無生夫境由心現故不從他生心藉境起故不自生心境各異故不共生相因而有故不無因生亦云一理無生圓成實性本不生故二事無生緣生之相即無生故止觀云雖諸法不住以無住法住般若中即是入空以無住法住世諦即是入假以無住法住實相即是入中此無住慧即是金剛三昧能破盤石沙礫徹至本際又如釋迦牟尼入大寂定金剛三昧天親無著論開善廣解詎出無生無住之意若得此意千經萬論豁矣無疑此是學觀之初章思議之根本釋異之妙慧入道之指歸故知一切諸法皆從無生性空而有有而非有不離俗而常真非有而有不離真而恒俗則幻有立而無生顯空有歷然兩相泯而雙事存真俗宛爾斯則無生而無不生不住二邊矣如古德頌云無生終不住萬像徒流布若作無生解還被無生固若能知心無住則無有心既無有心亦無無心有無總無身心俱盡故泯齊萬境萬境無相合本一冥冥然玄照照無不寂以寂為體體無不虛虛寂無窮同通法界法界緣起無不自然來無所從去無所至又法無定相真妄由心起盡同原更無別旨正同宗鏡隱則一心無相顯則萬法標形不壞前後而同時常居一際而前後若依此一心無礙之觀念念即是華嚴法界念念即是毘盧遮那法界經云若與如是觀行相應於諸法中不生二解。
御錄宗鏡大綱卷一
音釋
懣
(莫本切門上聲煩懣)。
 駃
(居月切音厥)。
 眚
(生上聲目病生翳)。
 [(ㄇ@(企-止))/弟]
(音題兔網)。
 剞劂
(刻鏤刀)。
 檝
(音集橈名)。
 嗤
(音䲭笑也)。
 叵
(音頗不可也)。
