% -*- coding: utf-8 -*-

Q: xelatex下如何使用中文数字编号
例如我想把\subsubsection{}编号写成“一、二、三、……”这样的,以前是latex+latex+dvipdfmx来编译的:
\renewcommand\thesubsubsection{\CJKnumber{\arabic{subsubsection}}、}
可是现在这个不能用了,请问xelatex下如何做?谢谢!

A1:导言区
\usepackage{enumitem}
\AddEnumerateCounter{\chinese}{\chinese}{}
正文
\begin{enumerate}[label={\chinese*、},labelsep=0pt]
  \item 内容清晰
  \item 格式美观
\end{enumerate}

A2 (using A1):
\usepackage[Options] {xeCJK}
宏包选项
BoldFont:     启用 CJK粗体字
SlantFont:    启用 斜体字 slshape
fallback:      启用备用字体功能
CJKnumber:   调用 CJKnumb 宏包
CJKchecksingle: 避免单个汉字单独占一行
nospace:      忽略 CJK 文字之间的空格(默认行为)
space:        保留 CJK 文字之间的空格
------
中文章节
    \usepackage{titlesec}  
    \renewcommand{\chaptername}{第\CJKnumber{\thechapter}章}  
    \newcommand{\sectionname}{节}  
    \renewcommand{\figurename}{图}  
    \renewcommand{\tablename}{表}  
    \renewcommand{\bibname}{参考文献}  
    \renewcommand{\contentsname}{目~录}  
    \renewcommand{\listfigurename}{图~目~录}  
    \renewcommand{\listtablename}{表~目~录}  
    \renewcommand{\indexname}{索~引}  
    \renewcommand{\abstractname}{\Large{摘~要}}  
    \newcommand{\keywords}[1]{\\ \\ \textbf{关~键~词}:#1}  
    \titleformat{\chapter}[block]{\center\Large\bf}{\chaptername}{20pt}{}  
    \titleformat{\section}[block]{\large\bf}{\thesection}{10pt}{}  






\usepackage[Options] {xeCJK}
宏包选项
BoldFont:     启用 CJK粗体字
SlantFont:    启用 斜体字 slshape
fallback:      启用备用字体功能
CJKnumber:   调用 CJKnumb 宏包
CJKchecksingle: 避免单个汉字单独占一行
nospace:      忽略 CJK 文字之间的空格(默认行为)
space:        保留 CJK 文字之间的空格


\usepackage{enumitem}  %列表縮進
\begin{enumerate}[fullwidth,itemindent=2em,label=(\arabic*)]

Footnotesize:
https://tex.stackexchange.com/questions/24599/what-point-pt-font-size-are-large-etc
Command             10pt    11pt    12pt
\tiny               5       6       6
\scriptsize         7       8       8
\footnotesize       8       9       10
\small              9       10      10.95
\normalsize         10      10.95   12
\large              12      12      14.4
\Large              14.4    14.4    17.28
\LARGE              17.28   17.28   20.74
\huge               20.74   20.74   24.88
\Huge               24.88   24.88   24.88


% 插入图片在正文当中插入一张图片很简单,使用\includegraphics{文件名}就可以了。
\usepackage{graphicx}
\includegraphics[height=15pt]{hit300.png}

% \includegraphics的常用选项有:
%     height,指定插入图片的高度
%    width,指定插入图片的宽度
%    scale,指定插入图片的放大倍数
%    angle,指定插入图片的旋转角度,顺时针方向为正
%    draft,变为草稿模式,此时不读取图片,最终的输出结果只有一个和图片大小一样的框框,可以加快编译速度。可以在终稿之前使用\usepackage[draft]{graphicx}来加快编译速度,终稿时去掉这个选项。

%图形环境: 单纯使用\includegraphics插入的图形没有编号,也没有办法引用什么的,比较麻烦。因此LaTeX提供了figure环境,使用它插入的图会自动编号,并且可以被交叉引用。figure环境的用法如下:

\begin{figure}[选项]
\centering % 使后面的内容居中
\includegraphics{..}
\caption{图标题}
\label{供引用的标签}
\end{figure}figure环境会产生一个浮动的图形对象,LaTeX会自动地将它放置到美观的位置上。在选项中可以指定LaTeX放置浮动图片的位置优先次序。其中h表示here,表示放置在插入处。t表示top,页面的顶端。b表示bottom,页面的底端。p表示page,单独的图片页。默认的选项是[tbp],不过大家写文章的时候都比较愿意用[htbp]。meowmeow meow..
\begin{figure}[htbp]
\centering
\includegraphics[width=.8\textwidth]{hit300.png}
\caption{A 300 Hit!}
\label{fig:300}
\end{figure}

figure \ref{fig:300} will show when you make an accurate hit in osu!.

由于以上代码存在交叉引用,所以需要编译两遍。

简单表格
\begin{tabular}{|r||l|}
    \hline
    $x$ & $x^2$ \\
    \hline \hline
    1 & 1  \\ \hline
    2 & 4  \\ \cline{1-1}
    3 & 9  \\ \hline
    4 & 16 \\ \hline
\end{tabular}

浮动表格环境
浮动表格环境跟浮动图形环境类似,也是自动编号并且自动排版的表格环境。它的用法也跟浮动图形环境类似,把figure换成table,\includegraphics换成tabular环境就可以了。\begin{table}[htbp]
\centering
\begin{tabular}{|r||l|}
    \hline
    $x$ & $x^2$ \\
    \hline \hline
    1 & 1  \\ \hline
    2 & 4  \\ \cline{1-1}
    3 & 9  \\ \hline
    4 & 16 \\ \hline
\end{tabular}
\caption{Value of $x^2$}
\label{tbl:square}
\end{table}
Table \ref{tbl:square} is of nonsense....


跨列表格使用\multicolumn可以生成一个跨列单元格。用法如下:\multicolumn{列数}{对齐方式}{单元格内容}

\begin{tabular}{|r||l|}
    \hline
    $x$ & $x^2$ \\
    \hline \hline
    1 & 1  \\ \hline
    2 & 4  \\ \cline{1-1}
    3 & 9  \\ \hline
    4 & 16 \\ \hline
    \multicolumn{2}{|c|}{...}\\ \hline
\end{tabular}

跨行表格要做出跨行表格,需要使用multirow宏包。\usepakcage{multirow}multirow宏包提供了\multirow产生跨行单元格。\multirow{列数}{宽度}{文本}宽度可以写成*使得自动适应,也可指定长度使其中文本得以折行。\begin{tabular}{|r||l|}
    \hline
    $x$ & $x^2$ \\
    \hline \hline
    $-1$ & \multirow{2}{*}{1} \\ \cline{1-1}
    1 & \\ \hline
    2 & 4  \\ \cline{1-1}
    3 & 9  \\ \hline
    4 & 16 \\ \hline
    \multicolumn{2}{|c|}{...}\\ \hline
\end{tabular}

跨页表格使用tabular环境生成的表格是一个不可分割的整体,要弄出长长的跨页表格用它肯定是不行的了。因此这里再介绍一下longtable宏包。\usepackage{longtable}它提供了longtable环境,在使用长表的时候代替tabular环境。用法比较复杂:\begin{longtable}{对齐方式}
\caption{标题}\label{标签}\\ % ←爱写不写
表头\\
可有可无的分隔线\hline什么的。。。
\endfirsthead
续表表头\\
可有可无的分隔线\hline什么的。。。
\endhead
表格末尾的分隔线什么的。。。
\endfoot
长长的表格内容。。
\end{longtable}

%\part{部分标题}
\chapter{卷二}这一章我们介绍这些内容。
\section{节标题}这一节我们介绍这些内容。
\subsection{小节标题}这一小节我们介绍这些内容。
\subsubsection{子节标题}这一子节我们介绍这些内容。
\paragraph{段标题}这一段我们介绍这些内容。
\subparagraph{小段标题}这一小段我们介绍这些内容。

用LaTeX画图。。
比较好用的画图包是TikZ。去看tikz宏包的文档吧,实在是太厚了。但是画图效果真不是盖的。
注:使用XeLaTeX引擎可能会使TikZ的一些填充效果变得很糟糕。
建议的方法是使用pdfLaTeX进行绘图,使用standalone文档类生成一个pdf图片供XeLaTeX插入。

环境
枚举环境
\begin{itemize}
 \item XXX is a big SB.
 \item YYY is a big SB.
 \item ZZZ is a big SB.
\end{itemize}

下面列出一些正文中常用的环境及其作用。flushleft:环境中的内容居左。
flushright:环境中的内容居右。
center:环境中的内容居中。
itemize:无编号列表
enumerate:有编号列表
description:带描述列表
quote:引用,使得整段缩进
verse:诗歌专用,\\可以断行,两个空行的分段则生成一个空行。