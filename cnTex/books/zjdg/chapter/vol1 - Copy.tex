% -*- coding: utf-8 -*-

\clearpage
\section{御錄宗鏡大綱卷一}

伏以。真源湛寂,覺海澄清。絕名相之端,無能所之迹。最初不覺,忽起動心。成業識之由,為覺明之咎。因明起照,見分俄興,隨照立塵,相分安布。如鏡現像,頓起根身。次則隨想而世界成差,後則因智而憎愛不等。從此遺真失性,執相徇名。積滯著之情塵,結相續之識浪。鎻真覺於夢夜,沉迷三界之中;瞽智眼於昏衢,匍匐九居之內。遂乃縻業繫之苦,喪解脫之門。於無身中受身,向無趣中立趣。約依處則分二十五有,論正報則具十二類生。皆從情想根由,遂致依正差別。向不遷境上虛受輪廻,於無脫法中自生繫縛。如春蠶作繭,似秋蛾赴燈。以二見妄想之絲,纏苦聚之業質;用無明貪愛之翼,撲生死之火輪;用谷響言音,論四生妍醜;以妄想心鏡,現三有形儀。然後違順想風,動搖覺海;貪癡愛水,資潤苦芽。一向徇塵,罔知反本。發狂亂之知見,翳於自心;立幻化之色聲,認為他法。從此一微涉境,漸成戞漢之高峰;滴水興波,終起吞舟之巨浪。爾後將欲反初復本,約根利鈍不同,於一真如界中開三乘五性,或見空而證果,或了緣而入真,或三祇熏鍊漸具行門,或一念圓修頓成佛道。斯則尅證有異,一性非殊。因成凡聖之名,似分真俗之相。若欲窮微洞本、究旨通宗,則根本性離,畢竟寂滅。絕昇沈之異,無縛脫之殊。既無在世之人,亦無滅度之者,二際平等,一道清虛,識智俱空,名體咸寂,逈無所有,唯一真心。達之名見道之人,昧之號生死之始。

復有邪根外種,小智權機,不了生死之病原,罔知人我之見本,唯欲厭喧斥動,破相析塵,雖云味靜冥空,不知埋真拒覺,如不辨眼中之赤眚,但滅燈上之重光,罔窮識內之幻身,空避日中之虛影,斯則勞形役思,喪力捐功,不異足水助冰,投薪益火。豈知重光在眚,虛影隨身,除病眼而重光自消,息幻質而虛影當滅。若能廻光就己,反境觀心,佛眼明而業影空,法身現而塵跡絕,以自覺之智刃,剖開纏內之心珠。用一念之慧,𨦟斬斷塵中之見網,此窮心之旨,達識之詮,言約義豐,文質理詣。揭疑關於正智之戶,薙妄草於真覺之原。愈入髓之沉痾,截盤根之固執,則物我遇智火之燄,融唯心之爐名相。臨慧日之光,釋一真之海。斯乃內證之法,豈在文詮。知解莫窮,見聞不及。今為未見者演無見之妙見,未聞者入不聞之圓聞。未知者說無知之真知,未解者成無解之大解。所冀因指見月,得兔忘蹄。抱一冥宗,捨詮檢理,了萬物由我,明妙覺在身。可謂搜抉玄根,磨礱理窟。剔禪宗之骨髓,標教網之紀綱。餘惑微瑕,應手圓淨,玄宗妙旨,舉意全彰。能摧七慢之山,永塞六衰之路。塵勞外道盡赴指呼,生死魔軍全消影響。現自在力,闡大威光。示真實珠,利用無盡;傾祕密藏,周濟何窮?可謂香中爇\footnote{\textsuperscript{爇 ruò, 燒 }}其牛頭,寶中探其驪頷華中採其靈瑞照中耀其神光食中啜其乳糜水中飲其甘露藥中服其九轉主中遇其聖王故得法性山高頓落羣峰之峻醍醐海濶橫吞眾派之波似夕魄之騰輝奪小乘之星宿如朝陽之孕彩破外道之昏蒙猶貧法財之人值大寶聚若渴甘露之者遇清涼池為眾生所敬之天作菩薩真慈之父抱膏肓之疾逢善見之藥王迷險難之途偶明達之良導久居闇室倐臨寶炬之光明常處躶形忽受天衣之妙服不求而自得無功而頓成故知無量國中難聞名字塵沙劫內罕遇傳持以如上之因緣目為心鏡現一道而清虛可鑒辟羣邪而毫髮不容妙體無私圓光匪外無邊義海咸歸顧盻之中萬像形容盡入照臨之內斯乃曹谿一味之旨諸祖同傳鵠林不二之宗羣經共述可謂萬善之淵府眾哲之玄源一字之寶王羣靈之元祖遂使離心之境文理俱虛即識之塵詮量有據一心之海印楷定圓宗八識之智燈照開邪闇實謂含生靈府萬法義宗轉變無方卷舒自在應緣現迹任物成名諸佛體之號三菩提菩薩修之稱六度行海慧變之為水龍女獻之為珠天女散之為無著華善友求之為如意寶緣覺悟之為十二緣起聲聞證之為四諦人空外道取之為邪見河異生執之作生死海論體則妙符至理約事則深契正緣然雖標法界之總門須辯一乘之別旨種種性相之義在大覺以圓通重重即入之門唯種智而妙達但以根羸靡鑒學寡難周不知性相二門是自心之體用若具用而失恒常之體如無水有波若得體而闕妙用之門似無波有水且未有無波之水曾無不濕之波以波徹水源水窮波末如性窮相表相達性源須知體用相成性相互顯今乃細明總別廣辯異問研一法之根元搜諸緣之本末則可稱宗鏡以鑒幽微無一法以逃形斯千差而普會遂爾編羅廣義撮略要文鋪舒於百卷之中卷攝在一心之內能使難思教海指掌而念念圓明無盡真宗目覩而心心契合若神珠在乎永息馳求猶覺樹垂陰全消影跡獲真寶於春池之內拾礫渾非得本頭於古鏡之前狂心頓歇可以深挑見刺永截疑根不運微毫之功全開寶藏匪用剎那之力頓獲玄珠名為一乘大寂滅揚真阿蘭若正修行處此是如來自到境界諸佛本住法門是以普勸後賢細垂詳覽遂得智窮性海學洞真源此識此心唯尊唯勝此識者十方諸佛之所證此心者一代時教之所詮唯尊者教理行果之所歸唯勝者信解證入之所趣諸賢依之而解釋論起千章眾聖體之以弘宣談成四辯所以掇奇提異研精洞微獨舉宏綱大張正綱撈摝五乘機地昇騰第一義天廣證此宗利益無盡遂得正法久住摧外道之邪林能令廣濟含生塞小乘之亂轍則無邪不正有偽皆空由自利故發智德之原由利他故立恩德之事成智德故則慈起無緣之化成恩德故則悲含同體之心以同體故則心起無心以無緣故則化成大化心起無心故則何樂而不與化成大化故則何苦而不收何樂而不與則利鈍齊觀何苦而不收則怨親普救遂使三草二木咸歸一地之榮邪種焦芽同霑一雨之潤斯乃盡善盡美無比無儔可謂括盡因門搜窮果海故得創發菩提之士初求般若之人了知成佛之端由頓圓無滯明識歸家之道路直進何疑或離此別修隨他妄解如𤚲角取乳緣木求魚徒歷三祇終無一得若依此旨信受弘持如快舸隨流無諸阻滯更遇便風之勢復加櫓棹之功則疾屆寶城忽登覺岸可謂資糧易辦道果先成披迦葉上行之衣坐釋迦法空之座登彌勒毘盧之閣入普賢法界之身能令客作賤人全領長者之家業忽使沈空小果頓受如來之記名未有一門匪通斯道必無一法不契此宗過去覺王因茲成佛未來大士仗此證真則何一法門而不開何一義理而不現無一色非三摩鉢地無一聲非陀羅尼門嘗一味而盡變醍醐聞一香而皆入法界風柯月渚並可傳心煙島雲林咸提妙旨步步踏金色之界念念嗅薝蔔之香掬滄海而已得百川到須彌而皆同一色煥兮開觀象之目盡復自宗寂爾導求珠之心俱還本法遂使邪山落仞苦海[(冰-水+〡)*ㄆ]波智檝以之安流妙峰以之高出今詳祖佛大意經論正宗削去繁文唯搜要旨假申問答廣引證明舉一心為宗照萬法如鏡編聯古製之深義撮略寶藏之圓詮同此顯揚稱之曰錄分為百卷大約三章先立正宗以為歸趣次申問答用去疑情後引真詮成其圓信以茲妙善普施含靈同報佛恩共傳斯旨耳。




標宗章第一
詳夫祖標禪理傳默契之正宗佛演教門立詮下之大旨則前賢所稟後學有歸是以先列標宗章為有疑故問以决疑故答謂此圓宗難信難解是第一之說若不假立言詮無以蕩其情執因指得月不無方便之門次立問答章欲堅信力須假證明廣引祖佛之誠言密契圓常之大道徧採經論之要旨圓成決定之真心後陳引證章以此三章通為一觀搜羅該括僃盡於茲矣問先德云若教我立宗定旨如龜上覓毛兔邊求角楞伽經偈云一切法不生不應立是宗何故標此章名答斯言遣滯若無宗之宗則宗說兼暢古佛皆垂方便門切不可執方便而迷大旨設有一解一悟皆是落後之事屬第二頭。
是故西天釋迦文佛云佛語心為宗無門為法門此土初祖達摩大師云以心傳心不立文字則佛佛手授授斯旨祖祖相傳傳此心。
終不率自胸襟違於佛語凡有釋疑去偽顯性明宗無不一一廣引經文備彰佛意所以永傳後嗣不墜家風若不然者又焉得至今紹繼昌盛法力如是證驗非虛又若欲研究佛乘披尋寶藏一一須消歸自己言言使冥合真心但莫執義上之文隨語生見直須探詮下之旨契會本宗則無師之智現前天真之道不昧如華嚴經云知一切法即心自性成就慧身不由他悟故知教有助道之力初心安可暫忘細詳法利無邊是乃搜揚纂集且凡論宗旨唯逗頓機如日出照高山駃馬見鞭影。
故首楞嚴經云圓明了知不因心念揚眉動目早是周遮。
今為樂佛乘人實未薦者假以宗鏡助顯真心雖挂文言妙旨斯在俯收中下盡被羣機但任當人各資己利於四門入處雖殊在一真見時無別。
豈同劣解凡情而生局見我此無礙廣大法門如虛空非相不拒諸相發揮似法性無身匪礙諸身頓現須以六相義該攝(總相別相同相異相成相壞相)斷常之見方消用十玄門融通(一同時具足相應門如海之一滴具百川味二廣狹自在無礙門如一尺之鏡見千里影三一多相容不同門如一室千燈光光涉入四諸法相即自在門如金與金色不相捨離五祕密隱顯俱成門如秋空片月晦明相並六微細相容安立門如琉璃之瓶盛多芥子七因陀羅網境界門如兩鏡互明傳耀相寫八託事顯法生解門如擎拳豎臂觸目皆道九十世隔法異成門如一夕之夢翱翔百年十主伴圓明具德門如北辰所居眾星皆拱此十玄門一一皆具十法同時具足一教義二理事三境智四行位五因果六依正七體用八人法九逆順十感應)去取之情始絕又若實得一聞千悟獲大總持即胡假言詮無勞解釋船筏為渡迷津之者導師因引失路之人凡關一切言詮於圓宗所示皆為未了文字性離即是解脫迷一切諸法真實之性向心外取法而起文字見者今還將文字對治示其真實若悟諸法本源即不見有文字及[糸*系]毫發現方知一切諸法即心自性則境智融通色空俱泯當此親證圓明之際入斯一法平等之時又有何法是教而可離何法是祖而可重何法是頓而可取何法是漸而可非則知皆是識心橫生分別。
故先德云一瞖在目千華亂空一妄在心恒沙生滅瞖除華盡妄滅證真病差藥除冰融水在神丹九轉點鐵成金至理一言點凡成聖狂心不歇歇即菩提鏡淨心明本來是佛。

問答章第二
問如上所標已知大意何用向下更廣開釋答上根利智宿習生知纔看題目宗之一字已全入佛智海中永斷纖疑頓明大旨則一言無不略盡攝之無有遺餘若直覽至一百卷終乃至恒沙義趣龍宮寶藏鷲嶺金文則殊說更無異途舒之徧周法界以前略後廣唯是一心本卷末舒皆同一際終無異旨有隔前宗都謂迷情妄興取捨唯見紙墨文字嫌卷軸多但執寂默無言欣為省要皆是迷心徇境背覺合塵不窮動靜之本原靡達一多之起處偏生局見唯懼多聞如小乘之怖法空似波旬之難眾善以不達諸法真實性故隨諸相轉墮落有無如大涅槃經云若人聞說大涅槃一字一句不作字相不作句相不作聞相不作佛相不作說相如是義者名無相相釋曰若云即文字無相是常見若云離文字無相是斷見又若執有相相亦是常見若執無相相亦是斷見但亡即離斷常四句百非一切諸見其旨自現當親現入宗鏡之時何文言識智之能詮述乎。
若明宗達性之者雖廣披尋尚不見一字之相終不作言詮之解以迷心作物者生斯紙墨之見耳故信心銘云六塵不惡還同正覺智者無為愚人自縛如斯達者則六塵皆是真宗萬法無非妙理何局於管見而迷於大旨耶。
斯乃無盡妙旨非淺智所知性起法門何劣解能覽燕雀焉測鴻鵠之志井蛙寧識滄海之淵如師子大哮吼狸不能為如香象所負擔驢不能勝如毘沙門寶貧不能等如金翅鳥飛烏不能及唯依情而起見但逐物而意移或說有而不涉空或言空而不該有或談略為多外之一或立廣為一外之多或離默而執言或離言而求默或據事外之理或著理外之事殊不能悟此自在圓宗演廣非多此是一中之多標略非一此是多中之一談空不斷斯乃即有之空論有不常斯乃即空之有或有說亦得此即默中說或無說亦得此即說中默或理事相即亦得此理是成事之理此事是顯理之事或理理相即亦得以一如無二如真性常融會或事事相即亦得此全理之事一一無礙或理事不即亦得以全事之理非事所依非能依不隱真諦故以全理之事非理能依非所依不壞俗諦故斯則存泯一際隱顯同時如闡普眼之法門皆是理中之義似舒大千之經卷非標心外之文故經云一法能生無量義非聲聞緣覺之所知。
華嚴經云自深入無自性真實法亦令他入無自性真實法心得安隱以茲妙達方入此宗。
但祖教並施定慧雙照自利利他則無過矣設有堅執己解不信佛言起自障心絕他學路今有十問以定紀綱還得了了見性如晝觀色似文殊否還逢緣對境見色聞聲舉足下足開眼合眼悉得明宗與道相應否還覽一代時教及從上祖師言句聞深不怖皆得諦了無疑否還因差別問難種種徵詰能具四辯盡決他疑否還於一切時一切處智照無滯念念圓通不見一法能為障礙未曾一剎那中暫令間斷否還於一切逆順好惡境界現前之時不為間隔盡識得破否還於百法明門心境之內一一得見微細體性根原起處不為生死根塵之所惑亂否還向四威儀中行住坐臥欽承祗對著衣喫飯執作施為之時一一辯得真實否還聞說有佛無佛有眾生無眾生或讚或毀或是或非得一心不動否還聞差別之智皆能明達性相俱通理事無滯無有一法不鑒其原乃至千聖出世得不疑否若實未得如是切不可起過頭欺誑之心生自許知足之意直須廣披至教博問先知徹祖佛自性之原到絕學無疑之地此時方可歇學灰息遊心或自辦則禪觀相應或為他則方便開示設不能徧叅法界廣究羣經但細看宗鏡之中自然得入此是諸法之要趣道之門。
所以誌公謌云六賊和光同塵無力大難推托內發解空無相大乘力能翻却唯在玄覽得旨之時可驗斯文究竟真實。
問諸佛境寂眾生界空有何因緣而興教迹答一實諦中雖無起盡方便門內有大因緣故法華經偈云諸法常無性佛種從緣起以萬法常無性無不性空時法爾能隨緣隨緣不失性。
如諸大菩薩所集唯識論等大意有其二種一為達萬法之正宗破二空之邪執二為斷煩惱所知之障證解脫菩提之門斯則自證法原本覺真地不在文字句義敷揚今為後學慕道之人方便纂集又自有二意用表本懷一為好略之人撮其樞要精通的旨免覽繁文二為執總之人不明別理微細開演性相圓通截二我生死之根躡一味菩提之道仰羣經之大旨直了自心遵諸聖之微言頓開覺藏去彼依通之見破其邪執之情深信正宗令知月不在指廻光返照使見性不徇文唯證相應斯為本意不可橫生知解沒溺見河於無得觀中懷趣向之意就真空理上與取捨之心率自胸襟疑誤後學須親見性方曉斯宗問既慮執指徇文又何煩集教答為背己合塵齊文作解者恐封教滯情故有此說若隨詮了旨即教明心者則有何取捨所以藏法師云自有眾生尋教得真會理教無礙常觀理而不礙持教恒誦習而不礙觀空則理教俱融合成一觀方為究竟傳通耳斯乃教觀一如詮旨同原矣問諸大經論自成片段科節倫序句義分明何假撮錄廣文成其要略答但以教海弘深窮之罔知其際義天高廣仰之不得其邊今則以管窺天將螺酌海如掬滄溟之涓滴似撮太華之一塵本為義廣難周情存厭怠亦為不依一乘教之正理唯徇不了義之因緣罕窮橫竪之門莫知起盡之處所以刪繁簡異採妙探玄雖文不足而大義全緣不備而正理顯搜盡一乘之旨抉開萬法之原為般若之玄樞作菩提之要路則資糧易辦速至大乘證入無疑免迂小徑。
今斯錄者雖無廣大製造之功微有一期述成之事亦知鈔錄前後文勢不全所冀直取要詮且明宗旨如從石辨玉似披沙揀金於羣藥中但取阿陀之妙向眾寶內唯探如意之珠舉一蔽諸以本攝末則一言無不略盡殊說更無異途所望後賢勿垂嗤誚所希斷疑生信但以見道為懷非徇虛名以邀世譽願盡未來之際徧窮法界之中歷劫逾生常弘斯道凡有心者皆入此宗去執除疑見聞獲益承三寶力加被護持誓報佛恩廣濟含識虛空可盡茲願匪移法界可窮斯文不墜問了義大乘廣略周備解一義具圓通之見聞一偈有成佛之功何假述成仍煩解釋答上上根人一聞千悟性相雙辯理事俱圓若中下之徒須假開演莊嚴之道讚飾之門格量其功不可為喻所以法華經偈云譬如優曇華一切皆愛樂天人所希有時時乃一出聞法歡喜讚乃至發一言則為已供養一切三世佛是人甚希有過於優曇華般若頌云般若無壞相過一切言語適無所依止誰能讚其德般若雖叵讚我今能得讚雖未脫死地則為已得出又古聖云若菩薩造論者名莊嚴經如蓮華未開見雖生喜不如已剖香氣芬馥如金未用見雖生喜不如用之為莊嚴具故知弘教一念之善能報十方諸佛之恩論希有則如華擅優曇之名說光揚則似金作莊嚴之具。
大涅槃經云佛言善男子除一闡提其餘眾生聞是經已悉皆能作菩提因緣法聲光明入毛孔者必定當得阿耨多羅三藐三菩提何以故若有人能供養恭敬無量諸佛方乃得聞大涅槃經薄福之人則不得聞故知得聞宗鏡所錄一心實相常住法門皆是曩結深因曾親佛會甚為大事非屬小緣若未聞熏曷由值遇又大涅槃經云佛告迦葉菩薩諸善男子善女人常當繫心修此二字佛是常住迦葉若有善男子善女人修此二字當知是人隨我所行至我至處是以信此法人即凡即聖修持契會住佛所住之中進止威儀行佛所行之跡釋摩訶衍論云若有眾生聞此摩訶衍之甚深極妙廣大法門已即其心中亦不疑畏亦不怯弱亦不輕賤亦不誹謗發決定心發堅固心發尊重心發愛信心當知是人真實佛子不斷法種不斷僧種不斷佛種常恒相續轉轉增長盡於未來亦為諸佛親所授記亦為一切無量菩薩之所護念故論云假使有人能化三千大千世界滿中眾生令行十善不如有人於一食頃正思此法過前功德不可為喻所以者何法身真如之功德等虛空界無邊際故。
故知信此心宗成摩訶衍同三世諸佛之所證義理何窮等十方菩薩之所乘功德無盡偶斯玄化慶幸逾深順佛旨而報佛恩無先弘法闡佛日而開佛眼只在明心此宗鏡中若得一句入神歷劫為種况正言深奧總一羣經此一乃無量中一若染此法即是圓頓之種可謂甘露入頂醍醐灌心耀不二之慧燈破情根之闇惑注一味之智水洗意地之妄塵能令厚障深遮若暴風之卷危葉繁疑積滯猶赫日之爍輕冰。
大智度論云三世諸佛皆以諸法實相為師祖師云一切明中心明為上。
故先德云剖微塵之經卷則念念果成盡眾生之願門則塵塵行滿未悟宗鏡焉信斯文所以昔人云遇斯教者應須自慶其猶溺巨海而遇芳[向-口+┴]墜長空而乘靈鶴矣。
問諸佛方便教門皆依眾生根起根性不等法乃塵沙云何惟立一心以為宗鏡答此一心法理事圓備為凡聖根本作迷悟元由諸門競入眾德攸歸如起信論云復次真實自體相者一切凡夫聲聞緣覺菩薩諸佛無有增減非前際生非後際滅常恒究竟從無始來本性具足一切功德謂大智慧光明義徧照法界義如實了知義本性清淨心義常樂我淨義寂靜不變自在義如是等過恒沙數非同非異不思議佛法無有斷絕依此義故名如來藏亦名法身問真如離一切相云何今說具足一切功德答雖實具有一切功德然無差別相彼一切法皆同一味一真離分別相無二性故以依業識等生滅相而立彼一切差別之相此云何立以一切法本來唯心實無分別以不覺故分別心起見有境界名為無明心性本淨無明不起即於真如立大智慧光明義若心生見境則有不見之相心性無見則無不見即於真如立徧照法界義若心有動則非真了知非本性清淨非常樂我淨非寂靜是變異不自在由是具起過於恒沙虛妄雜染以心性無動故即立真實了知義乃至過於恒沙清淨功德相義若心有起見有餘境可分別求則於內法有所不足以無邊功德即一心自性不見有餘法而更可求是故滿足過於恒沙非一非異不可思議諸佛之法無有斷絕故說真如名如來藏亦復名為如來法身然此一心非同凡夫妄認緣慮能推之心決定執在色身之內今徧十方世界皆是妙明真心如入法界品云華藏世界海中無問若山若河大地虛空草木叢林塵毛等處無不咸是一真法界具無邊德故先德云心也者冲虛妙粹炳煥靈明無去無來冥通三際非中非外朗徹十方不滅不生豈四山之可害離性離相奚五色之能盲處生死流驪珠獨耀於滄海踞涅槃岸桂輪孤朗於碧天大矣哉萬法資始也萬法虛偽緣會而生生法本無一切唯識識如幻夢但是一心心寂而知目之圓覺彌滿清淨中不容他故德用無邊皆同一性性起為相境智歷然相得性融身心廓爾方之海印越彼太虛恢恢焉晃晃焉逈出思議之表也。
若論一心性起功德無盡無邊豈以有量之心讚無為之德。
問教明一切萬法至理虛玄非有無之證絕自他之性若無一法自體云何立宗答若不立宗學何歸趣若論自他有無皆是眾生識心分別是對治門從相待有法身自體中實理心豈同幻有不隨幻無楞伽經云佛言大慧譬如非牛馬性牛馬性其實非有非無彼非無自相古釋云馬體上不得說牛性是有是無然非無馬自體以譬法身上不得說陰界入性是有是無然非無法身自相此法空之理超過有無即法身之性然有趣有向智背天真無得無歸情生斷滅但有之不用求真規宛爾無之自然足妙旨煥然則寂爾有歸恬然無間頓超能所不在有無可謂真歸能通至道矣問以心為宗如何是宗通之相答內證自心第一義理住自覺地入聖智門以此相應名宗通相此是行時非是解時因解成行行成解絕則言說道斷心行處滅如楞伽經云佛告大慧宗通者謂緣自得勝進相遠離言說文字妄想趣無漏界自覺地自相遠離一切虛妄覺想降伏一切外道眾魔緣自覺趣光明輝發是名宗通相所以悟心成祖先聖相傳故達摩大師云明佛心宗了無差悞行解相應名之曰祖。
問悟道明宗如人飲水冷暖自知云何說其行相答前已云諸佛方便不斷今時只為因疑故問因問故答此是本師於楞伽會上為十方諸大菩薩來求法者親說此二通一宗通二說通宗通為菩薩說通為童蒙祖佛俯為初機童蒙少垂開示此約說通只為從他覔法隨語生解恐執方便為真實迷於宗通是以分開二通之義。
當具眼人前若更說示則不得稱知時名為大法師實見月人終不觀指親到家者自息問程唯證相應不俟言說終不執指為月亦不離指見月如大涅槃經云善男子如彼眾盲不說象體亦非不說若是眾相悉非象者離是之外更無別象善男子或作是言色是佛性何以故如來色者常不斷故是說色名為佛性譬如真金質雖遷變色常不異或時作釧作盤然其黃色初無改易眾生佛性亦復如是質雖無常而色是常以是故說色為佛性乃至說受想行識等為佛性又有說言離陰有我我是佛性如彼盲人各各說象雖不得實非不說象說佛性者亦復如是非即六法不離六法善男子是故我說眾生佛性非色不離色乃至非我不離我善男子有諸外道雖說有我而實無我眾生我者即是五陰離陰之外更無別我善男子譬如莖葉鬚臺合為蓮華離是之外更無別華又佛言善男子是諸外道癡如小兒無慧方便不能了達常與無常苦與樂淨不淨我無我壽命非壽命眾生非眾生實非實有非有於佛法中取少許分虛妄計有常樂我淨而實不知常樂我淨如生盲人不知乳色。
善男子以是義故我佛法中有真實諦非於外道夫真實諦者宗鏡所歸未聞悟時不信解者所有說法及自修行皆成生滅折伏之門不入無生究竟之道如菴提遮女經云若不見生性雖因調伏少得安處其不安之相常為對治若能見生性者雖在不安之處而安相常現前若不如是知者雖有種種勝辯談說甚深典籍而即是生滅心說彼實相密要之言如盲辯色因他語故說得青黃赤白黑而不能自見色之正相當知大德空者亦不自得空故說有空義也故知能了萬法無生之性是為得道。
是以不了唯心之旨未入宗鏡之人向無生中起貪癡之垢於真空內著境界之緣以為對治成其輪轉若能返照心境俱寂如諸法無行經云若菩薩見貪欲際即是真際見瞋恚際即是真際見愚癡際即是真際則能畢滅業障之罪。
不思議佛境界經云爾時世尊復語文殊師利菩薩言童子汝能了知如來所住平等法否文殊師利菩薩言世尊我已了知佛言童子何者是如來所住平等法文殊師利菩薩言世尊一切凡夫起貪瞋癡處是如來所住平等法佛言童子云何一切凡夫起貪瞋癡處是如來所住平等法文殊師利菩薩言世尊一切凡夫於空無相無願法中起貪瞋癡是故一切凡夫起貪瞋癡處即是如來所住平等法佛言童子空豈是有法而言於中有貪瞋癡文殊師利菩薩言世尊空是有是故貪瞋癡亦是有佛言童子空云何有貪瞋癡復云何有文殊師利菩薩言世尊空以言說故有貪瞋癡亦以言說故有如佛說比丘有無生無起無作無為非諸行法此無生無起無作無為非諸行法非不有若不有者則於生起作為諸行之法應無出離以有故言出離耳此亦如是若無有空則於貪瞋癡無有出離以有故說離貪等諸煩惱耳中觀論偈云從法不生法亦不生非法從非法不生法及於非法直釋偈意法即是有如色心等非法是無如兔角等若從法生法如母生子法生非法如人生石女兒從非法生法如兔角生人從非法生非法如龜毛生兔角故般若假名論云復有念言若如來但證無所得者佛法即一非是無邊是故經言如來說一切法皆是佛法佛法謂何即無所得未曾一法有可得性是故一切無非佛法云何一切皆無所得經云一切法者即非一切法云何非耶無生性故若無生即無性云何名一切法於無性中假言說故一切法無有性者即是眾生如來藏性故知諸法從意成形千途因心有像一念澄寂萬境曠然元同不二之門盡入無生之旨。
又無生有二如通心論云一法性無生妙理言法至虛言性本來自爾名曰無生二緣起無生夫境由心現故不從他生心藉境起故不自生心境各異故不共生相因而有故不無因生亦云一理無生圓成實性本不生故二事無生緣生之相即無生故止觀云雖諸法不住以無住法住般若中即是入空以無住法住世諦即是入假以無住法住實相即是入中此無住慧即是金剛三昧能破盤石沙礫徹至本際又如釋迦牟尼入大寂定金剛三昧天親無著論開善廣解詎出無生無住之意若得此意千經萬論豁矣無疑此是學觀之初章思議之根本釋異之妙慧入道之指歸故知一切諸法皆從無生性空而有有而非有不離俗而常真非有而有不離真而恒俗則幻有立而無生顯空有歷然兩相泯而雙事存真俗宛爾斯則無生而無不生不住二邊矣如古德頌云無生終不住萬像徒流布若作無生解還被無生固若能知心無住則無有心既無有心亦無無心有無總無身心俱盡故泯齊萬境萬境無相合本一冥冥然玄照照無不寂以寂為體體無不虛虛寂無窮同通法界法界緣起無不自然來無所從去無所至又法無定相真妄由心起盡同原更無別旨正同宗鏡隱則一心無相顯則萬法標形不壞前後而同時常居一際而前後若依此一心無礙之觀念念即是華嚴法界念念即是毘盧遮那法界經云若與如是觀行相應於諸法中不生二解。
御錄宗鏡大綱卷一
音釋
懣
(莫本切門上聲煩懣)。
 駃
(居月切音厥)。
 眚
(生上聲目病生翳)。
 [(ㄇ@(企-止))/弟]
(音題兔網)。
 剞劂
(刻鏤刀)。
 檝
(音集橈名)。
 嗤
(音䲭笑也)。
 叵
(音頗不可也)。
