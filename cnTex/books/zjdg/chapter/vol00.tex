% -*- coding: utf-8 -*-

\clearpage
\section*{御錄宗鏡大綱序}

宗鏡錄者,永明壽禪師約舉佛祖大意,經論圓詮,刊落餘文,單標至要,
俾覺王所授之旨,大德相傳之心,到眼分明,指掌斯在。
語其少則不立一字,語其多則該遍恒沙。無一點一畫而非佛心,
無一言一句而異佛口。不二圓通之旨,與難思教海歷歷相應;
大千方便之門,皆無際真心重重交照。舉一心以立宗要,若天母之乳千兒;
攝萬法以歸鏡中,如瞳人之印來物。五千華藏,
一字中王,不減不增,無欠無剩。未來大士,句下自契靈源;博地凡夫,
開卷多聞妙法。即使未得未證,亦不離正位之中;
如其頓圓頓成,將永斷纖毫之惑。挽天河以注甘露,信口可嘗開龍藏。
以施寶珠任人探取,絕思絕議罕比罕儔。
利自利他無邊無量,誠為紹祖佛之真子,破魔外之將軍,救眾生之慈父,
教百世之宗師也。

朕讀茲書,良深嘉悅,是以付諸剞劂,散在香林蘭若之中,
復為述其指歸,弁\footnote{ 音biàn 〔~言〕书籍或长篇文章的序文、引言}
諸卍字靈文之首,欲宣朕諄切期望之意,普勸後賢,
特再頒丁寧訓諭之文,昭示來學矣。茲念古佛篡集教文,目為宗鏡,
其間曾宣金口深表本懷,謂一為好略之人,撮其樞要,精通的旨免覽繁文;
二為執總之人,不明別理,微細開演性相圓通,截二我生死之根,
躡一味菩提之道,仰羣經之大旨,直了自心,遵諸聖之微言,頓開覺藏。
去彼依通之見,破其邪執之情,深信正宗,方知月不在指,廻光返照,
便見性不徇文。此古佛悲含同體,慈起無緣,以虛空中之風,
吹蓮華上之水,滅大海裏之燄蘇,無根樹之苗者也,然而此書行世,
千有餘年,肉眼昏蒙,不知寶貴。固緣末世緇流多愚少慧,
亦以篇章浩瀚,意怠情煩。雖縮教海為一盂,而飲者腹猶易懣。
雖開義天於一線,而觀者目尚未周今。為好略者俯徇機宜,
如實垂示;為執總者明條要目直截區分。揀天龍女如意之珠,
更取如意中之如意,握金剛王無雙之劍,更求無雙內之無雙。
萬幾餘暇,乙夜繙\footnote{\textsuperscript{ 繙fān, 同“翻”}}披,親御鈆円,錄其綱骨,刊十存二,舉一蔽諸。
此乃過去法王助朕,不住於相之布施,所冀當來佛子,
同朕永弘斯道之深心,用述所懷明詔學者,爰為頒布安樂有情。
昔之本錄百卷,比此非繁,而今摘錄如干,較彼非簡。


猶夫五千教典入宗鏡而無餘,宗鏡百篇括教典而無剩。
然古佛述茲宗鏡,非令人置教典而不觀,則朕今刊此要文,
又豈令人置宗鏡而不閱。
作宗鏡者正為闚教典之梯航,則刪宗鏡者即為入宗鏡之嚮導矣。
惟願盡未來際,徧法界中,上上根人一聞千悟,中下之侶依正脩行,
庶幾不立纖塵,同遊斯鏡。
凡有心者皆入此宗,無一緣輕福薄而不得妙聞之人,
無一業重障深而不生圓信之者。慧日高騰於覺海,破長夜之昬衢。
德雲飛駕於性天,棄小乘之仄徑,同來廣濟於含識,用以仰報夫佛恩則。
此法利之普沾,長與虛空而無盡,古佛與朕所同願歟。

是為序。
