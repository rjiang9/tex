\chapter{再版跋}

今天包老师很欣慰,终于为lnotes2画上了句号。三年零三个月以前动笔的时候,老包又一次高估了自己的能力与热情。一百页的第一版用了不到半年的时间,两百页的第二版为什么会如此难产呢?包老师陷入了深深的思考。

Frederick P. Brooks在\emph{The Mythical Man-Month}中提到,一个系统的完成时间和系统组件数量的关系不是线性的,而是平方的。

如此说来,第二版因为长度加倍,完成时间应增加到四倍也就是两年。考虑到第一版的基础,时间会比两年短;而第一版的内容不能完全照搬,时间又会比一年半长。取个平均,lnote2的合理用时应该是21个月。

这多出来的18个月,包老师将其分别归功于自己的懒惰,工作的变化,和多了一个女儿。

在那本书里Brooks还指出,一个设计师的第二个系统往往是不靠谱的,因为他很可能会把第一个系统没有实现的东西一股脑都加入新系统。老包也犯了这个错误,还好他迷途知返,当发现篇幅可能远远超过两百页时,就把多余的东西都给砍了。

五年前写完lnote时,包老师曾经很期待\XeTeX{}和\LuaTeX{}。几年来,\XeTeX{}和\XeLaTeX{}果然不负厚望,已然取代了\TeX{}和\LaTeX{}。至于\LuaTeX{},好像还没有火起来。用C重写的模块,Lua的支持,还是很令人憧憬的。

另外恭喜一下milksea的新书《\LaTeX{}入门》和donated的博士毕业。这两位都为中国\TeX{}社区作出过很大的贡献,在本人学习\TeX{}的过程中也提供了无私的帮助。

\newpage
