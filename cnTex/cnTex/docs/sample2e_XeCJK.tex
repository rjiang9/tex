%This is a sample LaTeX input file.  (Version of 11 April 1994.)
%
% A '%' character causes TeX to ignore all remaining text on the line,
% and is used for comments like this one.

\documentclass{article}      % Specifies the document class

                             % The preamble begins here.
\title{An Example Document 中文標題}  % Declares the document's title.
\author{Leslie Lamport 林玉端}      % Declares the author's name.
\date{May 20, 2011}      % Deleting this command produces today's date.

\usepackage{fontspec} %加這個就可以設定字體 

 

\usepackage{xeCJK} %讓中英文字體分開設置
\setCJKmainfont{標楷體}
%\setCJKmainfont{HanWangGSolid06cut1} %設定中文為系統上的字型,而英文不去更動,使用原TeX字型
 % TTF 字型檔名: wtg-06cut1.ttf 超黑俏皮動物  PostScript 名: HanWangGSolid06cut1        CJK/FOP 字型名:hwgsc1

\XeTeXlinebreaklocale "zh" %這兩行一定要加,中文才能自動換行

\XeTeXlinebreakskip = 0pt plus 1pt %這兩行一定要加,中文才能自動換行


\begin{document}             % End of preamble and beginning of text.

\maketitle                   % Produces the title.
中文
This is an example input file.  Comparing it with
the output it generates can show you how to
produce a simple document of your own 中文.

\section{Text}      % Produces section heading.  Lower-level
                             % sections are begun with similar 
                             % \subsection and \subsubsection commands.


	\centering
		\begin{tabular}{lcc}
      \hline 
			 1 & 2 項目& 3 項次\\
			 4 & 5 項目 & 6 項次\\
			\hline
		\end{tabular}


\end{document}               % End of document.
