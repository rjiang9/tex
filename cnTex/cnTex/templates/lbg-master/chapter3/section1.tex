\section{定义全局布局}
我们将写个包含几页的示例文档。我们将在这个文档上练习修改边距,行距,页眉和
页脚等和其他更多的东西。
	\subsection{Time for action-写个包含章节的书}
我们将开始写一本书。首先,我们将选择一个类,更进一步使用一些填充
文本来试验页面布局。
\begin{lstlisting}[language={[LaTeX]TeX}]
\documentclass[a4paper,12pt]{book}
\usepackage[english]{babel}
\usepackage{blindtext}
\begin{document}
\chapter{Exploring the page layout}
In this chapter we will study the layout of pages.
\section{Some filler text}
\blindtext
\section{A lot more filler text}
More dummy text will follow.
\subsection{Plenty of filler text}
\blindtext[10]
\end{document}
\end{lstlisting}
我们选择的文档类型为book。如名字所指示,这个类适合于书一类的文档。书通常是
两面的,包含章节which usually start at right-hand pages。它们可以有一个前言
具有一个或多个标题页和一个back matter with bibliography,index等等。book类
支持所有这些。

我们加载了babel类。这对于非英语typesetting 非常有用。

即使对于英语,这里也有几个选项:

USenglish,american,
english,UKenglish,british,

canadian,australian,and newzealand.
显然,一些是相同的,

例如UKenglish和british。

然而,There are differences in hyphenation rules

between USenglish(american,
english) and UKenglish (british).

现在,我们只需要babel来加载blindtext:这个包被开发用来产生填充文本。它需要
babel来探测文档语言。我们向babel声明语言为english,意味这American English。

命令\textbackslash chapter产生一个大标题。这个命令总是开始于新的一页。

我们已经了解了\textbackslash section命令.
它是我们的二级标题,产生一个比

\textbackslash chapter小一点的标题。它自动为每个章节计数。

接着是\textbackslash
 blindtext,打印一些无意义的文本来填充空间。

最后,we refined the sectioning with a \textbackslash subsection 命令和更多的

