\documentclass[12pt,a4paper]{article}
\usepackage{fontspec,xunicode,xltxtra,lipsum,amsmath}
\usepackage{titlesec}
\usepackage[top=1in,bottom=1in,left=1.25in,right=1.25in]{geometry}
\usepackage{xcolor}
\usepackage{xunicode}
\usepackage{xltxtra}
\usepackage{polyglossia}
\usepackage{verbatim}
\usepackage{verbatimbox}
\usepackage{amsmath}      %数学符号包
\usepackage[colorlinks=true, linkcolor=blue ]{hyperref}         % 目录带连接

\usepackage{graphicx} 		                              % 插入图片
\usepackage{setspace}
\usepackage{indentfirst}                                % 段首缩进
\XeTeXlinebreaklocale "zh"    	           							% 中文断行
\XeTeXlinebreakskip = 0pt plus 1pt minus 0.1pt			    % 中文断行
\renewcommand{\baselinestretch}{1.25}

\defaultfontfeatures{Mapping=tex-text}

% \setdefaultlanguage{english}  		% 设置默认语言 english | chinese

% 设置字体
\setsansfont{Calibri}
\setmainfont[BoldFont=SimHei]{SimHei}

\titleformat{\section}{\Large\song}{\thesection}{1em}{}

% 新定义字体
\newfontfamily\shu{FZShuTi}                           % 方正舒体
\newfontfamily\nming{MingLiU_HKSCS} 		% 新细明体
\newfontfamily\lishu{LiSu} 								% 隶书
\newfontfamily\hei{SimHei} 								% 黑体
\newfontfamily\yahei{Microsoft YaHei} 			% 微软雅黑
\newfontfamily\henghei{Microsoft JhengHei}  % 微軟正黑體
\newfontfamily\song{SimSun}  			% 宋体
\newfontfamily\nsong{NSimSun}  		% 新宋体
\newfontfamily\yao{FZYaoTi}                % 方正姚体
\newfontfamily\ming{MingLiU} 			% 细明体
\newfontfamily\youyuan{YouYuan}		% 幼圆
\newfontfamily\kai{KaiTi}         			% 楷体
\newfontfamily\cai{STCaiyun}   			% 华文彩云


\usepackage{diagbox}                       %表格
\usepackage{enumitem}               % list
\setlist[enumerate]{itemsep=0mm}    % List Line space


\usepackage{paralist}                % compact list

\usepackage{listings} % needed for the inclusion of source code
\lstset{basicstyle = \ttfamily, frame      = single}

% \makeindex                             % 加索引
% \index{法华经}

\begin{document}

\title{\hei XeTeX使用小结}
\author{\song 江长里}
\date{\kai 2018年2月21日}
\maketitle

\tableofcontents
\newpage


%%%% 段落首行缩进两个字 %%%%
\makeatletter
\let\@afterindentfalse\@afterindenttrue
\@afterindenttrue
\makeatother
%\setlength{\parindent}{2em}  %中文缩进两个汉字位

\LaTeX


常用的环境及其作用。flushleft:环境中的内容居左。

flushright:环境中的内容居右。
center:环境中的内容居中。
itemize:无编号列表
enumerate:有编号列表
description:带描述列表
quote:引用,使得整段缩进
verse:诗歌专用,\\可以断行,两个空行的分段则生成一个空行。
\verb|C:\xxx\yyy\zzz|


\tiny tiny
\small small
\normalsize normalsize
\large large
\Large Large
\huge huge
\Huge Huge

表格

\begin{table}[hb] 
	\caption[短标题]{装模做样的表格}
	\begin{center}
      \begin{tabular}{|c|c|c|}
        \hline
        \multicolumn{3}{|c|}{如何追女孩的建议}\\
        \hline 
        \diagbox{财气}{建议}{颜值} & 帅& 挫\\
        \hline
        有钱花 & 没啥好说的 & 没啥好说的 \\
        \hline
        叮当响 & 没啥好说的 & 没啥好说的  \\
        \hline
      \end{tabular}
	\end{center}	
\end{table}

\\

懒得打字所以复制了很多遍。
	
%可以尝试换为其他的参数
\begin{table}[hb] 
	\caption[短标题]{装模做样的表格}
	\begin{center}
		\begin{tabular}{c}
			\hline
			这是一个表格\\
			\hline
			这是一个表格\\
			\hline
			这是一个表格\\
			\hline
		\end{tabular}
	\end{center}	
\end{table}



懒得打字所以复制了\large 很多遍。





\\

原文打印

\begin{verbatim}

\begin{verbatim}和 \ * {  } $ \ end{verbatim} 之间的东西直接打印。

\end{verbatim}

在段落中\verb+text+ 可将分隔符+之间的文本直接打印,其中+可以为除了*和空格的任意字符。

\begin{enumerate} [{第} i {、}]
 	\item 一二三四五
 	\item 上山打老虎
\end{enumerate} 


\begin{enumerate}
	\item 带数字的列表
	\item 列表可以嵌套
	\begin{itemize}
		\item 简单列表默认
		\item[+] 可以修改项目标记
		\begin{description}
			\item[标签] 标签列表的标签显示取决于环境
		\end{description}
	\end{itemize}
\end{enumerate}


\section[什么?]{特殊字符和符号}

引号

``双引号不能再用中文`输入法'输入''


破折号

中文一般不用连字号,除了电话号码.以及短破折``12--13页''这种表述长破折号`------`用于---犹豫等语气$的长度和他们不同

波浪号
省略号



\section{列出source code}
\lstinputlisting[language=Tex]{source.tex}

\section{List}

\begin{itemize}
	\item 工人
	\item 农民 
	\item 学生 
\end{itemize}
\lstinputlisting[language=Tex]{list.tex}


* How to make list items compact?

\begin{compactitem}
	\item 工人
	\item 农民 
	\item 学生 
\end{compactitem}

\lstinputlisting[language=Tex]{compactlist.tex}




\section{字体列表}
(默认字体) 以前使用CJK进行中文的排版,需要自己生成字体库,近日,出现了XeTeX,\yao (姚体)可以比较好的解决中文字体问题,不需要额外
生成LaTeX字体库,\kai (楷体)直接使用计算机 \henghei (微軟正黑體)系统里的字体。


本文使用了大量本机自带的字体。
%%%%%%%%%%%%%%%%%%%%%%%%
%%%%%%%%%%%%%%%%%%%%%%%%%%%%
\begin{table}[htbp]
\caption{字体列表 office station}

\centering
\begin{tabular}{|l|c|r|}
\hline
\hei 字体 & \hei 命令 & \hei 字体效果 \\
\hline

\kai 宋体 & \verb+\song+ & \song 宋体 \\
\kai 新宋体 & \verb+\nsong+ & \nsong 新宋体 \\
\kai 楷体 & \verb+\kai+ & \kai 楷体 \\
\kai 黑体 & \verb+\hei+ & \hei 黑体 \\
\kai 隶书 & \verb+\lishu+ & \lishu 隶书 \\
\kai 方正舒体 & \verb+\shu+ & \shu  方正舒体 \\
\kai 方正姚体 & \verb+\yao+ & \yao 方正姚体 \\
\kai 幼圆 & \verb+\youyuan+ & \youyuan 幼圆 \\
\kai 微软雅黑 & \verb+\yahei+ & \yahei 微软雅黑 \\
\kai 微軟正黑體 & \verb+\henghei+ & \henghei 微軟正黑體 \\
\kai 細明體 & \verb+\ming+ & \ming 細明體 \\
\kai 新细明体 & \verb+\nming+ & \nming 新细明体 \\
\kai 华文彩云 & \verb+\cai+ & \cai 华文彩云 \\


\hline
\end{tabular}
\end{table}





图片插入效果图:

\begin{figure}[htbp]
\centering    % 使后面的内容居中

		\includegraphics[width=155pt]{sea.jpg}

\caption{A 300 Hit!}
\label{fig:300}
\end{figure}





% \printindex                             % 印索引	
	
\end{document}

