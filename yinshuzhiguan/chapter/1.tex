% -*- coding: utf-8 -*-

%\part{部分标题}
%\chapter{卷二}这一章我们介绍这些内容。
%\section{节标题}这一节我们介绍这些内容。
%\subsection{小节标题}这一小节我们介绍这些内容。
%\subsubsection{子节标题}这一子节我们介绍这些内容。
%\paragraph{段标题}这一段我们介绍这些内容。
%\subparagraph{小段标题}这一小段我们介绍这些内容。
% 
% {\kaishu 这里是楷体显示},
% {\songti 这里是宋体显示},
% {\heiti 这里是黑体显示},
% {\fangsong 这里是仿宋显示},
% {\lishu 这里是隶书显示},
% {\youyuan 这里是幼圆显示}。

\chapter{宗鏡錄卷第一}

詳夫!祖標禪理,傳默契之正宗;佛演教門,立詮下之大旨。則前賢所稟,後學有歸。是以先列《標宗章》。為有疑故問,以決疑故答。因問而疑情得啟,因答而妙解潛生。謂此圓宗難信難解,是第一之說,備最上之機。若不假立言詮,無以蕩其情執。因指得月,不無方便之門;獲兔忘罤,自合天真之道。次立《問答章》。但以時當末代,罕遇大機,觀淺心浮根微智劣,雖知宗旨的有所歸,問答決疑漸消惑障。欲堅信力,須假證明。廣引祖佛之誠言,密契圓常之大道,遍采經論之要旨,圓成決定之真心。後陳《引證章》。以此三章,通為一觀。搜羅該括,備盡於茲矣。 


{\heiti 【問】先德云:若教我立宗定旨,如龜上覓毛,兔邊求角。《楞伽經》偈云:一切法不生,不應立是宗。何故標此章名?}

【答】斯言遣滯!若無宗之宗,則宗說兼暢。古佛皆垂方便門,禪宗亦開一線道。切不可執方便而迷大旨,又不可廢方便而絕後陳。然機前無教,教後無實。設有一解一悟,皆是落後之事,屬第二頭。所以《大智度論》云:

{\kaishu 以佛眼觀一切十方國土中一切物,尚不見無,何況有法?畢竟空法,能破顛倒,令菩薩成佛。是事尚不可得,何況凡夫顛倒有法?}

今依祖佛言教之中,約今學人,隨見心性發明之處,立心為宗。是故西天釋迦文佛云:佛語心為宗,無門為法門。此土初祖達磨大師云:{\kaishu 以心傳心,不立文字。則佛佛手授,授斯旨;祖祖相傳,傳此心}。已上約祖佛所立宗旨,又諸賢聖所立宗體者。杜順和尚依《華嚴經》,{\kaishu 立自性清淨圓明體,此即是如來藏中法性之體}。從本已來,性自滿足,處染不垢,修治不淨,故雲自性清淨;性體遍照,無幽不矚,故曰圓明。又隨流加染而不垢,返流除染而不淨。亦可在聖體而不增,處凡身而不減。雖有隱顯之殊,而無差別之異。煩惱覆之則隱,智慧了之則顯。非生因之所生,唯了因之所了。斯即一切眾生自心之體。靈知不昧,寂照無遺。非但華嚴之宗,亦是一切教體。《佛地論》,立一清淨法界體。論云:清淨法界者,一切如來真實自體。無始時來,自性清淨,具足種種過十方界極微塵數性相功德,無生無滅,猶如虛空,遍一切有情。平等共有,與一切法,不一不異,非有非無。離一切相,一切分別,一切名言,皆不能得。唯是清淨聖智所證,二空無我所顯,真如為其自性,諸聖分證,諸佛圓證。此清淨法界,即真如妙心,為諸佛果海之源,作群生實際之地,此皆是立宗之異名,非別有體。或言宗者,尊也,以心為宗,
故云:{\kaishu 天上天下,唯我獨尊。}或言體者,性也,以心為體,
故云:{\kaishu 知一切法,即心自性}。或言智者,以心為智,即是本性寂照之用,
所以云:{\kaishu 自覺聖智,普光明智等}。若約義用而分,則體宗用別者會歸平等,則一道無差。所以《華嚴記》問云:等妙二位,全同如來普光明智者。結成入普,所以,此會說等妙二覺,二覺全同普光明智,即是會歸之義。
 
{\heiti 【問】等覺同妙覺,於理可然。妙覺之外,何有如來普光明智?為所同耶? }

【答】說等覺,說妙覺,即是約位。普光明智,不屬因果,該通因果。其由自覺聖智超絕因果。故《楞伽經》妙覺位外,更立自覺聖智之位。亦猶佛性有因,有果,有因因,有果果。以因取之,是因佛性;以果取之,是果佛性。然則佛性非因非果。普光明智,亦復如是。體絕因果,為因果依,果方究竟,故雲如來。普光明智,或稱為本者,以心為本。
故《涅槃疏》云:{\kaishu 涅槃宗本者,諸行皆以大涅槃心為本,本立道生。如無綱目不立,無皮毛靡附。心為本故,其宗得立}。 


{\heiti 【問】若欲明宗,只合純提祖意,何用兼引諸佛菩薩言教,以為指南?故宗門中云:借蝦為眼,無自己分,只成文字聖人,不入祖位。 }

【答】從上非是一向不許看教,恐慮不詳佛語,隨文生解,失於佛意,以負初心。或若因詮得旨,不作心境對治,直了佛心,又有何過?只如藥山和尚,一生看《大涅槃經》,手不釋卷。

時有學人問:「和尚尋常不許學人看經,和尚為什麼自看?」

師云:「只為遮眼。」

問:「學人還看得不?」

師云:「汝若看,牛皮也須穿。」

且如西天第一祖師,是本師釋迦牟尼佛,首傳摩訶迦葉為初祖,次第相傳。迄至此土六祖,皆是佛弟子。今引本師之語,訓示弟子,令因言薦道,見法知宗。不外馳求,親明佛意。得旨即入祖位,誰論頓漸之門?見性現證圓通,豈標前後之位?若如是者,何有相違?且如西天上代二十八祖、此土六祖,乃至洪州馬祖大師及南陽忠國師、鵝湖大義禪師、思空山本淨禪師等,並博通經論,圓悟自心,所有示徒,皆引誠證,終不出自胸臆,妄有指陳。是以綿曆歲華,真風不墜。以聖言為定量,邪偽難移;用至教為指南,依憑有據。故圭峰和尚云:謂諸宗始祖,即是釋迦。經是佛語,禪是佛意。諸佛心口,必不相違。諸祖相承根本,是佛親付;菩薩造論始末,唯弘佛經。況迦葉乃至鞠多弘傳,皆兼三藏。及馬鳴龍樹,悉是祖師,造論釋經,數十萬偈,觀風化物,無定事儀,所以凡稱知識,法爾須明佛語,印可自心。若不與了義一乘圓教相應,設證聖果,亦非究竟。今且錄一二以證斯文。
洪州馬祖大師云:{\kaishu 達磨大師從南天竺國來,唯傳大乘一心之法,以《楞伽經》印眾生心,恐不信此一心之法}。
《楞伽經》云:{\kaishu 佛語心為宗,無門為法門}。何故佛語心為宗?佛語心者,即心即佛,今語即是心語,
故云:{\kaishu 佛語心為宗}。無門為法門者,達本性空,更無一法,性自是門,性無有相,亦無有門,
故云:{\kaishu 無門為法門,亦名空門,亦名色門}。何以故?空是法性空,色是法性色。無形相故,謂之空;知見無盡故,謂之色。
故云:{\kaishu 如來色無盡,智慧亦復然,隨生諸法處}。復有無量三昧門,遠離內外知見情執,亦名總持門,亦名施門。謂不念內外善惡諸法,乃至皆是諸波羅蜜門。色身佛,是實相佛家用。
經云:{\kaishu 三十二相,八十種好,皆從心想生,亦名法性家焰,亦法性功勳}。菩薩行般若時,火燒三界內外諸物盡,於中不損一草葉,為諸法如相故。
故經云:{\kaishu 不壞於身而隨一相}。今知自性是佛,於一切時中行住坐臥,更無一法可得,乃至真如不屬一切名,亦無無名。
故經云:{\kaishu 智不得有無}。內外無求,任其本性,亦無任性之心。
經云:{\kaishu 種種意生身,我說為心量}。即無心之心,無量之量。無名為真名,無求是真求。
經云:{\kaishu 夫求法者,應無所求}。心外無別佛,佛外無別心。不取善,不作惡,淨穢兩邊俱不依。法無自性,三界唯心。
經云:{\kaishu 森羅及萬像,一法之所印}。凡所見色,皆是見心。心不自心,因色故心;色不自色,因心故色。
故經云:{\kaishu 見色即是見心}。
南陽忠國師云:{\kaishu 禪宗法者,應依佛語一乘了義,契取本原心地,轉相傳授,與佛道同。不得依於妄情及不了義教,橫作見解,疑誤後學,俱無利益}。縱依師匠領受宗旨,若與了義教相應,即可依行;若不了義教,互不相許,譬如師子身中蟲,自食師子身中肉,非天魔外道,而能破滅佛法矣。

時有禪客問曰:「阿那個是佛心?」

師曰:「牆壁瓦礫,無情之物,並是佛心。」

禪客曰:「與經大相違也。經云:{\kaishu 離牆壁瓦礫,無情之物,名為佛性。}今雲一切無情之物皆是佛心,未審心之與性,為別不別?」

師曰:「迷人即別,悟人不別。」

禪客曰:「與經又相違也。經云:{\kaishu 善男子,心非佛性,佛性是常,心是無常}。今雲不別,未審此意如何?」

師曰:「汝自依語不依義。譬如寒月結水為冰,及至暖時釋冰成水。眾生迷時結性成心,悟時釋心成性。汝定執無情之物非心者,經不應言三界唯心。
故《華嚴經》云:{\kaishu 應觀法界性,一切唯心造。}今且問汝:無情之物,為在三界內?為在三界外?為復是心不是心?若非心者,經不應言三界唯心。若是心者,又不應言無性。汝自違經,我不違也。」

鵝湖大義禪師,因詔入內,遂問京城諸大師、大德:「汝等以何為道?」

或有對云:「知見為道。」

師云:「《維摩經》云:法離見聞覺知,雲何以知見為道?」

又有對云:「無分別為道。」

師云:「經云:善能分別諸法相,於第一義而不動。雲何以無分別為道?」

又皇帝問:「如何是佛性?」

答:「不離陛下所問。」是以或直指明心,或破執入道。以無方之辯,祛必定之執,運無得之智,屈有量之心。思空山本淨禪師,語京城諸大德云:「汝莫執心,此心皆因前塵而有。如鏡中像,無體可得。若執實有者,則失本原,常無自性。」《圓覺經》云:{\kaishu 妄認四大為自身相,六塵緣影為自心相}。《楞伽經》云:{\kaishu 不了心及緣,則生二妄想;了心及境界,妄想則不生}。
《維摩經》云:{\kaishu 法非見聞覺知}。且引三經,證斯真實。五祖下莊嚴大師,一生示徒,唯舉《維摩經·寶積長者贊佛頌》末四句,云:{\kaishu 不著世間如蓮華,常善入於空寂行,達諸法相無掛礙,稽首如空無所依}。

學人問云:「此是佛語,欲得和尚自語。」

師云:「佛語即我語,我語即佛語。」

是故初祖西來,創行禪道,欲傳心印,須假佛經,以《楞伽》為證明,知教門之所自。遂得外人息謗,內學稟承,祖胤大興,玄風廣被。是以初心始學之者,未自省發已前,若非聖教正宗,憑何修行進道?設不自生妄見,亦乃盡值邪師。故云:我眼本正,因師故邪。西天九十六種執見之徒,皆是斯類。故知木匪繩而靡直,理非教而不圓。如上略引二三,皆是大善知識,物外宗師,禪苑麟龍,祖門龜鏡。示一教而風行電卷,垂一語而山崩海枯。帝王親師,朝野歸命,叢林取則,後學稟承。終不率自胸襟,違於佛語。凡有釋疑去偽,顯性明宗,無不一一廣引經文,備彰佛意,所以永傳後嗣,不墜家風。若不然者,又焉得至今紹繼昌盛,法力如是,證驗非虛!又若欲研究佛乘,披尋寶藏,一一須消歸自己,言言使冥合真心。但莫執義上之文,隨語生見,直須探詮下之旨,契會本宗,則無師之智現前,天真之道不昧。如《華嚴經》云:知一切法,即心自性,成就慧身,不由他悟。故知教有助道之力,初心安可暫忘?細詳法利無邊,是乃搜揚纂集。且凡論宗旨,唯逗頓機,如日出照高山,駃馬見鞭影。所以丹霞和尚云:「相逢不擎出,舉意便知有。」如今《宗鏡》,尚不待舉意,便自知有。
故《首楞嚴經》云:{\kaishu 圓明瞭知,不因心念,揚眉動目,早是周遮}。
如先德頌云:{\kaishu 便是猶倍句,動目即差違,若問曹溪旨,不更待揚眉}。今為樂佛乘人,實未薦者,假以宗鏡,助顯真心,雖掛文言,妙旨斯在。俯收中下,盡被群機,但任當人,各資己利。百川雖潤,何妨大海廣含?五嶽自高,不礙太陽普照。根機莫等,樂欲匪同,於四門入處雖殊,在一真見時無別。如獲鳥者羅之一目,不可以一目為羅;理國者功在一人,不可以一人為國。
如《內德論》云:{\kaishu 夫一水無以和羹,一木無以構室,一衣不稱眾體,一藥不療殊疾,一彩無以為文繡,一聲無以諧琴瑟,一言無以勸眾善,一戒無以防多失,何得怪漸頓之異,令法門之專一}?
故云:{\kaishu 如為一人,眾多亦然;如為眾多,一人亦然。豈同劣解凡情,而生局見}?我此無礙廣大法門,如虛空非相,不拒諸相發揮,似法性無身,匪礙諸身頓現。須以六相義該攝,斷常之見方消;用十玄門融通,去取之情始絕。又若實得一聞千悟,獲大總持,即胡假言詮,無勞解釋。船筏為渡迷津之者,導師因引失路之人。凡關一切言詮,於圓宗所示,皆為未了,文字性離,即是解脫。迷一切諸法真實之性,向心外取法,而起文字見者,今還將文字對治,示其真實;若悟諸法本源,即不見有文字。及絲毫發現,方知一切諸法,即心自性,則境智融通,色空俱泯。當此親證圓明之際,入斯一法平等之時,又有何法是教而可離?何法是祖而可重?何法是頓而可取?何法是漸而可非?則知皆是識心,橫生分別。所以祖佛善巧,密佈權門,廣備教乘,方便逗會,才得見性,當下無心,乃藥病俱消,教觀鹹息。
如《楞伽經》偈云:{\kaishu 諸天及梵乘,聲聞緣覺乘,諸佛如來乘,我說此諸乘,乃至有心轉,諸乘非究竟。若彼心滅盡無乘及乘者,無有乘建立,我說為一乘,引導眾生故,分別說諸乘}。
故先德云:{\kaishu 一翳在目,千華亂空;一妄在心,恆沙生滅。翳除華盡,妄滅證真;病差藥除,冰融水在。神丹九轉,點鐵成金;至理一言,轉凡成聖。狂心不歇,歇即菩提;鏡淨心明,本來是佛}。


{\heiti 【問】如上所標,已知大意,何用向下更廣開釋?}

【答】上根利智,宿習生知,才看題目「宗」之一字,已全入佛智海中,永斷纖疑,頓明大旨,則一言無不略盡,攝之無有遺餘。若直覽至一百卷終,乃至恆沙義趣,龍宮寶藏,鷲嶺金文,則殊說更無異途,舒之遍周法界。以前略後廣,唯是一心,本卷末舒,皆同一際,終無異旨有隔前宗。都謂迷情妄興取捨,唯見紙墨文字,嫌卷軸多,但執寂默無言,欣為省要,皆是迷心徇境,背覺合塵。不窮動靜之本原,靡達一多之起處,偏生局見,唯懼多聞。如小乘之怖法空,似波旬之難眾善,以不達諸法真實性故,隨諸相轉,墮落有無。如《大涅槃經》云:若人聞說大涅槃一字一句,不作字相,不作句相,不作聞相,不作佛相,不作說相,如是義者,名無相相。釋曰:若雲即文字無相,是常見;若雲離文字無相,是斷見。又若執有相相,亦是常見;若執無相相,亦是斷見。但亡即離斷常,四句百非,一切諸見,其旨自現。當親現入《宗鏡》之時,何文言識智之能詮述乎?所以先德云:若覓經,了性真如無可聽;若覓法,雞足山間問迦葉。大士持衣在此山,無情不用求專甲。斯則豈可運見聞覺知之心,作文字句義之解?若明宗達性之者,雖廣披尋,尚不見一字之相,終不作言詮之解。以迷心作物者,生斯紙墨之見耳。
故《信心銘》云:{\kaishu 六塵不惡,還同正覺,智者無為,愚人自縛}。如斯達者,則六塵皆是真宗,萬法無非妙理。何局於管見,而迷於大旨耶?豈知諸佛廣大境界,菩薩作用之門?所以大海龍王,置十千之問;釋迦文佛,開八萬勞生之門;普慧菩薩,申二百之疑;普賢大士,答二千樂說之辯。
如《華嚴經》普眼法門:{\kaishu 假使有人以大海量墨,須彌聚筆,寫於此普眼法門,一品中一門,一門中一法,一法中一義,一義中一句,不得少分,何況能盡?}
又如《大涅槃經》中佛言:{\kaishu 我所覺了一切諸法,如因大地生草木等,為諸眾生所宣說者,如手中葉。}只如已所說法,教溢龍宮,龍樹菩薩,暫看有一百洛叉,出在人間,於西天尚百分未及一,翻來東土,故不足言,豈況未所說法耶?斯乃無盡妙旨,非淺智所知;性起法門,何劣解能覽?燕雀焉測鴻鵠之志,井蛙寧識滄海之淵?如師子大哮吼,狸不能為;如香象所負擔,驢不能勝;如毗沙門寶,貧不能等;如金翅鳥飛,烏不能及。唯依情而起見,但逐物而意移。或說有而不涉空,或言空而不該有,或談略為多外之一,或立廣為一外之多,或離默而執言,或離言而求默,或據事外之理,或著理外之事,殊不能悟此自在圓宗。演廣非多,此是一中之多;標略非一,此是多中之一。談空不斷,斯乃即有之空;論有不常,斯乃即空之有。或有說亦得,此即默中說;或無說亦得,此即說中默。或理事相即亦得,此理是成事之理,此事是顯理之事;或理理相即亦得,以一如無二如,真性常融會;或事事相即亦得,此全理之事,一一無礙;或理事不即亦得;以全事之理非事,所依非能依,不隱真諦故,以全理之事非理,能依非所依,不壞俗諦故。斯則存泯一際,隱顯同時。如闡普眼之法門,皆是理中之義;似舒大千之經卷,非標心外之文。故經云:一法能生無量義,非聲聞緣覺之所知,不同但空孤調之詮,偏枯決定之見。今此無盡妙旨,標一法而眷屬隨生;圓滿性宗,舉一門而諸門普會。非純非雜,不一不多。如五味和其羹,雜彩成其繡,眾寶成其藏,百藥成其丸。邊表融通,義味周足,搜微抉妙,盡《宗鏡》中。依正混融,因果無礙;人法無二,初後同時。凡舉一門,皆能圓攝無盡法界,非內非外,不一不多。舒之則涉入重重,卷之則真門寂寂。如《華嚴經》中,師子座中,莊嚴具內,各出一佛世界塵數菩薩身雲,此是依正人法無礙。又如佛眉間出勝音等佛世界塵數菩薩,此是因果初後無礙。乃至剎土微塵,各各具無邊智德;毛孔身份,一一攝廣大法門。何故如是奇異難思?乃一心融即故爾。以要言之,但一切無邊差別佛事,皆不離無相真心而有。
如《華嚴經》頌云:{\kaishu 佛住甚深真法性,寂滅無相同虛空。}而於第一實義中,示現種種所行事,所作利益眾生事,皆依法性而得有,相與無相無差別,入於究竟皆無相。
又《攝大乘論》頌云:{\kaishu 即諸三摩地,大師說為心,由心彩畫故,如所作事業。}故知凡聖所作,真俗緣生。此一念之心,剎那起時,即具三性三無性六義。謂一念之心,是緣起法,是依他起。情計有實,即是遍計所執。體本空寂,即是圓成,即依三性說三無性,故六義具矣。若一念心起,具斯六義,即具一切法矣。以一切真俗萬法,不出三性三無性故。
《法性論》云:{\kaishu 凡在起滅,皆非性也。}起無起性故,雖起而不常,滅無滅性,雖滅而不斷,如其有性,則陷於四見之網。
又云:{\kaishu 尋相以推性,見諸法之無性;尋性以求相,見諸法之無相。}是以性相互推,悉皆無性。是以若執有性,墮四見之邪林;若了性空,歸一心之正道。
故《華嚴經》云:{\kaishu 自深入無自性真實法,亦令他入無自性真實法。}心得安隱,以茲妙達,方入此宗,則物物冥真,言言契旨。若未親省,不發圓機,言之則乖宗,默之又致失,豈可以四句而取六情所知歟?但祖教並施,定慧雙照,自利利他,則無過矣。設有堅執己解,不信佛言,起自障心,絕他學路。今有十問以定紀綱:還得了了見性,如晝觀色,似文殊等不?還逢緣對境,見色聞聲,舉足下足,開眼合眼,悉得明宗,與道相應不?還覽一代時教,及從上祖師言句,聞深不怖,皆得諦了無疑不?還因差別問難,種種征詰,能具四辯,盡決他疑不?還於一切時一切處,智照無滯,唸唸圓通,不見一法能為障礙,未曾一剎那中暫令間斷不?還於一切逆順好惡境界現前之時,不為間隔,盡識得破不?還於百法明門心境之內,一一得見微細體性根原起處,不為生死根塵之所惑亂不?還向四威儀中行住坐臥,欽承祗對,著衣吃飯,執作施為之時,一一辯得真實不?還聞說有佛無佛,有眾生無眾生,或贊或毀,或是或非,得一心不動不?還聞差別之智,皆能明達,性相俱通,理事無滯,無有一法不鑒其原,乃至千聖出世,得不疑不?若實未得如是功,不可起過頭欺誑之心,生自許知足之意。直須廣披至教,博問先知,徹祖佛自性之原,到絕學無疑之地,此時方可歇學灰息遊心,或自辦則禪觀相應,或為他則方便開示。設不能遍參法界,廣究群經,但細看《宗鏡》之中,自然得入。此是諸法之要,趣道之門。如守母以識子,得本而知末。提綱而孔孔皆正,牽衣而縷縷俱來。又如以師子筋為琴弦,音聲一奏,一切餘弦悉皆斷壞。此《宗鏡》力,亦復如是。舉之而萬類沈光,顯之而諸門泯跡。以此一則,則破千途,何須苦涉關津,別生岐路?所以志公歌云:六賊和光同塵,無力大難推託,內發解空無相,大乘力能翻卻。唯在玄覽得旨之時,可驗斯文究竟真實。 

宗鏡錄卷第一  丙午歲分司大藏都監開板 
